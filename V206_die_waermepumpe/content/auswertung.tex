\section{Auswertung}
\label{sec:Auswertung}
Die Messreihen für die Temperaturen, Drücke und die Leistungsaufnahme $P$ des Kompressors
in Abhängigkeit der Zeit sind in Tabelle \ref{tab:messwerte} dargestellt.

\begin{table}[H]
  \centering
  \caption{Messwerte und aus diesen berechnete Werte}
  \label{tab:messwerte}
  \begin{tabular}{c c c c c c c c}
    \toprule
    $t/$s & $T_\text{warm}/$K & $T_\text{kalt}/$K & $p_warm/(10^{5}\symup{Pa})$ & $p_kalt/(10^{5}\symup{Pa})$ &
    $P$/W & $1/T_1/\left(10^{-3}\symup{\frac{1}{K}}\right)$ & $\ln\left(\frac{p_b}{p_0}\right)$ \\
    \midrule
      60  &  295.5 &  295.4 & 4.6  &  6.6 & 120 & 3.4 & 1.8586 \\
     120  &  296.8 &  294.8 & 4.6  &  6.6 & 125 & 3.4 & 1.8739 \\
     180  &  297.9 &  293.8 & 4.8  &  7.0 & 127 & 3.4 & 1.8739 \\
     240  &  299.2 &  292.5 & 4.8  &  7.1 & 129 & 3.4 & 1.9327 \\
     300  &  300.5 &  291.5 & 4.8  &  7.4 & 129 & 3.3 & 1.9469 \\
     360  &  301.8 &  290.6 & 4.7  &  7.7 & 128 & 3.3 & 1.9883 \\
     420  &  303.0 &  289.8 & 4.0  &  7.9 & 125 & 3.3 & 2.0281 \\
     480  &  304.3 &  289.0 & 4.5  &  8.0 & 125 & 3.3 & 2.0537 \\
     540  &  305.4 &  288.3 & 4.4  &  8.4 & 125 & 3.3 & 2.0663 \\
     600  &  306.6 &  287.8 & 4.4  &  8.6 & 125 & 3.3 & 2.1151 \\
     660  &  307.8 &  287.0 & 4.2  &  8.8 & 125 & 3.3 & 2.1386 \\
     720  &  308.8 &  286.3 & 4.2  &  9.0 & 125 & 3.2 & 2.1616 \\
     780  &  309.8 &  285.8 & 4.1  &  9.1 & 125 & 3.2 & 2.1841 \\
     840  &  310.8 &  285.1 & 4.1  &  9.4 & 125 & 3.2 & 2.1951 \\
     900  &  311.8 &  284.5 & 4.0  &  9.6 & 125 & 3.2 & 2.2275 \\
     960  &  312.6 &  283.9 & 4.0  &  9.9 & 125 & 3.2 & 2.2486 \\
    1020  &  313.5 &  283.4 & 3.9  & 10.0 & 125 & 3.2 & 2.2794 \\
    1080  &  314.4 &  282.9 & 3.9  &  9.1 & 125 & 3.2 & 2.2894 \\
    1140  &  315.2 &  282.4 & 3.8  & 10.4 & 125 & 3.2 & 2.3319 \\
    1200  &  316.0 &  282.0 & 3.8  & 10.8 & 125 & 3.2 & 2.3286 \\
    1260  &  316.8 &  281.9 & 3.8  & 10.8 & 125 & 3.2 & 2.3664 \\
    1320  &  317.6 &  281.1 & 3.8  & 11.0 & 125 & 3.2 & 2.3664 \\
    1380  &  318.3 &  280.8 & 3.8  & 11.0 & 125 & 3.1 & 2.3847 \\
    1440  &  319.0 &  280.5 & 3.7  & 11.2 & 125 & 3.1 & 2.3847 \\
    1500  &  319.8 &  280.1 & 3.7  & 11.5 & 125 & 3.1 & 2.4028 \\
    1560  &  320.3 &  279.8 & 3.7  & 11.6 & 125 & 3.1 & 2.4292 \\
    1620  &  321.0 &  279.5 & 3.7  & 11.9 & 125 & 3.1 & 2.4378 \\
    1680  &  321.6 &  279.3 & 3.6  & 12.0 & 125 & 3.1 & 2.4634 \\
    1740  &  322.2 &  279.0 & 3.6  & 12.0 & 125 & 3.1 & 2.4717 \\
    1800  &  322.8 &  278.8 & 3.6  & 12.1 & 125 & 3.1 & 2.4800 \\
    \bottomrule
  \end{tabular}
\end{table}

Außerdem sind bereits berechnete Werte,
die später in der Auswertung noch benötigt werden, zu sehen. Als Einheiten werden
die üblichen Einheiten des SI-Einheitensystems verwendet und die Größen, wenn nötig,
umgerechnet.

\subsection{Graphische Darstellung der Temperaturverläufe}
Die gemessenen Werte der Temperaturen werden in \ref{fig:tempgraphen} graphisch dargestellt.

%\begin{figure}
%  \centering
%  \includegraphics{data/HIERDENDATEINAMENEINFUEGEN.pdf}
%  \caption{Darstellung der gemessenen Temperaturen und Graphen der Ausgleichsfunktion}
%  \label{fig:tempgraphen}
%\end{figure}
