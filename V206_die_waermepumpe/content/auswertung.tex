\section{Auswertung}
\label{sec:Auswertung}
Die Messreihen für die Temperaturen, Drücke und die Leistungsaufnahme $P$ des Kompressors
in Abhängigkeit der Zeit sind in Tabelle \ref{tab:messwerte} dargestellt. Dabei
werden die Werte für die Zeit in Sekunden und die Werte für die Temperaturen in Kelvin
umgerechnet. Zu den gemessenen Werten für die Drücke $p_{\symup{a}}$ und
$p_{\symup{a}}$ wird noch der Umgebungsdruck von $\SI{10e5}{\pascal}$ hinzuaddiert.

\begin{table}[H]
  \centering
  \caption{Messwerte und aus diesen berechnete Werte}
  \label{tab:messwerte}
  \begin{tabular}{c c c c c c c c}
    \toprule
    $t/$s & $T_{\text{warm}}/$K & $T_{\text{kalt}}/$K & $p_{\text{kalt}}/(10^{5}\symup{Pa})$ & $p_{\text{warm}}/(10^{5}\symup{Pa})$ &
    $P$/W & $1/T_1/\left(10^{-3}\symup{\frac{1}{K}}\right)$ & $\ln\left(\frac{p_b}{p_0}\right)$ \\
    \midrule
    0	    &  293.35	 & 	293.45 	&	 5.0	&  4.6   &  115  &  3.41	&   1.5261  \\
    60    &	 293.75  & 	293.45 	&	 4.3	&  6.0   & 	115	 &  3.40 	&   1.7918  \\
    120   &	 294.95  & 	293.05	&	 4.4	&  6.2   & 	119  &	3.39 	&   1.8245 \\
    180	  &  296.05  & 	292.05	&	 4.6	&  6.4   & 	122	 &  3.38 	&   1.8563  \\
    240	  &  297.05  & 	290.85	&	 4.6	&  6.7	 &  125	 &  3.37 	&   1.9021 \\
    300	  &  298.45  & 	289.95	&	 4.5	&  7.0	 &  124  &	3.35	&   1.9459 \\
    360	  &  299.45  & 	289.05	&	 4.4	&  7.1 	 &  123	 &  3.34	&   1.9601 \\
    420	  &  300.75  & 	288.25	&	 4.4	&  7.3	 &  122  &	3.33	&   1.9879 \\
    480	  &  301.95  & 	287.55	&	 4.3	&  7.6 	 &  121	 &  3.31	&   2.0281 \\
    540	  &  302.95  & 	286.85	&	 4.2	&  7.8 	 &  121  &	3.30	&   2.0541  \\
    600   &	 304.15  & 	286.05	&	 4.1	&  8.0	 &  120	 &  3.29	&   2.0794  \\
    660   &	 305.25  & 	285.35	&	 4.0	&  8.2 	 &  120	 &  3.28	&   2.1041 \\
    720   &	 306.25  & 	284.65	&	 3.9	&  8.4	 &  120  &	3.27	&   2.1282 \\
    780   &	 307.25  &	284.05	&	 3.8	&  8.6 	 &  120  & 	3.25	&   2.1518  \\
    840   &	 308.25  & 	283.45	&	 3.8	&  8.8 	 &  121  &	3.24	&   2.1748  \\
    900   &	 309.15  &	282.85	&	 3.8	&  9.0 	 &  121  &  3.23	&   2.1972  \\
    960   &	 310.05  &  282.25	&	 3.7	&  9.2 	 &  122  &  3.23	&   2.2192 \\
    1020  &	 310.95  &  281.65	&	 3.6	&  9.3 	 &  122  &  3.22	&   2.2300 \\
    1080  &	 311.75  &  281.05	&	 3.6	&  9.6 	 &  122	 &  3.21	&   2.2618 \\
    1140  &	 312.65  &  280.55	&	 3.6	&  9.8 	 &  124  &  3.20	&   2.2824 \\
    1200  &	 313.55  &  280.05	&	 3.6	&  10.0  &	124  &  3.19	&   2.3026  \\
    1260  &	 314.35  &  279.65	&	 3.5	&  10.1  &	124  &  3.18	&   2.3125  \\
    1320	&  315.05  &  279.25	&	 3.5	&  10.3  &  124  &  3.17	&   2.3321  \\
    1380	&  315.85  &  278.85	&	 3.5	&  10.5  &	124  &  3.17	&   2.3514 \\
    1440	&  316.55  &	278.45	&	 3.4	&  10.8  &	124	 &  3.16	&   2.3795  \\
    1500	&  317.25  & 	278.05	&	 3.4	&  10.9  &	124	 &  3.15	&   2.3888 \\
    1560	&  317.95  &	277.75	&	 3.4	&  11.0  &	124  &  3.15	&   2.3979  \\
    1620	&  318.65  & 	277.45	&	 3.4	&  11.2  &	124  &  3.14	&   2.4159 \\
    1680	&  319.25  & 	277.15	&	 3.4	&  11.3  &	124  &  3.13	&   2.4248 \\
    1740	&  319.85  & 	276.85	&  3.4	&  11.5  &	124  &  3.13	&   2.4423 \\
    1800	&  320.55  & 	276.55	&  3.4	&  11.8  &	124  &  3.12	&   2.4681 \\
    1860	&  321.15  &	276.35	&  3.3	&  11.9  &	124  &  3.11	&   2.4765  \\
    1920	&  321.75  &	276.15  &	 3.3	&  12.0  &  125  &  3.11	&   2.4849  \\
    1980	&  322.25  &  275.95  &	 3.3	&  12.1  &  124  &  3.10	&   2.4932 \\
    2040	&  322.85  &  275.75  &	 3.3	&  12.2  &  124  &  3.10	&   2.5014 \\

    \bottomrule
  \end{tabular}
\end{table}

Außerdem sind bereits berechnete Werte,
die später in der Auswertung noch benötigt werden, zu sehen. Als Einheiten werden
die üblichen Einheiten des SI-Einheitensystems verwendet und die Größen, wenn nötig,
umgerechnet.

\subsection{Graphische Darstellung der Temperaturverläufe}
Die gemessenen Werte der Temperaturen werden in den Abbildungen \ref{fig:temp1} und \ref{fig:temp2}
graphisch dargestellt. Zudem wird zu beiden Messreihen eine nicht-lineare
Ausgleichsrechnung der Form
\begin{equation}
  T(t)=At^2+Bt+C
\end{equation}
durchgeführt. Dabei wird Python wie in Kapitel \ref{sec:Fehlerrechnung} beschrieben
verwendet.

\begin{figure}
  \centering
  \includegraphics{build/T1.pdf}
  \caption{Darstellung der gemessenen Temperatur im wärmeren Reservoire
   und Graph der Ausgleichsfunktion}
  \label{fig:temp1}
\end{figure}

\begin{figure}
  \centering
  \includegraphics{build/T2.pdf}
  \caption{Darstellung der gemessenen Temperatur im kälteren Reservoir
   und Graph der Ausgleichsfunktion}
  \label{fig:temp2}
\end{figure}

Die Parameter $A$, $B$ und $C$ ergeben sich für das wärmere Reservoir zu:
\begin{align*}
  A_1=\SI{-3.08(010)e-6}{\kelvin\per\second\squared}  \,, \\
  B_1=\SI{21.2(02)e-3}{\kelvin\per\second} \,,  \\
  C_1=\SI{292.55(009)}{\kelvin}  \,.
\end{align*}

Für das kältere Wärmereservoir lauten die Parameter
\begin{align*}
  A_2=\SI{-3.24(011)e-6}{\kelvin\per\second\squared}  \,, \\
  B_2=\SI{15.7(02)e-3}{\kelvin\per\second} \,,  \\
  C_2=\SI{294.33(010)}{\kelvin}  \,.
\end{align*}



Nun sollen mithilfe dieser brechneten Parameter die Differentialquotienten
$\frac{\symup{d}T_1}{\symup{d}t}$ und $\frac{\symup{d}T_2}{\symup{d}t}$ für
vier verschiedene Temperaturen berechnet werden. Hierfür werden die Temperaturen
bei $t=\SI{300}{second}$, $t=\SI{900}{second}$, $t=\SI{1200}{second}$ und
$t=\SI{1800}{second}$ verwendet. Die Temperaturen zu diesen Zeiten können Tabelle
\ref{tab:messwerte} entnommen werden.

Die Differentialquotienten lassen sich durch
\begin{equation}
  \frac{\symup{d}T}{\symup{d}t}=2At+B
\end{equation}
aus den Parametern der Ausgleichsrechnungen bestimmen.

Für das warme Reservoir $T_{\symup{warm}}$ ergibt sich
\begin{align*}
  \biggl()\frac{\symup{d}T_{\symup{warm}}}{\symup{d}t}\biggr)_{\symup{300}} =
  \biggl()\frac{\symup{d}T_{\symup{warm}}}{\symup{d}t}\biggr)_{\symup{900}} =
  \biggl()\frac{\symup{d}T_{\symup{warm}}}{\symup{d}t}\biggr)_{\symup{1200}} =
  \biggl()\frac{\symup{d}T_{\symup{warm}}}{\symup{d}t}\biggr)_{\symup{1800}} =
\end{align*}


Für das kalte Reservoir ergeben sich die Werte
\begin{align*}
  \biggl()\frac{\symup{d}T_{\symup{kalt}}}{\symup{d}t}\biggr)_{\symup{300}} =
  \biggl()\frac{\symup{d}T_{\symup{kalt}}}{\symup{d}t}\biggr)_{\symup{900}} =
  \biggl()\frac{\symup{d}T_{\symup{kalt}}}{\symup{d}t}\biggr)_{\symup{1200}} =
  \biggl()\frac{\symup{d}T_{\symup{kalt}}}{\symup{d}t}\biggr)_{\symup{1800}} =
\end{align*}
