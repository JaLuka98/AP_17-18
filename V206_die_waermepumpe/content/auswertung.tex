\section{Auswertung}
\label{sec:Auswertung}
Die Messreihen für die Temperaturen, Drücke und die Leistungsaufnahme $P$ des Kompressors
in Abhängigkeit der Zeit sind in Tabelle \ref{tab:messwerte} dargestellt. Dabei
werden die Werte für die Zeit in Sekunden und die Werte für die Temperaturen in Kelvin
umgerechnet. Zu den gemessenen Werten für die Drücke $p_{\symup{a}}$ und
$p_{\symup{a}}$ wird noch der Umgebungsdruck von $\SI{10e5}{\pascal}$ hinzuaddiert.

\begin{table}[H]
  \centering
  \caption{Messwerte und aus diesen berechnete Werte}
  \label{tab:messwerte}
  \begin{tabular}{c c c c c c c c}
    \toprule
    $t/$s & $T_{\text{warm}}/$K & $T_{\text{kalt}}/$K & $p_{\text{kalt}}/(10^{5}\symup{Pa})$ & $p_{\text{warm}}/(10^{5}\symup{Pa})$ &
    $P$/W & $1/T_1/\left(10^{-3}\symup{\frac{1}{K}}\right)$ & $\ln\left(\frac{p_b}{p_0}\right)$ \\
    \midrule
    0	    &  293,35	 & 	293,45 	&	 5,0	&  4,6   &  115  &  3,41	&   1,526  \\
    60    &	 293,75  & 	293,45 	&	 4,3	&  6,0   & 	115	 &  3,40 	&   1,792  \\
    120   &	 294,95  & 	293,05	&	 4,4	&  6,2   & 	119  &	3,39 	&   1,825 \\
    180	  &  296,05  & 	292,05	&	 4,6	&  6,4   & 	122	 &  3,38 	&   1,856  \\
    240	  &  297,05  & 	290,85	&	 4,6	&  6,7	 &  125	 &  3,37 	&   1,902 \\
    300	  &  298,45  & 	289,95	&	 4,5	&  7,0	 &  124  &	3,35	&   1,946 \\
    360	  &  299,45  & 	289,05	&	 4,4	&  7,1 	 &  123	 &  3,34	&   1,960 \\
    420	  &  300,75  & 	288,25	&	 4,4	&  7,3	 &  122  &	3,33	&   1,988 \\
    480	  &  301,95  & 	287,55	&	 4,3	&  7,6 	 &  121	 &  3,31	&   2,028 \\
    540	  &  302,95  & 	286,85	&	 4,2	&  7,8 	 &  121  &	3,30	&   2,054  \\
    600   &	 304,15  & 	286,05	&	 4,1	&  8,0	 &  120	 &  3,29	&   2,079  \\
    660   &	 305,25  & 	285,35	&	 4,0	&  8,2 	 &  120	 &  3,28	&   2,104 \\
    720   &	 306,25  & 	284,65	&	 3,9	&  8,4	 &  120  &	3,27	&   2,128 \\
    780   &	 307,25  &	284,05	&	 3,8	&  8,6 	 &  120  & 	3,25	&   2,152  \\
    840   &	 308,25  & 	283,45	&	 3,8	&  8,8 	 &  121  &	3,24	&   2,175  \\
    900   &	 309,15  &	282,85	&	 3,8	&  9,0 	 &  121  &  3,23	&   2,197  \\
    960   &	 310,05  &  282,25	&	 3,7	&  9,2 	 &  122  &  3,23	&   2,219 \\
    1020  &	 310,95  &  281,65	&	 3,6	&  9,3 	 &  122  &  3,22	&   2,230 \\
    1080  &	 311,75  &  281,05	&	 3,6	&  9,6 	 &  122	 &  3,21	&   2,262 \\
    1140  &	 312,65  &  280,55	&	 3,6	&  9,8 	 &  124  &  3,20	&   2,282 \\
    1200  &	 313,55  &  280,05	&	 3,6	&  10,0  &	124  &  3,19	&   2,303  \\
    1260  &	 314,35  &  279,65	&	 3,5	&  10,1  &	124  &  3,18	&   2,313  \\
    1320	&  315,05  &  279,25	&	 3,5	&  10,3  &  124  &  3,17	&   2,332  \\
    1380	&  315,85  &  278,85	&	 3,5	&  10,5  &	124  &  3,17	&   2,351 \\
    1440	&  316,55  &	278,45	&	 3,4	&  10,8  &	124	 &  3,16	&   2,380  \\
    1500	&  317,25  & 	278,05	&	 3,4	&  10,9  &	124	 &  3,15	&   2,389 \\
    1560	&  317,95  &	277,75	&	 3,4	&  11,0  &	124  &  3,15	&   2,398  \\
    1620	&  318,65  & 	277,45	&	 3,4	&  11,2  &	124  &  3,14	&   2,416 \\
    1680	&  319,25  & 	277,15	&	 3,4	&  11,3  &	124  &  3,13	&   2,425 \\
    1740	&  319,85  & 	276,85	&  3,4	&  11,5  &	124  &  3,13	&   2,442 \\
    1800	&  320,55  & 	276,55	&  3,4	&  11,8  &	124  &  3,12	&   2,468 \\
    1860	&  321,15  &	276,35	&  3,3	&  11,9  &	124  &  3,11	&   2,477  \\
    1920	&  321,75  &	276,15  &	 3,3	&  12,0  &  125  &  3,11	&   2,485  \\
    1980	&  322,25  &  275,95  &	 3,3	&  12,1  &  124  &  3,10	&   2,493 \\
    2040	&  322,85  &  275,75  &	 3,3	&  12,2  &  124  &  3,10	&   2,501 \\

    \bottomrule
  \end{tabular}
\end{table}

Außerdem sind bereits berechnete Werte,
die später in der Auswertung noch benötigt werden, zu sehen. Als Einheiten werden
die üblichen Einheiten des SI-Einheitensystems verwendet und die Größen, wenn nötig,
umgerechnet.

\subsection{Graphische Darstellung der Temperaturverläufe und Ausgleichsrechnung}
Die gemessenen Werte der Temperaturen werden in den Abbildungen \ref{fig:temp1} und \ref{fig:temp2}
graphisch dargestellt. Zudem wird zu beiden Messreihen eine nicht-lineare
Ausgleichsrechnung der Form
\begin{equation}
  T(t)=At^2+Bt+C
\end{equation}
durchgeführt. Dabei wird Python wie in Kapitel \ref{sec:Fehlerrechnung} beschrieben
verwendet.

\begin{figure}
  \centering
  \includegraphics{build/T1.pdf}
  \caption{Darstellung der gemessenen Temperatur im wärmeren Reservoir
   und Graph der Ausgleichsfunktion}
  \label{fig:temp1}
\end{figure}

\begin{figure}
  \centering
  \includegraphics{build/T2.pdf}
  \caption{Darstellung der gemessenen Temperatur im kälteren Reservoir
   und Graph der Ausgleichsfunktion}
  \label{fig:temp2}
\end{figure}

Die Parameter der Ausgleichsfunktion sind in Tabelle \ref{tab:parameter} zu finden.

\begin{table}[H]
  \centering
  \caption{Parameter der Ausgleichsrechnung für das wärmere und das kältere Reservoir}
  \label{tab:parameter}
  \begin{tabular}{c c c c}
    \toprule
    $ $& $A/\frac{\symup{K}}{\symup{s^2}}$ & $B/\frac{\symup{K}}{\symup{s}}$ & $C/$K \\
    \midrule
    Wämeres Reservoir   & \num{-3.08(010)e-6} & \num{21.2(02)e-3} & \num{292.55(009)} \\
    Kälteres Reservoir  & \num{3.24(011)e-6}  & \num{-15.7(02)e-3} & \num{294.33(010)} \\
    \bottomrule
  \end{tabular}
\end{table}
%\begin{align*}
%  A_\text{warm}&=\SI{-3.08(010)e-6}{\kelvin\per\second\squared}  \,, \\
%  B_\text{warm}&=\SI{21.2(02)e-3}{\kelvin\per\second} \,,  \\
%  C_\text{warm}&=\SI{292.55(009)}{\kelvin}  \,.
%\end{align*}

%Für das kältere Wärmereservoir lauten die Parameter
%\begin{align*}
%  A_\text{kalt}&=\SI{3.24(011)e-6}{\kelvin\per\second\squared}  \,, \\
%  B_\text{kalt}&=\SI{15.7(02)e-3}{\kelvin\per\second} \,,  \\
%  C_\text{kalt}&=\SI{294.33(010)}{\kelvin}  \,.
%\end{align*}

Im weiteren Verlauf der Auswertung werden einige Zeitpunkte im Speziellen untersucht.
Die Wahl dieser Zeitpunkte und die Temperaturen zu diesen Zeiten lassen sich Tabelle
\ref{tab:ausgewaehltemesswerte} entnehmen.

\begin{table}[h]
		\centering
    \caption{Darstellung der gewählten Zeitpunkte und der Temperaturen}
		\label{tab:ausgewaehltemesswerte}
		\begin{tabular}{lcccc}
			\toprule
			& $T(\SI{300}{\second}) / \text{K}$ & $T(\SI{900}{\second}) / \text{K}$ & $T(\SI{1200}{\second}) / \text{K}$ & $T(\SI{1800}{\second}) / \text{K}$\\
			\midrule
			Wärmeres Reservoir & 298,45 & 309,15 & 313,55 & 320,55 \\
			Kälteres Reservoir & 289,95 & 282,85 & 280,05 & 276,55 \\
			\bottomrule
		\end{tabular}
	\end{table}

Nun sollen mithilfe deraus der Ausgleichsrechnung gewonnenen Parameter die Differenzialquotienten
$\dot{T}_\text{warm}$ und $\dot{T}_\text{kalt}$ für vier verschiedene Temperaturen berechnet werden.

Die Differentialquotienten lassen sich durch
\begin{equation}
  \frac{\symup{d}T}{\symup{d}t}=2At+B
\end{equation}
aus den Parametern der Ausgleichsrechnungen bestimmen. Dabei muss die Gauß'sche
Fehlerfortpflanzung in der Form
\begin{equation}
  \sigma_{\dot{T}_\text{warm}} = \sqrt{4 t^2 \sigma_{A_\text{warm}}^{2} + \sigma_{B_\text{warm}}^{2} t^{2}}
\end{equation}
berücksichtigt werden.

Die berechneten Werte sind in Tabelle \ref{tab:differentialquotienten} zu finden.

\begin{table}[H]
  \centering
  \caption{Berechnete Werte für die Differenzialquotienten $\dot{T}_\text{warm}$ und $\dot{T}_\text{kalt}$
  zu den ausgewählten Zeitpunkten.}
  \label{tab:differentialquotienten}
  \begin{tabular}{c c c c c}
    \toprule
    $ $& $\dot{T}_{\symup{300}}/\frac{\symup{\num{e-3}K}}{\symup{s}}$
    &$\dot{T}_{\symup{900}}/\frac{\symup{\num{e-3}K}}{\symup{s}}$
    &$\dot{T}_{\symup{1200}}/\frac{\symup{\num{e-3}K}}{\symup{s}}$
    &$\dot{T}_{\symup{1800}}/\frac{\symup{\num{e-3}K}}{\symup{s}}$\\
    \midrule
    Wämeres Reservoir   & \num{19.27(021)} & \num{15.57(027)} & \num{13.72(031)}    & \num{10.0(04)}  \\
    Kälteres Reservoir  & \num{-13.75(023)}  & \num{-9.87(029)} & \num{-7.93(034)}  &  \num{4.0(04)} \\
    \bottomrule
  \end{tabular}
\end{table}

%Für das warme Reservoir $T_{\symup{warm}}$ ergibt sich
%\begin{align*}
%  \biggl(\frac{\symup{d}T_{\symup{warm}}}{\symup{d}t}\biggr)_{\symup{300}} = 0,01927\pm0,00021 \,,\\
%  \biggl(\frac{\symup{d}T_{\symup{warm}}}{\symup{d}t}\biggr)_{\symup{900}} =  0,01557\pm0,00027\,,\\
%  \biggl(\frac{\symup{d}T_{\symup{warm}}}{\symup{d}t}\biggr)_{\symup{1200}} = 0,01372\pm0,00031\,,\\
%  \biggl(\frac{\symup{d}T_{\symup{warm}}}{\symup{d}t}\biggr)_{\symup{1800}} = 0,0100\pm0,0004\,.\\
%\end{align*}
%Für das kalte Reservoir ergeben sich die Werte
%\begin{align*}
%  \biggl(\frac{\symup{d}T_{\symup{kalt}}}{\symup{d}t}\biggr)_{\symup{300}} = -0,01375\pm0,00023 \,,\\
%  \biggl(\frac{\symup{d}T_{\symup{kalt}}}{\symup{d}t}\biggr)_{\symup{900}} =  -0,00987\pm0,00029\,,\\
%  \biggl(\frac{\symup{d}T_{\symup{kalt}}}{\symup{d}t}\biggr)_{\symup{1200}} = -0,00793\pm0,00034\,,\\
%  \biggl(\frac{\symup{d}T_{\symup{kalt}}}{\symup{d}t}\biggr)_{\symup{1800}} = -0,0040\pm0,0004\,.\\
%\end{align*}

\subsection{Bestimmung der Güteziffern}

Aus diesen Werten soll nun die reale Güteziffer der Wärmepumpe berechnet werden. Dies geschieht gemäß der Gleichungen
\ref{eqn:waermewarmprozeit} und \ref{eqn:nureal}. Dabei ist die Wärmekapazität
der Apparatur ablesbar und hat den Wert
\begin{equation}
  m_{\symup{k}}c_{\symup{k}}=\SI{750}{\joule\per\kelvin} \,. \nonumber
\end{equation}
Die Wärmekapazität des sich im wärmeren Reservoir befindenden Wassers lässt sich
mithilfe der spezifischen Wärmekapazität $c_{\symup{w}}=\SI{4187}{\joule\per\kilo\gram\kelvin}$
und der Masse $m_{\symup{w}}=\SI{4}{\kilo\gram}$ des Wassers zu
\begin{equation}
  m_{\symup{w}}c_{\symup{w}}=\SI{16748}{\joule\per\kelvin}  \nonumber
\end{equation}
berechnen.
Der Fehler bei der realen Güteziffer ergibt sich dabei durch die Gauß'sche Fehlerfortpflanzung
\begin{equation}
  \sigma_{\nu_\text{real}} = \frac{m_\text{warm} c_\text{W} + m_\text{K} c_\text{K}}{P}
  \sqrt{4 t^2 \sigma_{A_\text{warm}}^2 + \sigma_{B_\text{warm}}^2} \,.
\end{equation}
Die ideale Güteziffer der Wärmepumpe hingegen berechnet sich durch Gleichung
\ref{eqn:nuid}. Die berechneten Werte für die reale und die ideale Güteziffer
sind in Tabelle \ref{tab:güteziffer} dargestellt.

\begin{table}[H]
  \centering
  \caption{Berechnete Werte für die ideale und die reale Güteziffer zu den ausgewählten
  Zeitpunkten und relative Abweichung der realen Werte von den idealen Werten.}
  \label{tab:güteziffer}
  \begin{tabular}{c c c c}
    \toprule
    $t/$s & $\nu_{\symup{real}}$ & $\nu_{\symup{ideal}}$ & relative Abweichung in \%\\
    \midrule
    300   & \num{2.72(003)} & 35.11 & -92,3 \\
    900   & \num{2.25(004)} & 11.75 & -80,9 \\
    1200  & \num{1.94(004)} & 9.36  & -79,3 \\
    1800  & \num{1.41(006)} & 7.29  & -80,7 \\
    \bottomrule
  \end{tabular}
\end{table}
%\begin{align*}
%  \nu_{\symup{real, 300}} &= \SI{2.72(003)}{einheit?} \,, & \nu_{\symup{real, 300}} &= \SI{35.11}{...}  \,,\\
%  \nu_{\symup{real, 900}} &= \SI{2.25(004)}{einheit?} \,, & \nu_{\symup{real, 900}} &= \SI{11.75}{}  \,,\\
%  \nu_{\symup{real, 1200}} &= \SI{1.94(004)}{einheit?} \,,& \nu_{\symup{real, 1200}} &= \SI{9.36}{}  \,,\\
%  \nu_{\symup{real, 1800}} &=  \SI{1.41(006)}{einheit?}\,,& \nu_{\symup{real, 1800}} &= \SI{7.29}{}  \,.\\
%\end{align*}

Auffällig ist hier, dass die reale Güteziffer der Wärmepumpe zu allen betrachteten
Zeitpunkten stark von der idealen Güteziffer abweicht.
Mögliche Gründe hierfür sind die schlechte Isolierung der Wärmereservoire, sowie
der nicht ideale WIrkungsgrad des Kompressors. Die Wärmereservoire bestanden
aus lediglich durch eine Styroporschicht isolierten und oben nur dürftig durch Holzplatten
abgedeckten Eimern. Zudem waren sämtliche Rohre nicht oder nur sehr schwach isoliert.

\subsection{Bestimmung des Massendurchsatzes}
Es wird zunächst die molare Verdampfungswärme $L_n$ des Transportgases
Dichlordifluormethan mithilfe einer Dampfdruck-Kurve bestimmt
\footnote{Die theoretische Grundlage hierfür kann in \cite{dampfdruck} nachgelesen
werden.} und daraus die auf die Masse bezogene Verdampfungswärme $L_m$ errechnet, da
diese für die Berechnung des Massendurchsatzes benötigt wird.
Der gemessene Druck im wärmeren Reservoir wird dafür logarithmisch gegen die reziproke Temperatur im
wärmeren Reservoir aufgetragen und anschließend wird eine lineare Ausgleichsrechnung
der Form
\begin{equation}
  \ln\left(\frac{p(T)}{p_0}\right)= \frac{A_L}{R T} + B_L
\end{equation}
in den Parametern $A_L$ und $B_L$ und durchgeführt. Dabei bezeichnet $R$ die universelle Gaskonstante mit
$R = \SI{8.3144598}{\joule\per\mol\kelvin}$. Da ihre geringe Unsicherheit in der Größenordnung
von $1 \cdot 10^{-7}$ die Genauigkeit der Ergebnisse nicht signifikant beeinflusst, wird
sie hier als exakt angesehen.
Für den Umgebungsdruck $p_0$ wird der Näherungswert $\SI{1}{\bar}$ verwendet.
Die Dampfdruck-Kurve ist in Abbildung \ref{fig:dampf} dargestellt.

\begin{figure}
  \centering
  \includegraphics{build/L1.pdf}
  \caption{Logarithmische Darstellung des gemessenen Druckes im wärmeren Reservoir
   in Abhängigkeit von der Temperatur und Graph der Ausgleichsfunktion.}
  \label{fig:dampf}
\end{figure}

Die Parameter der konkreten Ausgleichsrechnung ergeben sich zu
\begin{align*}
  A_L &= \SI{20.3(07)}{\kilo\joule\per\mol}\,, \\
  B_L &= \SI{10.09(026)} \,.
\end{align*}
Wird der y-Achsenabschnitt $B_L$ als vernachlässigbar angesehen, so lässt sich die
Steigung $A_L$ mit der molaren Verdampfungswärme $L_n$ identifizieren.
Die Verdampfungswärme pro Masse wird dann mit der molaren Masse $M = \SI{120.9}{\gram\per\mol}$
nach \cite{molaremasse} zu
\begin{equation*}
  L_m = \SI{168(5)}{\kilo\joule\per\kilogram}
\end{equation*}
bestimmt. Mit der Gauß'schen Fehlerfortpflanzung nach \eqref{eqn:gaussfehler} ergibt sich
die Unsicherheit von $L_m$ zu
\begin{equation*}
  \sigma_{L_m} = \frac{1}{M} \sigma_{L_n}\,.
\end{equation*}
Da nun die auf die Masse bezogene Verdampfungswärme bekannt ist, ist es möglich, den
als Differenzialquotienten ausgedrückten Massendurchsatz $\dot{m}$ mit den Gleichungen
\eqref{eqn:massendurchsatz} und \eqref{eqn:waermekaltprozeit} zu berechnen, wobei die
Diskretisierung in Ableitungen übergeht.
Der an den ausgewählten Punkten berechnete Massendurchsatz $\dot{m}$ ist zusammen mit
der zur Berechnung nötigenen Zwischenwerte $\dot{Q}_\text{kalt}$ in Tabelle
\ref{tab:tabmassendurchsatz} aufgeführt.
\begin{table}
		\centering
    \caption{Darstellung der Werte für den Wärmestrom am kalten Reservoir und den Massendurchsatz}
    \label{tab:tabmassendurchsatz}
		\begin{tabular}{ccc}
			\toprule
			$t/$s & $\dot{Q}_\text{kalt}/(\text{K} \cdot \text{s}^{-1})$ & $\dot{m}/(\text{g} \cdot \text{s}^{-1})$ \\
			\midrule
      300  & -241 $\pm$ 4 & -1,43 $\pm$ 0,05 \\
      900  & -173 $\pm$ 5 & -1,03 $\pm$ 0,05 \\
      1200 & -139 $\pm$ 6 & -0.83 $\pm$ 0,04 \\
      1800 &  -71  $\pm$ 8 & -0.42 $\pm$ 0,05 \\
			\bottomrule
		\end{tabular}
	\end{table}
Die Unsicherheiten ergeben sich unter Berücksichtigung von Gleichung \eqref{eqn:gaussfehler} dann zu
\begin{align*}
  \sigma_{\dot{Q}_\text{kalt}} &= (m_2 c_\text{w} + m_\text{k} c_\text{k}) \sigma_{\dot{T}_\text{kalt}} \,\\
  \sigma_{\dot{m}} &= \frac{1}{L_m} \sqrt{\dot{Q}_\text{kalt}^2 \sigma_{L_m}^2 + \sigma_{\dot{Q}_\text{kalt}}^2}\,.
\end{align*}

\subsection{Bestimmung der mechanischen Kompressorleistung}
Für die Berechnung der mechanischen Leistung $N_\text{mech}$ mithilfe von Gleichung
\eqref{eqn:kompressorleistung} des Kompressors an den vier ausgewählten Zeitpunkten
wird die Dichte des Transportgases zu diesen Zeitpunkten benötigt.
Diese wird nach Gleichung \eqref{eqn:rho} mit Werten aus Tabelle \ref{tab:messwerte}
berechnet.
Zur Berechnung sind auch die Größen $\rho_0 = \SI{5.51}{\gram\per\litre}$ als Dichte
des Transportgases bei einer Temperatur von $\SI{273.15}{\kelvin}$ und einem Druck von
$\SI{1e5}{\pascal}$ und der Adiabatenkoeffizient $\kappa$ = 1,14 nötig.
Die berechneten Zwischenergebnisse für $\rho$ und die mechanische Kompressorleistung
$N_\text{mech}$ zu den vier ausgewählten Zeitpunkten ist in Tabelle \ref{tab:datalast}
eingetragen.

\begin{table}
	\centering
	\caption{Die Dichte des Gases, die theoretische Kompressorleistung und das Verhältnis $N_\text{mech}/N$ in Abhängigkiet von der Zeit.}
  \label{tab:datalast}
	\begin{tabular}{ccc}
		\toprule
		$t/$s & $\rho/(\text{g} \cdot \text{l}^{-1}$) & $N_\text{mech} / \text{W}$ \\
    \midrule
		300 & 23,36 & -11,0 $\pm$ 0,4 \\
		900 & 20,22 & -15,4 $\pm$ 0,7 \\
	  1200 & 19,35 & -14,7 $\pm$ 0,8 \\
		1800 & 18,50 & -9,1 $\pm$ 1,0 \\
		\bottomrule
	\end{tabular}
\end{table}

Die Unsicherheit von $N_\text{mech}$ ergibt durch Anwendung der Gauß'schen Fehlerfortpflanzung
\eqref{eqn:gaussfehler} zu
\begin{equation}
  \sigma_{N_\text{mech}} = \frac{1}{1 - \kappa} \left(p_\text{warm} \sqrt[\leftroot{-1}\uproot{-1}\scriptstyle \kappa]{\frac{p_\text{kalt}}{p_\text{warm}}} - p_\text{kalt} \right) \frac{1}{\rho} \sigma_{\dot{m}}\,.
\end{equation}
