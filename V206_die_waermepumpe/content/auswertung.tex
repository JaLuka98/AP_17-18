\section{Auswertung}
\label{sec:Auswertung}
Die Messreihen für die Temperaturen, Drücke und die Leistungsaufnahme $P$ des Kompressors
in Abhängigkeit der Zeit sind in Tabelle \ref{tab:messwerte} dargestellt. Dabei
werden die Werte für die Zeit in Sekunden und die Werte für die Temperaturen in Kelvin
umgerechnet. Zu den gemessenen Werten für die Drücke $p_{\symup{a}}$ und
$p_{\symup{a}}$ wird noch der Umgebungsdruck von $\SI{10e5}{\pascal}$ hinzuaddiert.

\begin{table}[H]
  \centering
  \caption{Messwerte und aus diesen berechnete Werte}
  \label{tab:messwerte}
  \begin{tabular}{c c c c c c c c}
    \toprule
    $t/$s & $T_{\text{warm}}/$K & $T_{\text{kalt}}/$K & $p_{\text{kalt}}/(10^{5}\symup{Pa})$ & $p_{\text{warm}}/(10^{5}\symup{Pa})$ &
    $P$/W & $1/T_1/\left(10^{-3}\symup{\frac{1}{K}}\right)$ & $\ln\left(\frac{p_b}{p_0}\right)$ \\
    \midrule
    0	    &  293,35	 & 	293,45 	&	 5,0	&  4,6   &  115  &  3,41	&   1,526  \\
    60    &	 293,75  & 	293,45 	&	 4,3	&  6,0   & 	115	 &  3,40 	&   1,792  \\
    120   &	 294,95  & 	293,05	&	 4,4	&  6,2   & 	119  &	3,39 	&   1,825 \\
    180	  &  296,05  & 	292,05	&	 4,6	&  6,4   & 	122	 &  3,38 	&   1,856  \\
    240	  &  297,05  & 	290,85	&	 4,6	&  6,7	 &  125	 &  3,37 	&   1,902 \\
    300	  &  298,45  & 	289,95	&	 4,5	&  7,0	 &  124  &	3,35	&   1,946 \\
    360	  &  299,45  & 	289,05	&	 4,4	&  7,1 	 &  123	 &  3,34	&   1,960 \\
    420	  &  300,75  & 	288,25	&	 4,4	&  7,3	 &  122  &	3,33	&   1,988 \\
    480	  &  301,95  & 	287,55	&	 4,3	&  7,6 	 &  121	 &  3,31	&   2,028 \\
    540	  &  302,95  & 	286,85	&	 4,2	&  7,8 	 &  121  &	3,30	&   2,054  \\
    600   &	 304,15  & 	286,05	&	 4,1	&  8,0	 &  120	 &  3,29	&   2,079  \\
    660   &	 305,25  & 	285,35	&	 4,0	&  8,2 	 &  120	 &  3,28	&   2,104 \\
    720   &	 306,25  & 	284,65	&	 3,9	&  8,4	 &  120  &	3,27	&   2,128 \\
    780   &	 307,25  &	284,05	&	 3,8	&  8,6 	 &  120  & 	3,25	&   2,152  \\
    840   &	 308,25  & 	283,45	&	 3,8	&  8,8 	 &  121  &	3,24	&   2,175  \\
    900   &	 309,15  &	282,85	&	 3,8	&  9,0 	 &  121  &  3,23	&   2,197  \\
    960   &	 310,05  &  282,25	&	 3,7	&  9,2 	 &  122  &  3,23	&   2,219 \\
    1020  &	 310,95  &  281,65	&	 3,6	&  9,3 	 &  122  &  3,22	&   2,230 \\
    1080  &	 311,75  &  281,05	&	 3,6	&  9,6 	 &  122	 &  3,21	&   2,262 \\
    1140  &	 312,65  &  280,55	&	 3,6	&  9,8 	 &  124  &  3,20	&   2,282 \\
    1200  &	 313,55  &  280,05	&	 3,6	&  10,0  &	124  &  3,19	&   2,303  \\
    1260  &	 314,35  &  279,65	&	 3,5	&  10,1  &	124  &  3,18	&   2,313  \\
    1320	&  315,05  &  279,25	&	 3,5	&  10,3  &  124  &  3,17	&   2,332  \\
    1380	&  315,85  &  278,85	&	 3,5	&  10,5  &	124  &  3,17	&   2,351 \\
    1440	&  316,55  &	278,45	&	 3,4	&  10,8  &	124	 &  3,16	&   2,380  \\
    1500	&  317,25  & 	278,05	&	 3,4	&  10,9  &	124	 &  3,15	&   2,389 \\
    1560	&  317,95  &	277,75	&	 3,4	&  11,0  &	124  &  3,15	&   2,398  \\
    1620	&  318,65  & 	277,45	&	 3,4	&  11,2  &	124  &  3,14	&   2,416 \\
    1680	&  319,25  & 	277,15	&	 3,4	&  11,3  &	124  &  3,13	&   2,425 \\
    1740	&  319,85  & 	276,85	&  3,4	&  11,5  &	124  &  3,13	&   2,442 \\
    1800	&  320,55  & 	276,55	&  3,4	&  11,8  &	124  &  3,12	&   2,468 \\
    1860	&  321,15  &	276,35	&  3,3	&  11,9  &	124  &  3,11	&   2,477  \\
    1920	&  321,75  &	276,15  &	 3,3	&  12,0  &  125  &  3,11	&   2,485  \\
    1980	&  322,25  &  275,95  &	 3,3	&  12,1  &  124  &  3,10	&   2,493 \\
    2040	&  322,85  &  275,75  &	 3,3	&  12,2  &  124  &  3,10	&   2,501 \\

    \bottomrule
  \end{tabular}
\end{table}

Außerdem sind bereits berechnete Werte,
die später in der Auswertung noch benötigt werden, zu sehen. Als Einheiten werden
die üblichen Einheiten des SI-Einheitensystems verwendet und die Größen, wenn nötig,
umgerechnet.

\subsection{Graphische Darstellung der Temperaturverläufe und Ausgleichsrechnung}
Die gemessenen Werte der Temperaturen werden in den Abbildungen \ref{fig:temp1} und \ref{fig:temp2}
graphisch dargestellt. Zudem wird zu beiden Messreihen eine nicht-lineare
Ausgleichsrechnung der Form
\begin{equation}
  T(t)=At^2+Bt+C
\end{equation}
durchgeführt. Dabei wird Python wie in Kapitel \ref{sec:Fehlerrechnung} beschrieben
verwendet.

\begin{figure}
  \centering
  \includegraphics{build/T1.pdf}
  \caption{Darstellung der gemessenen Temperatur im wärmeren Reservoir
   und Graph der Ausgleichsfunktion}
  \label{fig:temp1}
\end{figure}

\begin{figure}
  \centering
  \includegraphics{build/T2.pdf}
  \caption{Darstellung der gemessenen Temperatur im kälteren Reservoir
   und Graph der Ausgleichsfunktion}
  \label{fig:temp2}
\end{figure}

Die Parameter der Ausgleichsfunktion ergeben sich für das wärmere Reservoir zu:
\begin{align*}
  A_\text{warm}&=\SI{-3.08(010)e-6}{\kelvin\per\second\squared}  \,, \\
  B_\text{warm}&=\SI{21.2(02)e-3}{\kelvin\per\second} \,,  \\
  C_\text{warm}&=\SI{292.55(009)}{\kelvin}  \,.
\end{align*}

Für das kältere Wärmereservoir lauten die Parameter
\begin{align*}
  A_\text{kalt}&=\SI{-3.24(011)e-6}{\kelvin\per\second\squared}  \,, \\
  B_\text{kalt}&=\SI{15.7(02)e-3}{\kelvin\per\second} \,,  \\
  C_\text{kalt}&=\SI{294.33(010)}{\kelvin}  \,.
\end{align*}



Nun sollen mithilfe dieser brechneten Parameter die Differenzialquotienten
$\frac{\symup{d}T_\text{warm}}{\symup{d}t}$ und $\frac{\symup{d}T_\text{kalt}}{\symup{d}t}$ für
vier verschiedene Temperaturen berechnet werden. Hierfür werden die Temperaturen
bei $t=\SI{300}{\second}$, $t=\SI{900}{\second}$, $t=\SI{1200}{\second}$ und
$t=\SI{1800}{\second}$ verwendet. Die Temperaturen zu diesen Zeiten können Tabelle
\ref{tab:messwerte} entnommen werden.

Die Differentialquotienten lassen sich durch
\begin{equation}
  \frac{\symup{d}T}{\symup{d}t}=2At+B
\end{equation}
aus den Parametern der Ausgleichsrechnungen bestimmen. Dabei muss die Gauß'sche
Fehlerfortpflanzung in der Form
\begin{equation}
  \sigma_{\dot{T}_\text{warm}} = \sqrt{4 t^2 \sigma_{A_\text{warm}}^{2} + \sigma_{B_\text{warm}}^{2} t^{2}}
\end{equation}
berücksichtigt werden.

Für das warme Reservoir $T_{\symup{warm}}$ ergibt sich
\begin{align*}
  \biggl(\frac{\symup{d}T_{\symup{warm}}}{\symup{d}t}\biggr)_{\symup{300}} = 0,01927\pm0,00021 \,,\\
  \biggl(\frac{\symup{d}T_{\symup{warm}}}{\symup{d}t}\biggr)_{\symup{900}} =  0,01557\pm0,00027\,,\\
  \biggl(\frac{\symup{d}T_{\symup{warm}}}{\symup{d}t}\biggr)_{\symup{1200}} = 0,01372\pm0,00031\,,\\
  \biggl(\frac{\symup{d}T_{\symup{warm}}}{\symup{d}t}\biggr)_{\symup{1800}} = 0,0100\pm0,0004\,.\\
\end{align*}


Für das kalte Reservoir ergeben sich die Werte
\begin{align*}
  \biggl(\frac{\symup{d}T_{\symup{kalt}}}{\symup{d}t}\biggr)_{\symup{300}} = -0,01375\pm0,00023 \,,\\
  \biggl(\frac{\symup{d}T_{\symup{kalt}}}{\symup{d}t}\biggr)_{\symup{900}} =  -0,00987\pm0,00029\,,\\
  \biggl(\frac{\symup{d}T_{\symup{kalt}}}{\symup{d}t}\biggr)_{\symup{1200}} = -0,00793\pm0,00034\,,\\
  \biggl(\frac{\symup{d}T_{\symup{kalt}}}{\symup{d}t}\biggr)_{\symup{1800}} = -0,0040\pm0,0004\,.\\
\end{align*}

Aus diesen Werten soll nun die reale Güteziffer der Wärmepumpe berechnet werden. Dies geschieht gemäß der Gleichungen
\ref{eqn:waermewarmprozeit} und \ref{eqn:nureal}. Dabei ist die spezifische Wärme
der Apparatur ablesbar und hat den Wert
\begin{equation}
  m_{\symup{k}}c_{\symup{k}}=\SI{750}{\joule\per\kelvin} \,. \nonumber
\end{equation}
Die spezifische Wärme des sich im wärmeren Reservoir befindenden Wassers lässt sich
mithilfe der spezifischen Wärmekapazität $c_{\symup{w}}=\SI{4187}{\joule\per\kilo\gram\kelvin}$
und der Masse $m_{\symup{w}}=\SI{4}{\kilo\gram}$ des Wassers zu
\begin{equation}
  m_{\symup{w}}c_{\symup{w}}=\SI{16748}{\joule\per\kelvin}  \nonumber
\end{equation}
berechnen.
Der Fehler bei der realen Güteziffer ergibt sich dabei durch die Gauß'sche Fehlerfortpflanzung
\begin{equation}
  \sigma_{\nu_\text{real}} = \frac{m_\text{warm} c_\text{W} + m_\text{K} c_\text{K}}{P}
  \sqrt{4 t^2 \sigma_{A_\text{warm}}^2 + \sigma_{B_\text{warm}}^2} \,.
\end{equation}
Die ideale Güteziffer der Wärmepumpe hingegen berechnet sich durch Gleichung
\ref{eqn:nuid}. Damit ergeben sich die Werte für die realen und idealen
Güteziffern zu den verschiedenen Zeitpunkten zu
\begin{align*}
  \nu_{\symup{real, 300}} &= \SI{2.72(003)}{einheit?} \,, & \nu_{\symup{real, 300}} &= \SI{35.11}{...}  \,,\\
  \nu_{\symup{real, 900}} &= \SI{2.25(004)}{einheit?} \,, & \nu_{\symup{real, 900}} &= \SI{11.75}{}  \,,\\
  \nu_{\symup{real, 1200}} &= \SI{1.94(004)}{einheit?} \,,& \nu_{\symup{real, 1200}} &= \SI{9.36}{}  \,,\\
  \nu_{\symup{real, 1800}} &=  \SI{1.41(006)}{einheit?}\,,& \nu_{\symup{real, 1800}} &= \SI{7.29}{}  \,.\\
\end{align*}
(FEHLER BERECHNEN!)
Auffällig ist hier, dass die reale Güteziffer der Wärmepumpe zu allen betrachteten
Zeitpunkten stark von der idealen Güteziffer abweicht.
Mögliche Gründe hierfür sind die schlechte Isolierung der Wärmereservoire, sowie
der nicht ideale WIrkungsgrad des Kompressors. Die Wärmereservoire bestanden
aus lediglich durch eine Styroporschicht isolierten und oben nur dürftig durch Holzplatten
abgedeckten Eimern. Zudem waren sämtliche Rohre nicht oder nur sehr schwach isoliert.





\subsection{Bestimmung der Güteziffern}

\subsection{Bestimmung des Massendurchsatzes}

Zur Bestimmung des Massendurchsatzes muss zunächst die Verdampfungswärme $L$ des
verwendeten Gases Dichloridfluormethan mithilfe einer Dampfdruck-Kurve bestimmt werden.
Dafür wird der gemessene Druck im wärmeren Reservoir logarithmisch gegen die Temperatur im
wärmeren Reservoir aufgetragen und anschließend wird eine lineare Ausgleichsrechnung
der Form
\begin{equation}
  \ln(p(T))=AT+B
\end{equation}
durchgeführt.
Die Dampfdruck-Kurve ist in Abbildung \ref{fig:dampf} dargestellt.

\begin{figure}
  \centering
  \includegraphics{build/L1.pdf}
  \caption{Logarithmische Darstellung des gemessenen Druckes im wärmeren Reservoir
   in Abhängigkeit von der Temperatur und Ausgleichsfunktion.}
  \label{fig:dampf}
\end{figure}

Die Parameter der Ausgleichsrechnung sind
\begin{align*}
  A = \SI{-2444.45(7910)}{1/K}\,, \\
  B = 10.09\pm0.26 \,.
\end{align*}
Die Verdampfungswärme lässt sich nun mithilfe des Zusammenahnges
%\begin{equation}
%
%\end{equation}
berechnen. Dabei muss darauf geachtet werden...(MOLARE MASSE)
