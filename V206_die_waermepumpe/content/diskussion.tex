\section{Diskussion}
\label{sec:Diskussion}

Insgesamt sind die Messergebnisse als konsistent zu bewerten. Sie zeigen, dass eine
reale Wärmepumpe deutlich schlechter arbeitet als eine ideale.
Bereits an den Graphen der Temperaturverläufe wird deutlich, dass die Temperatur
im jeweiligen Reservoir nicht steig ansteigt bzw. abfällt, sondern dass die
Steigung bzw. das Gefälle des Graphen mit einer steigenden Temperaturdifferenz
deutlich abflacht. Diese Erkenntnis bestätigen auch die zu verschiedenen Zeitpunkten
berechneten Differentialquotienten $\dot{T}_\text{warm}$ und $\dot{T}_\text{kalt}$.
Dieser Zusammenhang wird von der Theorie vorhergesagt, sodass geurteilt werden kann,
dass Experiment und Theorie hier in keinem Widerspruch stehen.

Auffällig ist zudem die stark von den idealen Werten abweichende Güteziffer $\nu_{\symup{real}}$.
Dennoch zeigt sie qualitativ ein Gefälle, wie es auch theoretisch zu erwarten ist.

Mögliche Gründe für die zuvor genannten Beobachtungen sind die schlechte Isolierung der Wärmereservoire, sowie
der nicht ideale Wirkungsgrad des Kompressors, was dadurch zu begründen ist, dass die Kompression des Gases
in der Realität nicht vollständig adiabatisch erfolgt. Die Wärmereservoire bestanden
aus lediglich durch eine Styroporschicht isolierten und oben nur dürftig durch Holzplatten
abgedeckten Eimern. Zudem waren sämtliche Rohre nicht oder nur sehr schwach isoliert.
Es kann also nicht von einem reversiblen Prozess ausgegangen werden, wie es in der
Theorie der Fall ist.

Außerdem muss angemerkt werden, dass die Messung insgesamt ungenau ist, da fünf Messwerte
zur gleichen Zeit abgelesen werden sollen, was jedoch praktisch nicht möglich ist.
Zudem waren sie Skalen der Manometer grob, sodass die Drücke nicht genau abgelesen
werden konnten.
