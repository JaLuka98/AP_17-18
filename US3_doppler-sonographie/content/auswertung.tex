\section{Auswertung}
\label{sec:Auswertung}

\subsection{Bestimmung der Strömungsgeschwindigkeit als Funktion des Dopplerwinkels}
\label{subsec:geschwindigkeit}

Zur Bestimmung der Strömungsgeschwindigkeit als Funktion des Dopplerwinkels müssen
zunächst die Dopplerwinkel gemäß Gleichung \eqref{eqn:dopplerwinkel} berechnet.
Es ergeben sich für die Prismenwinkel $\theta=15°$, $\theta=30°$ und $\theta=60°$
die Dopplerwinkel
\begin{align*}
  \alpha_{15}=80{,}06° \,,\\
  \alpha_{30}=70{,}53°\,,\\
  \alpha_{60}=54{,}74° \,.\\
\end{align*}
Dabei werden für die Schallgeschindigkeiten die Werte $c_{\symup{L}}=1800\,\symup{\frac{m}{s}}$
und $c_{\symup{P}}=2700\,\symup{\frac{m}{s}}$ der Versuchsanleitung \cite{Versuchsanleitung} entnommen.
Mithilfe dieser Winkel kann dann mit Gleichung \eqref{eqn:geschwindigkeit} aus den
Messwerten die Strömungsgeschwindigkeit $v$ berechnet werden. Die Messwerte und die
daraus berechneten Strömungsgeschwindigkeiten im Rohr mit einem Innendurchmesser von
16\,mm befinden sich in Tabelle \ref{tab:dick}, die für das Rohr mit einem
Innendurchmesser von 10\,mm in Tabelle \ref{tab:mittel} und die für das Rohr
mit einem Innendurchmesser von 7\,mm in Tabelle \ref{tab:dünn}.

\begin{table}[htp]
	\begin{center}
    \caption{Messwerte zum dicken Rohr ($d=16\,$mm) und daraus berechnete Werte.}
    \label{tab:dick}
		\begin{tabular}{ccccccc}
		\toprule
			{U/min} & {$\Delta \nu_{15}/$Hz} & {$\Delta \nu_{30}/$Hz} & {$\Delta \nu_{60}/$Hz} &
      {$v_{15}/\frac{\symup{m}}{\symup{s}}$} & {$v_{30}/\frac{\symup{m}}{\symup{s}}$} &
      {$v_{60}/\frac{\symup{m}}{\symup{s}}$}\\
			\midrule
			3000 & -73 & 110 & -159 & -0,19 & 0,15 & -0,12\\
			4000 & -98 & 183 & -269 & -0,26 & 0,25 & -0,21\\
			5000 & -134 & 262 & -378 & -0,35 & 0,35 & -0,29\\
			6000 & -195 & 366 & -568 & -0,51 & 0,49 & -0,44\\
			7000 & -256 & 476 & -732 & -0,67 & 0,64 & -0,57\\
			8000 & -293 & 549 & -800 & -0,76 & 0,74 & -0,62\\
		\bottomrule
		\end{tabular}
	\end{center}
\end{table}

\begin{table}[htp]
	\begin{center}
    \caption{Messwerte zum mittleren Rohr ($d=10\,$mm) und daraus berechnete Werte.}
    \label{tab:mittel}
		\begin{tabular}{ccccccc}
		\toprule
			{U/min} & {$\Delta \nu_{15}/$Hz} & {$\Delta \nu_{30}/$Hz} & {$\Delta \nu_{60}/$Hz} &
      {$v_{15}/\symup{\frac{m}{s}}$} & {$v_{30}/\symup{\frac{m}{s}}$} & {$v_{60}/\symup{\frac{m}{s}}$}\\
			\midrule
			3000 & -220 & 354 & -354 & -0,57 & 0,48 & -0,28\\
			4000 & -366 & 586 & -671 & -0,95 & 0,79 & -0,52\\
			5000 & -513 & 781 & -989 & -1,34 & 1,05 & -0,77\\
			6000 & -659 & 1001 & -1355 & -1,72 & 1,35 & -1,06\\
			7000 & -763 & 1202 & -1855 & -1,99 & 1,62 & -1,45\\
			8000 & -854 & 1251 & -2130 & -2,23 & 1,69 & -1,66\\
		\bottomrule
		\end{tabular}
	\end{center}
\end{table}

\begin{table}[htp]
	\begin{center}
    \caption{Messwerte zum dünnen ($d=7\,$mm) Rohr und daraus berechnete Werte.}
    \label{tab:dünn}
		\begin{tabular}{ccccccc}
		\toprule
			{U/min} & {$\Delta \nu_{15}/$Hz} & {$\Delta \nu_{30}/$Hz} & {$\Delta \nu_{60}/$Hz} &
      {$v_{15}/\symup{\frac{m}{s}}$} & {$v_{30}/\symup{\frac{m}{s}}$} & {$v_{60}/\symup{\frac{m}{s}}$}\\
			\midrule
			3000 & -433 & 629 & -720 & -1,13 & 0,85 & -0,56\\
			4000 & -653 & 958 & -1276 & -1,70 & 1,29 & -0,99\\
			5000 & -848 & 1270 & -1965 & -2,21 & 1,71 & -1,53\\
			6000 & -1099 & 1575 & -2698 & -2,87 & 2,13 & -2,10\\
			7000 & -1245 & 1825 & -3284 & -3,25 & 2,46 & -2,56\\
			8000 & -1331 & 1953 & -3772 & -3,47 & 2,64 & -2,94\\
		\bottomrule
		\end{tabular}
	\end{center}
\end{table}

Nun wird für jedes Rohr für jeden der Dopplerwinkel die Größe $\frac{\Delta \nu}{\cos(\alpha)}$
gegen die berechnete Strömungsgeschwindigkeit aufgetragen. Dies ist in den Abbildungen
\ref{fig:dick_15} bis \ref{fig:dünn_60} im Anhang zu sehen. In allen Abbildungen sind
lineare Zusammenhänge zu sehen.

\subsection{Bestimmung des Strömungsprofils der Dopplerflüssigkeit}
\label{subsec:strömungsprofil}
