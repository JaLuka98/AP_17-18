\section{Auswertung}
\label{sec:Auswertung}

\subsection{Bestimmung der Strömungsgeschwindigkeit als Funktion des Dopplerwinkels}
\label{subsec:geschwindigkeit}

Zur Bestimmung der Strömungsgeschwindigkeit als Funktion des Dopplerwinkels müssen
zunächst die Dopplerwinkel gemäß Gleichung \eqref{eqn:dopplerwinkel} berechnet werden.
Es ergeben sich für die Prismenwinkel $\theta=15°$, $\theta=30°$ und $\theta=60°$
die Dopplerwinkel
\begin{align*}
  \alpha_{15}=80{,}06° \,,\\
  \alpha_{30}=70{,}53°\,,\\
  \alpha_{60}=54{,}74° \,.\\
\end{align*}
Dabei werden für die Schallgeschwindigkeiten die Werte $c_{\symup{L}}=1800\,\symup{\frac{m}{s}}$
und $c_{\symup{P}}=2700\,\symup{\frac{m}{s}}$ der Versuchsanleitung \cite{Versuchsanleitung} entnommen.
Mithilfe dieser Winkel kann dann mit Gleichung \eqref{eqn:deltanu} aus den
Messwerten die Strömungsgeschwindigkeit $v$ berechnet werden. Die Messwerte und die
daraus berechneten Strömungsgeschwindigkeiten im Rohr mit einem Innendurchmesser von
16\,mm befinden sich in Tabelle \ref{tab:dick}, die für das Rohr mit einem
Innendurchmesser von 10\,mm in Tabelle \ref{tab:mittel} und die für das Rohr
mit einem Innendurchmesser von 7\,mm in Tabelle \ref{tab:dünn}.

\begin{table}[htp]
	\begin{center}
    \caption{Messwerte zum dicken Rohr ($d=16\,$mm) und daraus berechnete Werte.}
    \label{tab:dick}
		\begin{tabular}{ccccccc}
		\toprule
			{U/min} & {$\Delta \nu_{15}/$Hz} & {$\Delta \nu_{30}/$Hz} & {$\Delta \nu_{60}/$Hz} &
      {$v_{15}/\frac{\symup{m}}{\symup{s}}$} & {$v_{30}/\frac{\symup{m}}{\symup{s}}$} &
      {$v_{60}/\frac{\symup{m}}{\symup{s}}$}\\
			\midrule
			3000 & -73 & 110 & -159 & -0,19 & 0,15 & -0,12\\
			4000 & -98 & 183 & -269 & -0,26 & 0,25 & -0,21\\
			5000 & -134 & 262 & -378 & -0,35 & 0,35 & -0,29\\
			6000 & -195 & 366 & -568 & -0,51 & 0,49 & -0,44\\
			7000 & -256 & 476 & -732 & -0,67 & 0,64 & -0,57\\
			8000 & -293 & 549 & -800 & -0,76 & 0,74 & -0,62\\
		\bottomrule
		\end{tabular}
	\end{center}
\end{table}

\begin{table}[htp]
	\begin{center}
    \caption{Messwerte zum mittleren Rohr ($d=10\,$mm) und daraus berechnete Werte.}
    \label{tab:mittel}
		\begin{tabular}{ccccccc}
		\toprule
			{U/min} & {$\Delta \nu_{15}/$Hz} & {$\Delta \nu_{30}/$Hz} & {$\Delta \nu_{60}/$Hz} &
      {$v_{15}/\symup{\frac{m}{s}}$} & {$v_{30}/\symup{\frac{m}{s}}$} & {$v_{60}/\symup{\frac{m}{s}}$}\\
			\midrule
			3000 & -220 & 354 & -354 & -0,57 & 0,48 & -0,28\\
			4000 & -366 & 586 & -671 & -0,95 & 0,79 & -0,52\\
			5000 & -513 & 781 & -989 & -1,34 & 1,05 & -0,77\\
			6000 & -659 & 1001 & -1355 & -1,72 & 1,35 & -1,06\\
			7000 & -763 & 1202 & -1855 & -1,99 & 1,62 & -1,45\\
			8000 & -854 & 1251 & -2130 & -2,23 & 1,69 & -1,66\\
		\bottomrule
		\end{tabular}
	\end{center}
\end{table}

\begin{table}[htp]
	\begin{center}
    \caption{Messwerte zum dünnen ($d=7\,$mm) Rohr und daraus berechnete Werte.}
    \label{tab:dünn}
		\begin{tabular}{ccccccc}
		\toprule
			{U/min} & {$\Delta \nu_{15}/$Hz} & {$\Delta \nu_{30}/$Hz} & {$\Delta \nu_{60}/$Hz} &
      {$v_{15}/\symup{\frac{m}{s}}$} & {$v_{30}/\symup{\frac{m}{s}}$} & {$v_{60}/\symup{\frac{m}{s}}$}\\
			\midrule
			3000 & -433 & 629 & -720 & -1,13 & 0,85 & -0,56\\
			4000 & -653 & 958 & -1276 & -1,70 & 1,29 & -0,99\\
			5000 & -848 & 1270 & -1965 & -2,21 & 1,71 & -1,53\\
			6000 & -1099 & 1575 & -2698 & -2,87 & 2,13 & -2,10\\
			7000 & -1245 & 1825 & -3284 & -3,25 & 2,46 & -2,56\\
			8000 & -1331 & 1953 & -3772 & -3,47 & 2,64 & -2,94\\
		\bottomrule
		\end{tabular}
	\end{center}
\end{table}
Auffällig ist, dass die ermittelten Geschwindigkeiten $v_{\symup{i}}$ genau so wie die gemessenen
Werte für die Frequenzverschiebung $\Delta \nu_{\symup{i}}$ unterschiedliche Vorzeichen haben.
Der Grund dafür wird schnell klar, wenn noch ein Mal das Dopplerprisma aus Abbildung
\ref{fig:dopplerprisma} betrachtet wird. Die Winkel sind bezüglich der Strömungsgeschwindigkeit
unterschiedlich orientiert, sodass das Vorzeichen eine Aussage über die Strömungsgeschwindigkeit
in Relation zur Orientierung des Dopplerwinkels angibt. Unter Betrachtung des verwendeten
Dopplerprismas ist damit auch konsistent, dass die Frequenzverschiebung $\Delta \nu_{30}$
(und damit auch die daraus berechnete Strömungsgeschwindigkeit) ein anderes Vorzeichen hat,
als die beiden anderen.

Die Ergebnisse für die Strömungsgeschwindigkeiten $v_{\symup{i}}$ bei einem
konstanten Fluss werden gemäß den Gleichungen \eqref{eqn:mean} und \eqref{eqn:std}
gemittelt. Es ergeben sich die in Tabelle \ref{tab:mittelwerte} dargestellten Mittelwerte.
Dabei ist darauf zu achten, dass die Beträge der Störmungsgeschwindigkeiten verwendet werden.


\begin{table}[htp]
	\begin{center}
    \caption{Mittelwerte für die aus den verschiedenen Dopplerwinkeln berechneten
    Strömungsgeschwindigkeiten.}
    \label{tab:mittelwerte}
		\begin{tabular}{ccccccc}
		\toprule
			{U/min} & {$\overline{\nu_{{\text{dick}}}}/$Hz} & {$\sigma_{{\nu_{\text{dick}}}}/$Hz} &
      {$\overline{\nu_{\text{mittel}}}/$Hz} & {$\sigma_{{\nu_{\text{mittel}}}}/$Hz} &
      {$\overline{\nu_{\text{dünn}}}/$Hz} & {$\sigma_{{\nu_{\text{dünn}}}}/$Hz}\\
			\midrule
			3000 & 0,154 & 0,027 & 0,44 & 0,12 & 0,85 & 0,23\\
			4000 & 0,237 & 0,020 & 0,76 & 0,18 & 1,33 & 0,29\\
			5000 & 0,333 & 0,027 & 1,05 & 0,23 & 1,82 & 0,28\\
			6000 & 0,482 & 0,028 & 1,38 & 0,27 & 2,4 & 0,3\\
			7000 & 0,63  & 0,04 & 1,69 & 0,23 & 2,8 & 0,3\\
			8000 & 0,71  & 0,06 & 1,86 & 0,26 & 3,0 & 0,3\\
		\bottomrule
		\end{tabular}
	\end{center}
\end{table}

Nun wird für jedes Rohr für jeden der Dopplerwinkel die Größe $\frac{\Delta \nu}{\cos(\alpha)}$
gegen die berechnete Strömungsgeschwindigkeit aufgetragen. Dies ist in den Abbildungen
\ref{fig:dick_15} bis \ref{fig:duenn_60} im Anhang zu sehen. In allen Abbildungen sind
lineare Zusammenhänge zu erkennen.

\subsection{Bestimmung des Strömungsprofils der Dopplerflüssigkeit}
\label{subsec:strömungsprofil}

In Tabelle \ref{tab:profil} sind die mittlere Frequenzverschiebung,
die Streuintensität (std.) in Prozent und die aus der Frequenzverschiebung
berechnete mittlere Strömungsgeschwindigkeit in Abhängigkeit der Eindringtiefe,
ausgedrückt in µs, eingetragen. Es wurden zwei Messreihen aufgenommen, jeweils für
6100 und 3920 Umdrehungen pro Minute.

\begin{table}[htp]
        \begin{center}
          \caption{Werte für das Erstellen der Strömungsprofile.}
          \label{tab:profil}
                \begin{tabular}{ccccccc}
                \toprule
                & \multicolumn{3}{c}{$6100$rpm} & \multicolumn{3}{c}{$3920$rpm} \\\cmidrule(lr){2-4}\cmidrule(lr){5-7}
                        {$t/$µs} & {$\bar{\nu}/$Hz} & {std./\%} & {$\bar{v}/\symup{\frac{m}{s}}$} & {$\bar{\nu}/$Hz} & {std./\%} & {$\bar{v}/\symup{\frac{m}{s}}$}\\
                        \midrule
                        10,0 & -452 &  5,5 & -1,77 & -244 &  6,7 & -0,95\\
                        10,5 & -464 &  5,6 & -1,81 & -244 &  6,5 & -0,95\\
                        11,0 & -452 &  5,9 & -1,77 & -232 &  6,4 & -0,91\\
                        11,5 & -299 & 13,1 & -1,17 & -195 & 11,7 & -0,76\\
                        12,0 & -208 & 12,6 & -0,81 & -134 & 13,0 & -0,52\\
                        12,5 & -244 &  6,3 & -0,95 & -146 & 10,0 & -0,57\\
                        13,0 & -305 &  6,5 & -1,17 & -171 &  7,4 & -0,66\\
                        13,5 & -403 &  3,8 & -1,53 & -208 &  5,6 & -0,79\\
                        14,0 & -476 &  3,2 & -1,79 & -244 &  6,0 & -0,92\\
                        14,5 & -537 &  2,8 & -1,99 & -269 &  4,3 & -1,00\\
                        15,0 & -574 &  3,5 & -2,11 & -269 &  4,0 & -0,99\\
                        15,5 & -574 &  2,9 & -2,09 & -256 &  3,4 & -0,93\\
                        16,0 & -513 &  3,6 & -1,85 & -220 &  4,7 & -0,79\\
                        16,5 & -452 &  4,0 & -1,62 & -195 &  6,9 & -0,70\\
                        17,0 & -366 &  6,2 & -1,30 & -159 & 10,1 & -0,56\\
                        17,5 & -391 &  7,5 & -1,38 & -183 & 10,3 & -0,64\\
                        18,0 & -439 &  4,1 & -1,53 & -195 &  9,6 & -0,68\\
                        18,5 & -452 &  5,8 & -1,57 & -195 &  8,9 & -0,68\\
                        19,0 & -439 &  5,7 & -1,51 & -195 &  7,2 & -0,67\\
                        19,5 & -464 &  5,5 & -1,59 & -195 &  6,6 & -0,67\\
                \bottomrule
                \end{tabular}
        \end{center}
\end{table}

Die Geschwindigkeit wird mithilfe von Gleichung \eqref{eqn:geschwindigkeit} berechnet.
Es muss beachtet werden, dass die Messtiefe in µs angegeben wird. In Acryl, dem Material
des Prismas, entsprechen 4\,µs 10\,mm und in der Flüssigkeit entsprechen 4\,µs 6\,mm.
Da die Länge der Vorlaufstrecke im Glasprisma nach \cite{Versuchsanleitung} $30{,}7$\,mm
beträgt, beginnt das Rohr erst bei einer Eindringtiefe von $12{,}$28\,µs. Allgemein
sind so also die unterschiedlichen Anteile der Medien an der Laufstrecke zu berücksichtigen.
Dies kann geschehen, indem eine effektive Ultraschallgeschwindigkeit $c_\text{eff}$ in die Formel
eingesetzt wird, die für eine Eindringtiefe kleiner als $12{,}28$\,µs gleich $c_{\symup{P}}$
ist und für größere $t$ die unterschiedlichen Medien durch
\begin{equation}
  c_\text{eff} = \frac{12{,}28}{t} c_{\symup{P}} + \left(1-\frac{12{,}28}{t} c_{\symup{L}} \right)
\end{equation}
berücksichtigt. Es sei angemerkt, dass hier die Unterscheidung zwischen der Flüssigkeit im Rohr
und der Ummantelung nicht vorgenommen werden kann, da das Material der Hülle und damit auch nicht die
Ultraschallgeschwindigkeit im Medium bekannt ist.

Anhand dieser Daten können nun zwei Strömungsprofile erstellt werden, welche in den
Abbildungen \ref{fig:profil70} und \ref{fig:profil45} zu sehen sind. Es wird nur
der Betrag der Geschwindigkeit aufgetragen, da ein Vorzeichen nur eine andere Richtung der
Strömung anzeigt.


\begin{figure}
  \centering
  \includegraphics[width=15cm]{build/profil70.pdf}
  \caption{Betrag der mittleren Strömungsgeschwindigkeit und Streuintensität in Prozent
  in Abhängigkeit der Eindringtiefe für .. Umdrehungen pro Minute}
  \label{fig:profil70}
\end{figure}

\begin{figure}
  \centering
  \includegraphics[width=15cm]{build/profil45.pdf}
  \caption{Betrag der mittleren Strömungsgeschwindigkeit und Streuintensität in Prozent
  in Abhängigkeit der Eindringtiefe für .. Umdrehungen pro Minute}
  \label{fig:profil45}
\end{figure}
