\section{Diskussion}
\label{sec:Diskussion}

Die Messergebnisse zeigen im Allgemeinen die theoretisch zu erwartenden Zusammenhänge.
Die in den Tabellen \ref{tab:dick}, \ref{tab:mittel} und \ref{tab:dünn} dargestellten
Strömungsgeschwindigkeiten sind (wie auch theoretisch zu erwarten) im Rohr mit
dem kleinsten Innendurchmesser am größten und im Rohr mit dem größten Innendurchmesser
am kleinsten. Es ist jedoch auffällig, dass die bestimmten Strömungsgeschwindigkeiten
für einen Prismenwinkel von $\theta=15°$ stets am größten und die für einen Prismenwinkel
von $\theta=60°$ stets am kleinsten sind.

Wird $\frac{\Delta \nu}{\cos(\alpha)}$ gegen die berechneten Strömungsgeschwindigkeiten
augetragen, so ergeben sich stets lineare Zusammenhänge. Dies ist auch theoretisch zu
erwarten, da nach Gleichung \eqref{eqn:deltanu} die Geschwindigkeit und der oben genannte
Faktor proportional zueinander sein sollten.

Es ist anzumerken, dass keine Aussage darüber getroffen werden kann, ob die berechneten
Werte dem realen Wert enstprechen, da kein Vergleichswert vorliegt. In der
Rechnung wird eine ideale laminare Strömung angenommen.
Insbesondere bei hohen Strömungsgeschwindigkeiten im kleinen Rohr ist es jedoch
wahrscheinlich, dass Abweichungen durch Turbulenzen auftreten.
Zudem ist anzumerken, dass das Ablesen der Werte aus dem Programm teilweise nur
ungenau möglich war, da die Werte teils starke Schwankungen aufwiesen.

Die beiden angefertigten Strömungsprofile zeigen im Allgemeinen das erwartete Verhalten.
Werden beide miteinander verglichen, so fällt auf, dass die mittlere Strömungsgeschwindigkeit
bei 6100 Umdrehungen pro Minute durchweg höher als bei 3920 Umdrehungen pro Minute ist, was
zu erwarten war. Es muss davon ausgegangen werden, dass mindestens die ersten fünf Messpunkte
für die Beurteilung der Kurven nicht verwendet werden können, da diese im Prismamaterial
bzw. unter Umständen in der Ummantelung des Rohres genommen wurden und eine Untersuchung
einer Strömungsgeschwindigkeit dort nicht sinnvoll ist.\\
Zuerst seien die Streuintensitäten zu bewerten: Diese zeigen einen schlüssigen Verlauf,
da sie zur Mitte des Rohrs (bei ca. $15{,}61$\,µs) gering werden, da dort eine näherungsweise
laminare Strömung vorliegt. Zu den Rändern des Rohres nimmt sie hingegen zu, was mit zunehmender
Turbulenz erklärt wird.\\
Der Verlauf der mittleren Strömungsgeschwindigkeit ist ebenso konsistent. Diese
sollte in der Mitte des Rohrs maximal sein und zu den Rändern hin abnehmen, dies kann den Graphen
auch so entnommen werden.\\
Unklar bleibt der Einfluss der Hülle des Rohres auf die Beurteilung der Messung, wobei
jedoch geurteilt werden kann, dass auch bei der Benötigung von signifikanten Korrekturen
durch den Einfluss der Hülle auf die Schalllaufzeiten der grobe Verlauf der Messwerte
konsistent bleibt. Insofern ist dieser Teil des Versuches als erfolgreich zu bewerten, da
der Verlauf der Strömungsgeschwindigkeit im Rohr mit Ultraschall gut untersucht werden konnte.
