\section{Diskussion}
\label{sec:Diskussion}

Im Allgemeinen sind die Messergebnisse als gut zu bewerten, da die Messwerte
insgesamt in guter Näherung auf einer Geraden liegen und sie auch für die
Leistung in guter Näherung dem Verlauf der Theoriekurve folgen.

Bei den Messwerten für die Monozelle ist eine starke Sprungstelle zu verzeichnen.
Diese kann auch durch die Skalenwechsel nicht erklärt werden. Da der Wert, der
hier an der Sprungstelle liegt auch bei der Leistung deutlich von der
Theoriekurve abweicht, liegt nahe, dass hier ein Messfehler vorliegt. Der errechnete
Wert für die Leerlaufspannung der Monozelle liegt mit
$U_{\symup{0,mono}}=(1,337\pm0,018)\,\text{V}$ sehr nah an dem gemessenen
Wert $U_{\symup{0}}=1,335\,\text{V}$. Der systematische Fehler der direkten
Messung der Leerlaufspannung wurde bereits in der Auswertung diskutiert.

Bei der Messung des Innenwiderstandes $R_{\symup{i}}$ und der Leerlaufspannung
$U_{\symup{0}}$ bei angelegter Gegenspannung, liegt ein systematischer Fehler durch
den Skalenwechsel vor. Die beiden Fits haben mit
$a_{\symup{gegen}}=(6,138\pm0,213)\,\symup{\Omega}$ und
$a_{\symup{2,gegen}}=(5,852\pm0,239)\,\symup{\Omega}$ zwar annähernd
die gleiche Steigung, jedoch liegen die y-Achsenabschnitte mit
$b_{\symup{gegen}}=(1,600\pm0,036)\,\text{V}$ und
$b_{\symup{2,gegen}}=(1,281\pm0,014)\text{V}$
relativ weit auseinander. Es kann jedoch keine Aussage darüber getroffen werden,
bei welcher Skala der systematische Fehler entsteht. Ein möglicher Grund für den
entstandenen Fehler ist der Wechsel des Innenwiderstandes des Messgerätes beim
Umschalten der Skalen. Die Skalen scheinen nicht gut geiicht zu sein.

Die Messwerte für die Rechteckspannung liegen hingegen in guter Näherung auf einer Geraden,
sodass die durch die lineare Regression bestimmten Werte
$R_{\symup{i,rechteck}}=(67,73\pm2,33)\,\symup{\Omega}$ und
$U_{\symup{0,rechteck}}=(0,987\pm0,011)\,\text{V}$ als genau zu bewerten sind.

Die Messwerte für die gemessene Klemmenspannung $U_{\symup{K}}$ bei angelegter
Sinusspannung in Abhängigkeit von der Stromstärke $I$ liegen ebenfalls in guter Näherung auf
einer Geraden. Lediglich für hohe Klemmenspannungen weichen die Messwerte leicht
vom linearen Zusammenhang ab. Dies liegt daran, dass der Zusammenhang auch im Allgemeinen
nicht linear ist und der Gleichung \eqref{} folgt.

Die aus den Messwerten berechneten Werte für die Leistung am Belastungswiderstand
$R_{\symup{a}}$ in Abhängigkeit desselben folgen im Allgemeinen sehr gut dem Verlauf
der Theoriegkurve. Neben dem bereits diskutierten stark abweichenden Wert ist noch
auffällig, dass die Messwerte nahe dem Maximum der Funktion leicht über dem theoretischen
Erwartungswert liegen. Ein möglicher Grund hierfür ist, dass sich die Theoriekurve selbst
nur aus den selbst ermittelten, fehlerbehafteten Werten für die Leerlaufspannung
$U_{\symup{0}}$ und den Innenwiderstand $R_{\symup{i}}$ der Monozelle
berechnen lässt, sodass die Kurve selbst ungenau ist.
