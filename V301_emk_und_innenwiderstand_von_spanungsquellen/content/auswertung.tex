\section{Auswertung}
\label{sec:Auswertung}

Die direkte Messung der Leerlaufspannung $U_{\symup{0}}$ über das Voltmeter
mit einem Eingangswiderstand $R_{\symup{v}}\geq10$\,M$\symup{\Omega}$ ergibt
\begin{equation}
  %U_{\symup{0}}=1,335\,\text{V}\,.
  U_{\symup{0,exp}} = \SI{1.335}{\volt}\,.
\end{equation}


Nun sollen der Innenwiderstand $R_{\symup{i}}$ und die Leerlaufspannung $U_{\symup{0}}$
für alle vier Messreihen durch eine lineare Ausgleichsrechnung berechnet werden.
Dazu wird die Klemmenspannung $U_{\symup{K}}$ gegen die Stromstärke $I$ aufgetragen
und anschließend eine lineare Ausgleichsrechnung nach $f(x)=ax+b$ mit python scipy.optimize durchgeführt.
Nach Gleichung \eqref{eqn:klemmevoni} entspricht der Betrag der Steigung $a$ der Ausgleichsgeraden dabei dem Innenwiderstand des
Gerätes und die Leerlaufspannung dem y-Achsenabschnitt $b$. Für die Monozelle
ergibt sich der in Abbildung \ref{fig:monozelle} dargestellte Zusammenhang mit den
Parametern
\begin{align}
  R_{\symup{i,mono}}&=-a_{\symup{mono}} = \SI{6.446(0257)}{\ohm} \,, \\
  U_{\symup{0,mono}}&=b_{\symup{mono}} = \SI{1.337(0018)}{\volt} \,.
\end{align}
Die hierfür verwendeten Messdaten finden sich in Tabelle \ref{tab:monozelle}.

\begin{table}
  \centering
  \caption{Messdaten zur an der Monozelle gemessennen Klemmenspannung $U_{\symup{K}}$
  in Abhängigkeit von der durch den Widerstand geregelten Stromstärke $I$}
  \label{tab:monozelle}
  \begin{tabular}{c c}
    \toprule
    $U_{\symup{K}}/$V & $I/$mA\\
    \midrule
    0,25	&  170\\
    0,73	&  80\\
    0,87	&  79\\
    0,975	& 59,5\\
    1,065	& 46,0\\
    1,095	& 38,0\\
    1,131	& 33,0\\
    1,170	& 26,0\\
    1,170	& 25,0\\
    1,185	& 23,0\\
    1,193	& 21,0\\
    \bottomrule
  \end{tabular}
\end{table}

\begin{figure}
  \centering
  \includegraphics{build/monozelle.pdf}
  \caption{Auftragung der an der Monozelle gemessenen Klemmenspannung $U_{\symup{K}}$
  gegen die Stromstärke $I$}
  \label{fig:monozelle}
\end{figure}

Deutlich erkennbar ist hier eine Sprungstelle. Da diese jedoch nicht mit dem Wechsel
der Skalen der Messgeräte zusammenhängt, kann die Ausgleichsrechnung hier nicht besser durchgeführt
werden.

Nun soll der gemessene Wert für die Leerlaufspannung $U_{\symup{0}}$ mit dem
berechneten Wert verglichen werden. Da der Widerstand des Voltmeters entgegen
der theoretischen, idealen Erwartung endlich ist, liegt bei der direkten Messung
ein systematischer Fehler vor. Der Zusammenhang für die Leerlaufspannung folgt
nach Gleichung \eqref{eqn:klemmevoni} zu
\begin{equation}
  U_{\symup{0}}=U_{\symup{K}}+U_{\symup{K}}\frac{R_{\symup{i}}}{R_{\symup{v}}}\,.
\end{equation}
Dabei entspricht $U_{\symup{K}}$ dem gemessenen Wert für die Leerlaufspannung
und $R_{\symup{v}}$ dem Eingangswiderstand des Voltmeters.
Der Systematische Fehler bei der manuellen Messung der Leerlaufspannung ergibt
sich damit zu
\begin{equation}
  \frac{\Delta U}{U_{\symup{K}}}=\frac{R_{\symup{i}}}{R_{\symup{v}}} \leq \SI{6.45(026)}\cdot10^{-5}\%\,. \nonumber
\end{equation}

Auch wenn das Voltmeter hinter das Amperemeter gelegt wird, entstehen systematische
Fehler. In diesem Fall enthält die gemessene Spannung $U_{\symup{K}}$ zusätzlich
zur am Innenwiderstand der Spannungsquelle abfallenden Spannung noch die am Innenwiderstand
des Amperemeters abfallende Spannung.

Für die Messreihe der Monozelle mit einer angelgeten Gegenspannung ergibt sich
bei einem analogen Vorgehen der in Abbildung \ref{fig:gegenspannung} gezeigte Zusammenhang.
Die zugrundeliegenden Messwerte können Tabelle \ref{tab:gegenspannung} entnommen werden.
Dabei ist eine starke Sprungstelle beim Skalenwechsel zu verzeichnen, weswegen für
die Messwerte getrennte Ausgleichsrechnungen durchgeführt werden. Die Parameter ergeben sich zu
\begin{align*}
%  a_{\symup{gegen}}&=(6,138\pm0,213)\,\symup{\Omega}\,,\\
%  b_{\symup{gegen}}&=(1,600\pm0,036)\,\text{V}\,,\\
%  a_{\symup{2,gegen}}&=(5,852\pm0,239)\,\symup{\Omega}\,,\\
%  b_{\symup{2,gegen}}&=(1,281\pm0,014)\text{V}\,.
  a_{\symup{1,gegen}} &= \SI{6.14(021)}{\ohm}\,,\\
  b_{\symup{1,gegen}} &= \SI{1.60(004)}{\volt}\,,\\
  a_{\symup{2,gegen}} &= \SI{5.85(024)}{\ohm}\,,\\
  b_{\symup{2,gegen}} &= \SI{1.281(0014)}{\volt}\,.
\end{align*}

\begin{table}
  \centering
  \caption{Messdaten zur an der Monozelle gemessennen Klemmenspannung $U_{\symup{K}}$
  in Abhängigkeit von der durch den Widerstand geregelten Stromstärke $I$ bei angelegter
  Gegenspannung $U_{\symup{gegen}}$}
  \label{tab:gegenspannung}
  \begin{tabular}{c c}
    \toprule
    $U_{\symup{K}}/$V & $I/$mA\\
    \midrule
    3,19	&  260\\
    2,30	&  110\\
    1,95	&  60\\
    1,82	&  93\\
    1,70	&  70\\
    1,65	&  60,5\\
    1,58	&  51,5\\
    1,55	&  50\\
    1,50	&  39\\
    1,48	&  33\\
    1,46	&  29,5\\
    \bottomrule
  \end{tabular}
\end{table}

\begin{figure}
  \centering
  \includegraphics{build/gegenspannung.pdf}
  \caption{Auftragung der an der Monozelle gemessenen Klemmenspannung $U_{\symup{K}}$
  gegen die Stromstärke $I$ bei angelegter Gegenspannung $U_{\symup{gegen}}$ }
  \label{fig:gegenspannung}
\end{figure}

Da die Steigungen der Ausgleichsgeraden annähernd gleich sind, wird als Innenwiderstand $R_{\symup{i}}$
das Mittel dieser beiden Werte gebildet. Für die Leerlaufspannung wird ebenso verfahren.
Damit ergeben sich unter Beachtung der Gauß'schen Fehlerfortpflanzung die Werte zu
\begin{align*}
  %R_{\symup{i,gegen}}&=(6,00\pm0,16)\,\text{mA} \,,\\
  %U_{\symup{0,gegen}}&=(1,44\pm0,02)\,\text{V} \,.
  R_{\symup{i,gegen}} &= \SI{6.00(016)}{\ohm} \,,\\
  U_{\symup{0,gegen}} &= \SI{1.440(0019)}{\volt} \,.
\end{align*}
Die Gauß'sche Fehlerfortpflanzung wird dabei mit pyhton uncertainties.unumpy
berechnet.

Für die Sinus- und die Rechteckspannung wird analog zur Monozelle verfahren. Dabei
wird als Ausgangsspannung eine Spannung von $U_{\symup{0}}\approx1$\,V verwendet.
Daraus ergeben sich die in Abbildung \ref{fig:rechteck} und \ref{fig:sinus} dargestellten
Zusammenhänge. Die zugrundeliegenden Daten können Tabelle \ref{tab:re_sin} entnommen
werden.

\begin{table}
  \centering
  \caption{Messdaten zur Klemmenspannnung $U_{\symup{K}}$ in Abhängigkeit von der
  Stromstärke $I$ bei einer Rechteck- und bei einer Sinusspnnung.}
  \label{tab:re_sin}
  \begin{tabular}{c c c c}
    %\toprule
    \hline
    \multicolumn{2}{c}{Rechteckspannung} & \multicolumn{2}{c}{Sinusspannung} \\
    \cmidrule(lr){1-2}
    \cmidrule(lr){3-4}
    $U_{\symup{K}}/$V & $I/$mA & $U_{\symup{K}}/$V & $I/$mA \\
    \midrule
    0,849 &	1,9   & 2,130 	& 0,903 \\
    0,846	& 1,95  & 2,106	  & 0,903 \\
    0,834	& 2,17  & 2,076 	& 0,906 \\
    0,816	& 2,34  & 2,040 	& 0,921 \\
    0,801	& 2,75  & 1,998	  & 0,942 \\
    0,780	& 3,15  & 1,947	  & 0,990 \\
    0,750	& 3,68  & 1,848	  & 1,089 \\
    0,696	& 4,49  & 1,776	  & 1,173 \\
    0,633	& 5,50  & 1,635	  & 1,359 \\
    0,516	& 7,32  & 1,500	  & 1,536 \\
    0,330	& 9,20  & 1,296	  & 1,830 \\
    -     & -     & 0,837	  & 2,493 \\
    -     & -     & 0,756	  & 2,604 \\
    \bottomrule
  \end{tabular}
\end{table}

\begin{figure}
  \centering
  \includegraphics{build/rechteck.pdf}
  \caption{Auftragung der Klemmenspannung $U_{\symup{K}}$ gegen die Stromstärke $I$
  bei einer rechteckförmigen Spannung und Graph der linearen Ausgleichsfunktion}
  \label{fig:rechteck}
\end{figure}

\begin{figure}
  \centering
  \includegraphics{build/sinus.pdf}
  \caption{Auftragung der Klemmenspannung $U_{\symup{K}}$ gegen die Stromstärke $I$
  bei einer sinusförmigen Spannung und Graph der linearen Ausgleichsfunktion}
  \label{fig:sinus}
\end{figure}

Die Ausgleichsrechnung ergibt dabei bei der Rechteckspannug für den Innenwiderstand
und die Leerlaufspannung die Werte
\begin{align*}
  %R_{\symup{i,rechteck}}&=-a_{\symup{rechteck}}=(67,73\pm2,33)\,\symup{\Omega} \,, \\
  %U_{\symup{0,rechteck}}&=b_{\symup{rechteck}}=(0,987\pm0,011)\,\text{V}\,.
  R_{\symup{i,rechteck}}&=-a_{\symup{rechteck}} = \SI{67.7(23)}{\ohm} \,, \\
  U_{\symup{0,rechteck}}&=b_{\symup{rechteck}} = \SI{0.987(0011)}{\volt} \,.
\end{align*}
Für die Sinusspannung ergibt sich
\begin{align*}
  %R_{\symup{i,sinus}}&=-a_{\symup{sinus}}=(775,49\pm24,27)\,\symup{\Omega} \,, \\
  %U_{\symup{0,sinus}}&=b_{\symup{sinus}}=(2,741\pm0,036)\,\text{V}\,. \\
  R_{\symup{i,sinus}}&=-a_{\symup{sinus}} = \SI{775(24)}{\ohm} \,, \\
  U_{\symup{0,sinus}}&=b_{\symup{sinus}} = \SI{2.74(004)}{\volt} \,.
\end{align*}


Nun soll noch die bei der Monozelle im Belastungswiderstand $R_{\symup{a}}$ umgesetzte
Leistung $P$ in Abhängigkeit vom Belastungswiderstand untersucht werden.
Der Belastunsgwiderstand ist dabei
\begin{equation}
  R_{\symup{a}}=\frac{U_{\symup{K}}}{I}\,.
\end{equation}
Die Leistung beträgt
\begin{equation}
  P=U_{\symup{K}}I\,.
\end{equation}
Die aus den Messdaten errechneten Daten sind in Tabelle \ref{tab:leistung} zu finden und
werden in Abbildung \ref{fig:leistung} dargestellt. Zusätzlich ist in \ref{fig:leistung} eine
Theoriekurve eingezeichnet, die gemäß Gleichung \eqref{eqn:leistung} dem Zusammenhang
\begin{equation}
  P=I^2 R_{\symup{a}} = \frac{U_{\symup{0}}^2 R_{\symup{a}}}{(R_{\symup{a}}+R_{\symup{i}})^2}
\end{equation}
folgt.

\begin{table}
  \centering
  \caption{Messdaten zur an der Monozelle gemessennen Klemmenspannung $U_{\symup{K}}$
  in Abhängigkeit von der Stromstärke $I$ und die daraus errechneten Werte für
  die Leistung $P$ und den Belastungswiderstand $R_{\symup{a}}$}
  \label{tab:leistung}
  \begin{tabular}{c c c c}
    \toprule
    $U_{\symup{K}}/$V & $I/$mA & $P/$mW & $R_{\symup{a}}/\symup{\Omega}$\\
    \midrule
    0,25	&  170 &   42,50  &  1,5\\
    0,73	&  80  &   58,40  &  9,1\\
    0,87	&  79  &   68,73  &  11,0\\
    0,975	& 59,5 &   58.01  &  16,4\\
    1,065	& 46,0 &   48,90  &  23,2\\
    1,095	& 38,0 &   41,61  &  28,8\\
    1,131	& 33,0 &   37,32  &  34,3\\
    1,170	& 26,0 &   30,42  &  45,0\\
    1,170	& 25,0 &   29,25  &  46,8\\
    1,185	& 23,0 &   27,25  &  51,5\\
    1,193	& 21,0 &   25,05  &  56,8\\
    \bottomrule
  \end{tabular}
\end{table}

\begin{figure}
  \centering
  \includegraphics{build/leistung.pdf}
  \caption{Auftragung der Leistung $P$ am Belastungswiderstand gegen den Belastungswiderstand
  $R_{\symup{a}}$}
  \label{fig:leistung}
\end{figure}

Es ist erkennbar, dass die Messdaten dem Verlauf der Theoriekurve folgen.
Nur ein einziger Wert weicht stark von der Theoriekurve ab. Dies ist genau der
Wert, bei dem bereits bei den Messwerten zur Monozelle eine Sprungstelle zu verzeichnen
ist, obwohl dort kein Skalenwechsel stattfindet. Auffällig ist zudem, dass die anderen
Messwerte nahe des Maximums etwas über der Theoriekurve liegen.
Da die Werte jedoch insgesamt gut dem Verlauf der Theoriekurve folgen, lässt
sich darauf schließen, dass keine signifikanten systematischen Fehler bei der
Messung vorliegen.
