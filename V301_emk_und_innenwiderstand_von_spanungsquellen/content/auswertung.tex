\section{Auswertung}
\label{sec:Auswertung}

Die direkte Messung der Leerlaufspannung $U_{\symup{0}}$ über das Voltmeter
mit einem Innenwiderstand $R_{\symup{i}}\geq10$\,M$\symup{\Omega}$ ergibt
\begin{equation}
  U_{\symup{0}}=1,335\,\text{V}\,.
\end{equation}


Nun sollen der Innemwiderstand $R_{\symup{i}}$ und die Leerlaufspannung $U_{\symup{0}}$
für alle vier Messreihen durch eine lineare Ausgleichsrechnung berechnet werden.
Dazu wird die Klemmenspannung $U_{\symup{K}}$ gegen die Stromstärke $I$ aufgetragen
und anschließend gemäß (REFERENZ) eine Ausgleichsgerade ermittelt.
Die Steigung $a$ der Ausgleichsgeraden entspricht dabei dem Innenwiderstand des
Gerätes und die Leerlaufspannung dem y-Achsenabschnitt $b$. Für die Monozelle
ergibt sich der in Abbildung \ref{fig:monozelle} dargestellte Zusammenhang mit den
Parametern
\begin{align}
  R_{\symup{i,mono}}&=-a_{\symup{i,mono}}=(6,446\pm0,257)\,\symup{\Omega} \,, \\
  U_{\symup{0,mono}}&=b_{\symup{0,mono}}=(1,337\pm0,018)\,\text{V}\,.
\end{align}
Die hierfür verwendeten Messdaten finden sich in Tabelle \ref{tab:monozelle}.

\begin{table}
  \centering
  \caption{Messdaten zur an der Monozelle gemessennen Klemmenspannung $U_{\symup{K}}$
  in Abhängigkeit von der durch den Widerstand geregelten Stromstärke $I$}
  \label{tab:monozelle}
  \begin{tabular}{c c}
    \toprule
    $U_{\symup{K}}/$V & $I/$mA\\
    \midrule
    0,25	&  170\\
    0,73	&  80\\
    0,87	&  79\\
    0,975	& 59,5\\
    1,065	& 46,0\\
    1,095	& 38,0\\
    1,131	& 33,0\\
    1,170	& 26,0\\
    1,170	& 25,0\\
    1,185	& 23,0\\
    1,193	& 21,0\\
    \bottomrule
  \end{tabular}
\end{table}

\begin{figure}
  \centering
  \includegraphics{build/monozelle.pdf}
  \caption{Auftragung der an der Monozelle gemessenen Klemmenspannung $U_{\symup{K}}$
  gegen die Stromstärke $I$}
  \label{fig:monozelle}
\end{figure}

Deutlich erkennbar ist hier eine Sprungstelle. Da diese jedoch nicht mit dem Wechsel
der Skalen der Messgeräte zusammenhängt, kann der Fit hier nicht besser angepasst
werden.
Der Systematische Fehler bei der manuellen Messung der Leerlaufspannung ergibt
sich damit zu
\begin{equation}
  \frac{\Delta U_{\symup{0}}}{U_\symup{K}}=\frac{U_{\symup{0,mono}}-U_{\symup{0,manuell}}}
  {U_\symup{K}} = 0,20\%
\end{equation}

Für die Messreihe der Monozelle mit einer angelgeten Gegenspannung ergibt sich
bei einem analogem Vorgehen der in Abbildung \ref{fig:gegenspannung} Zusammenhang.
Die zugrundeliegenden Messwerte können Tabelle \ref{tab:gegenspannung} entnommen werden.
Dabei ist eine starke Sprungstelle beim Skalenwechsel zu verzeichnen, weswegen für
die Messwerte getrennte Fits erstellt werden. Die Parameter ergeben sich zu
\begin{align}
  a_{\symup{i,gegen}}&=(6,138\pm0,213)\,\symup{\Omega}\,,\\
  b_{\symup{0,gegen}}&=(1,600\pm0,036)\,\text{V}\,,\\
  a_{\symup{2,i,gegen}}&=(5,852\pm0,239)\,\symup{\Omega}\,,\\
  b_{\symup{2,0,gegen}}&=(1,281\pm0,014)\text{V}\,.
\end{align}

\begin{table}
  \centering
  \caption{Messdaten zur an der Monozelle gemessennen Klemmenspannung $U_{\symup{K}}$
  in Abhängigkeit von der durch den Widerstand geregelten Stromstärke $I$ bei angelegter
  Gegenspannung $U_{\symup{gegen}}$}
  \label{tab:gegenspannung}
  \begin{tabular}{c c}
    \toprule
    $U_{\symup{K}}/$V & $I/$mA\\
    \midrule
    3,19	&  260\\
    2,30	&  110\\
    1,95	&  60\\
    1,82	&  93\\
    1,70	&  70\\
    1,65	&  60,5\\
    1,58	&  51,5\\
    1,55	&  50\\
    1,50	&  39\\
    1,48	&  33\\
    1,46	&  29,5\\
    \bottomrule
  \end{tabular}
\end{table}

\begin{figure}
  \centering
  \includegraphics{build/gegenspannung.pdf}
  \caption{Auftragung der an der Monozelle gemessenen Klemmenspannung $U_{\symup{K}}$
  gegen die Stromstärke $I$ bei angelegter Gegenspannung $U_{\symup{gegen}}$ }
  \label{fig:gegenspannung}
\end{figure}

Da die Steigungen der Fits annähernd gleich sind, wird als Innenwiderstand $R_{\symup{i}}$
das Mittel dieser beiden Werte gebildet. Für die Leerlaufspannung wird ebenso verfahren.
Damit ergeben sich die Werte zu
\begin{align}
  R_{\symup{i,gegen}}&= \,,\\
  U_{\symup{0,gegen}}&= \,.
\end{align}

Für die Sinus- und die Rechteckspannung wird analog zur Monozelle verfahren. Dabei
wird als Ausgangsspannung eine Spannung von $U_{\symup{0}}\approx1$\,V verwendet.
Daraus ergeben sich die in Abbildung \ref{fig:rechteck} und \ref{fig:sinus} dargestellten
Zusammenhänge. Die zugrundeliegenden Daten können Tabelle \ref{tab:re_sin} entnommen
werden.

\begin{table}
  \centering
  \caption{Messdaten zur Klemmenspannnung $U_{\symup{K}}$ in Abhängigkeit von der
  Stromstärke $I$ bei einer Rechteck- und bei einer Sinusspnnung.}
  \label{tab:re_sin}
  \begin{tabular}{c c c c}
    %\toprule
    \hline
    \multicolumn{2}{c}{Rechteckspannung} & \multicolumn{2}{c}{Sinusspannung} \\
    \cmidrule(lr){1-2}
    \cmidrule(lr){3-4}
    $U_{\symup{K}}/$V & $I/$mA & $U_{\symup{K}}/$V & $I/$mA \\
    \midrule
    0,849 &	1,9   & 2,130 	& 0,903 \\
    0,846	& 1,95  & 2,106	  & 0,903 \\
    0,834	& 2,17  & 2,076 	& 0,906 \\
    0,816	& 2,34  & 2,040 	& 0,921 \\
    0,801	& 2,75  & 1,998	  & 0,942 \\
    0,780	& 3,15  & 1,947	  & 0,990 \\
    0,750	& 3,68  & 1,848	  & 1,089 \\
    0,696	& 4,49  & 1,776	  & 1,173 \\
    0,633	& 5,50  & 1,635	  & 1,359 \\
    0,516	& 7,32  & 1,500	  & 1,536 \\
    0,330	& 9,20  & 1,296	  & 1,830 \\
    -     & -     & 0,837	  & 2,493 \\
    -     & -     & 0,756	  & 2,604 \\
    \bottomrule
  \end{tabular}
\end{table}

\begin{figure}
  \centering
  \includegraphics{build/rechteck.pdf}
  \caption{Auftragung der Klemmenspannung $U_{\symup{K}}$ gegen die Stromstärke $I$
  bei einer rechteckförmigen Spannung und linearer Fit}
  \label{fig:rechteck}
\end{figure}

\begin{figure}
  \centering
  \includegraphics{build/sinus.pdf}
  \caption{Auftragung der Klemmenspannung $U_{\symup{K}}$ gegen die Stromstärke $I$
  bei einer sinusförmigen Spannung und linearer Fit}
  \label{fig:sinus}
\end{figure}

Die Ausgleichsrechnung ergibt dabei bei der Rechteckspannug für den Innenwiderstand
und die Leerlaufspannung die Werte
\begin{align}
  R_{\symup{i,rechteck}}&=-a_{\symup{i,rechteck}}=(67,73\pm2,33)\,\symup{\Omega} \,, \\
  U_{\symup{0,rechteck}}&=b_{\symup{0,rechteck}}=(0,987\pm0,011)\,\text{V}\,.
\end{align}
Für die Sinusspannung ergibt sich
\begin{align}
  R_{\symup{i,sinus}}&=-a_{\symup{i,sinus}}=(775,49\pm24,27)\,\symup{\Omega} \,, \\
  U_{\symup{0,sinus}}&=b_{\symup{0,sinus}}=(2,741\pm0,036)\,\text{V}\,.
\end{align}


Nun soll noch die bei der Monozelle im Belastungswiderstand $R_{\symup{a}}$ umgesetzte
Leistung $N$ in Abhängigkeit vom Belastungswiderstand untersucht werden.
Der Belastunsgwiderstand ist dabei
\begin{equation}
  R_{\symup{a}}=\frac{U_{\symup{K}}}{I}\,.
\end{equation}
Die Leistung ergibt sich zu
\begin{equation}
  N=U_{\symup{K}}I\,.
\end{equation}
Die aus den Messdaten errechneten Daten sind in Tabelle (REFERENZ) zu finden und
werden in Abbildung (REFERENZ) dargestellt. Zusätzlich ist in (REFERENZ) eine
Theoriekurve eingezeichnet, die dem Zusammenhang
\begin{equation}
  N=I^2 R_{\symup{a}} = \frac{U_{\symup{0}}^2 R_{\symup{a}}}{(R_{\symup{a}}+R_{\symup{i}})^2}
\end{equation}
folgt.

\begin{table}
  \centering
  \caption{Messdaten zur an der Monozelle gemessennen Klemmenspannung $U_{\symup{K}}$
  in Abhängigkeit von der durch den Widerstand geregelten Stromstärke $I$}
  \label{tab:monozelle}
  \begin{tabular}{c c c c}
    \toprule
    $U_{\symup{K}}/$V & $I/$mA & $N$/mW & $R_{\symup{a}}/\symup{\Omega}$\\
    \midrule
    0,25	&  170 &   42,50  &  1.5\\
    0,73	&  80  &   58,40  &  9.1\\
    0,87	&  79  &   68,73  &  11.0\\
    0,975	& 59,5 &   58.01  &  16.4\\
    1,065	& 46,0 &   48,90  &  23.2\\
    1,095	& 38,0 &   41,61  &  28.8\\
    1,131	& 33,0 &   37,32  &  34.3\\
    1,170	& 26,0 &   30,42  &  45.0\\
    1,170	& 25,0 &   29,25  &  46.8\\
    1,185	& 23,0 &   27,25  &  51.5\\
    1,193	& 21,0 &   25,05  &  56.8\\
    \bottomrule
  \end{tabular}
\end{table}

\begin{figure}
  \centering
  \includegraphics{build/leistung.pdf}
  \caption{Auftragung der Leistung $N$ am Belastungswiderstand gegen den Belastungswiderstand
  $R_{\symup{a}}$}
  \label{fig:leistung}
\end{figure}
