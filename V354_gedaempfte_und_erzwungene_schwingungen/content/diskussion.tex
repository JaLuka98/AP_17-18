\section{Diskussion}
\label{sec:Diskussion}
Die gemessenen Werte erscheinen im Allgemeinen sinnvoll, da sie insgesamt
sehr nah an den theoretisch berechneten Werten liegen.

Dennoch muss angemerkt werden, dass für die Messung des effektiven Widerstandes $R_{\symup{eff}}$
nur wenige Messwerte berücksichtigt wurden und diese nur aus einer manuell eingezeichneten
Einhüllenden abgelesen wurden.

Auffällig ist die große Abweichung des experimentell bestimmten Wertes für den
Widerstand $R_{\symup{ap}}$, bei dem der aperiodische Grenzfall eintritt, vom theoretisch
berechneten Wert. Mögliche Ursachen dafür sind die in der Messung nicht berücksichtigten
Innenwiderstände der anderen Geräte, sowie Ungenauigkeiten beim Ablesen vom Oszilloskop.
Es war nicht genau erkennbar, wann genau kein Überschwingen mehr stattfand. Außerdem
wurde die Messung nur ein einziges Mal durchgeführt.

Eine Weitere Quelle für systematische Fehler bei der Messung liegt in der Ungenauigkeit
des Oszilloskops. Das verwendete Gerät kann auf der gewählten Einstellung, bei der
das Zehnfache der angelegten Spannung angezeigt wird, die Spannung nur bis auf 0.2V genau
anzeigen. Dies führt zu Ungenauigkeiten beim Ablesen der Werte. Insbesondere
bei der Messung der Phasenverschiebung führen bereits geringe Änderungen der abgelesenen
Werte zu sehr großen Abweichungen des Ergebnisses. Dies ist eine mögliche Erklärung
für die Werte der Phasenverschiebung, die oberhalb von $\pi$ liegen. Außerdem zeigt
der Schwingkreis nicht zu vernachlässigende Phasenverschiebungen für kleine Frequenzen,
die auf systematische Fehler beim Ablesen und Eigenheiten der Schaltung zurückzuführen sind.
