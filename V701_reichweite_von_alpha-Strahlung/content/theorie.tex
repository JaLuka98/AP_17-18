\section{Theorie}
\label{sec:Theorie}
Eine experimentelle Bestimmung der Energie von Alphastrahlung ist durch Messung
ihrer Reichweite möglich. Alphateilchen verlieren in einem Medium ihre Energie
hauptsächlich durch Ionisationsprozesse sowie durch Anregung oder Dissoziation
von Molekülen. Elastische Stöße spielen nur eine untergeordnete Rolle. Der
Energieverlust wird mit $-\symup{d}E_\alpha / \symup{d}x$ bezeichnet und lässt sich
für hinreichend große Energien durch die Bethe-Bloch-Gleichung
\begin{equation}
  - \frac{\symup{d}E_\alpha}{\symup{d}x} = \frac{z^2 e^4}{4\pi \epsilon_0 m_\text{e}} \frac{n Z}{v^2} \ln{\left(\frac{2m_\text{e}v^2}{I}\right)}
  \label{eqn:bethebloch}
\end{equation}
beschreiben. Die Ladung der Alphateilchen wird durch $z$, ihre Geschwindigkeit mit $v$
bezeichnet. Die Ordnungszahl sei $Z$, $n$ ist die Teilchendichte und $I$ die Ionisierungsenergie
des Targetgases. Die Bethe-Bloch-Gleichung verliert für sehr kleine Energien ihre Gültigkeit,
weil dann Ladungsaustauschprozesse auftreten. \\
Die Reichweite eines Alphateilchens bezeichnet seine Wegstrecke bis zur vollständigen Abbremsung
und beträgt
\begin{equation}
  R = \int_0^{E_0} \frac{\symup{d}E_{\alpha}}{-\symup{d}E_{\alpha} / \symup{d}x}\,.
\end{equation}
Ein Alphateilchen verbringt bis zur vollständigen Abbremsung zwingend eine Zeit mit
niedriger Energie, in diesem Bereich versagt die Bethe-Bloch-Gleichung wie oben beschrieben.
Deswegen werden zur Bestimmung der mittleren Reichweite $R_m$ empirische Formeln
verwendet. So gilt zum Beispiel näherungsweise für die Reichweite von Alphastrahlung mit einer
Energie kleiner oder gleich $2{,}5$ Megaelektronenvolt in Luft
\begin{equation}
  R_m = 3,1 \cdot E_\alpha^{3/2}\,,
  \label{eqn:empirisch}
\end{equation}
wobei $R_m$ in Millimetern und $E_\alpha$ in Megaelektronenvolt anzugeben ist.\\
Für die Reichweite von Alphateilchen in Gasen gilt, dass sie bei konstanter Temperatur
und konstantem Volumen proportional zum Druck $p$ ist. Es gilt für einen festen Abstand $x_0$
zwischen Detektor und Quelle der Alphastrahlung
\begin{equation}
  x = x_0 \frac{p}{p_0}\,,
  \label{eqn:eff_weg}
\end{equation}
wobei $x$ die effektive Weglänge bezeichnet und $p_0 = 1013$\,mbar der Normaldruck ist.
