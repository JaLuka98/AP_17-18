\section{Diskussion}
\label{sec:Diskussion}

Die Messergebnisse sind insgesamt als ungenau zu bewerten. Die bestimmten Werte
für die mittleren Reichweiten und die mittleren Energien der Alpha-Strahlung
weichen mit
\begin{align*}
  R_{\symup{m,1}}&=\SI{19.186(0004)}{\milli\meter} \,, \\
  R_{\symup{m,2}}&=\SI{33.8(18)}{\milli\meter} \,, \\
  E_1&=\SI{3.3709(00005)}{\MeV} \,,\\
  E_2&=\SI{4.91(018)}{\MeV}
\end{align*}
sehr stark voneinander ab. Theoretisch wären hier für beide Abstände, bei denen gemessen
wird, die gleichen Ergebnisse zu erwarten, da sie charakteristisch für die verwendete
Alpha-Strahlung sind. Werden die Werte für die Energien mit den in Tabelle
\ref{tab:energie} verglichen, so zeigt sich, dass der Wert für $E_2$ deutlich zu
hoch ausfällt. Einen wahrscheinlichen Grund für diese Abweichungen stellen die in
Kapitel \ref{subsec:reichweite} durchgeführten linearen Ausgleichsrechnungen dar,
bei denen jeweils nur vier oder fünf Messwerte berücksichtigt wurden. Für die erste
Messreihe liegen die Messwerte dabei sehr gut auf der Ausgleichsgeraden, bei der
zweiten sind jedoch auch hier schon Abweichungen feststellbar. Da die Parameter
der linearen Ausgleichsrechnung die Grundelage für die berechnten mittleren Reichweiten
$R_{\symup{m,i}}$ und damit auch der mittleren Energien $E_{\symup{i}}$ sind, kann
es so schnell zu starken Abweichungen kommen.

Es ist anzumerken, dass für die linearen Ausgleichrechnungen bei der Bestimmung des
Energieverlustes nicht so viele Werte berücksichtigt werden konnten, wie vorgesehen,
da bei der ersten Messreihe des Drucks die Diskriminatorschwelle so eingestellt war, dass
für Drücke oberhalb von 850\,mbar keine Messwerte für den Channel, in dem die
meisten Counts auftreten, mehr aufgenmmen werden konnten. Für die zweite Messreihe
werden bei der Rechnung zwei Messwerte ausgenommen, die nicht auf der Geraden liegen.

Es zeigt sich außerdem, dass die berechneten Energieverluste mit
\begin{align*}
  -\frac{\symup{d}\,E_1}{\symup{d}\,x}&=\SI{0.959(028)}{\MeV\per\centi\meter} \,\\
  -\frac{\symup{d}\,E_2}{\symup{d}\,x}&=\SI{1.150(017)}{\MeV\per\centi\meter}
\end{align*}
nicht konsistent mit den oben aufgeführten Werten für die mittleren Reichweiten und
die zugehörigen Energien sind. Die oben angeführten Messergebnisse führen auf einen
Energieverlust von etwa $-\frac{\symup{d}\,E}{\symup{d}\,x}=1{,}5\,$cm.

Die Größenordnungen der Messergebnisse stimmen jedoch überein, sodass die theoretisch
zu erwartenden Zusammenhänge zumindest qualitativ gezeigt werden konnten.
