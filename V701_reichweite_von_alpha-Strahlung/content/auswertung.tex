\section{Auswertung}
\label{sec:Auswertung}

Die aufgenommenen Werte für die Zählraten und die Energie der \alpha-Strahlung, sowie
daraus berechnete werte befinden sich in den Tabellen \ref{tab:druck1} und \ref{tab:druck2}.
Gemessen wurde für zwei verschiedene Abstände $x_{0,\symup{i}}$ zwischen Strahler und
Detektor. Diese betrugen $x_{0,1}=2{,}2\,$cm und $x_{0,2}=1{,}8\,$cm
Aus den Messwerten werden gemäß Gleichung \eqref{eqn:eff_weg} die effektiven
Weglängen der \alpha-Strahlung berechnet. Außerdem werden aus den gesamten Anzahlen
der gezählten Ereignisse die Zählraten $Z=N_{\symup{ges}}/t$ berechnet.
\begin{table}[htp]
	\begin{center}
    \caption{Messwerte und daraus berechnete Werte für die Reichweite der Alphastrahlung
    in Abhängigkeit vom Druck mit einem Abstand von $x_{0,1}=2{,}2\,$cm.}
    \label{tab:druck1}
		\begin{tabular}{ccccc}
		\toprule
			{p/mbar} & {channel} & {$N_{ges}$} & {$x$/cm} & {Z}\\
			\midrule
			0 & 966 & 85483 & 0,00 & 712\\
			50 & 952 & 84772 & 0,11 & 706\\
			100 & 929 & 84944 & 0,22 & 708\\
			150 & 938 & 88255 & 0,33 & 735\\
			200 & 916 & 87268 & 0,43 & 727\\
			250 & 853 & 82128 & 0,54 & 684\\
			300 & 837 & 84315 & 0,65 & 703\\
			350 & 826 & 82626 & 0,76 & 689\\
			400 & 804 & 82835 & 0,87 & 690\\
			450 & 777 & 82689 & 0,98 & 689\\
			500 & 758 & 79876 & 1,09 & 666\\
			550 & 721 & 78694 & 1,19 & 656\\
			600 & 697 & 76527 & 1,30 & 638\\
			650 & 670 & 73763 & 1,41 & 615\\
			700 & 648 & 70818 & 1,52 & 590\\
			750 & 606 & 65117 & 1,63 & 543\\
			800 & 582 & 59842 & 1,74 & 499\\
			850 & 536 & 49709 & 1,85 & 414\\
			900 & 530 & 39311 & 1,95 & 328\\
			950 & 527 & 29029 & 2,06 & 242\\
			1000 & 524 & 22278 & 2,17 & 186\\
		\bottomrule
		\end{tabular}
	\end{center}
\end{table}

\begin{table}[htp]
	\begin{center}
    \caption{Messwerte und daraus berechnete Werte für die Reichweite der Alphastrahlung
    in Abhängigkeit vom Druck mit einem Abstand von $x_{0,2}=1{,}8\,$cm.}
    \label{tab:druck2}
		\begin{tabular}{ccccc}
		\toprule
			{p/mbar} & {channel} & {$N_{ges}$} & {$x$/cm} & {Z}\\
			\midrule
			0 & 1007 & 119924 & 0,00 & 999\\
			50 & 1083 & 117983 & 0,09 & 983\\
			100 & 1032 & 116688 & 0,18 & 972\\
			150 & 1014 & 116310 & 0,27 & 969\\
			200 & 991 & 115745 & 0,36 & 965\\
			250 & 963 & 114693 & 0,44 & 956\\
			300 & 942 & 113976 & 0,53 & 950\\
			350 & 925 & 113554 & 0,62 & 946\\
			400 & 882 & 113096 & 0,71 & 942\\
			450 & 856 & 112362 & 0,80 & 936\\
			500 & 852 & 109869 & 0,89 & 916\\
			550 & 820 & 109569 & 0,98 & 913\\
			600 & 802 & 108607 & 1,07 & 905\\
			650 & 762 & 110633 & 1,15 & 922\\
			700 & 742 & 107034 & 1,24 & 892\\
			750 & 721 & 105293 & 1,33 & 877\\
			800 & 690 & 104224 & 1,42 & 869\\
			850 & 658 & 100940 & 1,51 & 841\\
			900 & 624 & 99344 & 1,60 & 828\\
			950 & 594 & 96054 & 1,69 & 800\\
			1000 & 614 & 95223 & 1,78 & 794\\
		\bottomrule
		\end{tabular}
	\end{center}
\end{table}

Nun wird für beide Messreihen die Zählrate in Abhängigkeit von der effektiven Länge
aufgetragen. Es ergibt sich der in den Abbildungen \ref{fig:druck1} und \ref{fig:druck2}
dargestellte Verlauf.

\begin{figure}[H]
  \centering
  \includegraphics{build/druck1.pdf}
  \caption{Auftragung der Zählraten gegen die effektive Wegläge und lineare Ausgleichsrechnung
  zur Bestimmung der Reichweite für $x_{0,1}=2{,}2\,$cm.}
  \label{fig:druck1}
\end{figure}

\begin{figure}[H]
  \centering
  \includegraphics[width=\textwidth]{build/druck2.pdf}
  \caption{Auftragung der Zählraten gegen die effektive Wegläge und lineare Ausgleichsrechnung
  zur Bestimmung der Reichweite für $x_{0,2}=1{,}8\,$cm.}
  \label{fig:druck2}
\end{figure}

Dabei wird für die jeweils annähernd linear verlaufenden Teile eine lineare Ausgleichsrechnung
der Form
\begin{equation}
  f(x)=mx+b
\end{equation}
durchgeführt. Die Parameter der Ausgleichsrechnung ergeben sich zu
\begin{align*}
    \,.
\end{align*}
Daraus lässt sich durch
\begin{equation}

\end{equation}

die maximale Reichweite der \alpha-Strahlung bestimmen.

\subsection{Untersuchung der Statistik des radioaktiven Zerfalls}
\label{subsec:statistik}
Zuletzt soll die Statistik des radioaktiven Zerfalls, der theoretisch einer Poisson-Verteilung folgen sollte,
untersucht werden. Die Messwerte sind dabei dem Anhang zu entnehmen, sie werden aufgrund ihrer
hohen Anzahl hier nicht wiederholt. \\
Der Mittelwert der Stichprobe ergibt sich zu
\begin{equation*}
  \overline{N} = \SI{399.9(14)}{\per\second}\,.
\end{equation*}
Die Varianz der Stichprobe beträgt $205,27$.\\

Um den Zusammenhang zu veranschaulichen, wird ein Histogram der Daten
mit 20 Bins angefertigt. Darüberhinaus werden gaußverteilte Zufallszahlen mit den obigen
Werten für Mittelwert und Varianz erzeugt und ebenso in das Diagramm eingefügt.
Dieses ist in \ref{fig:hist} zu sehen.

\begin{figure}
  \centering
  \includegraphics{build/hist.pdf}
  \caption{Histogramm der gemessenen Zählraten und gaußverteilter Zufallszahlen.}
  \label{fig:hist}
\end{figure}
