\section{Auswertung}
\label{sec:Auswertung}

\subsection{Bestimmung der Reichweite von Alphastrahlung}
\label{subsec:reichweite}

Die aufgenommenen Werte für die Zählraten und die Energie der Alphastrahlung, sowie
daraus berechnete werte befinden sich in den Tabellen \ref{tab:druck1} und \ref{tab:druck2}.
Gemessen wurde für zwei verschiedene Abstände $x_{0,\symup{i}}$ zwischen Strahler und
Detektor. Diese betrugen $x_{0,1}=2{,}2\,$cm und $x_{0,2}=1{,}8\,$cm.
Aus den Messwerten werden gemäß Gleichung \eqref{eqn:eff_weg} die effektiven
Weglängen der Alphastrahlung berechnet. Außerdem werden aus den gesamten Anzahlen
der gezählten Ereignisse die Zählraten $Z=N_{\symup{ges}}/t$ berechnet.
\begin{table}[htp]
	\begin{center}
    \caption{Messwerte und daraus berechnete Werte für die Reichweite der Alphastrahlung
    in Abhängigkeit vom Druck mit einem Abstand von $x_{0,1}=2{,}2\,$cm.}
    \label{tab:druck1}
		\begin{tabular}{ccccc}
		\toprule
			{$p/$mbar} & {channel} & {$N_{ges}$} & {$x$/cm} & {$Z/\frac{1}{\text{s}}$}\\
			\midrule
			0 & 966 & 85483 & 0,00 & 712\\
			50 & 952 & 84772 & 0,11 & 706\\
			100 & 929 & 84944 & 0,22 & 708\\
			150 & 938 & 88255 & 0,33 & 735\\
			200 & 916 & 87268 & 0,43 & 727\\
			250 & 853 & 82128 & 0,54 & 684\\
			300 & 837 & 84315 & 0,65 & 703\\
			350 & 826 & 82626 & 0,76 & 689\\
			400 & 804 & 82835 & 0,87 & 690\\
			450 & 777 & 82689 & 0,98 & 689\\
			500 & 758 & 79876 & 1,09 & 666\\
			550 & 721 & 78694 & 1,19 & 656\\
			600 & 697 & 76527 & 1,30 & 638\\
			650 & 670 & 73763 & 1,41 & 615\\
			700 & 648 & 70818 & 1,52 & 590\\
			750 & 606 & 65117 & 1,63 & 543\\
			800 & 582 & 59842 & 1,74 & 499\\
			850 & 536 & 49709 & 1,85 & 414\\
			900 & 530 & 39311 & 1,95 & 328\\
			950 & 527 & 29029 & 2,06 & 242\\
			1000 & 524 & 22278 & 2,17 & 186\\
		\bottomrule
		\end{tabular}
	\end{center}
\end{table}

\begin{table}[htp]
	\begin{center}
    \caption{Messwerte und daraus berechnete Werte für die Reichweite der Alphastrahlung
    in Abhängigkeit vom Druck mit einem Abstand von $x_{0,2}=1{,}8\,$cm.}
    \label{tab:druck2}
		\begin{tabular}{ccccc}
		\toprule
			{$p$/mbar} & {channel} & {$N_{ges}$} & {$x$/cm} & {$Z/\frac{1}{\text{s}}$}\\
			\midrule
			0 & 1007 & 119924 & 0,00 & 999\\
			50 & 1083 & 117983 & 0,09 & 983\\
			100 & 1032 & 116688 & 0,18 & 972\\
			150 & 1014 & 116310 & 0,27 & 969\\
			200 & 991 & 115745 & 0,36 & 965\\
			250 & 963 & 114693 & 0,44 & 956\\
			300 & 942 & 113976 & 0,53 & 950\\
			350 & 925 & 113554 & 0,62 & 946\\
			400 & 882 & 113096 & 0,71 & 942\\
			450 & 856 & 112362 & 0,80 & 936\\
			500 & 852 & 109869 & 0,89 & 916\\
			550 & 820 & 109569 & 0,98 & 913\\
			600 & 802 & 108607 & 1,07 & 905\\
			650 & 762 & 110633 & 1,15 & 922\\
			700 & 742 & 107034 & 1,24 & 892\\
			750 & 721 & 105293 & 1,33 & 877\\
			800 & 690 & 104224 & 1,42 & 869\\
			850 & 658 & 100940 & 1,51 & 841\\
			900 & 624 & 99344 & 1,60 & 828\\
			950 & 594 & 96054 & 1,69 & 800\\
			1000 & 614 & 95223 & 1,78 & 794\\
		\bottomrule
		\end{tabular}
	\end{center}
\end{table}

Nun wird für beide Messreihen die Zählrate in Abhängigkeit von der effektiven Länge
aufgetragen. Es ergeben sich die in den Abbildungen \ref{fig:druck1} und \ref{fig:druck2}
dargestellten Verläufe.

\begin{figure}[H]
  \centering
  \includegraphics{build/druck1.pdf}
  \caption{Auftragung der Zählraten gegen die effektive Wegläge und lineare Ausgleichsrechnung
  zur Bestimmung der Reichweite für $x_{0,1}=2{,}2\,$cm.}
  \label{fig:druck1}
\end{figure}

\begin{figure}[H]
  \centering
  \includegraphics[width=\textwidth]{build/druck2.pdf}
  \caption{Auftragung der Zählraten gegen die effektive Wegläge und lineare Ausgleichsrechnung
  zur Bestimmung der Reichweite für $x_{0,2}=1{,}8\,$cm.}
  \label{fig:druck2}
\end{figure}

Dabei wird für die jeweils annähernd linear verlaufenden Bereiche eine lineare Ausgleichsrechnung
der Form
\begin{equation}
  f(x)=ax+b
\end{equation}
durchgeführt. Die Parameter der Ausgleichsrechnung ergeben sich zu
\begin{align*}
  a_1 &= \SI{-7.892(0027)e4}{\per\second\per\meter}\,, \\
  b_1 &= \SI{1870(5)}{\per\second}\,,  \\
  a_2 &= \SI{-1.85(017)e4}{\per\second\per\meter} \,,  \\
  b_2 &= \SI{1124(25)}{\per\second} \,.
\end{align*}
Daraus lässt sich durch
\begin{equation}
  R_{\symup{m}}=\frac{\frac{1}{2}Z_{\symup{max,i}}-b_{\symup{i}}}{a_{\symup{i}}}
\end{equation}
die mittlere Reichweite der Alphastrahlung bestimmen. Diese ergibt sich zu
\begin{align*}
  R_{\symup{m,1}}&=\SI{19.186(0004)}{\milli\meter} \,, \\
  R_{\symup{m,2}}&=\SI{33.8(18)}{\milli\meter} \,.
\end{align*}
Der Fehler der bestimmten mittleren Reichweiten folgt nach dem Gauß'schen Fehlerfortpflanzungsgesetz
\eqref{eqn:gaussfehler} dem Zusammenhang
\begin{equation}
  \sigma_{\symup{R_m}} = \sqrt{\sum\limits_{i = 1}^N\left(\frac{\partial R_{\symup{m}}}{\partial x_{\symup{i}}}
  \sigma_{\symup{i}} \right)^{\!\! 2}} = \sqrt{\sigma_{\symup{a}}^{2} \left(\frac{b}{a^{2}} -
  \frac{\frac{1}{2}Z_{\symup{max}}}{a^{2}}\right)^{2} + \frac{\sigma_{\symup{b}}^{2}}{a^{2}} } \,.
\end{equation}

Die zugehörigen Energien können mithilfe von Gleichung \eqref{eqn:empirisch} zu
\begin{align*}
  E_1&=\SI{3.3709(00005)}{\MeV} \,,\\
  E_2&=\SI{4.91(018)}{\MeV}
\end{align*}
bestimmt werden. Der zugehörige Fehler ergibt sich aus der Gauß'schen Fehlerfortpflanzung
\eqref{eqn:gaussfehler} zu
\begin{equation}
  \sigma_{E} = \sqrt{\sum\limits_{i = 1}^N\left( \frac{\partial E}{\partial x_{\symup{i}}}
   \sigma_{\symup{i}} \right)^{\!\! 2}} =
   \sqrt{\frac{\frac{2}{3}\left(\frac{10}{31}\right)^{2/3}\sigma_{\symup{a}}^{2}}{R_{\symup{m}}^{2/3}}}\,.
\end{equation}

\subsection{Bestimmung des Energieverlustes der Alphastrahlung}
\label{subsec:energie}

Zur Bestimmung des Energieverlustes der Alphastrahlung muss zunächst die Energie
der Strahlung bestimmt werden. Laut Versuchsanleitung \cite{Versuchsanleitung}
beträgt die Energie, die beim gemessenen Maximum bei einem Druck von $p=0\,$mbar
gemessen wird 4\,MeV. Damit ergibt sich für die Energie der Strahlung für den jeweiligen
Druck, bei dem gemessen wird zu
\begin{equation}
  E(p)=\frac{\text{channel}(p)\cdot4\,\text{MeV}}{\text{channel}(0)} \,.
\end{equation}
Die Berechneten Werte sind in Tabelle \ref{tab:energie} dargestellt.

\begin{table}[htp]
	\begin{center}
    \caption{Berechnete Werte für die Energien
    in Abhängigkeit vom Druck bei den beiden verschiedenen Abständen.}
    \label{tab:energie}
		\begin{tabular}{ccc}
		\toprule
			{$p$/mbar} & {$E_1$/MeV} & {$E_2$/MeV}\\
			\midrule
			0 & 4,00 & 4,00\\
			50 & 3,94 & 4,30\\
			100 & 3,85 & 4,10\\
			150 & 3,88 & 4,03\\
			200 & 3,79 & 3,94\\
			250 & 3,53 & 3,83\\
			300 & 3,47 & 3,74\\
			350 & 3,42 & 3,67\\
			400 & 3,33 & 3,50\\
			450 & 3,22 & 3,40\\
			500 & 3,14 & 3,38\\
			550 & 2,99 & 3,26\\
			600 & 2,89 & 3,19\\
			650 & 2,77 & 3,03\\
			700 & 2,68 & 2,95\\
			750 & 2,51 & 2,86\\
			800 & 2,41 & 2,74\\
			850 & 2,22 & 2,61\\
			900 & 2,19 & 2,48\\
			950 & 2,18 & 2,36\\
			1000 & 2,17 & 2,44\\
		\bottomrule
		\end{tabular}
	\end{center}
\end{table}

Nun wird die berechnete Energie für beide Messreihen gegen die effektive Weglänge aufgetragen
und es wird eine lineare Ausgleichsrechnung durchgeführt. Dies ist in den Abbildungen
\ref{fig:energie1} und \ref{fig:energie2} dargestellt. Dabei ist anzumerken, dass bei
den letzten drei Messwerten bei der ersten Messreihe keine zuverlässigen Werte für den
Channel mehr aufgenommen werden konnten, da die Diskriminatorschwelle das Maximum
unterdrückte. Daher werden diese Messwerte in diesem Teil der Auswertung nicht berücksichtigt.\\
Zudem ist auffällig, dass bei der zweiten Messreihe der erste und der letzte Wert stark von
der Geraden abweichen. Daher werden diese Werte für die lineare Ausgleichsrechnung nicht berücksichtigt.

\begin{figure}[H]
  \centering
  \includegraphics[width=\textwidth]{build/energie1.pdf}
  \caption{Auftragung der Energien gegen die effektive Wegläge und lineare Ausgleichsrechnung
  zur Bestimmung des Energieverlustes für $x_{0,1}=2{,}2\,$cm.}
  \label{fig:energie1}
\end{figure}

\begin{figure}[H]
  \centering
  \includegraphics[width=\textwidth]{build/energie2.pdf}
  \caption{Auftragung der Energien gegen die effektive Wegläge und lineare Ausgleichsrechnung
  zur Bestimmung des Energieverlustes für $x_{0,2}=1{,}8\,$cm.}
  \label{fig:energie2}
\end{figure}

Die Parameter der Ausgleichsrechnungen ergeben sich zu
\begin{align*}
  a_1 &= \SI{-95.9(28)}{\MeV\per\meter} \,,\\
  b_2 &= \SI{4.109(0030)}{\MeV} \,,\\
  a_2 &= \SI{-115.0(17)}{\MeV\per\meter} \,,\\
  b_2 &= \SI{4.357(0018)}{\MeV} \,.\\
\end{align*}
Der Energieverlust der Strahlung entspricht dem Betrag der Steigung der Geraden und ergibt sich
somit zu $-\frac{\symup{d}\,E_1}{\symup{d}\,x}=\SI{0.959(028)}{\MeV\per\centi\meter}$ und
$-\frac{\symup{d}\,E_2}{\symup{d}\,x}=\SI{1.150(017)}{\MeV\per\centi\meter}$.

%Der Fehler für die berechneten Energieverluste ergibt sich dabei gemäß der Gauß'schen
%Fehlerforpflanzung \eqref{eqn:gaussfehler} zu
%\begin{equation}
%  Fehlerformel...
%\end{equation}


\subsection{Untersuchung der Statistik des radioaktiven Zerfalls}
\label{subsec:statistik}
Zuletzt soll die Statistik des radioaktiven Zerfalls, der theoretisch einer Poisson-Verteilung folgen sollte,
untersucht werden. Die Messwerte sind dabei dem Anhang zu entnehmen, sie werden aufgrund ihrer
hohen Anzahl hier nicht wiederholt. \\
Der Mittelwert der Stichprobe ergibt sich näherungsweise zu
\begin{equation*}
  \overline{Z} = \SI{399.9(14)}{\per\second}\,.
\end{equation*}
Die Varianz der Stichprobe beträgt ungefähr $\SI{205.27}{\per\second\squared}$.
Für eine Poisson-Verteilung gilt der Zusammenhang
\begin{equation*}
	\symup{E}(X) = \symup{Var}(X)\,.
\end{equation*}
Das bedeutet, dass der Erwartungswert gleich der Varianz der Zufallsvariablen $X$
ist. Dies ist hier eindeutig nicht erfüllt, stattdessen beträgt die Varianz nur etwa halb so groß wie der Mittelwert.\\
Für eine Gauß-Verteilung gilt die Gleichung
\begin{equation*}
	\symup{Var}(X) = \sigma_X^2\,,
\end{equation*}
die Varianz ist also das Quadrat der Standardabweichung. Die empirische Standardabweichung
der Stichprobe beträgt hier circa $\SI{14.33}{\per\second}$. Dies wird durch Mutliplizieren
des Fehlers des Mittelwertes, welcher sich durch Gleichung \eqref{eqn:std} ergibt, mit $\sqrt{N}$ berechnet.
Das Quadrat der empirischen Standardabweichung ist ungefähr $\SI{205.35}{\per\second}$, was
fast genau gleich der empirischen Varianz ist. Insofern entspricht die gemessene Verteilung
eher einer Gauß- als einer Poisson-Verteilung.\\
Um den Zusammenhang zu veranschaulichen, wird ein normiertes Histogram der Daten
mit 20 Bins angefertigt. Darüberhinaus werden gaußverteilte Zufallszahlen mit den obigen
Werten für Mittelwert und Varianz erzeugt und ebenso in das Diagramm eingefügt.
Dieses ist in \ref{fig:hist} zu sehen.

\begin{figure}
  \centering
  \includegraphics{build/hist.pdf}
  \caption{Histogramm der gemessenen Zählraten und gaußverteilter Zufallszahlen.}
  \label{fig:hist}
\end{figure}
