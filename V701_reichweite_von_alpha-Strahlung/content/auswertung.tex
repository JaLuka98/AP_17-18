\section{Auswertung}
\label{sec:Auswertung}

\subsection{Untersuchung der Statistik des radioaktiven Zerfalls}
\label{subsec:statistik}
Zuletzt soll die Statistik des radioaktiven Zerfalls, der theoretisch einer Poisson-Verteilung folgen sollte,
untersucht werden. Die Messwerte sind dabei dem Anhang zu entnehmen, sie werden aufgrund ihrer
hohen Anzahl hier nicht wiederholt. \\
Der Mittelwert der Stichprobe ergibt sich zu
\begin{equation*}
  \overline{N} = \SI{399.9(14)}{\per\second}\,.
\end{equation*}
Die Varianz der Stichprobe beträgt $205,27$.\\

Um den Zusammenhang zu veranschaulichen, wird ein Histogram der Daten
mit 20 Bins angefertigt. Darüberhinaus werden gaußverteilte Zufallszahlen mit den obigen
Werten für Mittelwert und Varianz erzeugt und ebenso in das Diagramm eingefügt.
Dieses ist in \ref{fig:hist} zu sehen.

\begin{figure}
  \centering
  \includegraphics{build/hist.pdf}
  \caption{Histogramm der gemessenen Zählraten und gaußverteilter Zufallszahlen.}
  \label{fig:hist}
\end{figure}
