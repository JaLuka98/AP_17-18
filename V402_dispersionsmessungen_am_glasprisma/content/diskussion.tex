\section{Diskussion}
\label{sec:Diskussion}
Die Messung des Brechenden Winkels $\varphi$ wird als genau bewertet. Dieser wurde
zu ca. 60° bestimmt. Da die Flächen des Prismas gleichseitigen Dreiecken
ähneln, war dieser Wert auch zu erwarten.\\
Die Anpassung der erhaltenen Brechungsindizes durch die Dispersionskurve ist erfolgreich.
Durch die Methode der Abweichungsquadrate wird sich klar für die Kurve nach
Gleichung \eqref{eqn:fall1} entschieden. Wie an der Abbildung \ref{fig:kurve} zu
erkennen ist, passt die Ausgleichsfunktion die Werte gut an und die Abweichungen
sind statistischer, unter Umständen auch geringfügig systematischer Natur.\\
Die Abbesche Zahl ist mit einer Abweichung von ca. 6\% ebenfalls als genau zu bewerten.\\
Über die Genauigkeit des Auflösungsvermögens kann kein Urteil gefällt werden, da keine
Vergleichswerte bekannt sind. Die geringen relativen Unsicherheiten von unter einem Prozent
sprechen jedoch für eine tendenziell genaue Messung.\\
Die Absorptionsstelle $\lambda_1$ ist plausibel, da der experimentell bestimmte Wert von ca. 150\,nm im
Ultraviolettbereich liegt, wo er theoretisch zu erwarten ist.\\
Insgesamt sind die erhaltenen Werte als genau und der Versuch als erfolgreich
durchgeführt zu bewerten.
