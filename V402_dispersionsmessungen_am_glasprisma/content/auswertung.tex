\section{Auswertung}
\label{sec:Auswertung}
\subsection{Bestimmung des Brechenden Winkels}
\label{subsec:brechwinkel}
Zuerst wird der brechende Winkel $\varphi$ des Prismas bestimmt.
Die dazu aufgenommen Messwerte und die nach Gleichung \eqref{eqn:eta}
bestimmten $\varphi$ sind in Tabelle \ref{tab:phi} zu sehen.

\begin{table}[htp]
	\begin{center}
    \caption{Gemessene $\varphi$-Winkel und daraus erhaltene brechende Winkel $\varphi$.}
    \label{tab:phi}
		\begin{tabular}{ccc}
		\toprule
			{$\varphi_\text{l}/°$} & {$\varphi_\text{r}/°$} & {$\varphi/°$} \\
			\midrule
      72,5 & 192,5 & 60,00 \\
      82,8 & 202,6 & 59,90 \\
      80,2 & 200,2 & 60,00 \\
      83,3 & 203,5 & 60,10 \\
      81,4 & 201,4 & 60,00 \\
      79,4 & 199,5 & 60,05 \\
      78,5 & 198,7 & 60,10 \\
		\bottomrule
		\end{tabular}
	\end{center}
\end{table}

Die einzeln ermittelten brechenden Winkel lassen sich mitteln, um als Ergebnis
\begin{equation*}
  \varphi = \SI{60.021(26)}{\degree}
\end{equation*}
zu erhalten. Dieser Wert wird in späteren Rechnungen weiter verwendet.

\subsection{Bestimmung der Brechungsindizes und Ermittlung der Dispersionskurve}
\label{subsec:indizesunddispersion}
Nun werden die Ablenkungswinkel $\eta$ für die Wellenlängen $\lambda$ der Spektrallinien bestimmt.
Zuvor sind einige Anmerkungen angebracht: Es wurden zwei getrennte orangefarbene
Spektrallinien gesehen, sodass für beide getrennt gemessen wurde. Auf die Auswirkungen
dessen wird in der Diskussion eingegangen. Des Weiteren wurde das Prisma bei der konkreten
Durchführung den Werten zufolge nicht korrekt gedreht, da sich für $\eta$ nach Gleichung
\eqref{eqn:eta} negative Werte ergeben. Dies wird durch das Bilden des Betrags der Werte umgangen.
Die Wellenlängen der Spektrallinien werden \cite{nist} entnommen.
Die Werte der gemessenen Winkel $\Omega_\text{l}$ und $\Omega_\text{r}$ für die ebenso
dort notierten Spektrallinien sind gemeinsam mit den
bestimmten Brechungswinkeln in Tabelle \ref{tab:eta} eingetragen.
Die nach \eqref{eqn:nauswinkeln} ermittelten Werte für die Brechungsindizes bei den entsprechenden
Spektrallinien sind auch bereits in der Tabelle enthalten.

\begin{table}[htp]
	\begin{center}
    \caption{Übersicht über die gemessenen $\Omega$-Winkel und daraus berechnete Brechwinkel $\eta$.}
    \label{tab:eta}
		\begin{tabular}{cccccc}
		\toprule
			{Farbe (und Weiteres)} & {$\lambda$/nm} & {$\Omega_\text{l}/°$} & {$\Omega_\text{r}/°$} & {$\eta/°$} & {$n$}\\
			\midrule
      Rot (Cadmium)     & 643,85 & 92,7 & 332,8 & 60,1 & 1,7325 \pm 0,0005 \\
      Orange (1)        & 614,95 & 91,7 & 333,2 & 61,5 & 1,7446 \pm 0,0005 \\
      Orange (2)        & 614,64 & 91,7 & 333,2 & 61,5 & 1,7446 \pm 0,0005 \\
      Gelbgrün          & 597,07 & 91,4 & 333,5 & 62,1 & 1,7497 \pm 0,0005 \\
      Grün              & 546,08 & 91,1 & 333,8 & 62,7 & 1,7547 \pm 0,0005 \\
      Blaugrün          & 508,58 & 90,7 & 334,2 & 63,5 & 1,7614 \pm 0,0005 \\
      Blau (Cadmium)    & 467,81 & 90,5 & 334,3 & 63,8 & 1,7639 \pm 0,0005 \\
      Violett (stark)   & 435,83 & 89,9 & 335,0 & 65,1 & 1,7744 \pm 0,0005 \\
      Violett (schwach) & 404,66 & 89,1 & 335,7 & 66,6 & 1,7863 \pm 0,0006 \\
		\bottomrule
		\end{tabular}
	\end{center}
\end{table}

Es ist zu beachten, dass
der brechende Winkel $\varphi$ bereits fehlerbehaftet ist, sodass sich der Fehler
des Brechungsindizes zu
\begin{equation*}
  \sigma_n = \sqrt{
   \sigma_{\varphi}^{2} \left(\frac{\cos{\left (\frac{\eta}{2} + \frac{\varphi}{2} \right )}}{2 \sin{\left (\frac{\varphi}{2} \right )}} - \frac{\sin{\left (\frac{\eta}{2}
  + \frac{\varphi}{2} \right )} \cos{\left (\frac{\varphi}{2} \right )}}{2 \sin^{2}{\left (\frac{\varphi}{2} \right )}}\right)^{2}}
\end{equation*}
ergibt.

Im Folgenden soll untersucht werden, welche der beiden Dispersionskurven, die durch die
Gleichungen \eqref{eqn:fall1} und \eqref{eqn:fall2} beschrieben werden, die vorliegenden
Brechungsindizes, bzw. vielmehr ihre Quadrate besser anpasst. Dazu werden die konkreten
Ausgleichsrechnungen durchgeführt. Es ergeben sich die folgenden Werte für die Parameter:
\begin{align*}
  A_0 &= \SI{2.922(14)}{}\,,\\
  A_2 &= \SI{4.4(4)e4}{\meter\squared}\,,\\
  A_0' &= \SI{3.271(17)}{}\,,\\
  A_2' &= \SI{6.3(5)e-7}{\per\meter\squared}\,,.\
\end{align*}
Mithilfe der Gleichungen \eqref{eqn:abweichung1} und \eqref{eqn:abweichung2} lassen sich die Abweichungsquadrate
berechnen, um eine Aussage darüber treffen zu können, welche der beiden Dispersionskurven
geeigneter ist. Sie ergeben sich zu
\begin{align*}
  s_n^2 &= \SI{0.000169}{}\,,\\
  s_{n'}^2 &= \SI{0.19(3)}{}\,.
\end{align*}
Es sei darauf hingewiesen, dass sich der Fehler von $s_n^2$ in der Größenordnung
von $10^{12}$ befindet, sodass er nicht explizit ausgeschrieben wurde.
Da das Abweichungsquadrat für die erste Anpassungsfunktion deutlich geringer ist als für
die zweite, wird geurteilt, dass dise Kurve die Quadrate der Brechungsindizes besser anpasst,
sodass mit ihr auch im Folgenden weitergerechnet wird.
Die Quadrate der Brechungsindizes sind zusammen mit dem Graph der konkreten Anpassungsfunktion
in Abbildung \ref{fig:kurve} dargestellt.

\begin{figure}[htp]
  \centering
  \includegraphics[width=15cm]{build/kurve.pdf}
  \caption{Auftragung der Quadrate der Brechungsindizes gegen die Wellenlängen und Graph der Ausgleichsfunktion.}
  \label{fig:kurve}
\end{figure}

Es werden später noch direkt Brechungsindizes durch Einsetzen der Wellenlänge
in die Dispersionskurve mit den empirisch bestimmten Koeffizienten $A_0$ und $A_2$
berechnet. Die Unsicherheit ergibt sich dann zu
\begin{equation*}
  \sigma_n = \sqrt{\frac{\sigma_{A_{0}}^{2}}{4 A_{0} + \frac{4 A_{2}}{\lambda^{2}}} + \frac{\sigma_{A_{2}}^{2}}{4 \lambda^{4} \left(A_{0} + \frac{A_{2}}{\lambda^{2}}\right)}}\,.
\end{equation*}

\newpage
\subsection{Bestimmung der Abbeschen Zahl}
\label{subsec:abbe}
Zur Berechnung der Abbeschen Zahl wird Gleichung \eqref{eqn:abbe}. Die Größen $n_\text{C}$, $n_\text{D}$ und $n_\text{F}$ bezeichnen die Brechungsindizes
für die Wellenlängen $\lambda_\text{C} = \SI{656}{\nano\meter}$, $\lambda_\text{D} = \SI{589}{\nano\meter}$
und $\lambda_\text{F} = \SI{486}{\nano\meter}$.
Mit den in Kapitel \ref{subsec:indizesunddispersion} berechneten Parametern und
der Ausgleichsfunktion aus Gleichung \eqref{eqn:fall1} können die $n_i$ bestimmt werden.
Diese sind in Tabelle \ref{tab:abbe} eingetragen.

\begin{table}[htp]
	\begin{center}
    \caption{Brechungsindizes an den Fraunhoferschen Linien mit Wellenlängen $\lambda_i$.}
    \label{tab:abbe}
		\begin{tabular}{ccc}
		\toprule
			& {$\lambda/$nm} & {$n$} \\
			\midrule
      $\lambda_\text{C}$ & 656 & 1,7390 \pm 0,0019 \\
      $\lambda_\text{D}$ & 589 & 1,7460 \pm 0,0015 \\
      $\lambda_\text{F}$ & 486 & 1,7629 \pm 0,0013 \\
		\bottomrule
		\end{tabular}
	\end{center}
\end{table}

Mithilfe dieser Werte und Gleichung \eqref{eqn:abbe} kann nun die Abesche Zahl berechnet
werden. Sie ergibt sich zu
\begin{equation*}
  \nu = 31{,}2 \pm 2{,}6 \,.
\end{equation*}

Der Fehler ergibt sich durch Fortpflanzung der Unsicherheiten der Brechungsindizes zu
\begin{equation*}
  \sigma_{\nu}= \sqrt{\frac{\sigma_{n_C}^{2} \left(n_D - 1\right)^{2}}{\left(- n_C + n_F\right)^{4}}
  + \frac{\sigma_{n_D}^{2}}{\left(- n_C + n_F\right)^{2}} + \frac{\sigma_{n_F}^{2} \left(n_D - 1\right)^{2}}{\left(- n_C + n_F\right)^{4}}} \,.
\end{equation*}

Nach \cite{flint} beträgt die Abbe-Zahl für das verwendete Material (Schwerflintglas 18)
29{,}30. Die relative Abweichung beträgt 6{,}48\%.

\subsection{Bestimmung des Auflösungsvermögens}
\label{subsec:auflösung}
Wie in Kapitel \ref{subsec:messung} dargelegt wurde, ist es durch Gleichung \eqref{eqn:aufloesung} möglich,
mit Kenntnis der Absorptionskurve das theoretische Auflösungsvermögen $\frac{\lambda}{\Delta \lambda}$ zu bestimmen.
Da die Dispersionskurve hier experimentell bestimmt wurde, ist es also zumindest möglich.
das Auflösungsvermögen abzuschätzen. Unter Anwendung der obigen Gleichung ergibt sich dann
für das Auflösungsvermögen
\begin{equation}
  \frac{\lambda}{\Delta \lambda} = b \left|\frac{\symup{d}n(\lambda)}{\symup{d}\lambda}\right| = \frac{A_2}{\lambda^3\sqrt{A_0 + \frac{A_2}{\lambda^2}}}\,,
\end{equation}
wobei $b = \SI{3}{\centi\meter}$ die Basislänge des verwendeten Prismas ist.
Die Auflösungsvermögen an den Wellenlängen $\lambda_\text{C}$ und $\lambda_\text{F}$
ergeben sich dann zu
\begin{align*}
  A_\text{C} &= \SI{3690(30)}{}\,, \\
  A_\text{F} &= \SI{6500(50)}{}\,. \\
\end{align*}

Die Unsicherheit des Auflösungsvermögens ist
\begin{equation*}
  \sigma_A = \sqrt{\frac{A_2^{2} \sigma_{A_0}^{2}}{4 \lambda^{6} \left(A_0 + \frac{A_2}{\lambda^{2}}\right)^{3}}
  + \sigma_{A_2}^{2} \left(- \frac{A_2}{2 \lambda^{5} \left(A_0 + \frac{A_2}{\lambda^{2}}\right)^{\frac{3}{2}}}
  + \frac{1}{\lambda^{3} \sqrt{A_0 + \frac{A_2}{\lambda^{2}}}}\right)^{2}}
\end{equation*}

\subsection{Bestimmung der nächstgelegenen Absorptionsstelle}
\label{subsec:absorptionsstelle}
Durch Koeffizientenvergleich und Verwendung der Gleichungen \eqref{eqn:dispersion} (in zweiter Ordnung)
und \eqref{eqn:fall1} lässt sich die nächstgelegene Absorptionsstelle durch
\begin{equation*}
  \lambda_1 = \sqrt{\frac{A_2}{A_0-1}}
\end{equation*}
berechnen. Dies ergibt konkret durch Einsetzen der Ausgleichsparameter
\begin{equation*}
  \lambda_1 = \SI{151(7)}{\nano\meter}\,.
\end{equation*}
Die Unsicherheit beträgt dabei
\begin{equation*}
  \sigma_{\lambda_1} = \sqrt{\frac{A_{2} \sigma_{A_{0}}^{2}}{4 \left(A_{0} - 1\right)^{3}} + \frac{\sigma_{A_{2}}^{2}}{4 A_{2} \left(A_{0} - 1\right)}}\,.
\end{equation*}
