\section{Auswertung}
\label{sec:Auswertung}
\subsection{Bestimmung des Brechwinkels}
\label{subsec:brechwinkel}
Zuerst wird der brechende Winkel $\varphi$ des Prismas bestimmt.
Die dazu aufgenommen Messwerte und die nach REFERENZ!!!!!!!!!!!!!!!!!!!!!!!!!
bestimmten $\varphi$ sind in Tabelle \ref{tab:phi} zu sehen.

\begin{table}[htp]
	\begin{center}
    \caption{Gemessene $\varphi$-Winkel und daraus erhaltene brechende Winkel $\varphi$.}
    \label{tab:phi}
		\begin{tabular}{ccc}
		\toprule
			{$\varphi_\text{l}/°$} & {$\varphi_\text{r}/°$} & {$\varphi/°$} \\
			\midrule
      72,5 & 192,5 & 60,00 \\
      82,8 & 202,6 & 59,90 \\
      80,2 & 200,2 & 60,00 \\
      83,3 & 203,5 & 60,10 \\
      81,4 & 201,4 & 60,00 \\
      79,4 & 199,5 & 60,05 \\
      78,5 & 198,7 & 60,10 \\
		\bottomrule
		\end{tabular}
	\end{center}
\end{table}

Die einzeln ermittelten brechenden Winkel lassen sich mitteln, um als Ergebnis
\begin{equation*}
  \varphi = \SI{60.021(26)}{\degree}
\end{equation*}
zu erhalten. Dieser Wert wird in späteren Rechnungen weiter verwendet.

\subsection{Bestimmung der Brechungsindizes und Ermittlung der Dispersionskurve}
\label{subsec:indizesunddispersion}
Nun werden die Brechungswinkel $\eta$ für die Wellenlängen $\lambda$ der Spektrallinien bestimmt.
Zuvor sind einige Anmerkungen angebracht: Es wurden zwei getrennte orangefarbene
Spektrallinien gesehen, sodass für beide getrennt gemessen wurde. Auf die Auswirkungen
dessen wird in der Diskussion eingegangen. Des Weiteren wurde das Prisma bei der konkreten
Durchführung den Werten zufolge nicht korrekt gedreht, da sich für $\eta$ nach Gleichung
REFERENZ negative Werte ergeben. Dies wird durch das Bilden des Betrags der Werte umgangen.
Die Wellenlängen der Spektrallinien werden LITERATUR entnommen.
Die Werte der gemessenen Winkel $\Omega_\text{l}$ und $\Omega_\text{r}$ für die ebenso
dort notierten Spektrallinien sind gemeinsam mit den
bestimmten Brechungswinkeln in Tabelle \ref{tab:eta} eingetragen.
Die nach REFERENZ ermittelten Werte für die Brechungsindizes bei den entsprechenden
Spektrallinien sind auch bereits in der Tabelle enthalten.

\begin{table}[htp]
	\begin{center}
    \caption{Übersicht über die gemessenen $\Omega$-Winkel und daraus berechnete Brechwinkel $\eta$.}
    \label{tab:eta}
		\begin{tabular}{cccccc}
		\toprule
			{Farbe (und Weiteres)} & {$\lambda$/nm} & {$\Omega_\text{l}/°$} & {$\Omega_\text{r}/°$} & {$\eta/°$} & {$n$}\\
			\midrule
      Rot (Cadmium)     & 643,85 & 92,7 & 332,8 & 60,1 & 1,7325 \pm 0,0005 \\
      Orange (1)        & 614,95 & 91,7 & 333,2 & 61,5 & 1,7446 \pm 0,0005 \\
      Orange (2)        & 614,64 & 91,7 & 333,2 & 61,5 & 1,7446 \pm 0,0005 \\
      Gelbgrün          & 597,07 & 91,4 & 333,5 & 62,1 & 1,7497 \pm 0,0005 \\
      Grün              & 546,08 & 91,1 & 333,8 & 62,7 & 1,7547 \pm 0,0005 \\
      Blaugrün          & 508,58 & 90,7 & 334,2 & 63,5 & 1,7614 \pm 0,0005 \\
      Blau (Cadmium)    & 467,81 & 90,5 & 334,3 & 63,8 & 1,7639 \pm 0,0005 \\
      Violett (stark)   & 435,83 & 89,9 & 335,0 & 65,1 & 1,7744 \pm 0,0005 \\
      Violett (schwach) & 404,66 & 89,1 & 335,7 & 66,6 & 1,7863 \pm 0,0006 \\
		\bottomrule
		\end{tabular}
	\end{center}
\end{table}

Es ist zu beachten, dass
der brechende Winkel $\varphi$ bereits fehlerbehaftet ist, sodass sich der Fehler
des Brechungsindizes zu
\begin{equation*}
  \sigma_n = \sqrt{
   \sigma_{\phi}^{2} \left(\frac{\cos{\left (\frac{\eta}{2} + \frac{\phi}{2} \right )}}{2 \sin{\left (\frac{\phi}{2} \right )}} - \frac{\sin{\left (\frac{\eta}{2}
  + \frac{\phi}{2} \right )} \cos{\left (\frac{\phi}{2} \right )}}{2 \sin^{2}{\left (\frac{\phi}{2} \right )}}\right)^{2}}
\end{equation*}
ergibt.

Im Folgenden soll untersucht werden, welche der beiden Dispersionskurven, die durch die
Gleichungen REFERENZ und REFERENZ beschrieben werden,  die vorliegenden Brechungsindizes
besser anpasst.
