\section{Theorie}
\label{sec:Theorie}

Magnetische Felder werden von elektrischen Strömen und magnetischen Dipolen erzeugt.
Die Grundlage der Berechnung einer magnetischen Flussdichte $\symbf{B}$
am Ort $\symbf{r}$ durch einen stationären elektrischen Strom $I$ ist
das Biot-Savartsche Gesetz
\begin{equation}
  \symbf{B}(\symbf{r}) = \frac{\mu_0}{4 \symup{\pi}}
    \int_C \frac{I \symup{d} \symbf{s} \times \symbf{r}}{r^3}\,.
  \label{eqn:biotsavart}
\end{equation}
Dies ist ein Kurvenintegral über die stromdurchflossene Kurve $C$ mit ihrem
Wegelement $\symup{d}\symbf{s}$. Die magnetische Permeabilität des Vakuums $\mu_0$
hat den definierten Wert $\SI{4\pi e-7}{\henry\per\meter}$.

Nähert man für eine Spule ihre Länge als deutlich größer gegenüber ihrem Durchmesser,
so ergibt sich näherungsweise für den Betrag der magnetischen Flussdichte durch Berechnung mit \eqref{eqn:biotsavart}
im Inneren der Spule
\begin{equation}
  B = \mu \frac{n}{l} I\,.
  \label{langespuleinnen}
\end{equation}
Dabei gilt für die Permeabilität $\mu = \mu_0 \mu_r$ mit der materialabhängigen
relativen Permeabilität $\mu_r$, $n$ ist die Windungszahl der Spule und $l$ ihre Länge.
Die magnetische Flussdichte innerhalb der langen Spule ist ungefähr konstant, außerhalb
ist sie näherungsweise null.

Nahezu homogene Magnetfelder lassen sich mithilfe eines Helmholtz-Spulenpaars erzeugen.
Dieses besteht aus zwei in gleicher Richtung stromdurchflossenen Kreisspulen, die auf
auf einer gemeinsamen Achse aufgebaut werden. Ihr Abstand $d$ ist genau der gleiche Spulenradius $R$.
Auf der Symmetrieachse ist das Magnetfeld näherungsweise homogen. Sie lässt sich
ebenso mit dem Biot-Savartschen Gesetz \eqref{eqn:biotsavart} berechnen und ergibt sich zu


Die Messung magnetischer Felder geschieht in der Regel mithilfe einer Hall-Sonde.
In ihr befindet sich ein Metallplättchen, das von einem konstanten Strom durchflossen wird.
Ein äußeres magnetisches Feld übt Kraft auf die bewegten elektrischen Ladungen aus
und trennt sie. Dies erzeugt und eine Spannung und wird auch Hall-Effekt genannt.
Diese Spannung ist ein Maß für die Stärke der Flussdichte des äußeren magnetischen Feldes, welche
dann abgelesen werden kann. Je nach zu untersuchender Geometrie sind longitudinale und transversale Hall-Sonden
zu verwenden.
