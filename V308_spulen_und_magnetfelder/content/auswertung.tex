\section{Auswertung}
\label{sec:Auswertung}

\subsection{Magnetfeld einer kurzen und einer langen Spule}

Für das Magnetfeld einer Spule der Länge $l=\SI{0,16}{m}$ mit $n=300$ Windungen,
die von einem Strom $I=\SI{1,4}{A}$ durchflossen wird, ergeben sich die in
\ref{fig:lange_spule} dargestellten Messwerte. Die zugrundeliegenden Daten können
\ref{tab:lange_spule} entnommen werden. Der Theoriewert wird nach (REFERENZ) berechnet.

\begin{figure}
  \centering
  \includegraphics{build/lange_spule.pdf}
  \caption{Auftragung der Messwerte für die magnetische Feldstärke der langen Spule
  in Abhängigkeit der Position auf der Achse}
  \label{fig:lange_spule}
\end{figure}

Auffällig ist, dass der Theoriewert mit $B=3,30$mT zunächst sehr weit über dem
gemessenen Wert liegt. Da in der Theorie die magnetische Feldstärke für große Abstände
gegen Null geht, kann jedoch der Wert, dem sich die Messwerte annähern von dem theoretischen
Erwartungswert subtrahiert werden, sodass sich für den theoretischen Erwartungswert
nun $B=2.57$mT ergibt.


Für das Magnetfeld einer Spule der Länge $l=\SI{0,055}{m}$ mit $n=100$ Windungen,
die von einem Strom $I=\SI{1,2}{A}$ durchflossen wird, ergeben sich die in
\ref{fig:kurze_spule} dargestellten Messwerte. Die zugrundeliegenden Daten können
\ref{tab:kurze_spule} entnommen werden. Der Theoriewert wird nach (REFERENZ) berechnet.

\begin{figure}
  \centering
  \includegraphics{build/kurze_spule.pdf}
  \caption{Auftragung der Messwerte für die magnetische Feldstärke der kurzen Spule
  in Abhängigkeit der Position auf der Achse}
  \label{fig:kurze_spule}
\end{figure}

\begin{table}
  \centering
    \caption{Messdaten zum Magnetfeld der langen Spule}
    \label{tab:lange_spule}
    \begin{tabular}{c c}
      \toprule
      $r$/cm & $B/$mT \\
      \midrule
      -9	&  -0,63\\
      -8	&  -0,63\\
      -7	&  -0,62\\
      -6	&  -0,65\\
      -5	&  -0,62\\
      -4	&  -0,57\\
      -3	&  -0,53\\
      -2	&  -0,45\\
      -1	&  -0,24\\
      0	  &  0,04\\
      1	  &  0,65\\
      2	  &  1,50\\
      3	  &  2,01\\
      4	  &  2,28\\
      5	  &  2,37\\
      6   &  2,40\\
      7	  &  2,41\\
      8	  &  2,42\\
      9	  &  2,41\\
      10	&  2,40\\
      11	&  2,38\\
      12	&  2,35\\
      13	&  2,29\\
      14	&  2,20\\
      15	&  2,05\\
      16	&  1,69\\
      17	&  1,02\\
      19	&  -0,20\\
      20	&  -0,17\\
      21	&  -0,32\\
      22	&  -0,43\\
      23	&  -0,48\\
      24	&  -0,52\\
      25	&  -0,53\\
      26	&  -0,54\\
      27	&  -0,55\\
      28	&  -0,55\\
      \bottomrule
    \end{tabular}
\end{table}
\begin{table}
  \centering
  \caption{Messdaten zum Magnetfeld der kurzen Spule}
    \label{tab:kurze_spule}
    \begin{tabular}{c c}
      \toprule
      $r$/cm & $B/$mT \\
      \midrule
      -9  &	  -0.56\\
      -8  &	  -0.56\\
      -7	&  -0.56\\
      -6	&  -0.55\\
      -5	&  -0.53\\
      -4	&  -0.51\\
      -3	&  -0.46\\
      -2	&  -0.38\\
      -1	&  -0.24\\
      0	  &   0.06\\
      1	  &   0.65\\
      2	  &   1.28\\
      3	  &   1.68\\
      4	  &   1.77\\
      5	  &   1.59\\
      6	  &   1.13\\
      7	  &   0.47\\
      8	  &  -0.02\\
      9	  &  -0.31\\
      10	&  -0.42\\
      11	&  -0.49\\
      12	&  -0.51\\
      13	&  -0.53\\
      14	&  -0.54\\
      15	&  -0.55\\
      16	&  -0.56\\
      17	&  -0.56\\
      18	&  -0.57\\
      \bottomrule
    \end{tabular}
\end{table}

\subsection{Magnetfeld eines Helmholtzspulenpaares}

\begin{figure}
  \centering
  \includegraphics{build/spulenpaar_5.pdf}
  \caption{Auftragung der Messwerte für die magnetische Feldstärke des Helmholtzspulenpaares
  in Abhängigkeit der Position auf der Achse}
  \label{fig:spulenpaar_5}
\end{figure}

\begin{figure}
  \centering
  \includegraphics{build/spulenpaar_3.pdf}
  \caption{Auftragung der Messwerte für die magnetische Feldstärke des Helmholtzspulenpaares
  in Abhängigkeit der Position auf der Achse}
  \label{fig:spulenpaar_3}
\end{figure}






\begin{table}
  \centering
  \caption{Messdaten zum Magnetfeld des Helmholtzspulenpaares bei 5A ($B_{\symup{1}}$)
  und 3A ($B_{\symup{2}}$)}
  \label{tab:helmholtz}
  \begin{tabular}{c c c}
    \toprule
    $r/$cm & $B_{\symup{1}}/$mT & $B_{\symup{2}}/$mT\\
    \midrule
    -0.5	&  7.077 & 4.161\\
    -0.4	&  7.088 & 4.164\\
    -0.3	&  7.092 & 4.167\\
    -0.2	&  7.094 & 4.168\\
    -0.1	&  7.097 & 4.169\\
    0	    &  7.097 & 4.170\\
    0.1	  &  7.099 & 4.171\\
    0.2	  &  7.099 & 4.171\\
    0.3	  &  7.099 & 4.171\\
    0.4	  &  7.093 & 4.168\\
    0.5	  &  7.078 & 4.154\\
    6	    &  4.703 & 2.772\\
    6.5	  &  4.301 & 2.530\\
    7	    &  3.936 & 2.301\\
    7.5	  &  3.573 & 2.097\\
    8	    &  3.222 & 1.897\\
    8.5	  &  2.896 & 1.714\\
    9	    &  2.599 & 1.542\\
    9.5 	&  2.325 & 1.392\\
    10	  &  2.090 & 1.254\\
    10.5	&  1.864 & 1.128\\
    11	  &  1.682 & 1.017\\
    11.5	&  1.530 & 0.923\\
    12	  &  1.385 & 0.841\\
    12.5	&  1.252 & 0.770\\
    13	  &  1.135 & 0.702\\
    13.5	&  1.033 & 0.645\\
    14	  &  0.950 & 0.609\\
    14.5	&  0.869 & 0.544\\
    15	  &  0.797 & 0.500\\
    15.5	&  0.736 & 0.464\\
    16	  &  0.677 & 0.428\\
    16.5	&  0.627 & 0.397\\
    17	  &  0.580 & 0.370\\
    17.5	&  0.541 & 0.347\\
    18	  &  0.504 & 0.326\\
    18.5	&  0.472 & 0.307\\
    19	  &  0.443 & 0.288\\
    19.5	&  0.416 & 0.272\\
    20	  &  0.391 & 0.256\\
    20.5	&  0.367 & 0.242\\
    21	  &  0.346 & 0.230\\
    \bottomrule
  \end{tabular}
\end{table}

\begin{table}
  \centering
  \caption{Messdaten zum Magnetfeld des Helmholtzspulenpaares bei 5A ($B_{\symup{1}}$)
  und 3A ($B_{\symup{2}}$)(Fortsetzung)}
  \label{tab:helmholtz}
  \begin{tabular}{c c c}
    \toprule
    $r/$cm & $B_{\symup{1}}/$mT & $B_{\symup{2}}/$mT\\
    \midrule
    21.5	&  0.328 & 0.219\\
    22	  &  0.311 & 0.210\\
    22.5	&  0.296 & 0.199\\
    23	  &  0.278 & 0.190\\
    23.5	&  0.266 & 0.181\\
    \bottomrule
  \end{tabular}
\end{table}








\subsection{Bestimmung der Hysteresekurve einer Toroidspule mit Eisenkern}

Für die Remanenz $B_{\symup{r}}$ sind die Werte 128,8 mT und -126,1 mT ablesbar.
