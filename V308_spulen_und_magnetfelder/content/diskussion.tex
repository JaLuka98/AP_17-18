\section{Diskussion}
\label{sec:Diskussion}
Es lässt sich zunächst aussagen, dass die Messwerte in guter Näherung die theoreiwerte
beschreiben können, jedoch gibt es kleine Abweichungen.

Die gemessene magentische Feldstärke der einzelnen Spulen zeigt keinen komplett
symmetrischen Verlauf, wie es theoretisch zu erwarten ist. Mögliche Gründe hierfür sind
die magnetfelder der Kabel, die an einer Seite der Spule angeschlossen sind und somit
auch zum gesamten gemessenen Mangnetfeld beitragen, sowie die Erwärmung der Spule.
An einem Rand der Spule wurde eher gemessen als am anderen. Da sich die Spule
während des Versuchs erhitzt hat, ist es möglich, dass zum späteren Zeitpunkt aufgrund des
mit der Temperatur wachsenden Widerstandes in der Spule eine geringere magnetische Feldstärke
gemessen wird.

gemessen wird.




-asymmetrie bei einfachen spulen wegen Feld durch kabel

-Helmholtz mit 3A unter erwartungswert weil zweite messung -> zuvor erhitzt ->
höherer Widerstand
