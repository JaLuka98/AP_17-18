\section{Diskussion}
\label{sec:Diskussion}
Es lässt sich zunächst aussagen, dass die Messwerte in guter Näherung die theoreiwerte
beschreiben können, jedoch gibt es kleine Abweichungen.

Die gemessene magentische Feldstärke der einzelnen Spulen zeigt keinen komplett
symmetrischen Verlauf, wie es theoretisch zu erwarten ist. Mögliche Gründe hierfür sind
die magnetfelder der Kabel, die an einer Seite der Spule angeschlossen sind und somit
auch zum gesamten gemessenen Mangnetfeld beitragen, sowie die Erwärmung der Spule.
An einem Rand der Spule wurde früher gemessen als am anderen. Da sich die Spule
während des Versuchs erhitzt, ist es möglich, dass zum späteren Zeitpunkt aufgrund des
mit der Temperatur wachsenden Widerstandes in der Spule eine geringere magnetische
Feldstärke gemessen wird. Zudem erreichen die Messwerte bei der langen Spule nicht
den theoretisch erwarteten Wert. Dies kann auf Reibungseffekte zurückgeführt werden,
die den Stromfluss und damit auch das magnetische Feld schwächen.

Auffällig ist zudem, dass die Messwerte relativ zur Theoriekurve nach links, verschoben sind.
Mögliche Gründe hierfür sind erneut die Einflüsse der magnetischen Felder der Kabel oder
Einflüsse des daneben durchgeführten Versuchs. Auch Messfehler durch ein Verrutschen
der Spule während der Messung oder ungenaues Ablesen vom Lineal sind nicht auszuschließen.

An dieser Stelle muss angemerkt werden, dass die Messungen zu den beiden einzelnen Spulen
als relativ ungenau zu bewerten sind, da das verwendete Messgerät auch bei ausgeschaltetem
Spulenstrom Schwankungen der magnetischen Feldstärke $B$ von etwa $\SI{0,3}{\milli\tesla}$ anzeigte.
Gründe dafür könnten die parallel durchgeführten Experimente einer anderen Gruppe,
oder schlichtweg eine Ungenauigkeit des Messgerätes sein.

Bei dem Helmholtzspulenpaar, das von einem
Strom von $\SI{5}{\ampere}$ durchflossen wird, lässt sich die geringe Diskrepanz zwischen dem
theoretischem Erwartungswert und den gemessenen Werten durch Reibungseffekte erklären.

Bei der Messung, bei der das Helmholtzspulenpaar von einem Strom von $\SI{3}{\ampere}$ durchflossen
wird, spielen zusätzlich noch Effekte der Erwärmung eine Rolle. Die beiden Messungen
für die Helmholtzspulen wurden unmittlebar hintereinander durchgeführt, sodass sich die
Spulen bereits beim vorherigen Versuch erwärmt haben. Bedingt durch die Wärme
hatten diese beim zweiten Versuch einen höhrern Widerstand und konnten somit nur ein
geringeres magnetisches Feld induzieren.

Bei der Messung der Hysterese trat bei einem Messwert das Phänomen auf, dass sich
die magnetische Feldstärke bei konstantem Strom mit der Zeit immer weiter erhöhte.
Dabei stieg sie in einer Minute um etwa $\SI{1}{\milli\tesla}$. Da dies nur eine minimale Änderung
ist, hat sie auch keine nennenswerten Auswirkungen auf die Gestalt der Hysteresekurve.
Ein Grund hierfür könnten Ungenauigkeiten des Netzgerätes, das den Strom erzgeugt, sein,
sodass mit einem steigenden Strom auch die magnetische Feldstärke steigt. Es
könnten jedoch auch äußere Störfelder gewirkt haben.
