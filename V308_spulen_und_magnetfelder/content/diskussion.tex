\section{Diskussion}
\label{sec:Diskussion}
Es lässt sich zunächst aussagen, dass die Messwerte in guter Näherung die theoreiwerte
beschreiben können, jedoch gibt es kleine Abweichungen.

Die gemessene magentische Feldstärke der einzelnen Spulen zeigt keinen komplett
symmetrischen Verlauf, wie es theoretisch zu erwarten ist. Mögliche Gründe hierfür sind
die magnetfelder der Kabel, die an einer Seite der Spule angeschlossen sind und somit
auch zum gesamten gemessenen Mangnetfeld beitragen, sowie die Erwärmung der Spule.
An einem Rand der Spule wurde früher gemessen als am anderen. Da sich die Spule
während des Versuchs erhitzt, ist es möglich, dass zum späteren Zeitpunkt aufgrund des
mit der Temperatur wachsenden Widerstandes in der Spule eine geringere magnetische
Feldstärke gemessen wird.

Zudem erreichen die Messwerte bei der langen Spule nicht den theoretisch erwarteten
Wert. Dies kann auf Reibungseffekte zurückgeführt werden, die den Stromfluss und damit
auch das magnetische Feld schwächen. Auch bei dem Helmholtzspulenpaar, das von einem
Strom von $I=5$A durchflossen wird, lässt sich die geringe Diskrepanz zwischen dem
theoretischem Erwartungswert und den gemessenen Werten erklären.

Bei der Messung, bei der das Helmholtzspulenpaar von einem Strom von $I=3$A durchflossen
wird, spielen zusätzlich noch Effekte der Erwärmung eine Rolle. Die beiden Messungen
für die Helmholtzspulen wurden unmittlebar hintereinander durchgeführt, sodass sich die
Spulen bereits beim vorherigen Versuch erwärmt haben. Bedingt durch die Wärme
hatten diese beim zweiten Versuch einen höhrern Widerstand und konnten somit nur ein
geringeres magnetisches Feld induzieren.







-asymmetrie bei einfachen spulen wegen Feld durch kabel

-Helmholtz mit 3A unter erwartungswert weil zweite messung -> zuvor erhitzt ->
höherer Widerstand
