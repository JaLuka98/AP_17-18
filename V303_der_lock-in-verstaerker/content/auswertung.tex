\section{Auswertung}
\label{sec:Auswertung}

Bei der Untersuchung der Ausgänge am zentralen Gerät stellt sich heraus, dass
der Reference-Ausgang eine Spannung mit variabler Amplitude liefert. Die Amplitude
der Spannung des Oscillator-Ausgangs hat die feste Amplitude $\SI{3,32}{\volt}$,
wie an Abbildung \ref{fig:oscillator_output} ersichtlich ist, und eine fein einstellbare
Frequenz. Die Frequenz beider Signale kann grob durch einen gemeinsamen Regler verändert
werden. Die ersten beiden Abbildungen im Anhang sind Bilder dieser Signale.

Die Messwerte zur Abhängigkeit der Ausgangsspannung von der Phasenverschiebung
von Messsignal und Referenzsignal sind in Tabelle \ref{tab:phase} dargestellt.
Für die ersten fünf Werte finden sich die Graphen der Ausgangspannung im Anhang in
den Abbildungen \ref{fig:phase_0} bis \ref{fig:phase_90} für die Messung ohne hinzugefügtes
Rauschen und in den Abbildungen \ref{fig:rauschen_0} bis \ref{fig:rauschen_90} für
die Messung mit hinzugefügtem Rauschen.
An den Abbildungen ist zu erkennen, dass die Graphen der Ausgangsspannung
der verrauschten Signale fast identisch zu den Graphen der Ausgangsspannung
der nicht verrauschten Signale sind. Die Graphen der verrauschten Signale
weisen lediglich an einigen Stellen kleine Abweichungen auf.
Zudem kann an den Graphen bereits erkannt werden, wie sich die Ausgangsspannungen
bei Integration durch den Tiefpass verhalten. Dabei ist darauf zu achten, dass
die x-Achse des Oszilloskops nicht den Wert $y=0$ besitzt, da mittels automatischer
Einstellung des Oszilloskops die Skala immer so gewählt wurde, dass der Graph möglichst
groß angezeigt werden konnte. An den Abbildungen ist jedoch deutlich zu erkennen,
dass immer eine halbe Periode des Kosinus periodisch fortgesetzt wird und dass
der Teil dieser halben Periode mit steigender Phasenverschiebung verschoben wird.
Bei einer Phasenverschiebung von $\phi=0$ liegt eine gleichgerichtete Sinusspannung vor.
Diese entwickelt sich immer mehr zu einer stückweise punktsymmetrischen Spannung bei
$\phi=\frac{\pi}{2}$, die dem Verlauf eines Sinus auf dem Intervall von $-\frac{\pi}{2}$
bis $\frac{\pi}{2}$ folgt. Daran kann bereits erkannt werden, dass Betrag der integrierten
Spannung für $\phi=0$ maximal und für $\phi=\frac{\pi}{2}$ minimal wird.

\begin{table}
\centering
\caption{Messdaten zur Abhängigkeit der Ausgangsspannung von der Phasenverschiebung}
\label{tab:phase}
\begin{tabular}{c c c}
\toprule
$\phi/$grad & $U_\mathrm{out,1}(\phi)/$\,V & $U_\mathrm{out,2}(\phi)/$\,V \\
\midrule
  0	& 5,1	 & 4,9  \\
 30	&	4,75 & 4,8  \\
 45	&	3,9	 & 4,0  \\
 60	&	2,5	 & 2,4  \\
 90	&	0,2	 & 0,3  \\
120 &	-1,8 & -1,6 \\
135 &	-3,2 & -3,2 \\
150 &	-4,4 & -4,4 \\
180 &	-5   & -5   \\
270 &	-0,4 & -0,4 \\
\bottomrule
\end{tabular}
\end{table}

$U_\mathrm{out,1}$ wird ist in der folgenden Rechnung
die Ausgangsspannung bei Verstärkung des Messsignals
ohne hinzugefügtes Rauschen, $U_\mathrm{out,2}$ ist die Ausgangsspannung mit dazu geschaltetem
Rauschen. Die Werte des Messsignals $U_\mathrm{sig}$ sind in Kapitel \ref{sec:Durchführung}
aufgeführt.

Die Ausgangsspannung ist nach Gleichung \eqref{eqn:U_out_prop} proportional zum Kosinus der
Phasenverschiebung der gemischten Signale. Daher wird für die Ausgleichsfunktion
die Zuordnung
\begin{equation}
  U_{\mathrm{out,}i}(\phi) = U_{\mathrm{m,}i} \cos(\phi+\delta_i)+ U_{0,i}
  \label{eqn:fit}
\end{equation}
angesetzt. Dies entspricht einer allgemeinen Kosinusfunktion mit der Amplitude
$U_{\mathrm{m},i}$, dem Phasenparameter $\delta_i$ und der Nullspannung $U_{0,i}$ (Verschiebung
auf der Ordinate). Der Index $i=1,2$ zählt die beiden verschiedenen Messungen durch.

Eine Auftragung der Messwerte und des Graphen der Ausgleichsfunktion für das nicht
verrauschte Signal ist in Abbildung \ref{fig:nichtrausch} zu sehen.

\begin{figure}
  \centering
  \includegraphics{build/ohnerauschen.pdf}
  \caption{Messwerte und Graph der Ausgleichsfunktion zur Verstärkung des nicht verrauschten Signals}
  \label{fig:nichtrausch}
\end{figure}

Die Parameter der Ausgleichsfunktion ergeben sich dann konkret zu
\begin{align*}
  U_\mathrm{m,1} &= \SI{5,05(011)}{\volt} \,\\
  U_{0,1} &= \SI{0,01(009)}{\volt} \,\\
  \delta_1 &= \SI{-0,072(0027)}{\radian} \,.
\end{align*}

Die Ausgleichsrechnung mit \eqref{eqn:fit} wird auch für das verrauschte Signal durchgeführt.
Der Graph der Ausgleichsfunktion und die Messwerte sind in Abbildung \ref{fig:rausch}
zu sehen.

\begin{figure}
  \centering
  \includegraphics{build/mitrauschen.pdf}
  \caption{Messwerte und Graph der Ausgleichsfunktion zur Verstärkung des verrauschten Signals}
  \label{fig:rausch}
\end{figure}

Die Parameter der zweiten Ausgleichsrechnung ergeben sich dann zu
\begin{align*}
  U_\mathrm{m,2} &= \SI{5,01(014)}{\volt} \,\\
  U_{0,2} &= \SI{0,002(0124)}{\volt} \,\\
  \delta_2 &= \SI{-0,09(003)}{\radian} \,.
\end{align*}

Da während der gesamten Messung Gains eingeschaltet waren, müssen nun noch die
Signalspannungen zurückgerechnet werden. Die drei Gains betrugen 5, 10 und 10, also
muss die Signalspannung noch durch den Faktor 500 geteilt werden. Somit ergibt
sich für die Parameter
\begin{align*}
  U_\mathrm{m,1,rück} &= \SI{10,10(022)}{\milli\volt} \,\\
  U_\mathrm{m,2,rück} &= \SI{10,02(028)}{\milli\volt} \, \,.
\end{align*}
Dies ist entspricht auch in guter Näherung dem Wert für die  Amplitude der
Eingangsspannung.
