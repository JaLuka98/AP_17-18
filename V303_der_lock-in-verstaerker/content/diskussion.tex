\section{Diskussion}
\label{sec:Diskussion}
Der Lock-In-Verstärker ist laut Theorie in der Lage, schwache und eventuell
verrauschte Signale zu filtern. Diese Eigenschaft wird bei diesem Versuch eindeutig bestätigt.
Die Werte der Ausgangsspannung sind im Wesentlichen unabhängig davon, ob das Messsignal
verrauscht ist oder nicht. Die Abweichung hier liegt im Rahmen der Messunsichertheit.
Auffällig ist jedoch, dass die Unsicherheiten der Ergebnisse bei dem verrauschten
Signal geringfügig größer sind.

Die Abhängigkeit der Ausgangsspannung von dem Kosinus der Phasenverschiebung zwischen
Mess- und Referenzsignal wird in den Darstellungen der Messwerte und der Graphen
der Ausgleichsfunktionen gut deutlich. Die Parameter der Ausgleichsrechnung sind
nahezu deckungsgleich. Die nichtverschwindenden Phasenparameter $\delta_i$ sind
dadurch zu erklären, dass der Gang der Signale durch elektronische Bauteile diese
auch geringfügig verändert, schließlich wurde die Phasenverschiebung nur am
Anfang des Verstärkers am phase-shifter auf Null gestellt, was widerum auch nicht
garantiert, dass die Phasenverschiebung zu Beginn Null ist. Da die Parameter sehr
klein sind, können sie jedoch auch im Rahmen der  durch statistische Fehler enstehenden
Messungenauigkeit liegen.
Dass die Unsicherheiten der $U_{0,i}$ größer als die nominellen Werte sind, ist
hier unproblematisch, da der Theoriewert für diesen Parameter ohnehin null ist.
