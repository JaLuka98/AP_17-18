\section{Diskussion}
\label{sec:Diskussion}

Im Kurvenverlauf ist eindeutig ein starker Anstieg und eine anschließende
Sättigung erkennbar. Der Anstieg kommt dadurch zustande, dass die Gegenspannung
im Bereich zwischen $U_{\symup{G}}=-1\,$V und $U_{\symup{G}}=0\,$V niedrig genug
wird, sodass Elektronen mit einer hohen kinetischen Energie die Anode erreichen
können. Wird die Spannung nun weiter erhöht, so wirkt auf die Photoelektronen
eine Kraft, die sie zur Anode hin beschleunigt. Die Sättigung tritt ein, da
nun der Großteil der ausgelösten Elektronen die Anode erreicht.

Der Sättigungswert wird dabei durch die Anzahl der ausgelösten Elektronen
festgelegt. Diese widerum ist porportional zur Lichtentensität.
Um den Sättigungswert bereits bei endlichen Spannungen zu erhalten müsste die
Photozelle so konstruiert werden, dass die Anode eine Kugel um die Kathode bildet
und das Licht nur durch eine sehr kleine Öffnung in die Photozelle eintritt.

Der Strom fällt bei der Gegenspannung $U_{\symup{G}}$ nicht abrupt auf null ab,
da die Photoelektronen (wie bereits in Kaptitel \ref{subsec:Theorie_Messung}
beschrieben) aufgrund ihrer unterschiedlichen Energien im Festkörper
verschiedene kineitsche Energien besitzen und somit nicht alle gleichzeitig
bei einer festen Gegenspannung $U_{\symup{G}}$ abgebremst werden.

Wie an den Werten in Tabelle (REFERENZ) zu erkennen ist, wird für Beschleunigungspannungen
von der Anode zur Kathode hin ein geringer negativer Strom, also ein Strom von
der Anode zur Kathode hin gemessen. Dieser lässt sich durch die Eigenschaften der
Photokathode erklären. Laut Versuchsanleigun \cite{Versuchsanleitung} besteht die
Kathode aus einem Material, das bereits bei Raumtemperatur verdampft. Dadurch können
sich geringe Menden dieses Materials auch an der Anode ablagern. Wird nun die
Photozelle mit Licht bestrahlt, so treten auch aus der Anode Photolektronen aus.
Wird nun ein Feld angelegt, das dafür sorgt, dass keine Photoelektronen der Kathode
die Anode erreichen können aber die Phtotelektronen der Anode die Kathode erreichen,
so entsteht der gemessene negative Stromfluss.

Laut Versuchsanleitung \cite{Versuchsanleitung} tritt der negative Strom bereits
bei Einstrahlung energiearmen Lichts auf. Daraus kann gefolgert werden, dass die
Anode selbst eine geringe Austrittsarbeit $A_{\symup{a}}$ besitzt, die maximal die
Größe $hf$ besitzen kann, wobei $f$ die Frequenz des energiearmen Lichts ist.
Eine andere Interpretation dieser Beobachtung ist der oben genannte Effekt des
verdampfenden Kathodenmaterials, das sich auf der Anode ablagert.

Im Versuch zeigt sich, dass bereits bei kleinen Spannungsbeträgen ein Sättigungswert
erreicht werden kann. Dies kann dadurch erklärt werden, dass die kinetische Energie
der Photoelektronen relativ gering ist, sodass sie bereits durch ein schwaches
elektrisches Feld so abgelenkt werden können, dass sie die Anode erreichen.
