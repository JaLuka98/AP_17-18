\section{Diskussion}
\label{sec:Diskussion}
Die Messwerte folgen im Wesentlichen den theoretisch beschriebenen Zusammenhängen.
Insbesondere weisen die geringen relativen Unsicherheiten der Parameter der Ausgleichsfunktionen
darauf hin. Außerdem ist die Abweichung des experimentell bestimmten Wertes für
$h\e_0$ gering und der Literaturwert liegt in der Standardabweichung des experimentell
bestimmten Wertes. \\
Auffällig ist jedoch, dass sich bei jeder Messung keine exakt linearen Zusammenhänge
zwischen den Werten ergeben. Stattdessen legen sich die Messwerte in einer Art Welle
um die Ausgleichsgerade herum. Da dies bei jeder Messreihe auftritt, sind statistische
Unsicherheiten auszuschließen; nur systematische Fehler können diesen Effekt erklären.
Eine mögliche Ursache dafür besteht in einer ungenauen Anzeige der Stromstärke auf dem Messgerät, sodass
eine subjektive Tendenz beim Ablesen diesen Effekt hervorrufen könnte. Es sei darauf
hingewiesen, dass systematische Fehler auch oft durch Skalenwechsel hervorgerufen werden.\\
Außerdem konnte der Verlauf der Stromkurve in Kapitel \ref{subsec:drei} konsistent erklärt werden.
Insgesamt lässt sich die Durchführung dieses Versuchs also als erfolgreich bewerten.
