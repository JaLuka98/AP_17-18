\section{Auswertung}
\label{sec:Auswertung}
\subsection{Bestimmung der Grenzspannungen für das Licht der einzelnden Spektrallinien}
\label{subsec:grenzspannungen}
Wie bereits in Kapitel \ref{subsec:Theorie_Messung} dargelegt und in Gleichung \eqref{eqn:iu2} konkret
formuliert ist die Stromstärke nicht linear von der eingestellten Spannung abhängig.
Stattdessen ist der Photostrom ungefähr proportional zum Quadrat der Spannung. Dies
kann jedoch berücksichtigt werden, sodass zur Bestimmung der Gegenspannung $U_{\symup{g}}$
lediglich $\sqrt{I}$ gegen $U$ aufgetragen werden und der Schnittpunkt der Geraden
mit der $U$-Achse bestimmt werden muss. Dies wird im Folgenden stets durchgeführt.
Bei einer linearen Ausgleichsrechnung nach Gleichung
\eqref{eqn:ausgleichsgerade} berechnet sich die für $a \neq 0 \, \land \, b \neq 0$ eindeutige
Nullstelle der Geraden zu
\begin{equation*}
  x_\text{NS} = -b/a\,.
  \label{eqn:ns}
\end{equation*}

Die gemessene Stromstärke $I$ in Abhängigkeit der eingestellten Spannung $U$ für
den Einfall des Lichts der gelben Spektrallinie mit einer Wellenlänge von $\SI{578}{\nano\meter}$
\footnote{Die Wellenlängen des Lichts der Spektrallinien sind \cite{Versuchsanleitung} zu entnehmen.
Da dort einer Lichtfarbe manchmal mehrere Linien zugeordnet sind, werden die dafür angegebenen
Wellenlängen entsprechend ihrer Stärke gewichtet.}
ist in Tabelle \ref{tab:gelb} zu sehen \footnote{Solange keine Verwechslungsgefahr
besteht, seien alle Stromstärken und Spannungen stets leicht ungenau mit $I$ bzw. $U$ bezeichnet.
In diesem Abschnitt sind alle Spannungen außerdem als Gegenspannungen zu verstehen}. Die radizierten
Stromstärken sind ebenso eingetragen.

\begin{table}[htp]
        \begin{center}
          \caption{Messreihe und radizierte Stromstärken zur gelben Spektrallinie.}
          \label{tab:gelb}
                \begin{tabular}{ScS}
                \toprule
                        {$U/$V} & {$I/$A} & {$\sqrt{I}/\mathrm{\sqrt{pA}}$}\\
                        \midrule
                        0,54 &  0 & 0\\
                        0,50 &  1 & 1\\
                        0,45 &  4 & 2\\
                        0,40 &  7 & 2,65\\
                        0,35 & 12 & 3,46\\
                        0,30 & 19 & 4,36\\
                        0,25 & 26 & 5,1\\
                        0,20 & 36 & 6\\
                        0,15 & 46 & 6,78\\
                        0,10 & 54 & 7,35\\
                        0,05 & 62 & 7,87\\
                        0,00 & 68 & 8,25\\
                \bottomrule
                \end{tabular}
        \end{center}
\end{table}

Der Graph der Ausgleichsfunktion für diese Werte ist zusammen mit den Werten selbst
in Abbildung \ref{fig:gelb} zu sehen.

\begin{figure}
  \centering
  \includegraphics{build/gelb.pdf}
  \caption{Auftragung der radizierten Stromstärken gegen die eingestellte Spannung für die gelbe Spektrallinie und Graph der Ausgleichsfunktion.}
  \label{fig:gelb}
\end{figure}

Für die Gegenspannung für einfallendes Licht der gelben Spektrallinie ergibt sich
\begin{equation*}
  U_{g,\text{gelb}} = \SI{0.569(0011)}{\volt}\,.
\end{equation*}
Die relative Ungenauigkeit dieses Wertes beträgt 1,93\%.

In Tabelle \ref{tab:gruen} sind die Messwerte und radizierten Stromstärken
für das Licht der grünen Spektrallinie mit $\lambda = \SI{546}{\nano\meter}$
eingetragen.

\begin{table}[htp]
        \begin{center}
          \caption{Messreihe und radizierte Stromstärken zur grünen Spektrallinie.}
          \label{tab:gruen}
                \begin{tabular}{ScS}
                \toprule
                        {$U/$V} & {$I/$A} & {$\sqrt{I}/\mathrm{\sqrt{pA}}$}\\
                        \midrule
                        0.64 &   0 &  0   \\
                        0.60 &   4 &  2   \\
                        0.55 &  10 &  3.16\\
                        0.50 &  19 &  4.36\\
                        0.45 &  32 &  5.66\\
                        0.40 &  52 &  7.21\\
                        0.35 &  76 &  8.72\\
                        0.30 & 105 & 10.25\\
                        0.25 & 132 & 11.49\\
                        0.20 & 159 & 12.61\\
                        0.15 & 178 & 13.34\\
                        0.10 & 198 & 14.07\\
                        0.05 & 220 & 14.83\\
                        0.00   & 240 & 15.49\\
                \bottomrule
                \end{tabular}
        \end{center}
\end{table}

In Abbildung \ref{fig:grün} sind diese Messwerte aufgetragen und der Graph der Ausgleichsfunktion zu
diesen ist eingezeichnet.

\begin{figure}
  \centering
  \includegraphics{build/gruen.pdf}
  \caption{Auftragung der radizierten Stromstärken gegen die eingestellte Spannung für die grüne Spektrallinie und Graph der Ausgleichsfunktion.}
  \label{fig:grün}
\end{figure}

Die Gegenspannung für das einfallende grüne Licht ist
\begin{equation*}
  U_{g,\text{grün}} = \SI{0.685(0016)}{\volt}\,.
\end{equation*}
Die relative Ungenauigkeit dieses Wertes ist 2,34\%.

Die Werte zur grünblauen Spektrallinie sind Tabelle \ref{tab:grünblau} zu entnehmen.
Dabei hat das Licht der grünblauen Spektrallinie eine Wellenlänge von $\SI{492}{\nano\meter}$.

\begin{table}[htp]
        \begin{center}
          \caption{Messreihe und radizierte Stromstärken zur grünblauen Spektrallinie.}
          \label{tab:grünblau}
                \begin{tabular}{S[table-format=1.2]S[table-format=1.2]S[table-format=1.2]}
                \toprule
                        {$U/$V} & {$I/$A} & {$\sqrt{I}/\mathrm{\sqrt{pA}}$}\\
                        \midrule
                        0.86 &  0 & 0\\
                        0.80 &  0.5 & 0.71\\
                        0.75 &  1.5 & 1.22\\
                        0.70 &  2 & 1.41\\
                        0.65 &  2.5 & 1.58\\
                        0.60 &  3.5 & 1.87\\
                        0.55 &  4.5 & 2.12\\
                        0.50 &  6 & 2.45\\
                        0.45 &  7.5 & 2.74\\
                        0.40 &  9 & 3\\
                        0.35 & 10.5 & 3.24\\
                        0.30 & 12.5 & 3.54\\
                        0.25 & 14 & 3.74\\
                        0.20 & 16 & 4.00\\
                        0.15 & 18 & 4.24\\
                        0.10 & 19 & 4.36\\
                        0.05 & 20.5 & 4.53\\
                        0.00 & 22 & 4.69\\
                \bottomrule
                \end{tabular}
        \end{center}
\end{table}

Die Auftragung der Messwerte und der Graph der Ausgleichsfunktion sind in Abbildung \ref{fig:grünblau} zu sehen.

\begin{figure}
  \centering
  \includegraphics{build/gruenblau.pdf}
  \caption{Auftragung der radizierten Stromstärken gegen die eingestellte Spannung für die grünblaue Spektrallinie und Graph der Ausgleichsfunktion.}
  \label{fig:grünblau}
\end{figure}

Für einfallendes grünblaues Licht ergibt sich die Gegenspannung zu
\begin{equation*}
  U_{g,\text{grünblau}} = \SI{0.952(0018)}{\volt}\,.
\end{equation*}
Die relative Ungenauigkeit dieses Wertes beträgt 1,89\%.

In Tabelle \ref{tab:violett1} sind die Messwerte und radizierten Stromstärken
für das Licht der ersten violetten Spektrallinie mit $\lambda = \SI{435}{\nano\meter}$
eingetragen.

\begin{table}[htp]
        \begin{center}
          \caption{Messreihe und radizierte Stromstärken zur ersten violetten Spektrallinie.}
          \label{tab:violett1}
                \begin{tabular}{SSS}
                \toprule
                        {$U/$V} & {$I/$A} & {$\sqrt{I}/\mathrm{\sqrt{pA}}$}\\
                        \midrule
                        1.17 & 0 & 0\\
                        1.15 & 1,5 & 1.22\\
                        1.10 & 6 & 2.45\\
                        1.05 & 11 & 3.32\\
                        1.00 & 18 & 4.24\\
                        0.95 & 27 & 5.2\\
                        0.90 & 38 & 6.16\\
                        0.85 & 52 & 7.21\\
                        0.80 & 68 & 8.25\\
                        0.75 & 86 & 9.27\\
                        0.70 & 105 & 10.25\\
                        0.65 & 130 & 11.4\\
                        0.60 & 155 & 12.45\\
                        0.55 & 185 & 13.6\\
                        0.50 & 210 & 14.49\\
                        0.45 & 240 & 15.49\\
                        0.40 & 270 & 16.43\\
                        0.35 & 285 & 16.88\\
                        0.30 & 300 & 17.32\\
                        0.25 & 340 & 18.44\\
                        0.20 & 365 & 19.10\\
                        0.15 & 390 & 19.75\\
                        0.10 & 415 & 20.37\\
                        0.05 & 440 & 20.98\\
                        0.00 & 460 & 21.45\\
                \bottomrule
                \end{tabular}
        \end{center}
\end{table}

In Abbildung \ref{fig:violett1} sind diese Messwerte aufgetragen und der Graph der Ausgleichsfunktion zu
diesen ist eingezeichnet.

\begin{figure}
  \centering
  \includegraphics{build/violett1.pdf}
  \caption{Auftragung der radizierten Stromstärken gegen die eingestellte Spannung für die erste violette Spektrallinie und Graph der Ausgleichsfunktion.}
  \label{fig:violett1}
\end{figure}

Die Gegenspannung für das einfallende Licht der ersten violetten Spektrallinie beträgt
\begin{equation*}
  U_{g,\text{violett1}} = \SI{1.241(0015)}{\volt}\,.
\end{equation*}
Die relative Ungenauigkeit dieses Wertes ist 1,21\%.

Die eingestellten Spannungen und die gessenen Photoströme für das Licht
der zweiten violetten Spektrallinie mit $\lambda = \SI{406}{\nano\meter}$ befinden sich zusammen mit
den radizierten Stromstärken in Tabelle \ref{tab:violett2}.

\begin{table}[htp]
        \begin{center}
          \caption{Messreihe und radizierte Stromstärken zur zweiten violetten Spektrallinie.}
          \label{tab:violett2}
                \begin{tabular}{SSS}
                \toprule
                        {$U/$V} & {$I/$A} & {$\sqrt{I}/\mathrm{\sqrt{pA}}$}\\
                        \midrule
                        1.36 & 0 & 0\\
                        1.30 & 2 & 1.41\\
                        1.25 & 4 & 2\\
                        1.20 & 6 & 2.45\\
                        1.15 & 9.5 & 3.08\\
                        1.10 & 13 & 3.61\\
                        1.05 & 17 & 4.12\\
                        1.00 & 22 & 4.69\\
                        0.95 & 28 & 5.29\\
                        0.90 & 35 & 5.92\\
                        0.85 & 42 & 6.48\\
                        0.80 & 51 & 7.14\\
                        0.75 & 60 & 7.75\\
                        0.70 & 69 & 8.31\\
                        0.65 & 78 & 8.83\\
                        0.60 & 89 & 9.43\\
                        0.55 & 100 & 10\\
                        0.50 & 110 & 10.49\\
                        0.45 & 125 & 11.18\\
                        0.40 & 135 & 11.62\\
                        0.35 & 145 & 12.04\\
                        0.30 & 155 & 12.45\\
                        0.25 & 165 & 12.85\\
                        0.20 & 177 & 13.32\\
                        0.15 & 190 & 13.78\\
                        0.10 & 200 & 14.14\\
                        0.05 & 210 & 14.49\\
                        0.00 & 220 & 14.83\\
                \bottomrule
                \end{tabular}
        \end{center}
\end{table}

Die Auftragung der Messwerte und der Graph der Ausgleichsfunktion sind in Abbildung \ref{fig:violett2} zu sehen.

\begin{figure}
  \centering
  \includegraphics{build/violett2.pdf}
  \caption{Auftragung der radizierten Stromstärken gegen die eingestellte Spannung für die erste zweite violette Spektrallinie und Graph der Ausgleichsfunktion.}
  \label{fig:violett2}
\end{figure}

Die Gegenspannung für das einfallende Licht der zweiten violetten Spektrallinie ist
\begin{equation*}
  U_{g,\text{violett2}} = \SI{1.439(0013)}{\volt}\,.
\end{equation*}
Die relative Ungenauigkeit dieses Wertes beträgt 0.90\%.

In Tabelle \ref{tab:ultraviolett} sind die Messwerte und radizierten Stromstärken
für das Licht der ersten ultravioletten Spektrallinie mit $\lambda = \SI{365.5}{\nano\meter}$
eingetragen.

\begin{table}[htp]
        \begin{center}
          \caption{Messreihe und radizierte Stromstärken zur ultravioletten Spektrallinie.}
          \label{tab:ultraviolett}
                \begin{tabular}{SSS}
                \toprule
                        {$U/$V} & {$I/$A} & {$\sqrt{I}/\mathrm{\sqrt{pA}}$}\\
                        \midrule
                        1.17 & 0 & 0\\
                        1.15 & 1,5 & 1.22\\
                        1.10 & 6 & 2.45\\
                        1.05 & 11 & 3.32\\
                        1.00 & 18 & 4.24\\
                        0.95 & 27 & 5.2\\
                        0.90 & 38 & 6.16\\
                        0.85 & 52 & 7.21\\
                        0.80 & 68 & 8.25\\
                        0.75 & 86 & 9.27\\
                        0.70 & 105 & 10.25\\
                        0.65 & 130 & 11.4\\
                        0.60 & 155 & 12.45\\
                        0.55 & 185 & 13.6\\
                        0.50 & 210 & 14.49\\
                        0.45 & 240 & 15.49\\
                        0.40 & 270 & 16.43\\
                        0.35 & 285 & 16.88\\
                        0.30 & 300 & 17.32\\
                        0.25 & 340 & 18.44\\
                        0.20 & 365 & 19.10\\
                        0.15 & 390 & 19.75\\
                        0.10 & 415 & 20.37\\
                        0.05 & 440 & 20.98\\
                        0.00 & 460 & 21.45\\
                \bottomrule
                \end{tabular}
        \end{center}
\end{table}

In Abbildung \ref{fig:ultraviolett} sind diese Messwerte aufgetragen und der Graph der Ausgleichsfunktion zu
diesen eingezeichnet.

\begin{figure}
  \centering
  \includegraphics{build/ultraviolett.pdf}
  \caption{Auftragung der radizierten Stromstärken gegen die eingestellte Spannung für die ultraviolette Spektrallinie und Graph der Ausgleichsfunktion.}
  \label{fig:ultraviolett}
\end{figure}

Die Gegenspannung für das einfallende Licht der ultravioletten Spektrallinie beträgt
\begin{equation*}
  U_{g,\text{ultraviolett}} = \SI{1.809(0021)}{\volt}\,.
\end{equation*}
Die relative Ungenauigkeit dieses Wertes ist 1,16\%.

\subsection{Untersuchung der Beziehung zwischen Grenzspannungen und Frequenzen}

Gemäß Gleichung \eqref{eqn:hfeUA} lässt sich das Verhältnis $\frac{h}{e_0}$ und die
materialspezifische Austrittsarbeit $A_\text{k}$ bestimmen, wenn eine lineare Ausgleichsrechnung
mit den Grenzspannungen und den Frequenzen der Spektrallinien durchgeführt wird.
Die Werte sind in Tabelle \ref{tab:spannungfrequenz} zusammengefasst.
Dabei wird die Frequenz des Lichts durch $f = c/\lambda$ mit der Lichtgeschwindigkeit $c$ im Vakuum berechnet.

\begin{table}[htp]
        \begin{center}
          \caption{Werte der Grenzspannungen mit dazugehörigen Wellenlängen und Frequenzen.}
          \label{tab:spannungfrequenz}
                \begin{tabular}{cSc}
                \toprule
                        {$U_\text{g}/$V} & {$\lambda/$nm} & {$f \cdot 10^{-14}/$Hz}\\
                        \midrule
                        $\SI{0.569(0011)}{}$ & 578   & 5,19 \\
                        $\SI{0.685(0016)}{}$ & 546   & 5,49 \\
                        $\SI{0.952(0018)}{}$ & 492   & 6,09 \\
                        $\SI{1.241(0015)}{}$ & 435   & 6,89 \\
                        $\SI{1.439(0013)}{}$ & 406   & 7,38 \\
                        $\SI{1.809(0021)}{}$ & 365,5 & 8,20 \\
                \bottomrule
                \end{tabular}
        \end{center}
\end{table}

Im Folgenden werden die in \ref{subsec:grenzspannungen} bestimmten Grenzspannungen
als nicht fehlerbehaftet angenommen, da die Unsicherheiten durchgehend gering sind.
Wird nun eine lineare Ausgleichsrechnung gemäß $y = ax+b$ angesetzt, wobei die Grenzspannungen
die y-Werte und die Frequenzen die x-Werte sind, so ist der Parameter $a$ der experimentell
bestimmte Wert für den Quotienten $h/e_0$ und der Betrag von $b$ entspricht der Austrittsarbeit
$A_\text{k}$. Die Werte und der Graph der konkreten Ausgleichsfunktion ist in  Abbildung \ref{fig:spannungfrequenz}
dargestellt.

\begin{figure}
  \centering
  \includegraphics{build/dickfett.pdf}
  \caption{Auftragung der zuvor bestimmten Gegenspannungen gegen die Frequenzen des Lichts der Spektrallinien und Graph der Ausgleichsfunktion.}
  \label{fig:spannungfrequenz}
\end{figure}

Der experimentelle Schätzwert für $h/e_0$ beträgt $\SI{4.06(007)e-15}{\volt\second}$ mit einer relativen Unsicherheit von 1,72\%.
Der Literaturwert beträgt ungefähr $\SI{4.13577e-15}{\volt\second}$ \footnote{Dieser berechnet
sich aus den in \cite{hunde_0} angegeben Werten für das Plancksche Wirkungsquantum $h$ und der Elementarladung $e$.},
sodass die relative Abweichung zu diesem -1.83\% ist.
Der experimentell bestimmte Wert für die Austrittsarbeit $A_\text{k}$ beträgt $\SI{1.54(005)}{\electronvolt}$.
Die relative Unsicherheit dieses Wertes ist 3,25\%.

\newpage
\subsection{Untersuchung des Photostroms über ein großes Intervall anliegender Spannung}
\label{subsec:drei}

Das Verhalten des Photostroms in Abhängigkeit der Spannung für einfallendes Licht
der gelben Spektrallinie ist über ein großes Intervall von $\SI{-19}{\volt}$ bis $\SI{19}{\volt}$
zu untersuchen. Hier sind negative Spannungen explizit als Bremsspannungen und positive als
beschleunigende Spannungen zu verstehen. Die Messwerte sind in Tabelle \ref{tab:spannung}
zu sehen.

\begin{table}[htp]
        \begin{center}
          \caption{Messwerte des Photostroms in Abhängigkeit der angelegten Spannung für die gelbe Spektrallinie über das große Intervall.}
          \label{tab:spannung}
                \begin{tabular}{SS}
                \toprule
                        {$U/$V} & {$I/$A}\\
                        \midrule
                        -19 & -6\\
                        -18 & -6\\
                        -17 & -6\\
                        -16 & -6\\
                        -15 & -6\\
                        -14 & -6\\
                        -13 & -6\\
                        -12 & -6\\
                        -11 & -6\\
                        -10 & -5\\
                         -9 & -5\\
                         -8 & -5\\
                         -7 & -5\\
                         -6 & -5\\
                         -5 & -5\\
                         -4 & -4\\
                         -3 & -4\\
                         -2 & -3\\
                         -1 & -3\\
                          0 &  95\\
                          1 & 235\\
                          2 & 340\\
                          3 & 460\\
                          4 & 560\\
                          5 & 620\\
                          6 & 670\\
                          7 & 720\\
                          8 & 770\\
                          9 & 800\\
                         10 & 820\\
                         11 & 860\\
                         12 & 880\\
                         13 & 900\\
                         14 & 920\\
                         15 & 920\\
                         16 & 940\\
                         17 & 960\\
                         18 & 970\\
                         19 & 980\\
                \bottomrule
                \end{tabular}
        \end{center}
\end{table}

Zur Veranschaulichung sind diese Werte in Abbildung \ref{fig:letztes} grafisch dargestellt.

\begin{figure}
  \centering
  \includegraphics{build/spannung.pdf}
  \caption{Auftragung des Photostroms in Abhängigkeit der eingestellten Spannung in einem großen Intervall.}
  \label{fig:letztes}
\end{figure}

Im Kurvenverlauf ist ein starker Anstieg und eine anschließende
Sättigung erkennbar. Der Anstieg kommt dadurch zustande, dass die Gegenspannung
im Bereich zwischen $U_{\symup{G}}=-1\,$V und $U_{\symup{G}}=0\,$V niedrig genug
wird, sodass Elektronen mit einer hohen kinetischen Energie die Anode erreichen
können. Wird die Spannung nun weiter erhöht, so wirkt auf die Photoelektronen
eine Kraft, die sie zur Anode hin beschleunigt. Die Sättigung tritt ein, da
nun der Großteil der ausgelösten Elektronen die Anode erreicht.

Der Sättigungswert wird dabei durch die Anzahl der ausgelösten Elektronen
festgelegt. Diese widerum ist porportional zur Lichtentensität.
Um den Sättigungswert bereits bei endlichen Spannungen zu erhalten müsste die
Photozelle so konstruiert werden, dass die Anode eine Kugel um die Kathode bildet
und das Licht nur durch eine sehr kleine Öffnung in die Photozelle eintritt. Dann
wäre gewährleistet, dass stehts nahezu alle Phtotelektronen die Anode erreichen.

Der Strom fällt bei der Gegenspannung $U_{\symup{G}}$ nicht abrupt auf null ab,
da die Photoelektronen (wie bereits in Kaptitel \ref{subsec:Theorie_Messung}
beschrieben) aufgrund ihrer unterschiedlichen Energien im Festkörper
verschiedene kineitsche Energien besitzen und somit nicht alle gleichzeitig
bei einer festen Gegenspannung $U_{\symup{G}}$ abgebremst werden.

Wie an den Werten in Tabelle \ref{tab:spannung} zu erkennen ist, wird für Beschleunigungsspannungen
von der Anode zur Kathode hin ein geringer negativer Strom, also ein Strom von
der Anode zur Kathode hin gemessen. Dieser lässt sich durch die Eigenschaften der
Photokathode erklären. Laut Versuchsanleitung \cite{Versuchsanleitung} besteht die
Kathode aus einem Material, das bereits bei Raumtemperatur verdampft. Dadurch können
sich geringe Mengen dieses Materials auch an der Anode ablagern. Wird nun die
Photozelle mit Licht bestrahlt, so treten auch aus der Anode Photolektronen aus.
Wird nun ein Feld angelegt, das dafür sorgt, dass keine Photoelektronen der Kathode
die Anode erreichen können aber die Photoelektronen der Anode die Kathode erreichen,
so entsteht der gemessene negative Stromfluss.

Laut Versuchsanleitung \cite{Versuchsanleitung} tritt der negative Strom bereits
bei Einstrahlung energiearmen Lichts auf. Daraus kann gefolgert werden, dass die
Anode selbst eine geringe Austrittsarbeit $A_{\symup{a}}$ besitzt, die kleiner sein muss
als die Größe $hf$, wobei $f$ die Frequenz des energiearmen Lichts ist.
Eine andere Interpretation dieser Beobachtung ist der oben genannte Effekt des
verdampfenden Kathodenmaterials, das sich auf der Anode ablagert.

Im Versuch zeigt sich, dass bereits bei kleinen Spannungsbeträgen ein Sättigungswert
erreicht werden kann. Dies kann dadurch erklärt werden, dass die kinetische Energie
der Photoelektronen relativ gering ist, sodass sie bereits durch ein schwaches
elektrisches Feld so abgelenkt werden können, dass sie die Anode erreichen.
