\section{Auswertung}
\label{sec:Auswertung}
\subsection{Bestimmung der Grenzspannungen für das Licht der einzelnden Spektrallinien}
\label{subsec:grenzspannungen}
Wie bereits in \ref{subsec:Theorie_Messung} dargelegt und in Gleichung \eqref{eqn:iu2} konkret
formuliert ist die Stromstärke nicht linear von der eingestellten Spannung abhängig.
Stattdessen ist der Photostrom ungefähr proportional zum Quadrat der Spannung. Dies
kann jedoch berücksichtigt werden, sodass zur Bestimmung der Gegenspannung $U_{\symup{g}}$
lediglich $\sqrt{I}$ gegen $U$ aufgetragen werden und der Schnittpunkt der Geraden
mit der $U$-Achse bestimmt werden muss. Dies wird im Folgenden stets durchgeführt.
Bei einer linearen Ausgleichsrechnung nach Gleichung
\eqref{eqn:ausgleichsgerade} berechnet sich die für $a \neq 0 \, \land \, b \neq 0$ eindeutige
Nullstelle der Geraden zu
\begin{equation}
  x_\text{NS} = -b/a\,.
  \label{eqn:ns}
\end{equation}

Die gemessene Stromstärke $I$ in Abhängigkeit der eingestellten Spannung $U$ für
den Einfall des Lichts der gelben Spektrallinie mit einer Wellenlänge von $\SI{577}{\nano\meter}$
bzw. $\SI{579}{\nano\meter}$ ist in Tabelle \ref{tab:gelb} zu sehen \footnote{Solange keine Verwechslungsgefahr
besteht, seien alle Stromstärken und Spannungen stets leicht ungenau mit $I$ bzw. $U$ bezeichnet.
In diesem Abschnitt sind alle Spannungen außerdem als Gegenspannungen zu verstehen}. Die radizierten
Stromstärken sind ebenso eingetragen.

\begin{table}[htp]
        \begin{center}
          \caption{Messreihe und radizierte Stromstärken zur gelben Spektrallinie.}
          \label{tab:gelb}
                \begin{tabular}{ccc}
                \toprule
                        {$U/$V} & {$I/$A} & {$\sqrt{I}/\mathrm{\sqrt{pA}}$}\\
                        \midrule
                        0.64 & 0.00 & 0.00\\
                        0.60 & 4.00 & 2.00\\
                        0.55 & 10.00 & 3.16\\
                        0.50 & 19.00 & 4.36\\
                        0.45 & 32.00 & 5.66\\
                        0.40 & 52.00 & 7.21\\
                        0.35 & 76.00 & 8.72\\
                        0.30 & 105.00 & 10.25\\
                        0.25 & 132.00 & 11.49\\
                        0.20 & 159.00 & 12.61\\
                        0.15 & 178.00 & 13.34\\
                        0.10 & 198.00 & 14.07\\
                        0.05 & 220.00 & 14.83\\
                        0.00 & 240.00 & 15.49\\
                \bottomrule
                \end{tabular}
        \end{center}
\end{table}

Der Graph der Ausgleichsfunktion für diese Werte ist zusammen mit den Werten selbst
in Abbildung \ref{fig:gelb} zu sehen.

\begin{figure}
  \centering
  \includegraphics{build/gelb.pdf}
  \caption{Auftragung der radizierten Stromstärken gegen die eingestellte Spannung für die gelbe Spektrallinie und Graph der Ausgleichsfunktion.}
  \label{fig:gelb}
\end{figure}

Für die Gegenspannung für einfallendes Licht der gelben Spektrallinie ergibt sich
\begin{equation}
  U_{g,\text{gelb}} = \SI{0.685(0016)}{\volt}\,.
\end{equation}
Die relative Ungenauigkeit dieses Wertes beträgt 2.34\%.

In Tabelle \ref{tab:gruen} sind die Messwerte und radizierten Stromstärken
für das Licht der grünen Spektrallinie mit $\lambda = \SI{546}{\nano\meter}$
eingetragen.

In Abbildung \ref{fig:grün} sind diese Messwerte aufgetragen und der Graph der Ausgleichsfunktion zu
diesen eingezeichnet.

\begin{figure}
  \centering
  \includegraphics{build/grün.pdf}
  \caption{Auftragung der radizierten Stromstärken gegen die eingestellte Spannung für die gelbe Spektrallinie und Graph der Ausgleichsfunktion.}
  \label{fig:gelb}
\end{figure}
