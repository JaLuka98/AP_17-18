\section{Ziel}
\label{sec:Ziel}

Ziel dieses Versuchs ist die Aufnahme und Untersuchung verschiedener Emissions- und
Absorptionsspektren.

Zur Bestimmung der Halbwertsbreite werden die $K_{\symup{\alpha}}$ und die
$K_{\symup{\beta}}$ Linie vergrößert dargestellt. Um eine Aussage über die
Halbwertsbreite zu treffen können, die nicht lediglich einer Abschätzung entspricht,
werden die Messwerte miteinander verbunden. Dies ist in Abbildung \ref{fig:hbreite}
zu sehen.

\begin{figure}
  \centering
  \includegraphics{build/peaks.pdf}
  \caption{Skizzen zur Bestimmung der Halbwertsbreiten der $K_{\symup{\alpha}}$
  Linie (rechts) und der $K_{\symup{\beta}}$ Linie (links).}
  \label{fig:hbreite}
\end{figure}

Dabei beschreiben die blau gestrichelten Linien das Minimum und das Maximum des
jeweiligen Peaks. Die grüne gestrichelte Linie befindet sich in der Mitte der
beiden blauen und beschreibt somit die halbe Höhe des Peaks. Die magenta farbenen
vertikalen linien zeigen die Frequenzen, an denen die Verbindungsgeraden zwischen
den Messwerten gerade die halbe Höhe des Peaks besitzen. Die Differenz der beiden
Frequenzen ergibt in guter Näherung die Halbwertsbreite des jeweiligen Peaks.
Es ergeben sich die Werte
\begin{align*}
  \theta_{\symup{\alpha,1}}=22{,}09° \,, \\
  \theta_{\symup{\alpha,2}}=22{,}52° \,, \\
  \theta_{\symup{\beta,1}}=19{,}85° \,, \\
  \theta_{\symup{\beta,2}}=20{,}28° \,.
\end{align*}
Damit ergibt sich für die Halbwertsbreiten
\begin{align*}
  b_{\symup{\alpha}}=0{,}43° \,,\\
  b_{\symup{\beta}}=0{,}43°  \,.
\end{align*}

Das Auflösungsvermögen des gegebenen Versuchsaufbaus ist dann gegeben durch die
Energiedifferenzen $\Delta E_{\symup{\alpha}}=E_{\theta_{\symup{\alpha,1}}}
-E_{\theta_{\symup{\alpha,2}}}$ und $\Delta E_{\symup{\beta}}=
E_{\theta_{\symup{\beta,1}}}-E_{\theta_{\symup{\beta,2}}}$ der Linien. Diese
ergeben sich nach Gleichung \eqref{eqn:energieauswinkel} zu
\begin{align*}
  \Delta E_{\symup{\alpha}}=(8{,}191-8{,}043)\,\symup{keV}=0{,}148\,\symup{keV} \,, \\
  \Delta E_{\symup{\beta}}=(9{,}072-8{,}888)\,\symup{keV}=0{,}184\,\symup{keV} \,.
\end{align*}
