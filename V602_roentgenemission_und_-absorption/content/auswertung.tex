\section{Auswertung}
\label{sec:Auswertung}
\subsubsection{Überprüfung der Bragg-Bedingung}
\label{subsec:bragg}
Die Werte, die zur Überprüfung der Bragg-Bedingung aufgenommen wurden, sind in Tabelle
\ref{tab:bragg} zu sehen. Es wurden ursprünglich die Impulse pro Sekunde aufgezeichnet,
im Folgenden wird diese Messgröße jedoch als Intensität $I$ bezeichnet.

\begin{table}[htp]
        \begin{center}
          \caption{Messwerte der Intensität in Abhängigkeit des Winkels des Geiger-Müller Zählrohrs zur Überprüfung der Bragg-Bedingung.}
          \label{tab:bragg}
                \begin{tabular}{SS}
                \toprule
                        {$\alpha_\text{GM}/$Grad} & {$I$} \\
                        \midrule
                        26.0 & 31\\
                        26.1 & 31\\
                        26.2 & 31\\
                        26.3 & 32\\
                        26.4 & 31\\
                        26.5 & 28\\
                        26.6 & 34\\
                        26.7 & 31\\
                        26.8 & 38\\
                        26.9 & 42\\
                        27.0 & 44\\
                        27.1 & 51\\
                        27.2 & 55\\
                        27.3 & 68\\
                        27.4 & 76\\
                        27.5 & 84\\
                        27.6 & 98\\
                        27.7 & 106\\
                        27.8 & 105\\
                        27.9 & 121\\
                        28.0 & 123\\
                        28.1 & 141\\
                        28.2 & 136\\
                        28.3 & 136\\
                        28.4 & 129\\
                        28.5 & 123\\
                        28.6 & 115\\
                        28.7 & 114\\
                        28.8 & 104\\
                        28.9 & 85\\
                        29.0 & 80\\
                        29.1 & 72\\
                        29.2 & 57\\
                        29.3 & 45\\
                        29.4 & 45\\
                        29.5 & 33\\
                        29.6 & 31\\
                        29.7 & 30\\
                        29.8 & 28\\
                        29.9 & 27\\
                        30.0 & 28\\
                \bottomrule
                \end{tabular}
        \end{center}
\end{table}

Die Messreihe ist grafisch in \ref{fig:bragg} dargestellt.

\begin{figure}
  \centering
  \includegraphics{build/bragg.pdf}
  \caption{Auftragung der Intensität in Abhängigkeit des Winkels des Geiger-Müller Zählrohrs.}
  \label{fig:bragg}
\end{figure}

Wie an den Messwerten und am Diagramm zu erkennen ist, befindet sich das Maximum
der Intensität näherungsweise bei $28{,}1°$.

Der theoretische Wert beträgt $28°$, da der verwendete LiF-Kristall für diese Messreihe
auf einen festen Kristallwinkel von $14°$ eingestellt wurde. Der relative Fehler bei
der experimentellen Bestimmung beträgt demnach $0{,}36\%$.

\subsection{Untersuchung des Emissionsspektrums der Röntgenröhre}
\label{subsec:emission}

Es soll nun das Emissionsspektrum einer Röntgenröhre untersucht werden. Die Anode besteht aus
Kupfer. Die Messreihe ist in Tabelle \ref{tab:emission} zu finden.

\begin{longtable}{S[table-format=3.1] S[table-format=3.0]}
                \caption{Messwerte der Intensität in Abhängigkeit des doppelten Kristallwinkels zur Untersuchung des Emissionsspektrums.}\\
                \label{tab:emission}
                {$2 \cdot \theta/$Grad} & {$I$}\\
                \midrule
                8.0 & 16.0\\
                8.4 & 16.0\\
                8.8 & 18.0\\
                9.2 & 17.0\\
                9.6 & 24.0\\
                10.0 & 21.0\\
                10.4 & 24.0\\
                10.8 & 33.0\\
                11.2 & 39.0\\
                11.6 & 41.0\\
                12.0 & 50.0\\
                12.4 & 58.0\\
                12.8 & 69.0\\
                13.2 & 67.0\\
                13.6 & 75.0\\
                14.0 & 78.0\\
                14.4 & 88.0\\
                14.8 & 91.0\\
                15.2 & 102.0\\
                15.6 & 111.0\\
                16.0 & 118.0\\
                16.4 & 134.0\\
                16.7 & 122.0\\
                17.2 & 134.0\\
                17.6 & 134.0\\
                18.0 & 137.0\\
                18.4 & 146.0\\
                18.8 & 140.0\\
                19.2 & 151.0\\
                19.6 & 161.0\\
                20.0 & 159.0\\
                20.4 & 162.0\\
                20.8 & 168.0\\
                21.2 & 164.0\\
                21.6 & 178.0\\
                22.0 & 183.0\\
                22.4 & 157.0\\
                22.8 & 168.0\\
                23.2 & 167.0\\
                23.6 & 173.0\\
                24.0 & 164.0\\
                24.4 & 163.0\\
                24.8 & 168.0\\
                25.2 & 169.0\\
                25.6 & 160.0\\
                26.0 & 140.0\\
                26.4 & 137.0\\
                26.8 & 129.0\\
                27.2 & 123.0\\
                27.6 & 120.0\\
                28.0 & 113.0\\
                28.4 & 116.0\\
                28.8 & 111.0\\
                29.2 & 126.0\\
                29.6 & 111.0\\
                30.0 & 114.0\\
                30.4 & 113.0\\
                30.8 & 105.0\\
                31.2 & 100.0\\
                31.6 & 106.0\\
                32.0 & 107.0\\
                32.4 & 100.0\\
                32.8 & 98.0\\
                33.2 & 103.0\\
                33.6 & 102.0\\
                34.0 & 95.0\\
                34.4 & 91.0\\
                34.8 & 87.0\\
                35.2 & 80.0\\
                35.5 & 89.0\\
                36.0 & 81.0\\
                36.4 & 75.0\\
                36.8 & 75.0\\
                37.2 & 83.0\\
                37.5 & 77.0\\
                38.0 & 81.0\\
                38.4 & 84.0\\
                38.8 & 87.0\\
                39.2 & 85.0\\
                39.5 & 131.0\\
                40.0 & 1021.0\\
                40.4 & 675.0\\
                40.8 & 135.0\\
                41.2 & 104.0\\
                41.5 & 93.0\\
                42.0 & 96.0\\
                42.4 & 89.0\\
                42.8 & 86.0\\
                43.2 & 93.0\\
                43.6 & 108.0\\
                44.0 & 201.0\\
                44.4 & 3195.0\\
                44.8 & 3069.0\\
                45.2 & {-}\\
                45.5 & {-}\\
                46.0 & 81.0\\
                46.4 & 73.0\\
                46.8 & 65.0\\
                47.2 & 57.0\\
                47.6 & 57.0\\
                48.0 & 54.0\\
                48.4 & 59.0\\
                48.8 & 50.0\\
                49.2 & 45.0\\
                49.6 & 47.0\\
                50.0 & 49.0\\
                50.4 & 41.0\\
                50.8 & 46.0\\
                51.2 & 40.0\\
                51.6 & 43.0\\
                52.0 & 42.0\\
                \bottomrule
\end{longtable}
