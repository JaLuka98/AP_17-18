\section{Auswertung}
\label{sec:Auswertung}
\subsection{Überprüfung der Bragg-Bedingung}
\label{subsec:bragg}
Zunächst sei erwähnt, dass der verwendete LiF-Kristall einen Netzebendenabstand von
$d = \SI{201.4}{\pico\meter}$ besitzt. Dieser Wert wird stets für $d$ angenommen.
Außerdem werden Werte stets nach Ermessen sinnvoll gerundet, auch beim Ablesen aus
Diagrammen kann keine perfekte Genauigkeit gewährleistet werden. Ohne später
explizit darauf hinzuweisen, sind deswegen alle experimentell bestimmten Werte als Näherungen zu verstehen.\\
Die Werte, die zur Überprüfung der Bragg-Bedingung aufgenommen wurden, sind in Tabelle
\ref{tab:bragg} zu sehen. Es wurden ursprünglich die Impulse pro Sekunde aufgezeichnet,
im Folgenden wird diese Messgröße jedoch als Intensität $I$ bezeichnet.

\begin{table}[htp]
        \begin{center}
          \caption{Messwerte der Intensität in Abhängigkeit des Winkels des Geiger-Müller Zählrohrs zur Überprüfung der Bragg-Bedingung.}
          \label{tab:bragg}
                \begin{tabular}{S[table-format=2.1] S[table-format=3.0]}
                \toprule
                        {$\alpha_\text{GM}/$°} & {$I$} \\
                        \midrule
                        26.0 & 31\\
                        26.1 & 31\\
                        26.2 & 31\\
                        26.3 & 32\\
                        26.4 & 31\\
                        26.5 & 28\\
                        26.6 & 34\\
                        26.7 & 31\\
                        26.8 & 38\\
                        26.9 & 42\\
                        27.0 & 44\\
                        27.1 & 51\\
                        27.2 & 55\\
                        27.3 & 68\\
                        27.4 & 76\\
                        27.5 & 84\\
                        27.6 & 98\\
                        27.7 & 106\\
                        27.8 & 105\\
                        27.9 & 121\\
                        28.0 & 123\\
                        28.1 & 141\\
                        28.2 & 136\\
                        28.3 & 136\\
                        28.4 & 129\\
                        28.5 & 123\\
                        28.6 & 115\\
                        28.7 & 114\\
                        28.8 & 104\\
                        28.9 & 85\\
                        29.0 & 80\\
                        29.1 & 72\\
                        29.2 & 57\\
                        29.3 & 45\\
                        29.4 & 45\\
                        29.5 & 33\\
                        29.6 & 31\\
                        29.7 & 30\\
                        29.8 & 28\\
                        29.9 & 27\\
                        30.0 & 28\\
                \bottomrule
                \end{tabular}
        \end{center}
\end{table}

Die Messreihe ist grafisch in Abbildung \ref{fig:bragg} dargestellt.

\begin{figure}
  \centering
  \includegraphics{build/bragg.pdf}
  \caption{Auftragung der Intensität in Abhängigkeit des Winkels des Geiger-Müller Zählrohrs.}
  \label{fig:bragg}
\end{figure}

Wie an den Messwerten und am Diagramm zu erkennen ist, befindet sich das Maximum
der Intensität bei $28{,}1°$.

Der theoretische Wert beträgt $28°$, da der verwendete LiF-Kristall für diese Messreihe
auf einen festen Kristallwinkel von $14°$ eingestellt wurde. Der relative Fehler bei
der experimentellen Bestimmung beträgt demnach $0{,}36\%$.

\subsection{Untersuchung des Emissionsspektrums der Röntgenröhre}
\label{subsec:emission}

Es soll nun das Emissionsspektrum einer Röntgenröhre untersucht werden. Die Anode besteht aus
Kupfer. Die Messreihe ist in Tabelle \ref{tab:emission} zu finden.

\begin{table}[htp]
        \begin{center}
          \caption{Messwerte der Intensität in Abhängigkeit des doppelten Kristallwinkels zur Untersuchung des Emissionsspektrums.}
          \label{tab:emission}
                \begin{tabular}{S[table-format=2.1] S[table-format=3.0] S[table-format=2.1] S[table-format=3.0] S[table-format=2.1] S[table-format=3.0]}
                \toprule
                        {$2 \cdot \theta/$°} & {$I$} & {$2 \cdot \theta/$°} & {$I$} & {$2 \cdot \theta/$°} & {$I$} \\
                        \midrule
                         8.0 &  16 & 24.4 &  163 & 40.8 &  135\\
                         8.4 &  16 & 24.8 &  168 & 41.2 &  104\\
                         8.8 &  18 & 25.2 &  169 & 41.5 &   93   \\
                         9.2 &  17 & 25.6 &  160 & 42.0 &   96   \\
                         9.6 &  24 & 26.0 &  140 & 42.4 &   89   \\
                        10.0 &  21 & 26.4 &  137 & 42.8 &   86   \\
                        10.4 &  24 & 26.8 &  129 & 43.2 &   93   \\
                        10.8 &  33 & 27.2 &  123 & 43.6 &  108  \\
                        11.2 &  39 & 27.6 &  120 & 44.0 &  201  \\
                        11.6 &  41 & 28.0 &  113 & 44.4 & 3195 \\
                        12.0 &  50 & 28.4 &  116 & 44.8 & 3069 \\
                        12.4 &  58 & 28.8 &  111 & 45.2 &  547  \\
                        12.8 &  69 & 29.2 &  126 & 45.5 &  130  \\
                        13.2 &  67 & 29.6 &  111 & 46.0 &   81   \\
                        13.6 &  75 & 30.0 &  114 & 46.4 &   73   \\
                        14.0 &  78 & 30.4 &  113 & 46.8 &   65   \\
                        14.4 &  88 & 30.8 &  105 & 47.2 &   57   \\
                        14.8 &  91 & 31.2 &  100 & 47.6 &   57   \\
                        15.2 & 102 & 31.6 &  106 & 48.0 &   54   \\
                        15.6 & 111 & 32.0 &  107 & 48.4 &   59   \\
                        16.0 & 118 & 32.4 &  100 & 48.8 &   50   \\
                        16.4 & 134 & 32.8 &   98 & 49.2 &   45   \\
                        16.7 & 122 & 33.2 &  103 & 49.6 &   47   \\
                        17.2 & 134 & 33.6 &  102 & 50.0 &   49   \\
                        17.6 & 134 & 34.0 &   95 & 50.4 &   41   \\
                        18.0 & 137 & 34.4 &   91 & 50.8 &   46   \\
                        18.4 & 146 & 34.8 &   87 & 51.2 &   40   \\
                        18.8 & 140 & 35.2 &   80 & 51.6 &   43   \\
                        19.2 & 151 & 35.5 &   89 & 52.0 &   42   \\
                        19.6 & 161 & 36.0 &   81 &      &          \\
                        20.0 & 159 & 36.4 &   75 &      &          \\
                        20.4 & 162 & 36.8 &   75 &      &          \\
                        20.8 & 168 & 37.2 &   83 &      &          \\
                        21.2 & 164 & 37.5 &   77 &      &          \\
                        21.6 & 178 & 38.0 &   81 &      &          \\
                        22.0 & 183 & 38.4 &   84 &      &          \\
                        22.4 & 157 & 38.8 &   87 &      &          \\
                        22.8 & 168 & 39.2 &   85 &      &          \\
                        23.2 & 167 & 39.5 &  131 &      &          \\
                        23.6 & 173 & 40.0 & 1021 &      &          \\
                        24.0 & 164 & 40.4 &  675 &      &          \\
                \bottomrule
                \end{tabular}
        \end{center}
\end{table}

Diese Messwerte werden in Abbildung \ref{fig:emission} aufgetragen.

\begin{figure}
  \centering
  \includegraphics{build/emission.pdf}
  \caption{Auftragung der Messwerte des Emissionsspektrums der Kupfer-Röntgenröhre.}
  \label{fig:emission}
\end{figure}

Aus diesem Diagramm und der Tabelle können spezifische Elemente abgelesen werden.
Der Bremsberg beginnt bei $\theta = 4°$, er erreicht sein Maximum bei 11° und beginnt danach abzuflachen.
Da ab dem Anfangswinkel von 4° die Intensität bis zum Maximum des Bremsbergs konstant
ansteigt, wird dieser Winkel als Grenzwinkel in Gleichung \eqref{eqn:bragg} eingesetzt, um als experimentell ermittelte minimale
Wellenlänge des Bremsspektrums $\SI{28.098}{\pico\meter}$ zu erhalten. Nach Gleichung
\eqref{eqn:lambdamin} ist das Minimum der Wellenlängen des Bremsspektrums theoretisch
bei $\SI{35.424}{\pico\meter}$ zu erwarten. Es ergibt sich ein relativer Fehler von
$20{,}68\%$ zum Theoriewert. Durch Gleichung \eqref{eqn:winkel} ergibt sich
die experimentell bestimmte maximale Energie des Bremsspektrums zu $\SI{44.126}{\kilo\electronvolt}$.
Da die Röntgenröhre mit einer Spannung von $\SI{35}{\kilo\volt}$ betrieben wurde,
beträgt der theoretisch zu erwartende Wert $\SI{35}{\kilo\electronvolt}$. Die Abweichung
zu diesem Wert beträgt $26{,}07\%$. \\
Die $K_\beta$-Linie wird bei $20°$, die $K_\alpha$-Linie bei $22.3°$ identifiziert.
Gemäß Gleichung \eqref{eqn:winkel} und dem Zusammenhang $\lambda = hc/E$ ergibt
sich dann für die experimentell bestimmten Energien der Übergänge
\begin{align*}
  E_{\alpha,\text{exp}} &= \SI{8.1118}{\kilo\electronvolt} \,\\
  E_{\beta,\text{exp}} &= \SI{8.9996}{\kilo\electronvolt} \,.
\end{align*}
Die theoretisch zu erwartenden Werte für die Energien werden \cite{xraydata} entnommen \footnote{
Bei Entnahme der Theoriewerte wird die Feinstrukturaufspaltung nicht beachtet, da zur Vereinfachung
jeweils nur die \enquote{Kalpha1}- und \enquote{Kbeta1}-Werte übernommen wurden.
} und
betragen
\begin{align*}
  E_{\alpha,\text{theo}} &= \SI{8.0481}{\kilo\electronvolt} \,,\\
  E_{\beta,\text{theo}} &= \SI{8.9069}{\kilo\electronvolt} \,.
\end{align*}
Diese werden als nicht fehlerbehaftet betrachtet.
Die relative Abweichung des experimentell bestimmten Wertes der Energie
der $K_\alpha$-Linie beträgt $0{,}79\%$, bei der $K_\beta$-Linie sind es $1{,}04\%$.

\subsection{Untersuchung des Absorptionsspektrums von Brom}
\label{subsec:brom}

Nun wird das Absorptionsspektrum eines Bromabsorbers untersucht, der sich vor
dem Geiger-Müller Zählrohr befindet. Die dazugehörigen Messwerte sind in
Tabelle \ref{tab:brom} zu sehen.
Die Messwerte sind in Abbildung \ref{fig:brom} in ein Diagramm eingezeichnet.

\begin{table}[htp]
        \begin{center}
          \caption{Messwerte der Intensität in Abhängigkeit des doppelten Kristallwinkels zur Untersuchung des Absorptionsspektrums von Brom.}
          \label{tab:brom}
                \begin{tabular}{S[table-format=2.1] S[table-format=2.0] S[table-format=2.1] S[table-format=2.0]}
                \toprule
                        {$2 \cdot \theta/$°} & {$I$} & {$2 \cdot \theta/$°} & {$I$} \\
                        \midrule
                        22.0 & 4 & 26.2 &  7\\
                        22.2 & 4 & 26.4 &  9\\
                        22.4 & 5 & 26.6 &  9\\
                        22.6 & 5 & 26.8 &  8\\
                        22.8 & 6 & 27.0 &  8\\
                        23.0 & 5 & 27.2 &  8\\
                        23.2 & 4 & 27.4 &  8\\
                        23.4 & 5 & 27.6 &  8\\
                        23.6 & 6 & 27.8 &  6\\
                        23.8 & 5 & 28.0 &  8\\
                        24.0 & 5 & 28.2 &  6\\
                        24.2 & 5 & 28.4 &  6\\
                        24.4 & 5 & 28.6 &  6\\
                        24.6 & 4 & 28.8 &  5\\
                        24.8 & 5 & 29.0 &  5\\
                        25.0 & 5 & 29.2 &  5\\
                        25.2 & 5 & 29.4 &  5\\
                        25.4 & 4 & 29.6 &  5\\
                        25.6 & 4 & 29.8 &  4\\
                        25.8 & 5 & 30.0 &  5\\
                        26.0 & 6 &      &  \\
                        \bottomrule
                \end{tabular}
        \end{center}
\end{table}

\begin{figure}
  \centering
  \includegraphics{build/brom.pdf}
  \caption{Auftragung der Messwerte des Absorptionsspektrums von Brom.}
  \label{fig:brom}
\end{figure}

Die K-Kante befindet sich bei dem Maximum der Intensität, das bei ungefähr
$\theta_{\text{K,exp}}^{\text{Br}} = 13{,}25°$ abgelesen wird. Gemäß Gleichung \eqref{eqn:winkel} ergibt sich die dazugehörige Energie
zu $E_{\text{K,exp}}^{\text{Br}} = \SI{13.43}{\kilo\electronvolt}$.
Stets können die Theoriewerte aus Tabelle \ref{tab:theoriewerte} entnommen werden,
sodass deren Werte in diesem Kapitel nicht wiederholt werden. Die relative Abweichung zum Theoriewert
beträgt $-0{,}11\%$. Durch Einsetzen in Gleichung \eqref{eqn:bindungsenergie}
folgt für die Abschirmkonstante
\begin{equation*}
  \sigma_{\text{K,exp}}^{\text{Br}} =  35 - \sqrt{\frac{E_{\text{K,exp}}^{\text{Br}}}{R_\infty}} = 3{,}58\,.
\end{equation*}
Dabei wurde für die eigentlich negative Bindungsenergie ein positives Vorzeichen angesetzt.
Die relative Abweichung zum theoretischen Wert beträgt $1{,}58\%$.

\subsection{Untersuchung des Absorptionsspektrums von Strontium}
\label{subsec:strontium}

Das Absorptionsspektrum von Stromtium wird anschließend untersucht.
Die Messwerte befinden sich in Tabelle \ref{tab:strontium}.
In Abbildung \ref{fig:strontium} ist die Messreihe aufgetragen.

\begin{table}[htp]
        \begin{center}
          \caption{Messwerte der Intensität in Abhängigkeit des doppelten Kristallwinkels zur Untersuchung des Absorptionsspektrums von Strontium.}
          \label{tab:strontium}
                \begin{tabular}{S[table-format=2.1] S[table-format=2.0] S[table-format=2.1] S[table-format=2.0]}
                \toprule
                        {$2 \cdot \theta/$°} & {$I$} & {$2 \cdot \theta/$°} & {$I$} \\
                        \midrule
                        16.0 & 35 & 21.2 &  26\\
                        16.2 & 32 & 21.4 &  27\\
                        16.4 & 33 & 21.6 &  39\\
                        16.7 & 33 & 21.8 &  59\\
                        16.8 & 33 & 22.0 &  68\\
                        17.0 & 32 & 22.2 &  82\\
                        17.2 & 33 & 22.4 &  80\\
                        17.4 & 31 & 22.6 &  82\\
                        17.6 & 30 & 22.8 &  82\\
                        17.7 & 33 & 23.0 &  77\\
                        18.0 & 30 & 23.2 &  77\\
                        18.2 & 30 & 23.4 &  74\\
                        18.4 & 28 & 23.6 &  75\\
                        18.6 & 28 & 23.8 &  71\\
                        18.8 & 28 & 24.0 &  73\\
                        19.0 & 32 & 24.2 &  70\\
                        19.2 & 28 & 24.4 &  70\\
                        19.4 & 26 & 24.6 &  67\\
                        19.6 & 26 & 24.8 &  65\\
                        19.8 & 28 & 25.0 &  63\\
                        20.0 & 26 & 25.2 &  64\\
                        20.2 & 25 & 25.4 &  59\\
                        20.4 & 24 & 25.6 &  58\\
                        20.6 & 26 & 25.8 &  57\\
                        20.8 & 24 & 26.0 &  52\\
                        21.0 & 25 &      &  \\
                        \bottomrule
                \end{tabular}
        \end{center}
\end{table}

\begin{figure}
  \centering
  \includegraphics{build/strontium.pdf}
  \caption{Auftragung der Messwerte des Absorptionsspektrums von Strontium.}
  \label{fig:strontium}
\end{figure}

Das Maximum der Intensität liegt bei circa $\theta_{\text{K,exp}}^{\text{Sr}} = 11{,}2°$.
Gemäß Gleichung \eqref{eqn:winkel} folgt für die Energie der K-Kante $E_{\text{K,exp}}^{\text{Sr}} = \SI{15.85}{\kilo\electronvolt}$.
Der relative Fehler zum Theoriewert ist $-1{,}67\%$. Mit Gleichung \eqref{eqn:bindungsenergie}
beträgt die Abschirmkonstante
\begin{equation*}
  \sigma_{\text{K,exp}}^{\text{Sr}} =  38 - \sqrt{\frac{E_{\text{K,exp}}^{\text{Sr}}}{R_\infty}} = 3{,}86\,.
\end{equation*}
Die relative Abweichung zum Theoriewert ist $8{,}25\%$.

\subsection{Untersuchung des Absorptionsspektrums von Zink}
\label{subsec:zink}

Jetzt wird das Absorptionsspektrum eines Zinkabsorbers untersucht.
Die Messreihe ist in Tabelle \ref{tab:zink} zu sehen. Es ist zu erkennen, dass
der Messbereich so weit gewählt wurde, dass auch noch die charakteristische
Strahlung aufgenommen wurde. Da diese für das Verhalten von Zink als Absorber nicht
relevant ist, wird der Bereich ab $2 \cdot \theta = 39.4°$ bis zum Ende des Messbereichs
nicht für die grafische Darstellung in Abbildung \ref{fig:zink} berücksichtigt.

\begin{table}[htp]
        \begin{center}
          \caption{Messwerte der Intensität in Abhängigkeit des doppelten Kristallwinkels zur Untersuchung des Absorptionsspektrums von Zink.}
          \label{tab:zink}
                \begin{tabular}{S[table-format=2.1] S[table-format=2.0] S[table-format=2.1] S[table-format=2.0]}
                \toprule
                        {$2 \cdot \theta/$°} & {$I$} & {$2 \cdot \theta/$°} & {$I$} \\
                        \midrule
                        32.0 & 40 & 37.2 &  52\\
                        32.2 & 37 & 37.4 &  56\\
                        32.4 & 37 & 37.5 &  58\\
                        32.5 & 41 & 37.8 &  57\\
                        32.8 & 39 & 38.0 &  53\\
                        33.0 & 36 & 38.2 &  53\\
                        33.2 & 33 & 38.4 &  53\\
                        33.4 & 36 & 38.6 &  56\\
                        33.6 & 36 & 38.8 &  54\\
                        33.8 & 35 & 39.0 &  55\\
                        34.0 & 35 & 39.2 &  52\\
                        34.2 & 34 & 39.4 &  67\\
                        34.4 & 35 & 39.5 &  92\\
                        34.5 & 36 & 39.8 &  496\\
                        34.8 & 34 & 40.0 &  692\\
                        35.0 & 34 & 40.2 &  654\\
                        35.2 & 32 & 40.4 &  477\\
                        35.4 & 31 & 40.5 &  281\\
                        35.5 & 32 & 40.8 &  93\\
                        35.8 & 32 & 41.0 &  79\\
                        36.0 & 30 & 41.2 &  71\\
                        36.2 & 30 & 41.4 &  65\\
                        36.4 & 31 &      &  \\
                        36.6 & 32 &      &  \\
                        36.8 & 41 &      &  \\
                        37.0 & 49 &      &  \\
                        \bottomrule
                \end{tabular}
        \end{center}
\end{table}

\begin{figure}
  \centering
  \includegraphics{build/zink.pdf}
  \caption{Auftragung der Messwerte des Absorptionsspektrums von Zink.}
  \label{fig:zink}
\end{figure}

Die K-Kante liegt erneut bei dem Maximum der Intensität, welches bei ungefähr
$\theta_{\text{K,exp}}^{\text{Zn}} = 18{,}75°$ abgelesen wird. Mit Gleichung \eqref{eqn:winkel} ergibt sich die dazugehörige Energie
zu $E_{\text{K,exp}}^{\text{Zn}} = \SI{9.58}{\kilo\electronvolt}$.
Die relative Abweichung zum Theoriewert beträgt $-0{,}11\%$. Durch Einsetzen in Gleichung \eqref{eqn:bindungsenergie}
ist die Abschirmkonstante dann
\begin{equation*}
  \sigma_{\text{K,exp}}^{\text{Zn}} =  30 - \sqrt{\frac{E_{\text{K,exp}}^{\text{Zn}}}{R_\infty}} = 3{,}46\,.
\end{equation*}
Die relative Abweichung zum theoretischen Wert beträgt $3{,}90\%$.

\subsection{Untersuchung des Absorptionsspektrums von Zirconium}
\label{subsec:zirconium}

Das Absorptionsspektrum von Zirconium wird zuletzt untersucht.
Die Messwerte befinden sich in Tabelle \ref{tab:zirconium}.
In Abbildung \ref{fig:zirconium} sind die Messwerte aufgetragen.

\begin{table}[htp]
        \begin{center}
          \caption{Messwerte der Intensität in Abhängigkeit des doppelten Kristallwinkels zur Untersuchung des Absorptionsspektrums von Zirconium.}
          \label{tab:zirconium}
                \begin{tabular}{S[table-format=2.1] S[table-format=2.0] S[table-format=2.1] S[table-format=2.0]}
                \toprule
                        {$2 \cdot \theta/$°} & {$I$} & {$2 \cdot \theta/$°} & {$I$} \\
                        \midrule
                        16.0 &  51 & 20.2 &  113\\
                        16.2 &  54 & 20.4 &  113\\
                        16.4 &  58 & 20.6 &  117\\
                        16.7 &  57 & 20.8 &  120\\
                        16.8 &  59 & 21.0 &  116\\
                        17.0 &  58 & 21.2 &  114\\
                        17.2 &  59 & 21.4 &  116\\
                        17.4 &  60 & 21.6 &  118\\
                        17.6 &  61 & 21.8 &  116\\
                        17.7 &  58 & 22.0 &  118\\
                        18.0 &  57 & 22.2 &  110\\
                        18.2 &  55 & 22.4 &  113\\
                        18.4 &  55 & 22.6 &  111\\
                        18.6 &  54 & 22.8 &  109\\
                        18.8 &  56 & 23.0 &  104\\
                        19.0 &  57 & 23.2 &  108\\
                        19.2 &  65 & 23.4 &  106\\
                        19.4 &  76 & 23.6 &  109\\
                        19.6 &  92 & 23.8 &  103\\
                        19.8 & 105 & 24.0 &  101\\
                        20.0 & 111 &      &  \\
                        \bottomrule
                \end{tabular}
        \end{center}
\end{table}

\begin{figure}
  \centering
  \includegraphics{build/zirkonium.pdf}
  \caption{Auftragung der Messwerte des Absorptionsspektrums von Zirconium.}
  \label{fig:zirconium}
\end{figure}

Das Maximum der Intensität liegt bei circa $\theta_{\text{K,exp}}^{\text{Zr}} = 10{,}4°$.
Gemäß Gleichung \eqref{eqn:winkel} folgt für die Energie der K-Kante $E_{\text{K,exp}}^{\text{Zr}} = \SI{17.05}{\kilo\electronvolt}$.
Der relative Fehler zum Theoriewert ist $-5{,}32\%$. Mit Gleichung \eqref{eqn:bindungsenergie}
beträgt die Abschirmkonstante
\begin{equation*}
  \sigma_{\text{K,exp}}^{\text{Zr}} =  40 - \sqrt{\frac{E_{\text{K,exp}}^{\text{Zr}}}{R_\infty}} = 4{,}59\,.
\end{equation*}
Die relative Abweichung zum Theoriewert ist $27{,}19\%$.

\subsection{Experimentelle Bestimmung der Rydbergenergie}

In Tabellle \ref{tab:zusammenfassung} sind die die experimentell bestimmten Absorptionsenergien
und die Ordnungszahlen der untersuchten Elemente zusammengefasst.

\begin{table}
	\begin{center}
    \caption{Prdnungszahlen und experimentell bestimmte Werte für die Absorptionsenergien
    für die K-Kanten der untersuchten Elemente.}
    \label{tab:zusammenfassung}
		\begin{tabular}{ccccc}
		\toprule
			& {$Z$} & {$E_K/$KeV} \\
			\midrule
			Zn &  30  & 9,58  \\
      Br &  35  & 13,43 \\
      Sr &  38  & 15,85 \\
      Zr &  40  & 17,05 \\
		\bottomrule
		\end{tabular}
	\end{center}
\end{table}

Gemäß dem Moseleyschen Gesetz ist die Wurzel aus der Absorptionsenergie proportional
zur Ordnungszahl des Elements. Dies kann durch
\begin{equation*}
  \sqrt{E_K} = R_\infty Z
\end{equation*}
ausgedrückt werden, wobei die Proportionalitätskonstante gerade die zu bestimmende
Rydbergenergie $R_\infty = \SI{13.6}{\electronvolt}$ ist.\\
Es wird eine lineare Ausgleichsrechnung mit den vorliegenden Werten durchgeführt.
Dabei soll $a$ die Steigung und $b$ den Ordinatenabschnitt der Ausgleichsgerade  Die Messwerte
und der Graph der Ausgleichsfunktion sind in \ref{fig:rydberg} dargestellt.

\begin{figure}
  \centering
  \includegraphics{build/rydberg.pdf}
  \caption{Auftragung der Messwerte und Graph der Ausgleichsfunktion zur Bestimmung der Rydbergenergie.}
  \label{fig:rydberg}
\end{figure}

Die Parameter der Ausgleichsfunktion ergeben sich zu
\begin{align*}
  a &= \SI{0.105(5)}{\sqrt{\kilo\eV}} \,,\\
  b &= \SI{-0.04(18)}{\sqrt{\kilo\eV}}\,.
\end{align*}
Daraus ergibt sich der experimentell bestimmte Wert für die Rydbergenergie zu
$a^2 = \SI{11.0(11)}{\eV}$. Die relative Abweichung zum Theoriewert beträgt $-19{,}12\%$.

\subsection{Untersuchung des Absorptionsspektrums von Gold}
\label{subsec:gold}

Jetzt wird das Absorptionsspektrum eines Zinkabsorbers untersucht.
Die Messreihe ist in Tabelle \ref{tab:gold} zu sehen. Die grafische Darstellung der Messwerte
findet sich in Abbildung \ref{fig:gold}.

\begin{table}[htp]
        \begin{center}
          \caption{Messwerte der Intensität in Abhängigkeit des doppelten Kristallwinkels zur Untersuchung des Absorptionsspektrums von Gold.}
          \label{tab:gold}
                \begin{tabular}{S[table-format=2.1] S[table-format=2.0] S[table-format=2.1] S[table-format=2.0]}
                \toprule
                        {$2 \cdot \theta/$°} & {$I$} & {$2 \cdot \theta/$°} & {$I$} \\
                        \midrule
                        22.0 & 104 & 27.2 & 80\\
                        22.2 & 103 & 27.4 & 75\\
                        22.4 & 101 & 27.5 & 78\\
                        22.6 & 103 & 27.8 & 71\\
                        22.8 & 101 & 28.0 & 72\\
                        23.0 & 100 & 28.2 & 72\\
                        23.2 & 101 & 28.4 & 76\\
                        23.4 &  99 & 28.6 & 72\\
                        23.6 &  95 & 28.8 & 67\\
                        23.8 &  99 & 29.0 & 69\\
                        24.0 &  97 & 29.2 & 71\\
                        24.2 & 100 & 29.4 & 75\\
                        24.4 &  99 & 29.6 & 75\\
                        24.6 &  97 & 29.8 & 76\\
                        24.8 &  97 & 30.0 & 84\\
                        25.0 &  98 & 30.2 & 86\\
                        25.2 & 102 & 30.4 & 81\\
                        25.4 &  98 & 30.6 & 84\\
                        25.6 & 101 & 30.8 & 82\\
                        25.8 & 103 & 31.0 & 81\\
                        26.0 & 100 & 31.2 & 76\\
                        26.2 &  88 & 31.4 & 78\\
                        26.4 &  85 & 31.6 & 72\\
                        26.6 &  89 & 31.8 & 74\\
                        26.8 &  80 & 32.0 & 71\\
                        27.0 &  79 &      &   \\
                        \bottomrule
                \end{tabular}
        \end{center}
\end{table}

\begin{figure}
  \centering
  \includegraphics{build/gold.pdf}
  \caption{Auftragung der Messwerte des Absorptionsspektrums von Gold.}
  \label{fig:gold}
\end{figure}
