\section{Durchführung}
\label{sec:Durchführung}

Es werden eine Lampe, ein Objekt, eine Linse und ein Schirm auf
einer Schiene befestigt und so justiert, dass der Strahlengang gerade durch die Anordnung
gehen kann.

Bei der ersten Messreihe wird eine Sammellinse mit bekannter Brennweite verwendet.
Linse und Schirm werden auf der Schiene so verschoben, dass auf dem Schirm ein
scharfes Bild zu sehen ist. Dann werden die Gegenstandsweite, die Bildweite und
die Bildgröße gemessen. Analog hierzu wird bei der zweiten Messreihe verfahren.
Bei dieser wird eine zuvor mit Wasser befüllte Linse verwendet, deren Brennweite
unbekannt ist. Die Messung der Bildgröße entfällt bei dieser Messreihe.

Daraufhin wird nach der Methode von Bessel gemessen. Wie bereits
in Kapitel \ref{subsec:Methoden} beschrieben, werden für verschiedene Abstände zwischen
Objekt und Schirm jeweils zwei Linsenpositionen gesucht, bei denen ein scharfes Bild
auf dem Schirm zu erkennen ist. Die jeweiligen Bild- und Gegenstandsweiten werden
gemessen und notiert.
Danach wird die Messung ein mal für rotes und ein mal für blaues Licht wiederholt.
Dafür wird jeweils ein passender Filter in den Strahlengang vor dem Objekt gebracht.

Bei der Messung mit der Methode nach Abbe wird, wie ebenfalls in Kapitel \ref{subsec:Methoden}
bereits erwähnt, ein Linsensystem aus einer Sammellinse und einer Zerstreuungslinse
gebildet. Dabei werden Linsen mit betragsmäßig gleicher Brennweite verwendet.
Erneut werden hier Schirm und Linse so verschoben, dass ein scharfes Bild auf
dem Schirm entsteht. Die Bild- und Gegenstandsweiten werden bezüglich der Position
der schirmnahen Linse gemessen. Während der gesamten Messreihe ist darauf zu achten,
dass der Abstand zwischen den beiden Linsen konstant bleibt.
