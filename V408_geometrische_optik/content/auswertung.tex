\section{Auswertung}
\label{sec:Auswertung}

\subsection{Messung der Brennweite durch Gegenstands- und Bildweite}
Die erste Methode besteht darin, die Brennweite direkt durch die Linsengleichung
zu bestimmen. Die gemessenen Werte für die Gegenstandsweite $g$, die Bildweite
$b$ und die Bildgröße $B$  bei der Messung mit der Linse mit einer bekannten
Brennweite von $f_1 = \SI{150}{\milli\meter}$ sind dabei in Tabelle
\ref{tab:bekannt} zu finden. Eingetragen ist darüber hinaus auch bereits der
Abbildungsmaßstab, der konkret gemäß \eqref{eqn:V} mit $V_1=B/G$ bzw. mit $V_2=b/g$
mit bekannter, fester Gegenstandsgröße $G = \SI{3}{\centi\meter}$ berechnet wurde.


\begin{table}[htp]
	\begin{center}
    \caption{Messwerte zur Linse einer bekannten Brennweite und daraus berechnete Werte.}
    \label{tab:bekannt}
		\begin{tabular}{cccccc}
		\toprule
			{$g/$cm} & {$b/$cm} & {$B/$cm} & {$V1/$cm} & {$V2/$cm} & {$f/$cm}\\
			\midrule
			44,80 & 21,10 & 1,40 & 0,47 & 0,47 & 14,34\\
			33,00 & 25,40 & 2,20 & 0,73 & 0,77 & 14,35\\
			33,10 & 25,40 & 2,20 & 0,73 & 0,77 & 14,37\\
			47,80 & 20,50 & 1,30 & 0,43 & 0,43 & 14,35\\
			26,80 & 31,00 & 3,30 & 1,10 & 1,16 & 14,37\\
			56,00 & 11,40 & 1,00 & 0,33 & 0,20 & 9,47\\
			64,30 & 18,60 & 0,80 & 0,27 & 0,29 & 14,43\\
			19,40 & 35,60 & 8,10 & 2,70 & 1,84 & 12,56\\
			24,40 & 34,80 & 4,10 & 1,37 & 1,43 & 14,34\\
			74,00 & 17,40 & 0,70 & 0,23 & 0,24 & 14,09\\
			88,00 & 17,00 & 0,60 & 0,20 & 0,19 & 14,25\\
		\bottomrule
		\end{tabular}
	\end{center}
\end{table}


Eine grafische Auswertung ist möglich. Dazu werden die gemessenen $g_i$ und
$b_i$ auf die Abszissen- bzw. Ordinatenachse aufgetragen und Messwertpaare verbunden.
Dies ist in Abbildung \ref{fig:bekanntgross} geschehen.

\begin{figure}%[h]
  \centering
  \includegraphics{build/bekanntgross.pdf}
  \caption{Linse mit $f_1$, Auftragung der Bildweiten auf der Ordinaten- und der Gegenstandsweiten auf der Abszissenachse mit Verbindungslinien.}
  \label{fig:bekanntgross}
\end{figure}

Die Geraden sollten sich im Rahmen der Messgenauigkeit näherungsweise in einem Punkt
schneiden, dessen beide Koordinaten die Brennweite $f$ sein sollten. Nimmt man die
beiden als Ausreißer zu erkennenden Messwertpaare heraus, ist dies
erfüllt. Der relevante Bereich ist in Abbildung \ref{fig:bekanntklein} dargestellt.

\begin{figure}%[h]
  \centering
  \includegraphics{build/bekanntklein.pdf}
  \caption{Linse mit $f_1$, Ausschnitt um den Schnittpunkt aus Abbildung \ref{fig:bekanntklein}.}
  \label{fig:bekanntklein}
\end{figure}

Aus dieser vergrößerten Darstellung lässt sich näherungsweise $\SI{14.4}{\centi\meter}$ als grafisch
bestimmte Brennweite ablesen.

Die auch in der Tabelle \ref{tab:bekannt} eingetragenen Werte für die Brennweite $f$ wurden gemäß der
Linsengleichung \eqref{eqn:linsengleichung} berechnet. Wird aus diesen Werten
der Mittelwert nach Gleichung \eqref{eqn:mean} und die empirische Standardabweichung
nach Gleichung \eqref{eqn:std} bestimmt, so ergibt sich ein Wert von $\SI{14.32(0010)}{\centi\meter}$
für die experimentell durch Mittelwertbildung bestimmte Brennweite. Dabei ist zu
beachten, dass die oben Ausreißer identifizierten Messwertpaare nicht für diese Rechnung
berücksichtigt werden. Die beiden Werte, die so für den Brennwert ermittelt wurden,
können zu
\begin{equation*}
  f_{\text{1,exp}} = \SI{14.36(0010)}{\centi\meter}
\end{equation*}
gemittelt werden. Dann beträgt die relative Abweichung zur Herstellerangabe von
$f_1 = \SI{150}{\milli\meter}$ -4,27\%.

Auch die Brennweite $f_2$ der Wasserlinse soll auf diese Art und Weise bestimmt werden.
Die Messwerte sind in \ref{tab:unbekannt} zu finden, auch die aus der Linsengleichung \eqref{eqn:linsengleichung}
bestimmten Brennweiten sind bereits aufgeführt.
Die gemessenen Gegenstands- und Bildweiten werden in \ref{fig:wassergross} auf die Achsen aufgetragen
und der relevante Bereich in \ref{fig:wasserklein} vergrößert dargestellt.

\begin{table}[htp]
	\begin{center}
    \caption{Messwerte zur Wasserlinse mit unbekannter Brennweite und daraus berechnet Werte.}
    \label{tab:unbekannt}
		\begin{tabular}{ccc}
		\toprule
			{$g/$cm} & {$b/$cm} & {$f/$cm}\\
			\midrule
			24,20 & 25,90 & 12,51\\
			32,40 & 20,80 & 12,67\\
			41,60 & 17,80 & 12,47\\
			19,00 & 40,40 & 12,92\\
			16,90 & 46,50 & 12,40\\
			62,30 & 15,50 & 12,41\\
			70,60 & 15,30 & 12,57\\
			50,40 & 16,70 & 12,54\\
			25,30 & 24,90 & 12,55\\
			53,70 & 16,40 & 12,56\\
		\bottomrule
		\end{tabular}
	\end{center}
\end{table}

\begin{figure}%[h]
  \centering
  \includegraphics{build/wassergross.pdf}
  \caption{Wasserlinse, Auftragung der Bildweiten auf der Ordinaten- und der Gegenstandsweiten auf der Abszissenachse mit Verbindungslinien.}
  \label{fig:wassergross}
\end{figure}

\begin{figure}%[h]
  \centering
  \includegraphics{build/wasserklein.pdf}
  \caption{Wasserlinse, Ausschnitt um den Schnittpunkt aus Abbildung \ref{fig:wasserklein}.}
  \label{fig:wasserklein}
\end{figure}

Aufgrund des näherungsweise guten Schnitts aller Geraden in einem Punkt werden alle
Messwertpaare verwendet. Dann ergibt sich $\SI{12.7}{\centi\meter}$ als grafisch bestimmte
Brennweite für die Wasserlinse. Durch Mittelwertbildung der in Tabelle \ref{tab:unbekannt}
aufgeführten Brennweiten ergibt sich eine Brennweite von $\SI{12.56(15)}{\centi\meter}$.
Die Mittelung dieser Größen ergibt durch Bestimmung wie zuvor $\SI{12.63(15)}{\centi\meter}$.

\subsection{Bestimmung der Brennweite durch die Methode von Bessel}

Auch für diese Methode wird die Linse mit der bekannten Brennweite $f_1 = \SI{150}{\milli\meter}$
verwendet. Die Messwerte der Gesamtlänge $e$, die der beiden Gegenstandsweiten $g_1$ und $g_2$
und die der beiden Bildweiten $b_1$ sind in \ref{tab:bessel} dargestellt. Auch sind
bereits die nach Gleichung \eqref{eqn:bessel} berechneten Werte für die Brennweite
der Linse eingetragen. Für die benötigte Größe $d$ werden beide Berechnungsmöglichkeiten
aus Gleichung \eqref{eqn:d} verwendet und das arithmetische Mittel genommen. Es wird hier keine
Fehlerrechnung durchgeführt, da dies mit zwei Werten zu keinen sinnvollen Ergebnissen
führt. Es lässt sich dann die Brennweite der Linse unter weißem Licht durch
übliche Mittelwertbildung zu $\SI{144.1(8)}{\milli\meter}$ bestimmen.

\begin{table}[htp]
	\begin{center}
    \caption{Messwerte zur Methode von Bessel mit weißem Licht und daraus berechnete Werte.}
    \label{tab:bessel}
		\begin{tabular}{cccccc}
		\toprule
			{$e/$cm} & {$g_1/$cm} & {$b_1/$cm} & {$g_2/$cm} & {$b_2/$cm} & {$f/$cm}\\
			\midrule
			60,00 & 23,50 & 36,50 & 35,90 & 24,10 & 14,36\\
			65,00 & 20,50 & 44,50 & 43,50 & 21,50 & 14,22\\
			70,00 & 20,20 & 49,80 & 49,80 & 20,20 & 14,37\\
			75,00 & 19,70 & 55,30 & 55,50 & 19,50 & 14,48\\
			80,00 & 19,00 & 61,00 & 61,10 & 18,90 & 14,46\\
			85,00 & 18,40 & 66,60 & 66,50 & 18,50 & 14,45\\
			90,00 & 18,10 & 71,90 & 71,80 & 18,20 & 14,49\\
			95,00 & 17,80 & 77,20 & 77,30 & 17,70 & 14,43\\
			100,00 & 17,50 & 82,50 & 82,50 & 17,50 & 14,44\\
			105,00 & 17,30 & 87,70 & 87,70 & 17,30 & 14,45\\
		\bottomrule
		\end{tabular}
	\end{center}
\end{table}

\newpage
Die Messreihen und genau wie zuvor berechneten Brennweiten für das blaue und das
rote Licht finden sich in den Tabellen \ref{tab:besselblau} und \ref{tab:besselrot}.
Werden die Brennweiten wie zuvor gemittelt,
ergibt sich eine Brennweite von $\SI{144.15(28)}{\milli\meter}$ für das blaue Licht.
Für das rote Licht folgt eine Brennweite von $\SI{144.5(4)}{\milli\meter}$.

\begin{table}[htp]
	\begin{center}
    \caption{Messwerte zur Methode von Bessel mit blauem Licht und daraus berechnete Werte.}
    \label{tab:besselblau}
		\begin{tabular}{cccccc}
		\toprule
			{$e/$cm} & {$g_1/$cm} & {$b_1/$cm} & {$g_2/$cm} & {$b_2/$cm} & {$f/$cm}\\
			\midrule
			60,00 & 23,90 & 36,10 & 36,20 & 23,80 & 14,37\\
			70,00 & 20,30 & 49,70 & 49,60 & 20,40 & 14,43\\
			80,00 & 18,90 & 61,10 & 61,10 & 18,90 & 14,43\\
			90,00 & 18,00 & 72,00 & 71,90 & 18,10 & 14,43\\
		  100,00 & 17,50 & 82,50 & 82,60 & 17,40 & 14,40\\
		\bottomrule
		\end{tabular}
	\end{center}
\end{table}

\begin{table}[htp]
	\begin{center}
    \caption{Messwerte zur Methode von Bessel mit rotem Licht und daraus berechnete Werte.}
    \label{tab:besselrot}
		\begin{tabular}{cccccc}
		\toprule
			{$e/$cm} & {$g_1/$cm} & {$b_1/$cm} & {$g_2/$cm} & {$b_2/$cm} & {$f/$cm}\\
			\midrule
			60,00 & 23,90 & 36,10 & 35,90 & 24,10 & 14,40\\
			70,00 & 20,40 & 49,60 & 49,50 & 20,50 & 14,48\\
			80,00 & 19,00 & 61,00 & 61,10 & 18,90 & 14,46\\
			90,00 & 18,00 & 72,00 & 71,90 & 18,10 & 14,43\\
			100,00 & 17,60 & 82,40 & 82,40 & 17,60 & 14,50\\
		\bottomrule
		\end{tabular}
	\end{center}
\end{table}

%Bereits in Kapitel \ref{sec:theorie} wurde die chromatische Abberation als Abbildungsfehler
%erläutert. Dies
 \newpage
\subsection{Bestimmung der Brennweite durch die Methode von Abbe}
Die Messreihen bei der Methode von Abbe wurden mit einer Sammel- und einer
Zerstreuungslinse mit den Brennweiten $f_{\pm} = \pm \SI{50}{\mm}$ in
einem Abstand von $d=\SI{6}{\centi\meter}$ aufgenommen.
Die Messwerte sind in \ref{tab:abbe} eingetragen. Für den Abbildungsmaßstab gilt
hier gilt $V=\frac{B}{G}$.

\begin{table}[htp]
	\begin{center}
    \caption{Messwerte zur Methode von Abbe und daraus berechnete Werte.}
    \label{tab:abbe}
		\begin{tabular}{cccc}
		\toprule
			{$g'$cm} & {$b'/$cm} & {$B/$cm} & {$V$}\\
			\midrule
			12,00 & 12,40 & 1,80 & 0,60\\
			12,50 & 12,30 & 1,70 & 0,57\\
			15,00 & 11,80 & 1,30 & 0,43\\
			17,50 & 11,50 & 1,05 & 0,35\\
			20,00 & 11,20 & 0,90 & 0,30\\
			22,50 & 11,10 & 0,75 & 0,25\\
			25,00 & 10,80 & 0,65 & 0,22\\
			27,50 & 10,70 & 0,60 & 0,20\\
			30,00 & 10,20 & 0,55 & 0,18\\
			32,50 & 10,40 & 0,45 & 0,15\\
		\bottomrule
		\end{tabular}
	\end{center}
\end{table}

Die gemessenen Gegenstands- und Bildweiten folgen den linearen Zusammenhängen
(6) und (7). Also werden sie gegen $1+\frac{1}{V}$ bzw. $1+V$ in einem Diagramm
aufgetragen und einer Ausgleichsrechnung unterzogen. Die Auftragung der Messwerte
und die Graphen der Ausgleichsfunktionen sind dabei in Abbildung \ref{fig:gstrich}
und \ref{fig:bstrich} zu sehen.

\begin{figure}%[h]
  \centering
  \includegraphics{build/abbegstrich.pdf}
  \caption{Auftragung der gemssenen Gegenstandsweite gegen $1+1/V$ und Graph der Ausgleichsfunktion}
  \label{fig:gstrich}
\end{figure}

\begin{figure}%[h]
  \centering
  \includegraphics{build/abbebstrich.pdf}
  \caption{Auftragung der gemssenen Bildweite gegen $1+V$ und Graph der Ausgleichsfunktion}
  \label{fig:bstrich}
\end{figure}

Die Parameter der Ausgleichsfunktion ergeben sich dann zu
\begin{align*}
	f_g &= \SI{4.32(15)}{\centi\meter}		&		h &= \SI{0.8(8)}{\centi\meter} \,, \\
	f_b &= \SI{4.6(4)}{\centi\meter}		&		h^\prime &= \SI{5.1(5)}{\centi\meter}\,,
\end{align*}
wobei der Index die Variable kennzeichnet, für die die Ausgleichsrechnung
angesetzt wurde. Das Mittel der beiden Brennwerte ergibt sich nach üblicher
Bestimmung zu $\SI{4.47(20)}{\centi\meter}$.

Der Theoriewert für die Brennweite des Linsensystems ergibt sich nach Gleichung
\eqref{eqn:system} zu $f_{\symup{theo}}=\SI{4.17}{\centi\meter}$. Die Abweichung
der aus der Ausgleichsrechnung gewonnenen Parameter vom Theoriewert ergibt sich
damit zu $3,60 \%$ für $f_g$ und $10,31 \%$ für $f_b$.
