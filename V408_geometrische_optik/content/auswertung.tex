\section{Auswertung}
\label{sec:Auswertung}

\subsection{Messung der Brennweite durch Gegenstands- und Bildweite}
Die erste Methode besteht darin, die Brennweite direkt durch die Linsengleichung
zu bestimmen. Die gemessenen Werte für die Gegenstandsweite $g$, die Bildweite
$b$ und die Bildgröße $B$  bei der Messung mit der Linse mit einer bekannten
Brennweite von $f_1 = \SI{150}{\milli\meter}$ sind dabei in Tabelle
\ref{tab:bekannt} zu finden. Eingetragen ist darüber hinaus auch bereits der
Abbildungsmaßstab, der konkret gemäß \eqref{eqn:V} mit $V_1=B/G$ bzw. mit $V_2=b/g$
mit bekannter, fester Gegenstandsgröße $G = \SI{3}{\centi\meter}$ berechnet wurde.
