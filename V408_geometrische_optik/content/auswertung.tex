\section{Auswertung}
\label{sec:Auswertung}

\subsection{Messung der Brennweite durch Gegenstands- und Bildweite}
Die erste Methode besteht darin, die Brennweite direkt durch die Linsengleichung
zu bestimmen. Die gemessenen Werte für die Gegenstandsweite $g$, die Bildweite
$b$ und die Bildgröße $B$  bei der Messung mit der Linse mit einer bekannten
Brennweite von $f_1 = \SI{150}{\milli\meter}$ sind dabei in Tabelle
\ref{tab:bekannt} zu finden. Eingetragen ist darüber hinaus auch bereits der
Abbildungsmaßstab, der konkret gemäß \eqref{eqn:V} mit $V_1=B/G$ bzw. mit $V_2=b/g$
mit bekannter, fester Gegenstandsgröße $G = \SI{3}{\centi\meter}$ berechnet wurde.


Eine grafische Auswertung ist möglich. Dazu werden die gemessenen $g_i$ und
$b_i$ auf die Abszissen- bzw. Ordinatenachse aufgetragen und Messwertpaare verbunden.
Dies ist in Abbildung \ref{fig:bekanntgross} geschehen.

\begin{figure}
  \centering
  \includegraphics{build/bekanntgross.pdf}
  \caption{Linse mit $f_1$, Auftragung der Bildweiten auf der Ordinaten- und der Gegenstandsweiten auf der Abszissenachse mit Verbindungslinien.}
  \label{fig:bekanntgross}
\end{figure}

Die Geraden sollten sich im Rahmen der Messgenauigkeit näherungsweise in einem Punkt
schneiden, dessen beide Koordinaten die Brennweite $f$ sein sollten. Nimmt man die
beiden als Ausreißer zu erkennenden Messwertpaare heraus, ist dies
erfüllt. Der relevante Bereich ist in Abbildung \ref{fig:bekanntklein} dargestellt.

\begin{figure}
  \centering
  \includegraphics{build/bekanntklein.pdf}
  \caption{Linse mit $f_1$, Ausschnitt um den Schnittpunkt aus Abbildung \ref{fig:bekanntklein}.}
  \label{fig:bekanntklein}
\end{figure}

Aus dieser vergrößerten Darstellung lässt sich näherungsweise 14,4 als grafisch
bestimmte Brennweite ablesen.

Die auch in der Tabelle \ref{tab:bekannt} eingetragenen Werte für die Brennweite $f$ wurden gemäß der
Linsengleichung \eqref{eqn:linsengleichung} berechnet. Wird aus diesen Werten
der Mittelwert nach Gleichung \eqref{eqn:mean} und die empirische Standardabweichung
nach Gleichung \eqref{eqn:std} bestimmt, so ergibt sich ein Wert von $\SI{14.32(0010)}$
für die experimentell durch Mittelwertbildung bestimmte Brennweite. Dabei ist zu
beachten, dass die oben Ausreißer identifizierten Messwertpaare nicht für diese Rechnung
berücksichtigt werden. Die beiden Werte, die so für den Brennwert ermittelt wurden,
können zu
\begin{equation*}
  f_{\text{1,exp}} = \SI{14.36(0010)}{\centi\meter}
\end{equation*}
gemittelt werden. Dann beträgt die relative Abweichung zur Herstellerangabe von
$f_1 = \SI{150}{\milli\meter}$ -4,27\%.

Auch die Brennweite $f_2$ der Wasserlinse soll auf diese Art und Weise bestimmt werden.
Die Messwerte dazu sind in \ref{tab:unbekannt} zu finden.
Die gemessenen Gegenstands- und Brennweiten werden in \ref{fig:wassergross} auf die Achsen aufgetragen
und der relevante Bereich in \ref{fig:wasserklein} vergrößert dargestellt.

\begin{figure}{h}
  \centering
  \includegraphics{build/wassergross.pdf}
  \caption{Wasserlinse, Auftragung der Bildweiten auf der Ordinaten- und der Gegenstandsweiten auf der Abszissenachse mit Verbindungslinien.}
  \label{fig:wassergross}
\end{figure}

\begin{figure}{h}
  \centering
  \includegraphics{build/wasserklein.pdf}
  \caption{Wasserlinse, Ausschnitt um den Schnittpunkt aus Abbildung \ref{fig:wasserklein}.}
  \label{fig:wassergross}
\end{figure}

\subsection{Bestimmung der Brennweite durch die Methode von Bessel}
