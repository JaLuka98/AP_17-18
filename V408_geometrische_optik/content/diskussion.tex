\section{Diskussion}
\label{sec:Diskussion}
Zunächst ist zu bemerken, dass die möglichen Fehlerquellen bei diesem Versuch groß sind.
Das Bild war durch Verschieben der optischen Instrumente auf der optischen Bank möglichst
scharf zu stellen. Dabei entstehen statistische Fehler. Es ist möglich, dass diese
signifikant zu Abweichung von den Theoriewerten bzw. wahren Werten führen, da nur fünf-
bis zehnmal gemessen wurde. Diese geringe Anzahl an Messungen kann nicht
garantieren, dass sich die statistischen Fehler nahezu herausheben. Des Weiteren können
bei der Messung systematische Fehler auftreten, da eine Verzerrung bei der systematischen
Beurteilung der Schärfe der Bilder nicht ausgeschlossen werden kann. Außerdem besteht die
Möglichkeit, dass das Licht nicht ausreichend fokussiert wurde und die paraxiale Näherung
signifikant verletzt wurde.\\
Es ist auffällig, dass bei sämtlichen Verfahren die experimentell bestimmte Brennweite der Linse im Vergleich
zur Herstellerangabe nach unten hin abweicht. Daher sind die in Kapitel \ref{sec:Auswertung}
aufgeführten relativen Abweichung stets nur eingeschränkt als Maß zur Güte der Messung
dienen können, da die einseitige Abweichung nach unten systematische Fehler nahe legt.
insbesondere kann über die Güte der Messung der Brennweite der Wasserlinse keine
Aussage gemacht werden, da keine theoretischen Werte zum Vergleich vorliegen.
Gleiches gilt für die chromatische Abberation bei der Methode von Bessel, da der Einfluss
dieses Abbildungsfehlers auf den gemessenen Brennwert nicht bekannt ist. Werden
die Nominalwerte der bestimmten Brennwerte der Linse für rotes und blaues Licht betrachtet,
so konnte zumindest qualitativ die Erwartung bestätigt werden, dass sich die Brennwerte
unterscheiden. Da jedoch die Brennweite für blaues Licht in der $1-\sigma-\text{Umgebung}$
der Brennweite für blaues Licht liegt, kann kein abschließendes Urteil über die
Qualität und Genauigkeit der Messung getroffen werden. \\
Da sich die experimentell bestimmten Werte für die Brennweiten bei der direkten Methode
über die Linsengleichung und bei der Methode von Bessel nicht signifikant unterscheiden,
sind beide Methoden als ähnlich genau im Rahmen dieses Versuchs zu bewerten.
Zuletzt wird darauf hingewiesen, dass die Messwerte qualitativ den erwarteten Zuordnungen
folgen. So schneiden sich die Geraden in den Diagrammen zur direkten Methode abgesehen
von den Außreißern näherungsweise in einem Punkt, außerdem folgen die Messwerte bei der
Methode von Abbe gut der theoretisch vorhergesagten linearen Zuordnungen.
