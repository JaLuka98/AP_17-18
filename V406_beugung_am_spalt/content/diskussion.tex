\section{Diskussion}
\label{sec:Diskussion}
Die Messung spiegelt die theoretisch hergeleiteten Zusammenhänge sehr gut wieder,
da die Messwerte gut den Verlauf der theoretisch hergeleiteten Funktion zeigen. Zudem
sind die Ergebnisse der Messung für die verschiedenen Spaltbreiten mit
\begin{align*}
  b_{\text{exp},1}&=\SI{0.1458(00005)}{\milli\meter} \,, \\
  b_{\text{exp},2}&=\SI{0.07720(000031)}{\milli\meter} \,,\\
  b_{\text{exp},3}&=\SI{0.151(0008)}{\milli\meter} \,
\end{align*}
sehr nah an den vom Hersteller angegebenen Werten.
Es ist jedoch anzumerken, dass für die Anpassungen jeweils Anfangsparameter festgelegt
wurden, sodass nicht gewährleistet ist, dass die berechneten Ausgleichsfunktionen
treffend sind. Da diese dem Verlauf der Messwerte jedoch sehr gut folgen, und da
die aus den Parametern der Ausgleichsrechnung berechneten Werte für die Spaltbreiten
nah an den jeweiligen Herstellerangaben liegt, ist jedoch
anzunehmen, dass die Ausgleichsrechnungen recht genau und sinnvoll erfolgt sind.

Auffällig ist, dass die Messwerte selbst nach Bereinigung um den Dunkelstrom nie
den Wert null Annehmen, wie es jedoch  theoretisch anzunehmen wäre. Ein möglicher Grund
hierfür ist eine unzureichende Schärfe des Interferenzbildes. Dies ist besonders beim
Doppelspalt zu beobachten.

Desweiteren ist anzumerken, dass Ungenauigkeiten dadurch entstehen, dass das Maximum
nicht genau mittig liegt. Dadurch gibt es Abweichungen der tatsächlichen Winkel
von den berechneten Winkeln. Diese sind aber aufgrund des großen Abstandes zwischen
Spalt und Detektor sehr gering.

Zudem ist auffällig, dass beim Doppelspalt die Parameter der Ausgleichrechnung höhere
relative Ungenauigkeiten aufweisen als bei beiden Einzelspalten.
An dieser Stelle ist die Messung als ungenau zu bewerten. Obwohl die Messintervalle
bei dieser Messreihe bereits kleiner gewählt wurden, liegen offenbar nicht genügend
und nicht ausreichend dicht genommene Messwerte vor, um eine genaue Ausgleichsrechnung durchführen zu können.

Diese Ungenauigkeit führt auch dazu, dass das Hauptmaximum der Ausgleichsfunktion des Intensitätsverlaufes des
Doppelspaltes höher ist als der des Einzelspaltes. Die Messwerte wurden zwar auf
eine gleiche Intensität angepasst, jedoch schwingt die Ausgleichsfunktion des Doppelspalts
über. Dies könnte daran liegen, dass wie oben erwähnt der Strom nie auf null sank, sodass
gut und genau gemessenen Maxima keine genau gemessenen Minima gegenüberstehen und
die Ausgleichskurve höher als eigentlich wird.
