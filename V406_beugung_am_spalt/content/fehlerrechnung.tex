\section{Fehlerrechnung}
\label{sec:Fehlerrechnung}
Der Mittelwert einer Stichprobe von $N$ Werten wird durch
\begin{equation}
  \overline{x} = \sum\limits_{i = 1}^N x_i
  \label{eqn:mean}
\end{equation}

berechnet.
Die empirische Standardabweichung dieser Stichprobe ist durch
\begin{equation}
  \sigma_x = \sqrt{\frac{1}{N-1}
    \sum\limits_{i = 1}^N
    (x_i-\overline{x})^2}
    \label{eqn:std}
\end{equation}

gegeben.
Ist $f$ eine Funktion, die von unsicheren Variablen $x_i$ mit
Standardabweichungen $\sigma_i$ abhängt, so ist die Unsicherheit von f
\begin{equation}
  \sigma_f = \sqrt{
    \sum\limits_{i = 1}^N
      \left( \frac{\partial f}{\partial x_i} \sigma_i \right)^{\!\! 2}
  }\,.
  \label{eqn:gaussfehler}
\end{equation}

Diese Formel bezeichnet man als "Gauß'sches Fehlerfortpflanzungsgesetz".

Bei einer linearen Regression folgt eine Ausgleichsgerade
\begin{equation}
  y(x) = ax+b\,
\end{equation}
mit der Steigung $a$ und dem Ordinatenabschnitt $b$. Liegen Fehler in y-Richtung
und nur in y-Richtung vor, dann sind die Parameter $a$ und $b$ selbst unsicher
und ergeben sich zu
\begin{align}
  a &= \frac{\overline{xy}-\overline{x} \cdot \overline{y}}{\overline{x^2}-\overline{x}^2}\,,\\
  b &= \frac{\overline{y}-\overline{x^2}-\overline{xy} \cdot \overline{x}}{\overline{x^2}-\overline{x}^2}\,.
\end{align}

%Wenn im Folgenden Mittelwerte, Standardabweichungen und Fehler von
%Funktionen unsicherer Größen berechnet werden, so werden stets die obigen
%Formeln verwendet.
Jegliche Rechnungen werden mit IPython 5.3.0 in Python 3.6.1 durchgeführt. Dabei
werden die Bibliotheken "numpy" \cite{numpy} und "scipy" \cite{scipy} verwendet.
Letztere dient insbesondere zur Erstellung von Ausgleichsrechnungen.
Die Ausführung von Fehlerrechnungen geschieht unter Verwendung des Pakets
"uncertainties" \cite{uncertainties}. Zur Erstellung von Graphen wird die Bibliothek
"matplotlib" \cite{matplotlib} verwendet.
