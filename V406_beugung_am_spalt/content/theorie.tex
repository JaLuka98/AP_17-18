\section{Theorie}
\label{sec:Theorie}
In der Wellenoptik kommt es zur Beugung von Wellen, wenn diese auf eine Öffnung oder
ein Hindernis in der Größenordnung ihrer Wellenlänge treffen. Beugung bedeutet hierbei,
dass die Wellen von ihrer geraden Ausbreitungsrichtung abgebracht werden und sich in
die bei geradliniger Ausbreitung versperrten Bereiche fortpflanzen. In diesem Versuch
wird das Phänomen der Beugung bei Licht untersucht. \\
Im Folgenden wird stets die Fraunhofer-Näherung angenommen. Das bedeutet, dass das Licht
von einer als punktförmig angenommenen Quelle, die sich weit hinter der Blende befindet,
ausgeht und dass der Abstand des Schirms von der Blende als groß angenommen wird. Dies
entspricht also einer Fernfeldnäherung. Schematisch ist dies in \ref{fig:fraunhofer}
dargestellt. Weil die Quelle als weit entfernt angenommen wird, können eintreffende
Strahlen näherungsweise als parallel angenommen werden.

\begin{figure}
  \centering
  \includegraphics[width=260pt]{data/fraunhofer.png}
  \caption{Skizze der Fraunhofer-Beugung \cite{Versuchsanleitung}}
  \label{fig:fraunhofer}
\end{figure}

Die dargestellte Sammellinse dient zur Fokussierung der Strahlen in einer Brennebene,
sodass der Winkel aller im Punkt P interferierenden Strahlen gleich wird.
Es stellt sich heraus, dass die komplexe Amplitude $U$ an dem Beobachtungsort
$\symbf{r}_p$ für Fraunhofersche Beugung an einer Öffnung der Fläche $A$
ohne phasenverschiebendes Material durch
\begin{equation}
  U(\symbf{r}_p) = C' \int_{A} \symup{d}x \, \symup{d}y \, e^{-i(\symbf{q} \cdot \symbf{r})}
  \label{eqn:beugungsintegral}
\end{equation}
gegeben ist. Dies ist die zweidimensionale asymmetrische Fouriertransformierte der
Blende. Der Streuvektor $\symbf{q} = {\symbf{k}}_\text{aus} - {\symbf{k}}_\text{ein}$
gibt die Differenz der Wellenzahlen der einfallenden und der ausfallenden Welle ein.
Die Konstante $C'$ enthält Informationen über die Phase der Welle und dass die Beugung
eine Kugelwelle hervorruft. Der Beobachtungsschirm befindet sich in der ($x$,$y$)-Ebene.
Ausführung der Integration über einen Spalt der Breite $2a$ führt auf die Amplitude
\begin{equation}
  U(\symbf{r}_p) = 4 \pi C' a \delta(q_y) \frac{\sin(q_x a)}{q_x a}\,.
\end{equation}
Die Deltadistribution beschreibt, dass die Ausbreitung der Welle in y-Richtung durch den
Spalt nicht beeinflusst wird. Die Amplitude der Welle ist nicht in einem Experiment nicht
direkt messbar, da ihre Frequenz zu hoch ist. Stattdessen wird die Intensität gemessen.
Diese ist proportional zum Quadrat der Amplitude. Wird der Ablenkwinkel $\beta$ mit
\begin{equation*}
  q_x a = k a \sin(\beta)
\end{equation*}
eingeführt, ergibt sich für die Intensität als Funktion des Ablenkwinkels
\begin{equation}
  I = I(\beta = 0) \frac{\sin^2(k a \sin(\beta))}{k a \sin(\beta)}\,.
\end{equation}
Dabei ist $I(\beta = 0)$ die maximale Intensität der nicht abgelenkten Welle.
