\section{Durchführung}
\label{sec:Durchführung}

Der Versuchsaufbau ist in \ref{fig:aufbau} zu sehen. Zunächst wird der Abstand $L$
vom Beugungsspalt zur Detektorblende gemessen. Der Detektor ist hier ein Photoelement.
Dieses misst elektromagnetische Wellen und erzeugt einen Strom, der dann gemessen wird.
Durch die Wärme im Photoelement entstehen freie Ladungsträger, die einen Strom erzeugen,
ohne dass elektromagnetische Wellen den Detektor erreichen. Nach Abdecken der Detektorblende
wird dieser sogenannte Dunkelstrom $I_\text{dunkel}$ vor jeder einzelnen Messung abgelesen.
Danach wird der Gegenstand entfernt. Zur Messung der Lichtintensität wird der
angezeigte Strom $I$ notiert.

\begin{figure}
  \centering
  \includegraphics[width=260pt]{data/aufbau.png}
  \caption{Skizze der Versuchsanordnung \cite{Versuchsanleitung}}
  \label{fig:aufbau}
\end{figure}
