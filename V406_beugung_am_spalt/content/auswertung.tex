\section{Auswertung}
\label{sec:Auswertung}

Der zuallererst gemessene Abstand vom Spalt zum Detektor beträgt $L = \SI{1}{\meter}$.
Es wurden drei Messreihen aufgenommen. Zunächst wurde die Beugung an einem Einzelspalt
der Breite $b_1 = \SI{0.15}{\milli\meter}$ untersucht, danach an einem Einzelspalt
der Breite $b_2 = \SI{0.075}{\milli\meter}$ und zuletzt an einem Doppelspalt mit
der jeweiligen Spaltbreite von $b_3 = \SI{0.15}{\milli\meter}$ und dem Spaltmittenabstand
$d = \SI{0.5}{\milli\meter}$. Der gemessene Dunkelstrom betrug beim ersten Einzelspalt
$\SI{3.4}{\nano\ampere}$, beim zweiten Einzelspalt und beim Doppelspalt
$\SI{3.2}{\nano\ampere}$. In Tabelle \ref{tab:messwerte} sind die Messreihen der Stromstärken
$I_i$ gegen die Detektorpositionen $x_i$ dargestellt. Dabei nummerieren stets wie bereits oben die Indizes 1,2 und 3 die
Spalte der Reihenfolge nach durch. Es wurde bereits der Dunkelstrom von den Messwerten
subtrahiert, sodass in der Tabelle die bereinigten Werte zu sehen sind.

%\begin{table}[htp]
        %\begin{center}
                %\begin{tabular}{cc}
                \begin{longtable}{S[table-format=3.2] S[table-format=3.2] S[table-format=3.2] S[table-format=3.2] S[table-format=3.2] S[table-format=3.2]}
                \caption{Messreihen zur Beugung an zwei Einzelspalten und einem Doppelspalt}\\
                \label{tab:messwerte}
                %\toprule
                {$x_1/$mm} & {$I_1/$nA} & {$x_2/$mm} & {$I_2/$nA} & {$x_3/$mm} & {$I_3/$nA} \\
                \midrule
                -13.00 & 9.60 & -13.00 & 16.80 & -12.00 & 14.80\\
                -12.50 & 10.60 & -12.75 & 17.80 & -11.75 & 15.80\\
                -12.00 & 12.60 & -12.50 & 18.80 & -11.50 & 15.80\\
                -11.50 & 15.10 & -12.25 & 18.80 & -11.25 & 18.80\\
                -11.00 & 16.10 & -12.00 & 18.80 & -11.00 & 29.80\\
                -10.50 & 15.10 & -11.75 & 18.80 & -10.75 & 43.80\\
                -10.00 & 12.10 & -11.50 & 18.55 & -10.50 & 48.80\\
                -9.50 & 9.10 & -11.25 & 17.80 & -10.25 & 41.80\\
                -9.00 & 10.10 & -11.00 & 16.80 & -10.00 & 33.80\\
                -8.75 & 13.10 & -10.75 & 15.80 & -9.75 & 30.80\\
                -8.50 & 17.60 & -10.50 & 14.80 & -9.50 & 39.80\\
                -8.25 & 19.60 & -10.25 & 12.80 & -9.25 & 42.80\\
                -8.00 & 32.60 & -10.00 & 11.80 & -9.00 & 36.80\\
                -7.75 & 40.60 & -9.75 & 10.30 & -8.75 & 26.80\\
                -7.50 & 47.60 & -9.50 & 8.80 & -8.50 & 21.80\\
                -7.25 & 55.60 & -9.25 & 7.80 & -8.25 & 20.80\\
                -7.00 & 60.60 & -9.00 & 6.80 & -8.00 & 22.80\\
                -6.75 & 64.60 & -8.75 & 5.80 & -7.75 & 23.80\\
                -6.50 & 62.60 & -8.50 & 5.80 & -7.50 & 28.30\\
                -6.25 & 58.60 & -8.25 & 6.30 & -7.25 & 46.30\\
                -6.00 & 52.60 & -8.00 & 7.30 & -7.00 & 76.80\\
                -5.75 & 43.60 & -7.75 & 9.30 & -6.75 & 101.80\\
                -5.50 & 33.60 & -7.50 & 11.80 & -6.50 & 96.80\\
                -5.25 & 26.60 & -7.25 & 15.30 & -6.25 & 76.80\\
                -5.00 & 21.60 & -7.00 & 19.30 & -6.00 & 71.80\\
                -4.75 & 22.60 & -6.75 & 24.80 & -5.75 & 96.80\\
                -4.50 & 32.60 & -6.50 & 30.80 & -5.50 & 116.80\\
                -4.25 & 54.60 & -6.25 & 38.30 & -5.25 & 111.80\\
                -4.00 & 91.60 & -6.00 & 44.80 & -5.00 & 76.80\\
                -3.75 & 136.60 & -5.75 & 55.80 & -4.75 & 46.80\\
                -3.50 & 211.60 & -5.50 & 65.80 & -4.50 & 31.80\\
                -3.25 & 281.60 & -5.25 & 76.80 & -4.25 & 31.80\\
                -3.00 & 376.60 & -5.00 & 88.80 & -4.00 & 38.80\\
                -2.75 & 486.60 & -4.75 & 101.80 & -3.75 & 54.80\\
                -2.50 & 616.60 & -4.50 & 116.80 & -3.50 & 116.80\\
                -2.25 & 716.60 & -4.25 & 129.30 & -3.25 & 261.80\\
                -2.00 & 826.60 & -4.00 & 141.80 & -3.00 & 436.80\\
                -1.75 & 936.60 & -3.75 & 156.80 & -2.75 & 526.80\\
                -1.50 & 1046.60 & -3.50 & 166.80 & -2.50 & 476.80\\
                -1.25 & 1096.60 & -3.25 & 166.80 & -2.25 & 496.80\\
                -1.00 & 1171.60 & -3.00 & 194.30 & -2.12 & 636.80\\
                -0.75 & 1196.60 & -2.75 & 204.30 & -2.00 & 896.80\\
                -0.50 & 1221.60 & -2.50 & 216.80 & -1.88 & 1221.8\\
                -0.25 & 1196.60 & -2.25 & 226.80 & -1.75 & 1546.8\\
                0.00 & 1146.60 & -2.00 & 234.30 & -1.50 & 1846.80\\
                0.25 & 1096.60 & -1.75 & 241.80 & -1.25 & 1496.80\\
                0.50 & 996.60 & -1.50 & 246.80 & -1.13 & 1246.80\\
                0.75 & 896.60 & -1.25 & 254.30 & -1.00 & 1096.80\\
                1.00 & 766.60 & -1.00 & 261.80 & -0.88 & 1096.80\\
                1.25 & 636.60 & -0.75 & 261.80 & -0.75 & 1346.80\\
                1.50 & 536.60 & -0.50 & 261.80 & -0.62 & 1746.80\\
                1.75 & 416.60 & -0.25 & 256.80 & -0.50 & 2246.80\\
                2.00 & 316.60 & 0.00 & 256.80 & -0.38 & 2596.80\\
                2.25 & 236.60 & 0.25 & 254.30 & -0.25 & 2746.80\\
                2.50 & 171.60 & 0.50 & 246.80 & -0.12 & 2596.80\\
                2.75 & 106.60 & 0.75 & 236.80 & 0.00 & 2246.80\\
                3.00 & 60.60 & 1.00 & 226.80 & 0.12 & 1746.80\\
                3.25 & 35.60 & 1.25 & 224.30 & 0.25 & 1346.80\\
                3.50 & 21.60 & 1.50 & 206.80 & 0.38 & 1146.80\\
                3.75 & 16.60 & 1.75 & 196.80 & 0.50 & 1121.80\\
                4.00 & 20.60 & 2.00 & 186.80 & 0.62 & 1346.80\\
                4.25 & 28.60 & 2.25 & 171.80 & 0.75 & 1646.80\\
                4.50 & 40.60 & 2.50 & 159.30 & 0.88 & 1896.80\\
                4.75 & 50.60 & 2.75 & 146.80 & 1.00 & 1996.80\\
                5.00 & 58.60 & 3.00 & 131.80 & 1.13 & 1946.80\\
                5.25 & 64.60 & 3.25 & 121.80 & 1.25 & 1646.80\\
                5.50 & 68.60 & 3.50 & 106.80 & 1.38 & 1296.80\\
                5.75 & 66.60 & 3.75 & 94.30 & 1.50 & 946.80\\
                6.00 & 62.60 & 4.00 & 81.80 & 1.63 & 646.80\\
                6.25 & 56.60 & 4.25 & 71.80 & 1.75 & 496.80\\
                6.50 & 47.60 & 4.50 & 61.80 & 2.00 & 546.80\\
                6.75 & 38.60 & 4.75 & 51.80 & 2.25 & 646.80\\
                7.00 & 30.60 & 5.00 & 43.80 & 2.50 & 546.80\\
                7.25 & 22.60 & 5.25 & 35.80 & 2.75 & 496.80\\
                7.50 & 16.10 & 5.50 & 29.80 & 3.00 & 126.80\\
                7.75 & 12.60 & 5.75 & 23.80 & 3.25 & 70.80\\
                8.00 & 10.10 & 6.00 & 18.80 & 3.50 & 56.80\\
                8.25 & 9.10 & 6.25 & 13.80 & 3.75 & 37.80\\
                8.50 & 9.10 & 6.50 & 11.80 & 4.00 & 24.80\\
                8.75 & 10.10 & 6.75 & 7.80 & 4.25 & 34.80\\
                9.00 & 11.10 & 7.00 & 5.80 & 4.50 & 63.80\\
                9.25 & 12.10 & 7.25 & 4.80 & 4.75 & 90.80\\
                9.50 & 13.10 & 7.50 & 3.80 & 5.00 & 86.80\\
                9.75 & 13.10 & 7.75 & 3.80 & 5.25 & 66.80\\
                10.00 & 12.60 & 8.00 & 3.60 & 5.50 & 56.80\\
                10.50 & 9.60 & 8.25 & 4.00 & 5.75 & 71.80\\
                11.00 & 6.20 & 8.50 & 4.70 & 6.00 & 101.80\\
                11.50 & 3.80 & 8.75 & 5.60 & 6.25 & 96.80\\
                12.00 & 3.80 & 9.00 & 5.80 & 6.50 & 71.80\\
                12.50 & 5.70 & 9.25 & 7.30 & 6.75 & 41.80\\
                13.00 & 9.10 & 9.50 & 8.30 & 7.00 & 32.80\\
                     &     & 9.75 & 9.30 & 7.25 & 29.80\\
                     &     & 10.00 & 10.30 & 7.50 & 25.80\\
                     &     & 10.25 & 10.30 & 7.75 & 23.80\\
                     &     & 10.50 & 11.80 & 8.00 & 22.80\\
                     &     & 10.75 & 11.80 & 8.25 & 22.80\\
                     &     & 11.00 & 12.30 & 8.50 & 26.80\\
                     &     & 11.25 & 11.80 & 8.75 & 36.80\\
                     &     & 11.50 & 11.80 & 9.00 & 36.80\\
                     &     & 11.75 & 11.30 & 9.25 & 24.80\\
                     &     & 12.00 & 10.80 & 9.50 & 20.80\\
                     &     & 12.25 & 10.30 & 9.75 & 28.80\\
                     &     & 12.50 & 9.30 & 10.00 & 36.80\\
                     &     & 12.75 & 8.80 & 10.25 & 30.80\\
                     &     & 13.00 & 7.80 & 10.50 & 19.80\\
                     &     &    &  & 10.75 & 12.80\\
                     &     &    &  & 11.00 & 11.80\\
                     &     &    &  & 11.25 & 11.80\\
                     &     &    &  & 11.50 & 11.30\\
                     &     &    &  & 11.75 & 10.80\\
                     &     &    &  & 12.00 & 13.80\\
                \bottomrule
                %\end{tabular}
        %\end{center}
\end{longtable}

Die Anpassungsfunktion für die Messreihen an den Einzelspalten ist in Analogie zu
\eqref{eqn:einzeltheo} und mit \eqref{eqn:phi}
\begin{equation*}
  I_i (x) = A_{0,i} \,\, b_{\text{exp},i} \,\, \symup{sinc}^2\left(\frac{\pi b \sin\left(\frac{x-x_{0,i}}{L}\right)}{\lambda}\right) \,.
\end{equation*}
Dabei sind die Parameter die Amplitudenfaktoren $A_{0,i}$, die experimentell bestimmten Spaltbreiten
$b_{\text{exp},i}$ und die experimentellen Zentrierungen $x_{0,i}$.
Für beide Regressionen wurden die folgenden Anfangsparameter vorgegeben:
\begin{equation*}
  \{A_0, b_{\text{exp}}, x_0\} = \{0.73, \SI{-0.5}{\milli\meter}, \SI{-1.5}{\milli\meter}\}\,.
\end{equation*}
In den Abbildungen \ref{fig:spalt1} und \ref{fig:spalt2} sind die Messreihen der Beugung an den Einzelspalten aufgetragen
und die Graphen der konkreten Ausgleichsfunktionen ist zu sehen.

\begin{figure}
  \centering
  \includegraphics{build/einfachspalt_1.pdf}
  \caption{Auftragung der Stromstärke gegen die Detektorposition und Graph der Ausgleichsfunktion, erster Einzelspalt}
  \label{fig:spalt1}
\end{figure}

\begin{figure}
  \centering
  \includegraphics{build/einfachspalt_2.pdf}
  \caption{Auftragung der Stromstärke gegen die Detektorposition und Graph der Ausgleichsfunktion, zweiter Einzelspalt}
  \label{fig:spalt2}
\end{figure}

Die Parameter, die sich durch die nichtlineare Ausgleichsrechnung ergeben, sind
in Tabelle \ref{tab:paramseinzel} zu sehen.

\begin{table}
\centering
\begin{tabular}{cccc}
\toprule
& $A_{0,1}$ & $x_{0,1}/$mm & $b_{\text{exp},1}/$mm \\
\midrule
%Parameter & $\SI{7.551(0020)}$ & $\SI{-0.569(0006)}$ & $\SI{0.1458(00005)}$ & $\SI{6.603(0022)}$ & $\SI{-0.560(0013)}$ & $\SI{0.07720(000031)}$ \\
Parameter & 7,551 $\pm$ 0,020 & -0,569 $\pm$ 0,006 & 0,1458 $\pm$ 0,0005 \\
Relative Ungenauigkeit & 0,26\% & 1,05\% & 0,34\%\\
\toprule
& $A_{0,2}$ & $x_{0,2}/$mm & $b_{\text{exp},1}/$mm \\
\midrule
Parameter & 6,603 $\pm$ 0.022 & -0,560 $\pm$ 0,013 & 0,07720 $\pm$ 0,00031 \\
Relative Ungenauigkeit & 0,33\% & 2,32\% & 0,40\%\\
\bottomrule
\end{tabular}
\caption{Parameter der Ausgleichsfunktionen, Einzelspalte}
\label{tab:paramseinzel}
\end{table}

Die relative Abweichung der experimentell ermittelten Spaltbreite des ersten Einzelspaltes zur am Anfang der
Auswertung angemerkten Herstellerangabe beträgt -2,80\%, für den zweiten Einzelspalt 2,93\%.

Die Messreihe zum Doppelspalt wird mit der Funktion
\begin{equation*}
  I_3 (x) = 4 \cos^2\left(\frac{\pi d_\text{exp} \sin\left(\frac{x-x_{0,3}}{L}\right)}{\lambda}\right) {A_{0,3}}^2 \,\, {b_{\text{exp},3}}^2 \,\,
  \symup{sinc}^2\left(\frac{\pi b \sin\left(\frac{x-x_{0,3}}{L}\right)}{\lambda}\right)
\end{equation*}
angepasst. Diese entspricht der theorisch berechneten Intensitätsverteilung \eqref{eqn:doppeltheo}.
Diese Anpassungsfunktion hat vier zu bestimmende Parameter: Eine Amplitudenfaktor $A_{0,3}$,
die experimentell bestimmte Spaltbreite der baugleichen Spalte $b_{\text{exp},3}$,
die experimentelle Zentrierung $x_{0,3}$ und der experimentelle Spaltmittenabstand
$d_\text{exp}$.
Für die Regression wurden die Anfangsparameter
\begin{equation*}
  \{A_0, b_{\text{exp}}, x_0, d_\text{exp}\} = \{0.73, \SI{-0.5}{\milli\meter}, \SI{-1.5}{\milli\meter}, \SI{1}{\milli\meter}\}\,,
\end{equation*}
gewählt.
Der Graph der Ausgleichsfunktion und die Auftragung der gemessenen Stromstärke gegen
die Verschiebung des Detektors ist in Abbildung \ref{fig:spalt3} veranschaulicht.

\begin{figure}
  \centering
  \includegraphics{build/doppelspalt.pdf}
  \caption{Auftragung der Stromstärke gegen die Detektorposition und Graph der Ausgleichsfunktion, Doppelspalt}
  \label{fig:spalt3}
\end{figure}

Die Ergebnisse für die Parameter der Ausgleichsrechnung finden sich in Tabelle \ref{tab:paramsdoppel}.

\begin{table}
\centering
\begin{tabular}{ccccc}
\toprule
& $A_{0,3}$ & $x_{0,3}/$mm & $b_{\text{exp},3}/$mm & $d_\text{exp}$ \\
\midrule
Parameter & 5,85 $\pm$ 0,26 & -0,254 $\pm$ 0,013 & -0,151 $\pm$ 0,008 & 0,466 $\pm$ 0,006 \\
Relative Ungenauigkeit & 4,44\% & 5,12\% & 5,30\% & 1,29\%\\
\bottomrule
\end{tabular}
\caption{Parameter der Ausgleichsfunktion, Doppelspalt}
\label{tab:paramsdoppel}
\end{table}

Die relative Abweichung der experimentell bestimmten Spaltbreite der beiden einzelnen
Spalte zur Herstellerangabe beträgt 0,67\% \footnote{Dass die numerisch bestimmte
experimentelle Spaltbreite negativ ist, ist nicht von Bedeutung, da die Spaltbreite
in der Funktion ohnehin quadriert wird.}, die Abweichung beim Spaltmittenabstand
-6,80\%.

Zuletzt soll noch das Beugungsbild eines Einzelspaltes mit dem eines Doppelspaltes
verglichen werden. Da der erste Einzelspalt und der Doppelspalt mit $b_1 = b_3 = \SI{0.15}{\milli\meter}$
die gleiche Spaltbreite haben, werden diese beiden Aperturen verglichen.
Wie an Gleichung \eqref{eqn:doppeltheo} zu erkennen ist und dort danach angemerkt
wurde, beschreibt die Intensitätsverteilung des Einzelspaltes die Einhüllende
der Verteilung des Doppelspaltes. Um die konkreten Verteilungen miteinander vergleichen
zu können, müssen sie angepasst werden. Alle Messwerte und die Werte der Ausgleichsfunktion
des Einzelspaltes werden mit $\frac{\text{max} I_3(s)}{\text{max} I_1(s)} = \frac{2750}{1225}$
multipliziert, sodass beide Verteilungen die gleiche Nullintensität haben. Außerdem
werden sie um null zentriert, indem für die experimentelle Zentrierung in der Formel
null eingesetzt wird und die Messwerte entsprechend verschoben werden.
Die Darstellung beider angepasster Verteilungen ist in Abbildung \ref{fig:doppeleinzel}
zu sehen.

\begin{figure}
  \centering
  \includegraphics{build/doppeleinzel.pdf}
  \caption{Vergleich der Beugungsbilder eines Einzel- und eines Doppelspalts}
  \label{fig:doppeleinzel}
\end{figure}

Es ist erkennbar, dass der Intensitätsverlauf des Einzelspaltes grob als Einhüllende
des Intensitätsverlaufes des Doppelspaltes angesehen werden kann, jedoch ist das Hauptmaximum
beim Doppelspalt deutlich höher als beim Einzelspalt.
