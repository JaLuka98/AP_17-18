\newpage
\section{Diskussion}
\label{sec:Diskussion}
Im Allgemeinen spiegeln die Messwerte gut die theoretisch zu erwartenden
Zusammenhänge wieder.

Lediglich beim A-Scan ist auffällig, dass die in Tabelle \ref{tab:messwerte}
dargestellten Differenzen $\Delta d$ zwischen den mithilfe der Schieblehre
und mithilfe des A-Scans bestimmten Werten ziemlich genau die Größe der
der Laufzeitkorrektur $\Delta h=0{,}4\,$cm besitzen. Ohne die Laufzeitkorrektur
wären die Abweichungen der mithilfe des A-Scans bestimmten Werte zu den mit der
Schieblehre bestimmten Werten sehr gering. \\
Das legt nahe, dass die Laufzeitkorrektur nicht korrekt ist. Da die Ungenauigkeit
in der letzten bei der Messung signifikanten Nachkommastelle liegt, sind Ungenauigkeiten
beim Ablesen der Werte für die Höhen $h_{\symup{S}}$ und $h_{\symup{US}}$ wahrscheinlich.

Beim B-Scan hingegen erscheint die Laufzeitkorrektur angemessen. Die in Tabelle
\ref{tab:b-scan} Werte für die Differenz $\Delta d$ beim B-Scan sind relativ gering
und wären ohne die Laufzeitkorrektur von $\Delta h=0{,}5\,$cm größer.

Die zur Bestimmung des Auflösungsvermögens aufgenommenen Bilder der Fehlstellen 1
und 2 zeigen das theoretisch zu Erwartende: Die 4\,MHz-Sonde besitzt ein größeres
Auflösungsvermögen als die 1\,MHz-Sonde, jedoch kann die 1\,MHz-Sonde mit geringerem
Amplitudenverlust durch das Material senden.

Beim TM-Scan des Herzmodells ist anzumerken, dass Ungenauigkeiten dadurch entstehen,
dass nicht mit perfekt konstanter Frequenz gepumpt werden kann. Zudem hatte das
Herzmodell ein kleines Loch, sodass die Herzsimulation nicht mehr exakt so funktioniert,
wie angedacht. Da dieses Experiment jedoch insgesamt nur zur Veranschaulichung der
Funktion eines TM-Scans dienen soll, sind diese Ungenauigkeiten im Versuchsaufbau
zu vernachlässigen. Insgesamt liegt das Ergebnis mit $V_{\symup{Herz}}=12{,}9\pm1{,}8\,
\symup{\frac{cm^3}{s}}$ in einem Bereich, der, wenn er mit den Abmessungen des
Versuchsaufbaus verglichen wird, als sinnvoll erscheint.
