\section{Theorie}
\label{sec:Theorie}
\subsection{Theoretische Betrachtungen zu Schall und Ultraschall}
\label{subsec:theorie1}
Ultraschall bezeichnet longitudinale Schallwellen im Frequenzbereich von ca.
20\,kHz bis 1\,GHz, der oberhalb der Hörschwelle des Menschen liegt.
Schall wird als longitudinale Welle angesehen, die sich gemäß
\begin{equation}
  p(x,t) = p_0 + v_0 Z \cos(\omega t - kx)
\end{equation}
durch Druckschwankung ausbreitet. Dabei bezeichnet $p(x,t)$ den Druck des Mediums
am Ort $x$ zur Zeit $t$, $p_0$ den Normaldruck des Mediums, $v_0$ die Amplitude der Teilchengeschwindigkeit,
$\omega$ die Kreisfrequenz der Welle, $k$ den Betrag des Wellenzahlvektors und $Z = c \rho$ ist die akustische
Impedanz, auch Schallkennwiderstand genannt. Sie beschreibt, wie stark das Medium
der Wellenausbreitung widersteht und ist von der Phasengeschwindigkeit $c$ und der
Dichte des Mediums $\rho$ abhängig. Da Schall sich im Allgemein mindestens teilweise
longitudinal ausbreitet, ist die Phasengeschwindigkeit abhängig vom Medium. Inbesondere
ist in Festkörpern auch transversale Ausbreitung möglich, sodass sich dort komplizierte Zusammenhänge
ergeben.\\
Im Allgemeinen erleidet eine Welle bei der Ausbreitung in einem Medium Energieverlust
durch Absorption. Es gilt das Exponentialgesetz
\begin{equation}
  I(x) = I_0 \cdot e^{\alpha x}
\end{equation}
für die Abnahme der Anfangsintensität $I_0$, wenn die Welle eine Strecke von $x$ durch
das Medium durchlaufen hat und $\alpha$ der Absorptionskoeffizient ist.
Luft absorbiert Ultraschall sehr stark, sodass in der Praxis ein Kontaktmittel
zum besseren Übertrag des Ultraschalls von Quelle zum zu untersuchenden Material
verwender wird.\\
Beim Übergang einer Schallwelle von einem Medium in ein anderes treten Reflexion und
Transmission auf. Der Reflexionskoeffizient $R$ gibt dabei das Verhältnis der reflektierten
zur einfallenden Intensität an und ist durch
\begin{equation}
R = \left( \frac{Z_1-Z_2}{Z_1+Z_2} \right)
\end{equation}
gegeben. Wieder bezeichnen die $Z_i$ die akustischen Impedanzen der Medien. Der
transmittierte Anteil beträgt $T = 1 - R$.\\
Die Erzeugung von Ultraschall kann zum Beispiel durch Anwendung des reziproken piezo-elektrischen
Effektes geschene. Zeigt eine polare Achse eines piezoelektrischen Kristalls in Richtung
eines elektrischen Wechselfeldes, so kann dieser zu Schwingungen angeregt werden.
Bei diesen werden Ultraschallwellen durch den Kristall abgestrahlt, Resonanzeffekte
sind bei übereinstimmender Eigen- und Anregungsfrequenz möglich. Auch das Empfangen
von Ultraschall ist mit piezoelektrischen Kristallen möglich, da diese zu Schwingungen
angeregt werden, wenn jener auf den Kristall trifft. Häufig werden Quarze zu diesen
Zwecken verwendet.

\subsection{Mögliche Scanverfahren mit Ultraschall}
\label{subsec:theorie2}
Grundsätzlich werden zwei verschiedene Verfahren bei der Untersuchung von Material
durch Ultraschall unterschieden.\\
Beim Durchschallungsverfahren nimmt ein Empfänger gegenüber des Senders die Intensität auf.
Aus einer gemessenen abgeschwächten Intensität kann dann auf Fehlstellen in der Probe geschlossen werden.
Hierbei kann jedoch keine Aussage darüber getroffen werden, wo sich die Fehlstelle befindet.\\
Wird der Sender auch gleichzeitig als Empfänger an Grenzflächen refelktierter Pulse verwendet, so wird vom Impuls-Echo-Verfahren
gesprochen. Die Höhe des Echos kann Rückschlüsse auf die Größe der Fehlstelle geben.
Ist die Schallgeschwindigkeit $c$ im Medium bekannt, so kann mit gemessener Laufzeit $t$
mithilfe von
\begin{equation}
  s = \frac{1}{2} ct
\end{equation}
die Lage der Fehlstelle bestimmt werden, wobei $s$ den Abstand von ihr zur Ultraschallsonde bezeichnet.\\
Die Messergebnisse der Amplitude und der Laufzeit können auf verschiedene Weisen aufgetragen werden.
Beim A-Scan wird die Amplitude in Abhängigkeit der Laufzeit aufgetragen. Dies stellt
ein eindimensionales Verfahren zur Abtastung dar. Der B-Scan stellt ein zweidimensionales
Bild dar. Dieser wird durch Bewegen der Sonde auf der Probe aufgenommen. Die Amplituden
werden durch Schattierungen dargestellt.
Beim TM-Scan erfolgt eine Aufnahme einer Bildfolge durch schnelle Abtastung, die eine Bewegung sichtbar machen kann.
