\section{Auswertung}
\label{sec:Auswertung}

\subsection{Bestimmung der Tiefe und der Größe der Störstellen mit einem A-Scan}
\label{subsec:A_scan_störstellen}

Zur Bestimmung der Eindringtiefen müssen diese zunächst Laufzeitkorrekturen, die durch
die Schutzschichten auf den Sonden entstehen, berücksichtigt werden. Dafür wird der
mit der Schieblehre bestimmte Wert von der Höhe $h_{\symup{S}}=8{,}03\,$cm des Blocks
von dem des mit Ultraschall bestimmten Wertes $h_{\symup{US}}=8,07\,$cm subtrahiert.
Es muss also von allen Werten in dieser Messreihe diese Differenz von $\Delta h=0{,}04\,$cm
subtrahiert werden. In Tabelle \ref{tab:messwerte} sind die  bereits bereinigten
Messwerte zur Messung mit dem A-Scan, sowie die mit der Schieblehre bestimmten
Werte dargestellt. Zudem befinden sich dort die berechneten Werte für die Dicken
der Störstellen.

\begin{table}[htp]
	\begin{center}
    \caption{Messwerte zur Messung der Störstellen mit einem A-Scan und mit einer Schieblehre,
    sowie daraus berechnete Werte.}
    \label{tab:messwerte}
		\begin{tabular}{cccccccc}
		\toprule
			{Störstelle} & {$s_{\symup{1,A}}/$mm} & {$s_{\symup{2,A}}/$mm} & {$d_{\symup{A}}/$mm} &
      {$s_{\symup{1,S}}/$mm} & {$s_{\symup{2,S}}/$mm} & {$d_{\symup{S}}/$mm} & {$\Delta d/mm$}\\
			\midrule
			3   & 15,2 & 63,0 & 2,1 & 13,2 & 61,1 & 6,0 & 3,9\\
			4   & 23,7 & 55,8 & 0,8 & 21,6 & 53,7 & 5,0 & 4,2\\
			5   & 32,2 & 48,3 & -0,2 & 30,0 & 46,3 & 4,0 & 4,2\\
			6   & 41,0 & 41,1 & -1,8 & 38,6 & 38,7 & 3,0 & 4,8\\
			7   & 49,0 & 33,0 & -1,7 & 46,7 & 30,8 & 3,0 & 4,7\\
			8   & 56,7 & 24,8 & -1,2 & 54,7 & 22,8 & 3,0 & 4,2\\
			9   & 64,7 & 16,8 & -1,2 & 62,7 & 14,9 & 3,0 & 4,2\\
			10  & 73,3 & 8,8  & -1,8 & 70,6 & 6,9 & 3,0 & 4,8\\
			11  & 17,0 & 57,5 & 5,80 & 15,0 & 55,5 & 10,0 & 4,2\\
		\bottomrule
		\end{tabular}
	\end{center}
\end{table}

Dabei werden die Strecken, die in Abbildung \ref{fig:acrylblock} unterhalb der Störstellen liegen
als $s_1$ und die Strekcne, die oberhalb der Störstellen liegen als $s_2$ bezeichnet.
Die Messwerte, die mit dem A-Scan aufgenommen wurden, werden mit dem Index "A"
und die, die mit der Schieblehre aufgenommen wurden mit dem Index "S" versehen.

Die Bestimmung der Dicke der Störstellen mithilfe des A-Scans erfolgt durch den
Zusammenhang
\begin{equation}
  d_{\symup{A}}=h_{\symup{S}}-s_{\symup{1,A}}-s_{\symup{2,A}} \,.
\end{equation}
Die Differenz $\Delta d$ ergibt sich durch
\begin{equation}
  \Delta d=|d_{\symup{A}}-d_{\symup{S}}| \,.
\end{equation}

Auffällig ist, dass diese Differenzen alle ungefähr die Größe der Laufzeitkorrektur
besitzen.

\subsection{Bestimmung des Auflösungsvermögens mit einem A-Scan}
\label{subsec:A_scan_auflösung}

Zur Bestimmung der Auflösung werden eine 1\,MHz-Sonde und eine 4\,MHz-Sonde verwendet die
Störstellen 1 und 2 werden von beiden Seiten ausgemessen.
Die augenommenen Bilder für die Störstellen 1 und 2 befinden sich in den Abbildung \ref{fig:auflösung1},
\ref{fig:auflösung2}, \ref{fig:auflösung3} und \ref{fig:auflösung4}.

\begin{figure}[H]
  \centering
  \includegraphics[width=\textwidth]{data/1mhzdoppelteFehlstellekurzelLaufzeitGedrehtwieinZeichnung.jpeg}
  \caption{Darstellung der Störstellen 1 und 2 mit einem A-Scan mit einer 1\,MHz-Sonde, wobei der Ultraschall
  zuvor die kurze Strecke durchlaufen musste.}
  \label{fig:auflösung1}
\end{figure}

\begin{figure}[H]
  \centering
  \includegraphics[width=\textwidth]{data/1mhzdoppelteFehlstellelangeLaufzeit.jpg}
  \caption{Darstellung der Störstellen 1 und 2 mit einem A-Scan mit einer 1\,MHz-Sonde, wobei der Ultraschall
  zuvor die lange Strecke durchlaufen musste.}
  \label{fig:auflösung2}
\end{figure}

\begin{figure}[H]
  \centering
  \includegraphics[width=\textwidth]{data/4mhzdoppelteFehlstellekurzelLaufzeitGedrehtwieinZeichnung.jpg}
  \caption{Darstellung der Störstellen 1 und 2 mit einem A-Scan mit einer 4\,MHz-Sonde, wobei der Ultraschall
  zuvor die kurze Strecke durchlaufen musste.}
  \label{fig:auflösung3}
\end{figure}

\begin{figure}[H]
  \centering
  \includegraphics[width=\textwidth]{data/4mhzdoppelteFehlstellelangeLaufzeit.jpg}
  \caption{Darstellung der Störstellen 1 und 2 mit einem A-Scan mit einer 4\,MHz-Sonde, wobei der Ultraschall
  zuvor die lange Strecke durchlaufen musste.}
  \label{fig:auflösung4}
\end{figure}

Deutlich erkennbar ist, dass bei beiden Sonden die Amplitude bei einer längeren
Laufzeit abnimmt. Bei der Messung mit der 4\,MHz-Sonde tritt dieser Effekt jedoch
deutlich stärker auf und die Maxima sind kaum noch zu erkennen. Dafür liefert die
4\,MHz-Sonde jedoch bei einer kurzen Laufzeit ein deutlich schärferes Bild als die
1\,MHz-Sonde.

\subsection{Bestimmung der Tiefe und der Größe der Störstellen mit einem B-Scan}
\label{subsec:B_scan_störstellen}

Zur Bestimmung der Lage und der Größe der Störstellen mit einem B-Scan muss zunächst
erneut die Laufzeitkorrektur durchgeführt werden. Der mithilfe des Ultraschalls bestimmte
Wert für die Höhe des Blocks beträgt $h=80{,}8\,$mm. Der Differenz zur mit der Schieblehre
bestimmten Wert beträgt also $\Delta h=0{,}5\,$mm. In Tabelle \ref{tab:b-scan}
sind die bereinigten Messwerte, sowie die daraus berechneten Werte dargestellt.
Für Störstelle 10 konnten dabei nicht alle Werte aufgenommen werden, da diese bei einer
Messung von Störstelle 11 überlagert wurde.

\begin{table}[htp]
	\begin{center}
    \caption{Messwerte zur Messung der Störstellen mit einem B-Scan und daraus berechnete Werte.}
    \label{tab:b-scan}
		\begin{tabular}{ccccc}
		\toprule
			{Störstelle} & {$s_{\symup{1,B}}/$mm} & {$s_{\symup{2,B}}/$mm} & {$d_{\symup{B}}/$mm} & {$\Delta d/mm$}\\
			\midrule
			3 & 12,8 & 61,1 & 6,4 & 0,4\\
			4 & 21,1 & 55,8 & 3,4 & 1,6\\
			5 & 29,7 & 48,3 & 2,3 & 1,7\\
			6 & 38,5 & 40,2 & 1,6 & 1,4\\
			7 & 46,5 & 31,9 & 1,9 & 1,1\\
			8 & 54,8 & 23,7 & 1,8 & 1,2\\
			9 & 62,6 & 15,4 & 2,3 & 0,7\\
			10 & {-} & 6,6 & {-} & {-}\\
			11 & 14,6 & 57,3 & 8,4 & 1,60\\
		\bottomrule
		\end{tabular}
	\end{center}
\end{table}

Die Rechnung erfolgt dabei analog zu der in Kapitel \ref{subsec:A_scan_störstellen}.
In Abbildung \ref{fig:b-scan} ist beispielhaft für einen B-Scan der B-Scan des
Acrylblocks von der oberen Kante aus dargestellt.
\begin{figure}[H]
  \centering
  \includegraphics[width=10cm]{data/Bscanrechtsnachlinkswieinzeichnung.jpg}
  \caption{B-Scan des Acrylblocks von der oberen Kante aus.}
  \label{fig:b-scan}
\end{figure}

\subsection{Bestimmung der Herzvolumens mit einem TM-Scan}
\label{subsec:Herzvolumen}


In Abbildung \ref{fig:herzvolumen} ist das im Versuch erstellte Diagramm der
simulierten Herschläge in Abhängigkeit von der Zeit dargestellt.

\begin{figure}[H]
  \centering
  \includegraphics[width=10cm]{data/Herzfrequenz.jpg}
  \caption{Abhängigkeit der simulierten Herzschläge in Abhängigkeit von der Zeit; augenommen
  mit einem TM-Scan.}
  \label{fig:herzvolumen}
\end{figure}

Aus diesem Diagramm werden die Amplituden $t_s$ der Herzschläge abgelesen. Dabei
ist zu beachten, dass während der Messung noch ein falscher Wert für die Schallgeschwindigkeit
eingetragen war. Deshalb müssen alle abgelesenen Werte mit dem Quotienten
$\frac{c_{\symup{W}}}{c_{\symup{A}}}$ der Schallgeschwindigkeiten von Wasser
und Acryl multipliziert werden.

Das Luftvolumen, von dem das Wasser verdrängt wird, wird als Kugelsegment angenähert.
Es gilt damit für das Volumen
\begin{equation}
  V_s = \frac{h\pi}{6}\cdot(3r^2 + h^2) \,,
\end{equation}
wobei $r$ der Radius des verwendeten Zylinders und $h=s_0-s$ ist. Dabei ist $s$
die gemessene Eindringtiefe beim Herzschlag und $s_0$ die gemessene Strecke in
der Ausganglage.

In Tabelle \ref{tab:herz} sind die aus dem Diagramm abgelesenen und bereits mit dem
Korrekturfaktor multiplizierten Werte, sowie daraus berechnete Werte zu sehen. Für den
ersten Schlag wird kein Wert abgelesen, da dieser nur sehr schlecht erkennbar ist.

\begin{table}[htp]
	\begin{center}
    \caption{Aus dem Diagramm abgelesene und mit dem Korrekturfaktor multiplizierte Werte und
    daraus berechnete Werte für die Höhe $h$ und das Volumen $V$ des Kugelsegments.}
    \label{tab:herz}
		\begin{tabular}{ccc}
		\toprule
			{$s/$mm} & {$h/$mm} & {$V/\symup{cm^3}$}\\
			\midrule
			31,3 & 12,8 & 11,9\\
			30,0 & 14,1 & 13,4\\
			30,1 & 14,0 & 13,3\\
			31,6 & 12,5 & 11,6\\
			30,1 & 14,0 & 13,3\\
			31,3 & 12,8 & 11,9\\
			31,6 & 12,5 & 11,6\\
			31,5 & 12,6 & 11,7\\
			31,1 & 13,0 & 12,2\\
			29,9 & 14,2 & 13,5\\
			31,9 & 12,2 & 11,3\\
			30,5 & 13,6 & 12,8\\
			32,9 & 11,2 & 10,2\\
			32,8 & 11,3 & 10,3\\
			34,5 & 9,6 & 8,5\\
			34,3 & 9,8 & 8,8\\
			34,1 & 10,0 & 9,0\\
		\bottomrule
		\end{tabular}
	\end{center}
\end{table}

Nach den Gleichungen \eqref{eqn:mean} und \eqref{eqn:std} wird der Mittelwert
der Volumina $V$ zu $V_{\symup{mittel}}=11{,}5\pm1{,}6\,\symup{cm^3}$ berechnet.

Die Frequenz des simulierten Herzschlags ist $f=1{,}125\,$Hz. Damit ergibt sich
das gesuchte Herzvolumen gemäß der Gleichung
\begin{equation}
  V_{\symup{Herz}} = V_{\symup{mittel}} f
\end{equation}
zu $V_{\symup{Herz}}=12{,}9\pm1{,}8\, \symup{\frac{cm^3}{s}}$.
