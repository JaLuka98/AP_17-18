\section{Diskussion}
\label{sec:Diskussion}

Im Allgemeinen spiegelt die Messung viele der theoretisch zu erwartenden Annahmen
wieder. Die Plateausteigung ist mit $a=(3{,}61 \pm 0{,}20) \symup{\frac{\%}{100V}}$
relativ gering und liegt im Rahmen der Ungenauigkeit, in der ein Zählrohr arbeitet.
Für reale Zählrohre ist im Allgemeinen auch eine geringe Plateausteigung zu erwarten,
da bei steigenden Spannungen immer einige Nachentladungen auftreten.

Der Verlauf der im Zählrohr ausgelösten Elektronen in Abhängigkeit von der angelegten
Spannung zeigt nicht ganz den theoretisch zu erwartenden Verlauf. Der Anstieg
der Messwerte zeigt eher einen linearen als einen exponentiellen Verlauf. Eine
mögliche Ursache für diese Abweichung stellt das Amperemeter dar. Da die maximal
gemessene Stromstärke bei etwa $I=0{,}9\,$µA liegt und dies nur im unteren Bereich
der Skala des Messgerätes liegt, konnten die Werte nur auf etwa $0{,}05\,$µA genau
abgelesen werden. Aufgrund dessen zeigen die Messwerte einen lineareren Verlauf
als die Theoriekurve.

Die beiden bestimmten Totzeiten liegen in einer $\sigma$ Umgebung der jeweils
anderen. Dies deutet darauf hin, dass beide Methonden relativ genau sind. Auffällig
ist jedoch der deutlich größere Fehler des Ergebnisses bei der Bestimmung der Totzeit
durch den Vergleich der Zählraten zweier Proben. Hierbei ist jedoch anzumerken,
dass der Fehler bei der Bestimmung der Totzeit mithilfe des Oszilloskops in Wahrheit
auch deutlich höher ist als hier angegeben, da beim Ablesen vom Oszilloskop bereits
Fehler auftreten. Die abgelesenen Werte wurden in der durchgeführten Rechnung als
fehlerlos angesehen.
