\section{Auswertung}
\label{sec:Auswertung}


\subsection{Ablenkung eines Elektronenstrahls im elektrischen Feld}
\label{subsec:efeld}


Zur Auswertung der Ablenkung des Elektronenstrahls werden zunächst die gemessenen Werte
für die Ablenkspannung $U_{\symup{d}}$ gegen die Ablenkung $D$ des Strahls aufgetragen.
Es wurden fünf Messreihen bei jeweils verschiedenen Beschleunigungsspannungen aufgenommen.
Diese betrugen $U_{\symup{b180}}=180$V, $U_{\symup{b230}}=230\,$V, $U_{\symup{b270}}=270\,$V,
$U_{\symup{b300}}=300\,$V und $U_{\symup{b350}}=350\,$V.
Für jede Messreihe wird eine lineare Ausgleichsrechnung der Form
\begin{equation*}
  y=a x + b
\end{equation*}
durchgeführt. Die Graphen sind in den Abbildungen \ref{fig:ablenkspannung} und
\ref{fig:ablenkspannung2} zu finden. Die zugrundeliegenden Messwerte befinden sich
in Tabelle \ref{tab:elektrisch}.

\begin{table}[]
	\begin{center}
    \caption{Messwerte zur Ablenkung des Elektronenstrahls im elektrischen Feld.}
    \label{tab:elektrisch}
		\begin{tabular}{cccccc}
		\toprule
			{$U_{\symup{d180}}/$V} & {$U_{\symup{d230}}/$V} & {$U_{\symup{d270}}/$V} & {$U_{\symup{d300}}/$V} & {$U_{\symup{d350}}/$V} & {$D/$mm}\\
			\midrule
			-17,48 & -22,50 & -28,90 & -29,10 & -34,10 & -24,00\\
			-14,41 & -18,66 & -23,70 & -24,20 & -28,60 & -18,00\\
			-11,30 & -14,47 & -18,36 & -19,12 & -23,10 & -12,00\\
			-7,92 & -10,55 & -13,52 & -13,76 & -16,28 & -6,00\\
			-4,90 & -6,38 & -8,50 & -8,35 & -10,25 & 0,00\\
			-1,74 & -2,54 & -3,44 & -3,11 & -3,86 & 6,00\\
			1,51 & 1,76 & 1,83 & 2,35 & 2,52 & 12,00\\
			5,10 & 6,08 & 7,13 & 7,97 & 8,87 & 18,00\\
			8,23 & 10,20 & 12,70 & 12,93 & 15,25 & 24,00\\
		\bottomrule
		\end{tabular}
	\end{center}
\end{table}

\begin{figure}[]
  \centering
  \includegraphics{build/ablenkspannung.pdf}
  \caption{Messdaten (Kreuze) und Ausgleichsfunktion (Geraden) für die Beschleunigungspannungen
  $U_{\symup{d180}}=180$V (rot), $U_{\symup{d230}}=230$V (blau) und $U_{\symup{d270}}=270$V (grün).}
  \label{fig:ablenkspannung}
\end{figure}

\begin{figure}[]
  \centering
  \includegraphics{build/ablenkspannung2.pdf}
  \caption{Messdaten (Kreuze) und Ausgleichsfunktion (Geraden) für die Beschleunigungspannungen
  $U_{\symup{d300}}=300$V (gelb) und $U_{\symup{d350}}=350$V (violett).}
  \label{fig:ablenkspannung2}
\end{figure}

Die Parameter $a_{\symup{i}}$ der linearen Ausgleichsrechnungen für die jeweiligen
Beschleunigungsspanungen sind
\begin{align*}
  a_{180} = \SI{1.863(0011)e-3}{\meter\per\volt} \,, \\
  a_{230} = \SI{1.466(0007)e-3}{\meter\per\volt}  \,, \\
  a_{270} = \SI{1.163(0007)e-3}{\meter\per\volt}  \,,\\
  a_{300} = \SI{1.131(0006)e-3}{\meter\per\volt}  \,,\\
  a_{350} = \SI{0.963(0009)e-3}{\meter\per\volt} \,.\\
\end{align*}
Diese entsprechen der Empfindlichkeit $D/U_{\symup{d_{\symup{i}}}}$ der Röhre bei
der jeweils eingestellten Beschleunigungsspannung.

Nun werden die errechneten Empfindlichkeiten gegen die reziproken Beschleunigungsspannungen
aufgetragen und es wird erneut eine lineare Ausgleichsrechnung durchgeführt. Der
zugehörige Graph ist in Abbildung \ref{fig:empfindlichkeiten} zu sehen.

\begin{figure}[]
  \centering
  \includegraphics{build/empfindlichkeiten.pdf}
  \caption{Auftragung der zuvor berechneten Empfindlichkeiten gegen die reziproken
  Beschleunigungsspannungen und lineare Ausgleichsrechnung.}
  \label{fig:empfindlichkeiten}
\end{figure}

Der Parameter $a$ der Ausgleichsrechnung beträgt hier
\begin{align*}
  a=\SI{0.3375(00212)}{\meter} \,.
\end{align*}
Der theoretische Wert hierzu beträgt $pL/2d=0.3575\,$m. Der Wert stellt eine
Konstante der Kathodenstrahlröhre dar, der die Abmessungen der einzelnen Bauteile
miteinander in Verbindung setzt. Dabei ist $p$ die
Länge der ablenkenden Kondensatorplatten, $L$ die Strecke vom Kondensator bis zum
Schirm und $d$ der Abstand der Kondensatoplatten zueinander. Die Werte können der
im Anhang beigefügten Skizze entnommen werden. Die relative Abweichung
des durch die Ausgleichsrechnung ermittelten Wertes vom theoretisch berechneten Wert
beträgt $-5.59\%$.


\subsection{Der Kathodenstrahl Oszillograph}
\label{subsec:kathodenstrahloszillograph}

Es wurden die in Tabelle \ref{tab:kat_os} dargestellten Messwerte für die Frequenz
$f$ der Sinusspannung bei den Frequenzverhältnissen $m/n$ aufgenommen.

\begin{table}[h!]
	\begin{center}
    \caption{Messwerte zum Kathodenstrahl Oszillographen.}
    \label{tab:kat_os}
		\begin{tabular}{cc}
		\toprule
			{$m/n$} & {$f/$Hz}\\
			\midrule
      0,5   &   39,94 \\
      1   &   79,87 \\
      2   &   159,70  \\
      3   &   239,50\\
		\bottomrule
		\end{tabular}
	\end{center}
\end{table}
Die Frequenzen werden mit dem Inversen des Frequenzverhältnisses multipliziert und
anschließend gemäß  den Gleichungen \eqref{eqn:mean} und \eqref{eqn:std} gemittelt.
Daraus ergibt sich für die Freqenz der Sinusspannung ein Wert von
\begin{equation*}
  f=\SI{79.86(02)}{\hertz} \,.
\end{equation*}


\subsection{Ablenkung eines Elektronenstrahls im magnetischen Feld}
\label{subsec:bfeld}

Zur Auswertung der Ablenkung des Elektronenstrahls im magnetischen Feld werden
zunächst aus den Messwerten die magnetische Flussdichte $B$ des Helmholtz-Spulenpaares
gemäß \eqref{eqn:helmholtz} und der Radius $r$ der Elektronenbahn nach \eqref{eqn:r} berechnet.
Gemessen wurde mit Beschleunigungspannungen von $U_{\symup{b250}}=250$V und
$U_{\symup{b400}}=400$V. Die Messwerte und
die berechneten Werte befinden sich in Tabelle \ref{tab:magnetisch}.

\begin{table}[htp]
	\begin{center}
    \caption{Messwerte zur Ablenkung des Elektronenstrahls im magnetischen Feld und
    daraus berechnete Werte bei einer Beschleunigungsspannung von
    $U_{\symup{b250}}=250$V und $U_{\symup{b400}}=400$V.}
    \label{tab:magnetisch}
		\begin{tabular}{cccccc}
		\toprule
			{$I_{250}/$A} & {$B_{250}/$µT} & {$I_{400}/$A} & {$B_{400}/$µT} & {$D/$mm} & {$r/$m}\\
			\midrule
			0,00 & 0,00  & 0,00 &  0,00 & 24,00 & -0,77\\
			0,29 & 18,49 & 0,46 & 29,33 & 18,00 & -0,58\\
			0,70 & 44,64 & 0,86 & 54,84 & 12,00 & -0,39\\
			1,00 & 63,77 & 1,25 & 79,71 & 6,00 & -0,20\\
			1,25 & 79,71 & 1,60 & 102,03 & 0,00 & 0,00\\
			1,60 & 102,03 & 2,00 & 127,54 & -6,00 & 0,20\\
			1,95 & 124,35 & 2,40 & 153,05 & -12,00 & 0,39\\
			2,23 & 141,89 & 2,85 & 181,75 & -18,00 & 0,58\\
			2,55 & 162,62 & 3,25 & 207,26 & -24,00 & 0,77\\
		\bottomrule
		\end{tabular}
	\end{center}
\end{table}

Nun wird der Radius $r$ der Elektronenbahn gegen die magnetische Feldstärke $B$
des Spulenpaares aufgetragen und eine lineare Ausgleichsrechnung durchgeführt.
Der zughörige Graph ist in Abbildung \ref{fig:bfeld} zu sehen.

\begin{figure}[h!]
  \centering
  \includegraphics{build/bfeld.pdf}
  \caption{Messdaten (Kreuze) und Ausgleichsfunktion (Geraden) für die Beschleunigungspannungen
  $U_{\symup{b250}}=250$V (blau) und $U_{\symup{b400}}=400$V (rot).}
  \label{fig:bfeld}
\end{figure}

Die Parameter $a_{\symup{i}}$ der Ausgleichsrechnung betragen hier
\begin{align*}
  a_{250} = \SI{9.504(0134)e3}{\meter\per\tesla}  \,, \\
  a_{400} = \SI{7.570(0088)e3}{\meter\per\tesla}  \,.
\end{align*}
Mithilfe dieser Werte soll nun unter Benutzung von Gleichung \eqref{eqn:elektronenladung}
die spezifische Elektronenladung $e/m_{\symup{e}}$ bestimmt werden. Es folgt der
Zusammenhang
\begin{equation*}
  \frac{e}{m_{\symup{e}}}= 8 a_{\symup{i}}^2 U_{\symup{bi}} \,.
\end{equation*}
Damit ergibt sich für die experimentell bestimmten spezifischen Elektronenladungen
\begin{align*}
  \left(\frac{e}{m_{\symup{e}}}\right)_{250}=\SI{1.81(005)e11}{\coulomb\per\kilo\per\gram}\,,\\
  \left(\frac{e}{m_{\symup{e}}}\right)_{400}=\SI{1.83(004)e11}{\coulomb\per\kilo\per\gram}\,.
\end{align*}

Der Literaturwert für die spezifische Elektronen beträgt
$\left(\frac{e}{m_{\symup{e}}}\right)_{\symup{lit}}=\SI{1.76e11}{\coulomb\per\kilo\per\gram}$
\cite{elektronenladung}.
Die relativen Abweichungen der ermittelten Ergebnisse vom Literaturwert betragen $2,84\%$ und
$3,97\%$.

\subsection{Bestimmung des Erdmagnetfeldes}
\label{subsec:erdmagnetfeld}

Der gemessene Winkel, den das Erdmagnetfeld mit der horizontalen einschließt, beträgt
$\phi=\SI{72}{\degree}$. Der notwendige Strom, um ein Gegenfeld mit dem Helmholtzspulenpaar
zu erzeugen, beträgt laut Messung $I_{\symup{gegen}}=\SI{50}{\milli\ampere}$. Nach Gleichung
\eqref{eqn:helmholtz} wird die magnetische Feldstärke für das Gegenfeld in horizontale
Richtung zu $B_{\symup{hor}}=\SI{3.18}{\micro\tesla}$ bestimmt. Mithilfe des Zusammenhanges
\begin{equation*}
  B_{\symup{ges}}=\frac{B_{\symup{hor}}}{\cos(\phi)}
\end{equation*}
lässt sich die gesamte magnetische Feldstärke des Erdmagnetfeldes zu
$B_{\symup{ges}}=\SI{10.32}{\micro\tesla}$ bestimmen. Der Literaturwert beträgt
$B_{\symup{ges}}=\SI{48.5}{\micro\tesla}$ \cite{erdmagnetfeld}.
Die relative Abweichung beträgt $-78,7\%$.
