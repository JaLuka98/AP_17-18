\section{Auswertung}
\label{sec:Auswertung}


\subsection{Ablenkung eines Elektronenstrahls im elektrischen Feld}
\label{subsec:efeld}


Zur Auswertung der Ablenkung des Elektronenstrahls werden zunächst die gemessenen Werte
für die Ablenkspannung $U_{\symup{d}}$ gegen die Ablenkung $D$ des Strahls aufgetragen.
Es wurden fünf Messreihen bei jeweils verschiedenen Beschleunigungsspannungen aufgenommen.
Diese betrugen $U_{\symup{d1}}=180$V, $U_{\symup{d12}}=230\,$V, $U_{\symup{d3}}=270\,$V,
$U_{\symup{d4}}=300\,$V und $U_{\symup{d5}}=300\,$V.
Für jede Messreihe wird eine lineare Ausgleichsrechnung der Form
\begin{equation*}
  y=a x + b
\end{equation*}
durchgeführt. Die Graphen sind in den Abbildungen \ref{fig:ablenkspannung} und
\ref{fig:ablenkspannung2} zu finden. Die zugrundeliegenden Messwerte befinden sich
und Tabelle \ref{tab:elektrisch}.

\begin{table}[htp]
	\begin{center}
    \caption{Messwerte zur Ablenkung des Elektronenstrahls im elektrischen Feld.}
    \label{tab:elektrisch}
		\begin{tabular}{cccccc}
		\toprule
			{$U_{\symup{d1}}/$V} & {$U_{\symup{d2}}/$V} & {$U_{\symup{d3}}/$V} & {$U_{\symup{d4}}/$V} & {$U_{\symup{d5}}/$V} & {$D/$mm}\\
			\midrule
			-17,48 & -22,50 & -28,90 & -29,10 & -34,10 & -24,00\\
			-14,41 & -18,66 & -23,70 & -24,20 & -28,60 & -18,00\\
			-11,30 & -14,47 & -18,36 & -19,12 & -23,10 & -12,00\\
			-7,92 & -10,55 & -13,52 & -13,76 & -16,28 & -6,00\\
			-4,90 & -6,38 & -8,50 & -8,35 & -10,25 & 0,00\\
			-1,74 & -2,54 & -3,44 & -3,11 & -3,86 & 6,00\\
			1,51 & 1,76 & 1,83 & 2,35 & 2,52 & 12,00\\
			5,10 & 6,08 & 7,13 & 7,97 & 8,87 & 18,00\\
			8,23 & 10,20 & 12,70 & 12,93 & 15,25 & 24,00\\
		\bottomrule
		\end{tabular}
	\end{center}
\end{table}

\begin{figure}[h]
  \centering
  \includegraphics{build/ablenkspannung.pdf}
  \caption{Messdaten und Ausgleichsfunktion für die Beschleunigungspannungen
  $U_{\symup{d1}}=180$V (rot), $U_{\symup{d2}}=230$V (blau)und $U_{\symup{d3}}=270$V (grün).}
  \label{fig:ablenkspannung}
\end{figure}

\begin{figure}[h]
  \centering
  \includegraphics{build/ablenkspannung2.pdf}
  \caption{Messdaten und Ausgleichsfunktion für die Beschleunigungspannungen
  $U_{\symup{d4}}=300$V (gelb) und $U_{\symup{d5}}=350$V (violett).}
  \label{fig:ablenkspannung2}
\end{figure}

Die Parameter $a_{\symup{i}}$ der linearen Ausgleichsrechnungen sind
\begin{align*}
  HIER WERTE EINFÜGEN!
\end{align*}
Diese entsprechen der Empfindlichkeit $D/U_{\symup{d_{\symup{i}}}}$ der Röhre bei
der jeweils eingestellten Beschleunigungsspannung.

Nun werden die errechneten Empfindlichkeiten gegen die reziproken Beschleunigungsspannungen
aufgetragen und es wird erneut eine lineare Ausgleichsrechnung durchgeführt. Der
zugehörige Graph ist in Abbildung \ref{fig:empfindlichkeiten} zu sehen.

\begin{figure}[h]
  \centering
  \includegraphics{build/empfindlichkeiten.pdf}
  \caption{Auftragung der zuvor berechneten Empfindlichkeiten gegen die reziproken
  Beschleunigungsspannungen und lineare Ausgleichsrechnung.}
  \label{fig:empfindlichkeiten}
\end{figure}

Der Parameter $a$ der Ausgleichsrechnung beträgt hier
\begin{align*}
  HIER WERTE EINFUEGEN \,.
\end{align*}




\subsection{Ablenkung eines Elektronenstrahls im magnetischen Feld}
\label{subsec:bfeld}

Zur Auswertung der Ablenkung des Elektronenstrahls im magnetischen Feld werden
zunächst aus den Messwerten die magnetische Feldstärke $B$ des Helmholtz-Spulenpaares
gemäß (REFERENZ) und der Radius $r$ der Elektronenbahn nach (REFERENZ) berechnet. Die Messwerte und
die berechneten Werte befinden sich in Tabelle \ref{tab:magnetisch}.

\begin{table}[htp]
	\begin{center}
    \caption{Messwerte zur Ablenkung des Elektronenstrahls im magnetischen Feld und
    daraus berechnete Werte.}
    \label{tab:magnetisch}
		\begin{tabular}{cccccc}
		\toprule
			{$I_1/$A} & {$B_1/$µT} & {$I_2/$A} & {$B_2/$µT} & {$D/$mm} & {$r/$m}\\
			\midrule
			0,00 & 0,00  & 0,00 &  0,00 & 24,00 & -0,77\\
			0,29 & 18,49 & 0,46 & 29,33 & 18,00 & -0,58\\
			0,70 & 44,64 & 0,86 & 54,84 & 12,00 & -0,39\\
			1,00 & 63,77 & 1,25 & 79,71 & 6,00 & -0,20\\
			1,25 & 79,71 & 1,60 & 102,03 & 0,00 & 0,00\\
			1,60 & 102,03 & 2,00 & 127,54 & -6,00 & 0,20\\
			1,95 & 124,35 & 2,40 & 153,05 & -12,00 & 0,39\\
			2,23 & 141,89 & 2,85 & 181,75 & -18,00 & 0,58\\
			2,55 & 162,62 & 3,25 & 207,26 & -24,00 & 0,77\\
		\bottomrule
		\end{tabular}
	\end{center}
\end{table}

Nun wird der Radius $r$ der Elektronenbahn gegen die magnetische Feldstärke $B$
des Spulenpaares aufgetragen und eine lineare Ausgleichsrechnung durchgeführt.
Der zughörige Graph ist in Abbildung \ref{fig:bfeld} zu sehen.

\begin{figure}[h]
  \centering
  \includegraphics{build/bfeld.pdf}
  \caption{Messdaten und Ausgleichsfunktion für die Beschleunigungspannungen
  $U_{\symup{d1}}=250$V (blau) und $U_{\symup{d2}}=400$V (rot).}
  \label{fig:bfeld}
\end{figure}
