\section{Diskussion}
\label{sec:Diskussion}

Im Allgemeinen ist festzustellen, dass die Messreihe zur Bestimmung der Röhrenkonstante
der Kathodenstrahlröhre und die Messung zur Bestimmung der spezifischen Elektronenladung
gute Werte liefern, die nah am Theorie- bzw. Literaturwert liegen. Lediglich die
Messung des Erdmagnetfeldes ist als sehr ungenau zu bewerten.

Als mögliche Fehlerquelle an der Kathodenstrahlröhre ist anzuführen, dass nie
ein idealer Punkt auf dem Schirm sichtbar wurde, sodass nicht genau festgestellt
werden kann, bei welcher Ablenkspannung der Strahl genau auf eine Linie fällt.
Zudem wird der Strahl durch die Ablenkung aufgefächert, was zu Unsicherheiten
in den Randbereichen führt.

Desweiteren ist anzumerken, dass die beiden Messreihen, die zur Bestimmung der
spezifischen Elektronenladung verwendet wurden, ohne eine Ausrichtung der Apparatur
in Richtung des Erdmagnetfeldes aufgenommen wurden. Dies führt zu geringen
systematischen Fehlern im Versuchsaufbau.

Der stark abweichende Wert für die Intensität des Erdmagnetfeldes ist darauf zurückzuführen,
dass die Bestimmung der Himmelsrichtungen im Versuch mithilfe des Deklinatorium-Inklinatoriums
nur sehr ungenau möglich war. Eine Kompass-App auf dem Smartphone lieferte ebenfalls
keine zuverlässigen Ergebnisse.
