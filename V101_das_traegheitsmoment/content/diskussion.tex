\section{Diskussion}
\label{sec:Diskussion}

Die Methode der Bestimmung des Trägheitsmoments durch Messung der Schwingungsdauer
der Figur auf einer mit einer Spiralfeder verbundenen Drillachse ist als ungenau
zu bewerten.
Die Werte für den Zylinder und die Kugel sind mit ... und ... negativ. Dies
deutet auf einen Fehler bei der Messung hin, da Trägheitsmomente skalare Größen sind
und somit in der Theorie keine negativen Werte annehmen können. Der Fehler entsteht
hierbei vermutlich bei der Messung des Trägheitsmomentes der Drillachse, das am Ende
von jedem gemessenen Trägheitsmoment subtrahiert werden muss. Ungenauigkeiten entstehen
hierbei dadurch, dass die Stange als masselos und die aufgesteckten Gewichte als
Punktmassen angenommen werden. Zudem ist die Schwingungsdauer bei diesem Versuch
im Vergleich zu den anderen Schwingungsdauern relativ groß, sodass hier der Umkehrpunkt
der Pendelbewegung nicht genau bestimmt werden kann, was zu weiteren Ungenauigkeiten führen kann.

Ein weiterer Faktor, der für die starke Abweichungden der Ergebnisse von den
Theoriewerten verantwortlich sein könnte, ist die Messung des Direktionsmomentes
der Drillachse. Zum einen kann es hier zu Ungenauigkeiten dadurch kommen, dass der
Kraftmesser nicht genau in Richtung des $e_\mathrm{\phi}$ Vektors zeigt, und zum
Anderen dadurch, dass kein perfekt linearer Zusammenhang zwischen Auslenkung und Kraft
besteht.

Bei der Messung des Trägheitsmoments der Puppe entstehen zudem weitere systematische
Unsicherheiten dadurch, dass die Puppe während der Messung nicht immer in der perfekt
gleichen Position bleibt. Die ausgestreckten Körperteile der Puppe sacken während des
Messvorgangs geringfügig ab, sodass sie sich nicht perfekt um ihre in der Theorie
angenommenen Achsen drehen.

Desweiteren stand die Puppe während des gesamten Versuchs leicht schief. Eine Messnung mit
angelegten Armen und Beinen war gar nicht möglich, da die Puppe nur "geeiert" ist. Das
bedeutet, dass sie sich während des gesamten versuchst nicht perfekt um eine Achse durch
ihren Schwerpunkt bgedreht hat, sondern um eine leicht zu dieser verschobenen Achse.
Somit sind auch hierfür die theoretischen Annahmen für die Drehachsen nicht genau.




Skizze:
\begin{itemize}
  \item Systematischer Fehler durch Annahme der masselosen Stange
  \item systematischer Fehler: Körperhaltung der Puppe nicht ideal konstant (leichtes
  absacken der Arme und Beine während der Schingung)
  \item Puppe stand leicht schief -> Drehachse ggf. nicht genau durch Schwerpunkt

\end{itemize}
