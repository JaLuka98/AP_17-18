\section{Diskussion}
\label{sec:Diskussion}

Die Methode der Bestimmung des Trägheitsmoments durch Messung der Schwingungsdauer
der Figur auf einer mit einer Spiralfeder verbundenen Drillachse ist als ungenau
zu bewerten.
Die Werte für den Zylinder und die Kugel sind mit ... und ... negativ. Dies
deutet auf einen Fehler bei der Messung hin, da Trägheitsmomente positive Werte haben. Der Fehler entsteht
hierbei vermutlich bei der Messung des Trägheitsmomentes der Drillachse, das am Ende
von jedem gemessenen Trägheitsmoment subtrahiert werden muss. Wie in der Auswertung
bereits angemerkt ergeben sich ohne diese Subtraktion konsistentere Werte. Ungenauigkeiten entstehen
hierbei dadurch, dass die Stange als masselos und die aufgesteckten Gewichte als
Punktmassen angenommen werden. Zudem ist die Schwingungsdauer bei diesem Versuch
im Vergleich zu den anderen Schwingungsdauern relativ groß, sodass hier der Umkehrpunkt
der Pendelbewegung nicht genau bestimmt werden kann, was zu weiteren Ungenauigkeiten führt.
Desweiteren sind die Massen der angehängten Gewichte leicht unterschiedlich, sodass
die Drehachse nicht genau durch den Schwerpunkt des Systems geht.

Ein weiterer Faktor, der für die starke Abweichungden der Ergebnisse von den
Theoriewerten verantwortlich sein könnte, ist die Messung des Direktionsmomentes
der Drillachse. Zum Einen kann es hier zu Ungenauigkeiten dadurch kommen, dass der
Kraftmesser nicht genau in Richtung des $e_\mathrm{\phi}$ Vektors zeigt, und zum
Anderen dadurch, dass kein perfekt linearer Zusammenhang zwischen Auslenkung und Kraft
besteht.

Die Messergebnisse für die Trägheitsmomente der Puppe weichen in beiden Haltungen
mit... und .... sehr stark von den theoretischen Erwartungswerten ab.

Bei der Messung enstehen systematische Fehler dadurch, dass die Puppe während
der Messung nicht immer in der gleichen Position bleibt. Die ausgestreckten
Körperteile der Puppe sacken während des Messvorgangs geringfügig ab, sodass sie sich
nicht perfekt um ihre in der Theorie angenommene Achse drehen.

Desweiteren stand die Puppe während des gesamten Versuchs leicht schief. Eine Messung mit
angelegten Armen und Beinen war gar nicht möglich, da die Puppe so nicht stabil rotierte.
Während des gesamten Versuchs rotierte die Puppe also nicht genau um eine Achse durch
ihren Schwerpunkt, sondern um eine leicht zu dieser verschobenen Achse.
Somit sind auch hierfür die theoretischen Annahmen für die Drehachsen nicht genau.

Eine weitere Unsicherheit stellt die Dichte des Materials der Puppe dar. Farblich
weist das Holz eine starke Ähnlichkeit zu Ahornholz auf, andere Holzarten sind jedoch
auch nicht auszuschließen.

Im Allgemeinen ist der theoretische Erwartungswert bei dieser Methode sehr ungenau,
da die Puppe als aus einer Kugel, einem Zylinder als Rumpf, zwei Zylindern als Arme
und zwei Zylindern als Beine angenommen wird. In der Realität ist jedoch die Form deutlich
komplexer. Zudem ergeben sich, wenn man für die Masse über das Volumen und die Dichte bestimmt,
für die einzelnen Körperteile starke Abhängigkeiten vom Radius von $r^5$ für den Kopf
und $r^4$ für die restlichen Körperteile. Eine geringfügige Änderung des Radiusses
kann also enorme Änderungen in dem berechneten Theoriewert hervorrufen. Da die gemessenen
Durchmesser nur geschätzte Mittelwerte für das jeweilige Körperteil sind, entstehen
hier bereits bei der Berechnung des Theoriewertes sehr große systematische Fehler.
