\section{Durchführung}
\label{sec:Durchführung}

Zur Bestimmung der Trägheitsmomente der verschiedenen Körper müssen zunächst die
Apparaturkonstanten, also die Winkelrichtgröße $\symbf{D}$ und das Trägheitsmoment
der Drillachse $\symbf{I_D}$, bestimmt werden.

Zur Bestimmung der Winkelrichtgröße wird ein Stab in die Apparatur eingeschraubt und
um verschiedene Winkel $\phi$ ausgelenkt. Mit einer Federwaage wird die im Abstand
$\symbf{r}$ wirkende rücktreibende Kraft gemessen. Dabei muss darauf geachtet werden,
dass der Auslenkungswinkel nicht größer als 360° ist, um inelastische Verformungen der
Spiralfeder zu vermeiden, und darauf, dass die Kraft immer in Richtung des Einheitsvektors
$\symbf{e_{\phi}}$ gemessen wird, da ansonsten der Zusammenhang aus \eqref{eqn:winkelrg}
nicht mehr gilt.
