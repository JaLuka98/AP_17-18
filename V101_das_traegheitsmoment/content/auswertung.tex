\section{Auswertung}
\label{sec:Auswertung}
\subsection{Bestimmung der Apparatekonstanten}
Im Folgenden werden die Winkelrichtgröße $D$ der Spiralfeder und das
Eigenträgheitsmoment $I_{\symup{D}}$ der Drillachse aus den aufgenommenen
Messwerten berechnet.
\subsubsection{Bestimmung der Winkelrichtgröße}
Die Messwerte, die zur Bestimmung der Winkelgrichtgröße $D$ herangezogen werden,
sind in Tabelle \ref{tab:winkelrichtgroesse} zusammengefasst. Bei der Messung wurde die
Federwaage in dem konstanten Abstand $r=\SI{9.1}{cm}$ angesetzt.

\begin{table}
\centering
\caption{Messdaten zur Bestimmung der Winkelrichtgröße}
\label{tab:winkelrichtgroesse}
\begin{tabular}{c c}
\toprule
$\varphi/$grad & $F/$N \\
\midrule
 60	& 0.19 \\
 60	& 0.19 \\
 90	& 0.31 \\
 90	& 0.29 \\
120 &	0.41 \\
120 &	0.41 \\
150 &	0.60 \\
150 &	0.61 \\
180 &	0.73 \\
180 &	0.75 \\
210 &	0.89 \\
210 &	0.88 \\
240 &	1.01 \\
240 &	1.02 \\
270 &	1.13 \\
270 &	1.19 \\
300 &	1.39 \\
300 &	1.31 \\
330 &	1.43 \\
330 &	1.44 \\
\bottomrule
\end{tabular}
\end{table}

Gemäß \eqref{eqn:winkelrg} kann die Winkelrichtgröße nun bestimmt werden, indem
die aufgenommenen Messwerte nach \eqref{eqn:mean} gemittelt werden und die
Abweichung gemäß \eqref{eqn:std} berechnet wird. Dies ergibt den Wert
\begin{equation}
  D = \SI{0.0207(0024)}{\newton\meter}\,,
\end{equation}
wobei die relative Unsicherheit 11,6\% ist.
\subsubsection{Bestimmung des Eigenträgheitsmoments der Drillachse}
Zuerst werden aus der Periodendauer $T$ der Schwingung der Massen und ihrem
Abstand $r$ von der Drehachse ihre Quadrate gebildet, um eine lineare Regression
von $T^2$ gegen $r^2$ zu bilden. Das ist möglich, da nach der Schwingungsgleichung
\eqref{eqn:schwingung} und $I=I_{\symup{D}}+2\overline{m}r^2$ durch Anwendung des
Steiner'schen Satzes \eqref{eqn:steiner}
\begin{equation}
  T^2 = \frac{4\pi^2}{D}(I_{\symup{D}}+\overline{m}r^2)\propto r^2
  \label{eqn:drillachsengleichung}
\end{equation}
gilt. Hierbei ist
\begin{equation}
  \overline{m} = \frac{m_1+m_2}{s} =\SI{227.45}{\gram}
\end{equation}
der Mittelwert der angehängten Massen, sodass \eqref{eqn:drillachsengleichung}
kompakt geschrieben werden kann. $r$ ist der Abstand bis zum Schwerpunktes der
als Punktmassen angenommenen Zylinder, der sich nach
\begin{equation}
  r = r_{\symup{bis Zylinder}}
      +\frac{h_{\symup{zylinder}}}{2}
\end{equation}
mit $h_{\symup{zylinder}} = \SI{2.98}{cm}$ berechnet.
Die lineare Regression der Werte aus Tabelle \ref{tab:drillachse} ist in
Abbildung \ref{fig:drillachse} zu sehen.

\begin{table}
\centering
\caption{Messdaten zur Bestimmung des Eigenträgheitsmoments der Drillachse}
\label{tab:drillachse}
\begin{tabular}{c c c c}
\toprule
$r/$cm & $T/$s & $r^2/$cm² & $T^2$/s² \\
\midrule
 1.96 & 2.38 &   3.84 &  5.66 \\
 3.96 & 2.69 &  15.68 &  7.24 \\
 5.97 & 2.98 &  35.58 &  8.88 \\
 7.97 & 3.31 &  63.52 & 10.99 \\
 9.97 & 3.85 &  99.40 & 14.82 \\
11.96 & 4.24 & 143.04 & 17.94 \\
13.97 & 4.80 & 195.16 & 23.04 \\
16.00 & 5.33 & 256.00 & 28.46 \\
17.95 & 5.75 & 322.21 & 33.01 \\
19.98 & 6.29 & 399.00 & 39.56 \\
\bottomrule
\end{tabular}
\end{table}

\begin{figure}
  \centering
  \includegraphics{build/Drillachse.pdf}
  \caption{Graph von $T^2$ gegen $r^2$ und Fit}
  \label{fig:drillachse}
\end{figure}

Die Regressionsgerade folgt der allgemeinen Geradengleichung
\begin{equation}
  T^2(r^2) = ar^2+b\,.
  \label{eqn:gerade}
\end{equation}
Die Regressionsparameter $a$ und $b$ ergeben sich hierbei zu
\begin{align}
  a &= \SI{757.48(797)}{\second\squared/\meter\squared}\,,\\
  b &= \SI{4.70(019)}{\second\squared}\,.
\end{align}
Durch Kombination von \eqref{eqn:gerade} mit \eqref{eqn:drillachsengleichung},
folgt für das Eigenträgheitsmoment $I_{\symup{D}}$ der Drillachse
\begin{equation}
  I_{\symup{D}} = \frac{2b\overline{m}}{a} = \SI{2.82(012)e-3}{\kilogram\meter\squared}
\end{equation}
mit einer relativen Unsicherheit von 4,3\%.
\subsection{Bestimmung der Trägheitsmomente zweier Körper}
\subsubsection{Bestimmung des Trägheitsmoments eines Zylinders}
Die Abmessungen des untersuchten Zylinders sind $h_{\symup{z}}=\SI{13.99}{\cm}$ und
$d_{\symup{z}}=\SI{10}{\cm}$, seine Masse $m_{\symup{z}}=\SI{2.3959}{\kilogram}$.
Die gemessenen Periodendauern sind in \ref{tab:zylinder} zusammengefasst.

\begin{table}[!h]
\centering
\caption{Periodendauern der Rotation des Zylinders}
\label{tab:zylinder}
\begin{tabular}{c c}
\toprule
$3T$/s & $T/$s \\
\midrule
6.73 & 2.24 \\
6.72 & 2.24 \\
6.80 & 2.27 \\
6.73 & 2.24 \\
6.73 & 2.24 \\
\bottomrule
\end{tabular}
\end{table}

Durch ihre Mittelung auf $\overline{T_{\text{z}}}=\SI{2.25(001)}{\second}$ und
\eqref{eqn:traegheitschwingung} lässt sich
\begin{equation}
  I_{\symup{z,exp}} = \frac{D\overline{T_{\text{z}}}^2}{4\pi^2}-I_{\symup{D}} = \SI{-0.18(033)e-3}{\kilogram\meter\squared}
\end{equation}
berechnen. Der theoretisch berechnete Wert für das Trägheitsmoment eines Zylinders
bei Rotation um die Symmetrieachse ist nach \eqref{eqn:traegheitzylinder}
\begin{equation}
  I_{\symup{z,theo}} = \SI{2.99e-3}{\kilogram\meter\squared}\,.
\end{equation}
Es ist auffällig, dass eine Berechnung nach \eqref{eqn:schwingung} zu einem
konsistenteren Wert von \SI{2.64(033)e-3}{\kilogram\meter\squared} mit einem
relativen Fehler zum Theoriewert von 11.7\% führt. Diese alternative Berechnung
bedeutet anschaulich eine Vernachlässigung des Eigenträgheitsmoments $I_{\symup{D}}$
der Drillachse.

\subsubsection{Bestimmung des Trägheitsmoments einer Kugel}
Der Durchmesser der untersuchten Kugel beträgt $d_{\symup{k}}=\SI{13.724(0014)}{\cm}$,
ist also nur minimal fehlerbehaftet. Die Masse der Kugel wurde bestimmt zu
$m_{\symup{k}}=\SI{812.5}{\gram}$. Die Messwerte der Periodendauern finden sich
in \ref{tab:kugel}, ihr Mittelwert ist $\overline{T_{\text{k}}}=\SI{1.69(002)}{\second}$.

\begin{table}
\centering
\caption{Periodendauern der Rotation der Kugel}
\label{tab:kugel}
\begin{tabular}{c c}
\toprule
$3T$/s & $T/$s \\
\midrule
5.03 & 1.68 \\
5.06 & 1.69 \\
5.09 & 1.70 \\
5.09 & 1.70 \\
5.24 & 1.75 \\
5.01 & 1.67 \\
5.12 & 1.71 \\
5.13 & 1.71 \\
4.96 & 1.65 \\
5.07 & 1.69 \\
\bottomrule
\end{tabular}
\end{table}

Wird erneut \eqref{eqn:traegheitschwingung} verwendet, ergibt sich
\begin{equation}
  I_{\symup{k,exp}} = \frac{D\overline{T_{\text{k}}}^2}{4\pi^2}-I_D = \SI{-1.32(021)e-3}{\kilogram\meter\squared}\,.
\end{equation}
Für das Trägheitsmoment der Kugel gemäß \eqref{eqn:traegheitkugel} folgt
\begin{equation}
  I_{\symup{k,theo}} = \SI{1.530(0003)e-3}{\kilogram\meter\squared}\,.
\end{equation}
Wird stattdessen der experimentelle Wert mit \eqref{eqn:schwingung} berechnet,
ergibt er sich zu \SI{1.50(018)e-3}{\kilogram\meter\squared} mit einem relativen
Fehler von 1,96\%.

\subsection{Bestimmung des Trägheitsmoments einer Modellpuppe}
Zuletzt ist das Trägheitsmoment einer Modellpuppe in zwei verschiedenen
Körperhaltungen experimentell zu bestimmen und mit theoretischen Berechnungen zu
vergleichen. Die Körperhaltungen sind dabei in \ref{sec:Durchführung} beschrieben.
Die Maße der Körperteile, die zur theoretischen Berechnung der Trägheitsmomente
herangezogen wurden, sind im Anhang zu finden. Um Massen berechnen zu können,
wurde für das Holz eine Dichte von $\rho=\SI{650}{\kilogram\per\meter\cubed}$
angenommen.
\subsubsection{Bestimmung des Trägheitsmoments der Puppe in der ersten Stellung}
\begin{table}
Die Messreihe findet sich in \ref{tab:puppe1}.

\centering
\caption{Periodendauern der Rotation der Puppe in erster Haltung}
\label{tab:puppe1}
\begin{tabular}{c c}
\toprule
$3T$/s & $T/$s \\
\midrule
2.30 & 0.77 \\
2.36 & 0.79 \\
2.38 & 0.79 \\
2.40 & 0.80 \\
2.29 & 0.76 \\
2.30 & 0.77 \\
2.40 & 0.80 \\
2.38 & 0.79 \\
2.33 & 0.78 \\
2.32 & 0.77 \\
\bottomrule
\end{tabular}
\end{table}

Daraus ergibt sich nach Mittelung der Periodendauern
$\overline{T_{\text{p1}}}=\SI{0.78(001)}{\second}$. Setzt man diesen Wert in
\eqref{eqn:traegheitschwingung} ein, erhält man für das Trägheitsmoment
\begin{equation}
  I_{\symup{p1,exp}} = \frac{D\overline{T_{\text{p1}}}^2}{4\pi^2}-I_D = \SI{-2.50(012)e-3}{\kilogram\meter\squared}\,.
\end{equation}
**************************************
HIER NOCH VERGLEICH MIT THEORIEWERT EINFÜGEN
**************************************
In Kapitel \ref{sec:Theorie} wurde bereits in \eqref{eqn:traegheitpuppe1} die
Formel für den theoretischen Wert des Trägheitsmoments angegeben.














\subsubsection{Bestimmung des Trägheitsmoments der Puppe in der zweiten Stellung}
\begin{table}
Die gemittelte Periodendauer der Rotation der zweiten Puppe lässt sich aus
\ref{tab:puppe2} errechnen zu $\overline{T_{\text{p2}}}=\SI{0.96(001)}{\second}$.

\centering
\caption{Periodendauern der Rotation der Puppe in zweiter Haltung}
\label{tab:puppe2}
\begin{tabular}{c c}
\toprule
$3T$/s & $T/$s \\
\midrule
2.93 & 0.98 \\
2.83 & 0.94 \\
2.86 & 0.95 \\
2.90 & 0.97 \\
2.86 & 0.95 \\
2.87 & 0.96 \\
2.90 & 0.97 \\
2.83 & 0.94 \\
2.84 & 0.95 \\
2.83 & 0.94 \\
\bottomrule
\end{tabular}
\end{table}
Schlussendlich folgt für das Trägheitsmoment in der zweiten Haltung nach Einsetzen
der gemittelten Periodendauer in \eqref{eqn:traegheitschwingung}
\begin{equation}
  I_{\symup{p2,exp}} = \frac{D\overline{T_{\text{p2}}}^2}{4\pi^2}-I_D = \SI{-2.35(001)e-3}{\kilogram\meter\squared}\,.
\end{equation}
***********************************
HIER NOCH VERGLEICH MIT THEORIEWERT EINFÜGEN
**********************************
