\section{Auswertung}
\subsection{Bestimmung der Apparatekonstanten}
\subsubsection{Bestimmung der Winkelrichtgröße}
Die Messwerte, die zur Bestimmung der Winkelgrichtgröße $D$ herangezogen werden,
sind in \ref{tab:winkelrichtgroesse} zusammengefasst. Bei der Messung wurde die
Federwaage in dem konstanten Abstand $r=\SI{9.1}{cm}$.
\begin{table}
\centering
\caption{Messdaten zur Bestimmung der Winkelrichtgröße}
\label{tab:winkelrichtgroesse}
\begin{tabular}{c c c}
\toprule
$\varphi/$grad & $r/$cm & $F/$N \\
\midrule
 60	& 0.19 \\
 60	& 0.19 \\
 90	& 0.31 \\
 90	& 0.29 \\
120 &	0.41 \\
120 &	0.41 \\
150 &	0.60 \\
150 &	0.61 \\
180 &	0.73 \\
180 &	0.75 \\
210 &	0.89 \\
210 &	0.88 \\
240 &	1.01 \\
240 &	1.02 \\
270 &	1.13 \\
270 &	1.19 \\
300 &	1.39 \\
300 &	1.31 \\
330 &	1.43 \\
330 &	1.44 \\
\bottomrule
\end{tabular}
\end{table}
Gemäß \eqref{eqn:winkelrg} kann die Winkelrichtgröße nun bestimmt werden, indem
die aufgenommenen Messwerte nach \eqref{eqn:mean} gemmittelt werden und die
Abweichung gemäß \eqref{eqn:std} berechnet wird. Dies ergibt den Wert
\begin{equation}
  D = \SI{0.0207(0024)}{\newton\meter},
\end{equation}
wobei die relative Unsicherheit 11,6\% ist.
\subsubsection{Bestimmung des Eigenträgheitsmoments der Drillachse}
Zuerst werden aus der Periodendauer $T$ der Schwingung der Massen und ihrem
Abstand $r$ von der Drehachse ihre Quadrate gebildet, um eine lineare Regression
von $T^2$ gegen $r^2$ zu bilden. Das ist möglich, da nach der Schwingungs
\eqref{eqn:schwingung} und $I=I_D+2\overline{m}r^2$ durch Anwendung des
Steiner'schen Satzes \eqref{eqn:steiner}
\begin{equation}
  T^2 = \frac{4\pi^2}{D}(I_D+\overline{m}r^2)\propto r^2
  \label{eqn:schwingungkonkret}
\end{equation}
gilt. Hierbei ist
\begin{equation}
  \overline{m} = \frac{m_1+m_2}{s} =\SI{227.45}{\gram}
\end{equation}
der Mittelwert der angehängten Massen, sodass \eqref{eqn:schwingungkonkret}
kompakt geschrieben werden kann. $r$ ist der Abstand bis zum Schwerpunktes der
als Punktmassen angenommenen Zylinder, der sich nach
\begin{equation}
  r = r_{\symup{bis Zylinder}}
      +\frac{h_{zylinder}}{2}
\end{equation}
mit $h_{\symup{zylinder}} = \SI{2.98}{cm}$ berechnet.
Die lineare Regression der Werte aus \ref{tab:drillachse} ist in
\ref{fig:drillachse} zu sehen.

\begin{table}
\centering
\caption{Messdaten zur Bestimmung Eigenträgheitsmoments der Drillachse}
\label{tab:drillachse}
\begin{tabular}{c c c c}
\toprule
$r/$cm & $T/$s & $r^2/$cm² & $T^2$/s² \\
\midrule
 1.96 & 2.38 &   3.84 &  5.66 \\
 3.96 & 2.69 &  15.68 &  7.24 \\
 5.97 & 2.98 &  35.58 &  8.88 \\
 7.97 & 3.31 &  63.52 & 10.99 \\
 9.97 & 3.85 &  99.40 & 14.82 \\
11.96 & 4.24 & 143.04 & 17.94 \\
13.97 & 4.80 & 195.16 & 23.04 \\
16.00 & 5.33 & 256.00 & 28.46 \\
17.95 & 5.75 & 322.21 & 33.01 \\
19.98 & 6.29 & 399.00 & 39.56 \\
\bottomrule
\end{tabular}
\end{table}
\begin{figure}
  \centering
  \includegraphics{build/Drillachse.pdf}
  \caption{Graph von $T^2$ gegen $r^2$ und Fit}
  \label{fig:drillachse}
\end{figure}
Die Regressionsgerade folgt der allgemeinen Geradengleichung
\begin{equation}
  T^2(r^2) = ar^2+b.
  \label{eqn:gerade}
\end{equation}
Die Regressionsparamter $a$ und $b$ ergeben sich hierbei zu
\begin{align}
  a &= \SI{757.48(797)}{\second\squared/\meter\squared}\\
  b &= \SI{4.70(019)}{\second\squared}.
\end{align}
Kombiniert man \eqref{eqn:gerade} mit \eqref{eqn:schwingungkonkret}, so erhält man für
das Eigenträgheitsmoment $I_D$ der Drillachse
\begin{equation}
  I_D = \frac{2b\overline{m}}{a} = \SI{2.82(012)e-3}{\kilogram\meter\squared}
\end{equation}
mit einer relativen Unsicherheit von 4,3\%.
\subsection{Bestimmung der Trägheitsmomente zweier Körper}
\subsubsection{Bestimmung des Trägheitsmoments eines Zylinders}
Die Abmessungen des untersuchten Zylinders sind $h=\SI{13.99}{\cm}$ und
$d=\SI{10}{\cm}$, seine Masse $m=\SI{2.3959}{\kilogram}$. Die gemessenen
Periodendauern sind in \ref{tab:zylinder} zusammengefasst.

\begin{table}
\centering
\caption{Periodendauern der Rotation des Zylinders}
\label{tab:zylinder}
\begin{tabular}{c c}
\toprule
$3T$/s & $T/$s \\
\midrule
6.73 & 2.24
6.72 & 2.24
6.80 & 2.27
6.73 & 2.24
6.73 & 2.24
\bottomrule
\end{tabular}
\end{table}

Durch ihre Mittelung und \label{eqn:schwingung} lässt sich
\begin{equation}
  I_z = frac{DT^2}{4\pi^2} = \SI{2.64(0.31)}{\kilogram\meter\squared}
\end{equation}
mit einer relativen Unsicherheit von 

\label{sec:Auswertung}
