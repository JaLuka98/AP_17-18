\section{Fehlerrechnung}
\label{sec:Fehlerrechnung}
Der Mittelwert einer Stichprobe von $N$ Werten wird durch
\begin{equation}
  \overline{x} = \sum\limits_{i = 1}^N x_i
\end{equation}
\label{eqn:mean}
berechnet.
Die empirische Standardabweichung dieser Stichprobe ist durch
\begin{equation}
  \sigma_x = \sqrt{\frac{1}{N-1}
    \sum\limits_{i = 1}^N
    (x_i-\overline{x})^2}
\end{equation}
\label{eqn:std}
gegeben.
Ist $f$ eine Funktion, die von unsicheren Variablen $x_i$ mit
Standardabweichungen $\sigma_i$ abhängt, so ist die Unsicherheit von f
\begin{equation}
  \sigma_f = \sqrt{
    \sum\limits_{i = 1}^N
      \left( \frac{\partial f}{\partial x_i} \sigma_i \right)^{\!\! 2}
  }.
\end{equation}
\label{eqn:gaussfehler}
Diese Formel bezeichnet man als "Gauß'sches Fehlerfortpflanzungsgesetz".
Wenn im Folgenden Mittelwerte, Standardabweichungen und Fehler von
Funktionen unsicherer Größen berechnet werden, so werden stets die obigen
Formeln verwendet.

HIER NOCH FORMELN FÜR GERADENFITS EINFÜGEN!!!!!!!!!!!!!!!!!!!!
111111
111111



11111111







11111111111
