\section{Theorie}
\label{sec:Theorie}
Um Rotationsbewegungen angemessen zu beschreiben, sollten zunächst Begriffe
eingeführt werden, die in Analogie zu den Begriffen der Translation stehen.

Die Trägheit eines Körpers gegen eine translatorische Beschleunigung $\symbf{a}$
durch eine Kraft $\symbf{F}=m\symbf{a}$ ist in der klassischen Mechanik die
Masse $m$ eines Körpers\footnote{Im folgenden wird
$\dfrac{\mathrm{d}m}{\mathrm{d}t}=0$ vorrausgesetzt, sodass
$\symbf{F}=\dfrac{\mathrm{d}\symbf{p}}{\mathrm{d}t}=m\symbf{a}$ gilt.}.
Diese Rolle spielt bei Rotationsbewegungen der Trägheitstensor, der sich in
einfacher Betrachtung bei Drehung um eine fest vorgegebene Achse auf das skalare
Trägheitsmoment $I$ reduziert.

Das Trägheitsmoment $I$ errechnet sich hierbei durch
\begin{equation}
  I=\int r_{\perp}^2 \,\symup{d}m\,,
  \label{eqn:traegheitallg}
\end{equation}
wenn die Drehachse durch den Schwerpunkt des Körpers geht. Das Integral ist hierbei
als Volumenintegral über den Körper zu verstehen. Verläuft die Drehachse nicht
durch den Schwerpunkt des Körpers, so muss nach dem Satz von Steiner noch ein
Summand hinzu addiert werden:
\begin{equation}
  I=I_{\symup{s}}+m\symbf{a}^2\,.
  \label{eqn:steiner}
\end{equation}
Dabei ist $m$ die Masse des Körpers, $I_{\symup{s}}$ das Trägheitsmoment bezüglich
der Drehachse durch den Schwerpunkt des Körpers und $a$ der Abstand der realen
Drehachse zu dieser. Mithilfe dieses Satzes lassen sich die Trägheitsmomente
komplexer Körper einfach berechnen, indem der Körper in mehrere einfacher zu berechnende
Körper aufgeteilt wird und deren Trägheitsmomente bezüglich der Drehachse berechnet
und aanschließend aufaddiert werden.

Wirkt auf den drehbaren Körper eine Kraft $\symbf{F}$ im Abstand $\symbf{r}$ und somit ein
äußeres Drehmoment der Größe $\symbf{M}=\symbf{F}\times\symbf{r}$, so wird er um
einen Winkel $\phi$ ausgelenkt, bis sich ein Gleichgewicht zwischen innerem Drehmoment,
das durch die Spiralfeder erzeugt wird, und äußerem Drehmoment einstellt. Für das innere
Drehmoment ergibt sich der Zusammenhang
\begin{equation}
  \symbf{M}=\symbf{D}\phi\,.
  \label{eqn:drehmoment_innen}
\end{equation}
Dabei ist $\phi$ der Winkel der Auslenkung und $\symbf{D}$ die Winkelrichtgröße.
Diese stellt ein Analogon zur Federkonstante bei translatorischen Bewegungen dar.
Zeigt die wirkende Kraft $\symbf{F}$ in Richtung des Einheitsvektors $\symbf{e}_{\phi}$
in Polarkoordinaten und steht somit senkrecht zum Hebelarm $\symbf{r}$, so lässt sich
die Winkelrichtgröße zu
\begin{equation}
  {D}=\frac{Fr}{\phi}\,.
  \label{eqn:winkelrg}
\end{equation}

Wird der Körper nun losgelassen, verschwindet das äußere Drehmoment. Das innere
rücktreibende Drehmoment sorgt nun dafür, dass der Körper harmonisch schwingt.
Für die Schwingungsdauer $T$ ergibt sich
\begin{equation}
  T=2\pi\sqrt{\frac{I}{D}}\,,
  \label{eqn:schwingung}
\end{equation}
wobei $I$ das gesamte Trägheitsmoment darstellt. Teilt man das gesamte Trägheitsmoment
$I$ nun in die einzelnen Trägheitsmomente $I_{\symup{D}}$ der Drillachse und
$I_{\symup{K}}$ des Körpers auf und stellt nach $I_{\symup{K}}$ um, so ergibt sich für
das Trägheitsmoment des Körpers
\begin{equation}
  I_{\symup{K}}=\frac{T^2D}{4\pi^2}-I_{\symup{D}}\,.
  \label{eqn:traegheitschwingung}
\end{equation}

Für die im Versuch zu bestimmenden Körper ergeben sich aus \eqref{eqn:traegheitallg}
und \eqref{eqn:steiner} folgende Formeln:
\begin{gather}
  I_{\symup{Kugel}}=\frac{2}{5}mr^2
  \label{eqn:traegheitkugel} \\
  I_{\symup{Zylinder}}=\frac{1}{2}mr^2
  \label{eqn:traegheitzylinder} \\
  \begin{split}
    I_{\symup{Puppe1}}=I_{\symup{K}}+I_{\symup{R}}
    +2\Bigl(I_{\symup{A}}+(R_{\symup{R}}+R_{\symup{A}})^2\Bigr)
    +2\Biggl(I_{\symup{B}}+\biggl(\frac{h_{\symup{B}}}{2}\biggr)^2\Biggr)
    \label{eqn:traegheitpuppe1}
  \end{split}
  \\
  \begin{split}
    I_{\symup{Puppe2}}=I_{\symup{K}}+I_{\symup{R}}
    +2\Biggl(I_{\symup{A}}+\biggl(R_{\symup{R}}+\frac{h_{\symup{A}}}{2}\biggr)^2\Biggr)
    +2\Biggl(I_{\symup{B}}+\biggl(\frac{h_{\symup{B}}}{2}\biggr)^2\Biggr)
    \label{eqn:traegheitpuppe2}
  \end{split}
\end{gather}
Dabei ist $I_{\symup{Puppe1}}$ das gesamte Trägheitsmoment der Puppe in der ersten
Haltung und $I_{\symup{Puppe2}}$ das gesamte Trägheitsmoment der Puppe in der
zweiten Körperhaltung. $I_{\symup{K}}$ ist das Trägheitsmoment des Kopfes, $I_{\symup{R}}$
das Trägheitsmoment des Rumpfes, $I_{\symup{A}}$ das Trägheitsmoment eines Armes und
$I_{\symup{B}}$ das Trägheitsmoment eines Beines. $R_{\symup{R}}$ und $R_{\symup{A}}$ sind die zugehörigen
Radien und $h_{\symup{A}}$ und $h_{\symup{B}}$ die zugehörigen Höhen. Die quadrierten Faktoren folgen
aus den geometrischen Abmessungen der Puppe. Zudem ist zu beachten, dass $I_{\symup{Arm}}$
in \eqref{eqn:traegheitpuppe1} und \eqref{eqn:traegheitpuppe2} unterschiedliche Werte
annimmt, da die Arme um verschiedene Achsen gedreht werden.
