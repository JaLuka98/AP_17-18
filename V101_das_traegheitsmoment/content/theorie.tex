\section{Theorie}
\label{sec:Theorie}
Um Rotationsbewegungen angemessen zu beschreiben, sollten zunächst Begriffe
eingeführt werden, die in Analogie zu den Begriffen der Translation stehen.

Die Trägheit eines Körpers gegen eine translatorische Beschleunigung $\symbf{a}$
durch eine Kraft $\symbf{F}=m\symbf{a}$ ist in der klassischen Mechanik die
Masse $m$ eines Körpers\footnote{Im folgenden wird
$\dfrac{\mathrm{d}m}{\mathrm{d}t}=0$ vorrausgesetzt, sodass
$\symbf{F}=\dfrac{\mathrm{d}\symbf{p}}{\mathrm{d}t}=m\symbf{a}$ gilt.}.
Diese Rolle spielt bei Rotationsbewegungen der Trägheitstensor, der sich in
einfacher Betrachtung bei Drehung um eine fest vorgegebene Achse auf das skalare
Trägheitsmoment $\symbf{I}$ reduziert.

Das Trägheitsmoment $\symbf{I}$ errechnet sich hierbei durch
\begin{equation}
  \symbf{I}=\int r^2 dm  ,
  \label{eqn:traegheitallg}
\end{equation}
wenn die Drehachse durch den Schwerpunkt des Körpers geht. Das Inegral ist hierbei
als Volumenintegral über den Körper zu verstehen. Verläuft die Drehachse nicht
durch den Schwerpunkt des Körpers, so muss nach dem Satz von Steiner noch ein
Summand hinzu addiert werden:
\begin{equation}
  \symbf{I}=\symbf{I_s}+ma^2\;.
  \label{eqn:steiner}
\end{equation}
Dabei ist $m$ die Masse des Körpers, $\symbf{I_s}$ das Trägheitsmoment bezüglich
der Drehachse durch den Schwerpunkt des Körpers und $a$ der Abstand der realen
Drehachse zu dieser.

Wirkt auf den drehbaren Körper eine Kraft $\symbf{F}$ im Abstand $\symbf{r}$ und somit ein
äußeres Drehmoment der Größe $\symbf{M}=\symbf{F}\times\symbf{r}$, so wird er um
einen Winkel $\phi$ ausgelenkt, bis sich ein Gleichgewicht zwischen innerem Drehmoment,
das durch die Spiralfeder erzeugt wird, und äußerem Drehmoment einstellt. Für das innere
Drehmoment ergibt sich der Zusammenhang
\begin{equation}
  \symbf{M}=\symbf{D}\phi\;.
  \label{eqn:drehmoment_innen}
\end{equation}
Dabei ist $\phi$ der Winkel der Auslenkung und $\symbf{D}$ die Winkelrichtgröße.
Diese stellt ein Analogon zur Federkonstante bei translatorischen Bewegungen dar.
Zeigt die wirkende Kraft $\symbf{F}$ in Richtung des Einheitsvektors $\symbf{e_{phi}}$
in Polarkoordinaten und steht somit senkrecht zum Hebelarm $\symbf{r}$, so lässt sich
die Winkelrichtgröße zu
\begin{equation}
  \symbf{D}=\frac{\symbf{F}\symbf{r}}{\phi}\;.
  \label{eqn:winkelrg}
\end{equation}

Wird der Körper nun losgelassen, verschwindet das äußere Drehmoment. Das innere
rücktreibende Drehmoment sorgt nun dafür, dass der Körper harmonisch schwingt.
Für die Schwingungsdauer $T$ ergibt sich
\begin{equation}
  T=2\pi\sqrt{\frac{\symbf{I}}{\symbf{D}}}\;,
  \label{eqn:schwingung}
\end{equation}
wobei I das gesamte Trägheitsmoment darstellt. Teilt man das gesamte Trägheitsmoment
$\symbf{I}$ nun in die einzelnen Trägheitsmomente $\symbf{I_D}$ der Drillachse und
$\symbf{I_K}$ des Körpers auf und stellt nach $\symbf{I_K}$ um, so ergibt sich für
das Trägheitsmoment des Körpers
\begin{equation}
  \symbf{I_K}=\frac{T^2 \symbf{D}}{4\pi^2}-\symbf{I_D}\;.
  \label{eqn:traegheitschwingung}
\end{equation}
