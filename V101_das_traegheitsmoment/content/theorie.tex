\section{Theorie}
\label{sec:Theorie}
Um Rotationsbewegungen angemessen zu beschreiben, sollten zunächst Begriffe
eingeführt werden, die in Analogie zu den Begriffen der Translation stehen.

Die Trägheit eines Körpers gegen eine translatorische Beschleunigung $\symbf{a}$
durch eine Kraft $\symbf{F}=m\symbf{a}$ ist in der klassischen Mechanik die
Masse $m$ eines Körpers\footnote{Im folgenden wird
$\dfrac{\mathrm{d}m}{\mathrm{d}t}=0$ vorrausgesetzt, sodass
$\symbf{F}=\dfrac{\mathrm{d}\symbf{p}}{\mathrm{d}t}=m\symbf{a}$ gilt.}.
Diese Rolle spielt bei Rotationsbewegungen der Trägheitstensor, der sich in
einfacher Betrachtung bei Drehung um eine fest vorgegebene Achse auf das skalare
Trägheitsmoment $\symbf{I}$ reduziert.
