\section{Durchführung}
\label{sec:Durchführung}
Zunächst soll die Zeitkonstante eines RC-Gliedes durch die Beobachtung und Auswertung
der Spannung am Kondensator bestimmt werden. Dafür wird die in (ABBILDUNG EINFÜGEN)
gezeigte Schaltung verwendet. An das RC-Glied wird dabei mit einem Multimeter eine
Rechteckspannung angelegt. Die Spannung $U_{\text{C}}$ am Kondensator wird abgegriffen und auf dem
Oszilloskop so dargestellt, dass eine abfallende Flanke möglichst genau zu erkennen ist.
Danach wird ein Bild des Graphen erstellt, aus dem Wertepaare für die Spannung $U$
und die Zeit $t$ abgelesen werden können.


Zur Bestimmung der Zeitkonstante des RC-Gliedes über die Messung der Amplitude $A$ in
Abhängigkeit von der Frequenz $f$, wird mithilfe eines Multimeters eine sinusförmige
Wechselspannung an das RC_Glied angelegt. Mithilfe des Multimeters kann dann die Amplitude
$A$ der Spannung am Kondensator gemessen werden. Nun wird die Amplitude für Frequenzen
über drei Größenordnungen hinweg gemessen. Zum Schluss sollte noch überprüft werden,
ob die vom Multimeter erzeugte Spannung eine konstante Amplitude hat.

Bei der Bestimmung der Zeitkonstanten des RC-Gliedes über die Messung der Phasenverschiebung
von der Generatorspannung $U_{\text{0}}$ und der Spannung $U_{\text{C}}$ am Kondensator
muss zunächst eine vom Multimeter erzeugte Spannung an das RC-Glied angelegt werden.
Auf dem Oszilloskop werden beide Spannungen angezeigt.
