\section{Theorie}
In diesem Versuch werden Relaxationsvorgänge betrachtet. Bei einer Relaxation kehrt
ein zuvor gestörtes System nicht-oszillatorisch in seinen Gleichgewichtszustand
zurück. Das System verfügt über eine charakteristische Größe $A$. Ist es linear,
so folgt die Änderungsgeschwindigkeit zum Zeitpunkt $t$ der Gleichung
\begin{equation}
  \frac{\symup{d}A}{\symup{d}t}(t) = c[A(t)-A(t \to \infty)]\,.
  \label{eqn:linearrelax}
\end{equation}
Dabei ist A(t \to \infty) der nur asymptotisch erreichbare Wert der Größe $A$ im
Gleichgewichtszustand, $c$ ist eine Konstante größer Null, sodass A beschränkt
bleibt.
Wird \eqref{eqn:linearrelax} nach der Zeit integriert, ergibt sich für $A$ die
exponentielle Zuordnung
\begin{equation}
  A(t) = A(t \to \infty) + [A(0) - A(t \to \infty)] \exp(ct)\,.
\end{equation}
Die in diesem Versuch untersuchten Relaxationsvorgänge sind der Auf- und Entladevorgang
eines Kondensators über einen Widerstand. Der Schaltplan hierzu ist in ... zu sehen.
Stets liegt auf den Kondensatorplatten zum Zeitpunkt $t$ eine Ladung $Q(t)$. Für
die am Kondensator anliegende Spannung $U_{\text{C}}$ gilt dann
\begin{equation}
  U_{\text{C}} = Q / C\,,
\end{equation}
wobei $C$ die Kapazität des Kondensators ist. Da der Kondensator mit dem Widerstand
in Reihe geschaltet ist, bewirkt die am Kondensator abfallende Spannung $U_{\text{C}}$
gemäß dem Ohmschen Gesetz den Strom
\begin{equation}
  I = U_{\text{C}} / R\,.
\end{equation}
Zusammen mit der Beziehung $\dot{Q} = -I$ lässt sich eine zu \eqref{eqn:linearrelax}
analoge Differentialgleichung für $Q(t)$ mit der asymptotischen Randbedingung
$Q(t \to \infty) = 0$, da es sich um einen Entladevorgang handelt, aufstellen. Die
Lösung dieser ist
\begin{equation}
  Q(t) = Q(0) \exp(-t/\tau)
\end{equation}
mit der Zeitkonstanten $\tau \coloneqq RC$ des RC-Glieds.
\label{sec:Theorie}
