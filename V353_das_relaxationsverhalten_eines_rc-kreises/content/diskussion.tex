\section{Diskussion}
\label{sec:Diskussion}
Zuallererst sei die gute Übereinstimmung der Messwerte mit den Ausgleichsfunktionen
und der Theoriekurve zu bemerken. Deswegen lassen sich grobe Fehler ausschließen,
statistische Fehler konnten gering gehalten werden.
Die ermittelten Werte für die Zeitkonstante sind zusammengefasst in Tabelle
dargestellt.

\begin{table}
\centering
\begin{tabular}{ccc}
\toprule
$\tau_1 / \si{\milli \second}$ & $\tau_2 / \si{\milli \second}$ & $\tau_3 / \si{\milli \second}$\\
\midrule
$0,616 \pm 0,027$ & $0,829 \pm 0,006$ & $0,822 \pm 0,010$\\
\bottomrule
\end{tabular}
\caption{Zusammenfassung der ermittelten Werte für die Zeitkonstante}
\end{table}

Aufällig ist, dass die letzten beiden Werte konsistent sind und signifikant vom
ersten Wert abweichen. Desweiteren ist die relative Unsicherheit der ersten Messung mit
4,38\% deutlich höher als die der zweiten und dritten Messung mit 0,72\% und 1,22\%.
Dementsprechend lässt sich annehmen, dass die Zeitkonstante des hier untersuchten
RC-Gliedes nah an \tau_2 und \tau_3 liegt.
Eine mögliche Erklärung für die große Abweichung bei der direkten Messung liegt
in der Nichtbeachtung des Innenwiderstands des Generators. Es ist außerdem zu bemerken,
dass nur recht wenige Messwerte für die direkte Messung verwendet wurden. Die Messwerte
wurden anhand des Ausdrucks abgelesen und nicht die Cursor-Funktion des Oszilloskops verwendet.
Dieses Verfahren erscheint aufgrund der nicht unwesentlichen Streuung der Messwerte um die
Ausgleichsgerade als zu ungenau, es sollten die Messwerte stattdessen direkt bei der
Versuchsdurchführung mit den Funktionen des Oszilloskops abgelesen werden.
Außerdem treten Fehler bei der direkten Methode auf, da die Auf- und Entladung nur
näherungsweise vollständig geschehen kann. Dadurch beträgt der Ordinatenabschnitt nicht,
wie theoretisch erwartet, null.
Bei der Verwendung des RC-Gliedes als Integrator wurde das aus der Theorie zu erwartende Ergebnis
in guter Näherung erzielt.
