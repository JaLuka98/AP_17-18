\section{Auswertung}
\label{sec:Auswertung}
\subsection{Bestimmung der Zeitkonstante durch die Entladung eines Kondensators}
Die aus der mit dem Oszilloskop aufgenommenen Entladekurve des Kondensators
entnommenen Wertepaare sind in Tabelle \ref{tab:entladung} zu finden.

\begin{table}
\centering
\caption{Messdaten zur Entladung des Kondensators über den Widerstand}
\label{tab:entladung}
\begin{tabular}{c c}
\toprule
$U_\text{C}(t)/$mV & $t/$ms \\
\midrule
1088 &  0   \\
 560 &  0.5 \\
 304 &  1   \\
 152 &  1.5 \\
 80  &  2   \\
 36  &  2.5 \\
  8  &  3   \\
  4  &  3.5 \\
  2  &  4   \\
\bottomrule
\end{tabular}
\end{table}
Nach Teilen durch $U_\text{C}(0)$ und Logarithmieren von \eqref{eqn:kondensatorurelax} folgt
\begin{equation}
  \ln{\left(\frac{U_\mathrm{C}(t)}{U_0}\right)} = -\frac{t}{\tau}\,.
\end{equation}
Die graphische Darstellung der Messwerte und die Ausgleichsgerade ist in Abbildung \ref{fig:entladung}
zu sehen.
\begin{figure}
  \centering
  \includegraphics{build/uc.pdf}
  \caption{Graph von $U_\text{C}$ gegen $t$ und Ausgleichsfunktion}
  \label{fig:entladung}
\end{figure}
Die Ausgleichsgerade folgt allgemein \eqref{eqn:gerade}, sodass die Steigung der
Ausgleichsgerade $a$ mit der Zeitkonstanten $\tau$ des RC-Gliedes in dem Zusammenhang
$a = -\frac{1}{\tau}$ steht. Hier lässt sich $a$ zu
\begin{align}
  %a = \SI{-1.62(007){\per\milli\second}}
  \SI{2.82(012)e-3}{\kilogram\meter\squared}
\end{align}
berechnen und für $\tau$ folgt dann
\begin{align}
%  \tau = RC = \SI{0.616(0027)e-3{\second}}\,.
\end{align}
...
\subsection{Bestimmung der Zeitkonstante durch Analyse der Frequenzabhängigkeit der Kondensatorspannung}

\subsection{Der RC-Kreis als Integrator}

Für hohe angelegte Frequenzen sollte der RC-Kreis die angelegte Spannung integrieren.
Das bedeutet, dass der Graph der am Kondensator gemessenen Spannung dem Graphen der
Stammfunktion der angelegten Generatorspannung entsprechen sollte. In diesem Versuchsteil
wurde für alle angelegten Spannungen die Frequenz $f=191.8$ kHz gewählt.
Die erste an den RC-Kreis angelegte Spannung ist eine Reckteckspannung. Die Stammfunktion
zu dieser sollte gleichmäßig ansteigende Flanken dort haben, wo die Recheckspannung
positiv ist und gleichmäßig abfallende Flanken dort, wo die Rechteckspannung
negativ ist. Genau dieses Verhalten ist in \ref{fig:rechteck} zu sehen. Der blaue Graph beschreibt
dabei die an das RC-Glied angelegte Spannung und der gelbe Graph die am Kondensator
abgegriffene Spannung.
\begin{figure}
  \centering
  \includegraphics{data/integration_rechteck.PNG}
  \caption{Graphen der Generatorspannung und der Spannung am Kondensator bei anglegter
  Rechteckspannung}
  \label{fig:rechteck}
\end{figure}
Wird an das RC-Glied eine Sägezahnspannung angelegt, so sollte die am Kondensator
abgegriffene Spannung eine sich periodisch wiederholende Funktion beschreiben, die
Extrema an den Nullstellen sowie Wendepunkten an den Extrema des Graphen der
Generatorspannung besitzt. Die Graphen der Messung \ref{fig:saegezahn} bestätigen diesen Zusammenhang.
Der blaue Graph beschreibt erneut den Verlauf der angelegten Generatorspannung und
der gelbe Graph den Verlauf der am Kondensator abgegriffenen Spannung.
\begin{figure}
  \centering
  \includegraphics{data/integration_saegezahn.PNG}
  \caption{Graphen der Generatorspannung und der Spannung am Kondensator bei anglegter
  Sägezahnspannung}
  \label{fig:saegezahn}
\end{figure}
Wird an den RC-Kreis eine sinusförmige Spannung angelegt, so sollte sich nach bei der
Integration gemäß
\begin{equation}
  \int \sin(t) \, \symup{d}t=-\cos(t)
\end{equation}
eine Spannung ergeben, die dem Graphen von $-\cos(t)$ entspricht. Der Verlauf der
im Versuch aufgenommenen Graphen \ref{fig:sinus} folgt dieser Beziehung. Hier beschreibt der blaue Graph
wieder den Verlauf der angelegten Generatorspannung und der gelbe Graph den Verlauf der
am Kondensator abgegriffenen Spannung.
\begin{figure}
  \centering
  \includegraphics{data/integration_sinus.PNG}
  \caption{Graphen der Generatorspannung und der Spannung am Kondensator bei anglegter
  Sinusspannung}
  \label{fig:sinus}
\end{figure}
