\section{Auswertung}
\label{sec:Auswertung}
\subsection{Bestimmung der Zeitkonstante durch die Entladung eines Kondensators}
Die aus der mit dem Oszilloskop aufgenommenen Entladekurve des Kondensators
entnommenen Wertepaare sind in Tabelle \ref{tab:entladung} zu finden.

\begin{table}
\centering
\caption{Messdaten zur Entladung des Kondensators über den Widerstand}
\label{tab:entladung}
\begin{tabular}{c c}
\toprule
$U_\text{C}(t)/$mV & $t/$ms \\
\midrule
1088 &  0   \\
 560 &  0.5 \\
 304 &  1   \\
 152 &  1.5 \\
 80  &  2   \\
 36  &  2.5 \\
  8  &  3   \\
  4  &  3.5 \\
  2  &  4   \\
\bottomrule
\end{tabular}
\end{table}
Nach Teilen durch $U_\text{C}(0)$ Logarithmieren von \eqref{eqn:kondensatorurelax} folgt
\begin{equation}
  \ln{\left(\frac{U_\mathrm{C}(t)}{U_0}\right)} = -t/\tau\,.
\end{equation}
Die Ausgleichsgerade folgt allgemein \eqref{eqn:gerade}, sodass 
