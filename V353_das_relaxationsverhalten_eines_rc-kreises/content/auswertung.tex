\section{Auswertung}
\label{sec:Auswertung}
\subsection{Bestimmung der Zeitkonstante durch die Entladung eines Kondensators}
Die aus der mit dem Oszilloskop aufgenommenen Entladekurve des Kondensators
entnommenen Wertepaare sind in Tabelle \ref{tab:entladung} zu finden.

\begin{table}
\centering
\caption{Messdaten zur Entladung des Kondensators über den Widerstand}
\label{tab:entladung}
\begin{tabular}{c c}
\toprule
$U_\text{C}(t)/$mV & $t/$ms \\
\midrule
1088 &  0   \\
 560 &  0.5 \\
 304 &  1   \\
 152 &  1.5 \\
 80  &  2   \\
 36  &  2.5 \\
  8  &  3   \\
  4  &  3.5 \\
  2  &  4   \\
\bottomrule
\end{tabular}
\end{table}
Nach Teilen durch $U_\text{C}(0)$ und Logarithmieren von \eqref{eqn:kondensatorurelax} folgt
\begin{equation}
  \ln{\left(\frac{U_\mathrm{C}(t)}{U_0}\right)} = -\frac{t}{\tau}\,.
\end{equation}
Die graphische Darstellung der Messwerte und die Ausgleichsgerade ist in Abbildung \ref{fig:entladung}
zu sehen.
\begin{figure}
  \centering
  \includegraphics{build/uc.pdf}
  \caption{Graph von $U_\text{C}$ gegen $t$ und Ausgleichsfunktion}
  \label{fig:entladung}
\end{figure}
Die Ausgleichsgerade folgt allgemein \eqref{eqn:gerade}, sodass die Steigung der
Ausgleichsgerade $a$ mit der Zeitkonstanten $\tau$ des RC-Gliedes in dem Zusammenhang
$a = -\frac{1}{\tau}$ steht. Hier lässt sich $a$ zu
\begin{align}
  %a = \SI{-1.62(007){\per\milli\second}}
  \SI{2.82(012)e-3}{\kilogram\meter\squared}
\end{align}
berechnen und für $\tau$ folgt dann
\begin{align}
%  \tau = RC = \SI{0.616(0027)e-3{\second}}\,.
\end{align}
...
\subsection{Bestimmung der Zeitkonstante durch Analyse der Frequenzabhängigkeit der Kondensatorspannung}
