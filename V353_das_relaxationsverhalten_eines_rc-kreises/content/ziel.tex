\section{Ziel}
Ziel dieses Versuchs ist die Ermmitlung der Zeitkonstante $RC$ eines RC-Glieds.
Diese ist zum Einen direkt über den Auf- oder Entladevorgang eines Kondensators,
dann über die Frequenzabhängigkeit der Amplitude der Spannung an einem
Kondensator und zuletzt durch Beachtung der Phasenverschiebung zwischen
anliegender Wechsel- und resultierender Kondensatorspannung zu bestimmen. Die verschiedenen
Verfahren sollen verglichen werden. Außerdem ist die Abhängigkeit der
Relativamplitude von der Phase des Wechselstroms in einem Polarkoordinatensystem
zu veranschaulichen. Auch ist zu zeigen, dass es unter Verwendung eines RC-Glieds
möglich ist, Spannungen zu integrieren.
\label{sec:Ziel}
