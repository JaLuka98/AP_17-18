\section{Diskussion}
\label{sec:Diskussion}

Die Messung ist insgesamt als höchst ungenau zu bewerten. Die aus den Messdaten errechneten
Werte liegen zwar in der gleichen Größenordnung wie die theoretisch berechneten,
jedoch sind die in Tabelle \ref{tab:abweichung} dargestellten Abweichungen sehr groß.
Lediglich die experimentell bestimmte Güte des Selektivverstärkers weist nur eine
geringe Abweichung auf.

Da die Abweichungen für die Methode der Messung des Widerstandes konsistent nach
unten abweichen, liegt die Vermutung nahe, dass die Abweichungen hier durch
systematische Fehler hervorgerufen werden.

Die aus den über die Spannung ermittelten Messwerten gewonnenen Ergebnisse zeigen
keine klare Tendenz und sehr große Abweichungen. Hier ist der Versuchsaufbau zu
kritisieren, da, sobald die Probe etwas zu fest in die Spule geschoben wurde bereits
große Änderungen der Spannung entstanden. Mit einer besser fixierten Spule könnten
hier vermutlich genauere Ergebnisse erzielt werden. Desweiteren ist anzumerken,
dass für Neodymoxid, bei welchem die größte Abweichung auftritt, nur sehr
geringe Spannungsunterschiede zu beobachten waren. Deswegen entstehen hier weitere
große Unsicherheiten durch das Ablesen vom Messgerät.
