\section{Theorie}
\label{sec:Theorie}
Bei Anwesenheit von Materie hängen magnetische FLussdichte $\symbf{B}$, externe
magnetische Feldstärke $\symbf{H}$, magnetische Permeabilität des Vakuums $\mu_0$
und Magnetisierung des Materials $\symbf{M}$ durch
\begin{equation}
  \symbf{B} = \mu_0 \symbf{H} + \symbf{M}
  \label{eqn:bhm}
\end{equation}
zusammen. Die Magnetisierung $\symbf{M}$ beschreibt hierbei die Dichte der
magnetischen Dipolmomente im Material und hängt durch
\begin{equation}
  \symbf{M} = \mu_0 \chi \symbf{H}
\end{equation}
von dem externen Magnetfeld $\symbf{H}$ ab. Dabei ist die Proportionalitätskonstante
$\chi$ die magnetische Suszeptibilität. Im einfachsten Fall ist sie eine materialabhängige
Konstante, kann aber im allgemeinen Fall vom externen Magnetfeld und der Temperatur $T$
des Materials abhängen.
In diesem Versuch wird die Suszeptibilität paramagnetischer Substanzen untersucht.
Nur Atome, Moleküle oder Ionen mit nicht verschwindendem Drehimpuls weisen
Paramagnetismus auf. Dieser beruht darauf, dass sich magnetische Momente parallel zu
einem äußeren Magnetfeld ausrichten. Da thermische Bewegungen dies stören, ist der Effekt
temperaturabhängig. Da der atomare Drehimpuls mit dem magnetischen Moment gekoppelt ist,
soll dieser Zusammenhang erläutert werden.
Der Gesamtdrehimpuls $\symbf{J}$ eines Atoms setzt sich aus dem Bahndrehimpuls
$\symbf{L}$ der Elektronenhülle, dem Eigendrehimpuls, dem sogenannten Spin, $\symbf{S}$
und dem Kernspin, der für paramagnetische Effekte vernachlässigt werden kann, zusammen.
