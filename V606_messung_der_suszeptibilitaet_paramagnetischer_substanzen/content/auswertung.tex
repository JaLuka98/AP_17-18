\section{Auswertung}
\label{sec:Auswertung}

\subsection{Bestimmung der Güte des Selektivverstärkers}
\label{subsec:guete}

Zur Bestimmung der Güte wird die Ausgangsspannung $U_{\symup{a}}$ gegen die Frequenz
$f$ aufgetragen. Die Messwerte hierzu befinden sich in Tabelle \ref{tab:guete}.
Es zeigt sich der in Abbildung \ref{fig:guete} dargestellte Verlauf.

\begin{table}[htp]
	\begin{center}
    \caption{Messwerte zur Bestimmung der Güte des Selektivverstärkers.}
    \label{tab:guete}
		\begin{tabular}{cc}
		\toprule
			{$f/$kHz} & {$U_a/$V}\\
			\midrule
			20,00 & 0,01\\
			21,00 & 0,01\\
			22,00 & 0,02\\
			23,00 & 0,02\\
			24,00 & 0,02\\
			25,00 & 0,03\\
			26,00 & 0,03\\
			27,00 & 0,03\\
			28,00 & 0,04\\
			29,00 & 0,05\\
			30,00 & 0,06\\
			30,50 & 0,07\\
			31,00 & 0,08\\
			31,50 & 0,09\\
			32,00 & 0,09\\
			32,50 & 0,12\\
			33,00 & 0,15\\
			33,50 & 0,20\\
			34,00 & 0,29\\
			34,10 & 0,26\\
			34,20 & 0,28\\
			34,30 & 0,36\\
			34,40 & 0,42\\
			34,50 & 0,49\\
			34,60 & 0,54\\
			34,70 & 0,66\\
			34,80 & 0,76\\
			34,90 & 0,99\\
			35,00 & 1,00\\
			35,10 & 1,50\\
			35,20 & 1,85\\
			35,30 & 1,70\\
			35,40 & 1,40\\
			35,50 & 0,80\\
			35,60 & 0,85\\
			35,70 & 0,66\\
			35,80 & 0,58\\
			35,90 & 0,46\\
			36,00 & 0,42\\
			36,50 & 0,23\\
			37,00 & 0,14\\
			37,50 & 0,09\\
			38,00 & 0,06\\
			39,00 & 0,02\\
			40,00 & 0,06\\
		\bottomrule
		\end{tabular}
	\end{center}
\end{table}

\begin{figure}
  \centering
  \includegraphics{build/glocke.pdf}
  \caption{Auftragung der gemessenen Ausgangsspannung $U_{\symup{a}}$ gegen die angelegte
  Frequenz f und Vergrößerung des Peaks.}
  \label{fig:guete}
\end{figure}

Die Güte des Selektivverstärkers ist gegeben durch den Zusammenhang
\begin{equation}
  Q=\frac{f_0}{f_{+} - f_{-}} \,.
\end{equation}
Dabei ist $f_0$ die Durchlassfrequenz und $f_{+}$ und $f_{-}$ sind die Frequenzen,
bei denen die Ausgangsspannung $\frac{1}{\sqrt{2}}$ des Scheitelwerts beträgt. Durch
graphische Auswertung erhält man die Werte
\begin{align*}
  f_0=      \,\\
  f_{+}=     \, \\
  f_{-}=      \.\\
\end{align*}
Die Güte des Selektivverstärkers berechnet sich dann zu $Q= $.


\subsection{Bestimmung der Suszeptibilitäten der Proben}
\label{subsec:suszept}

Es werden die Materialien Dysprosiumoxid, Trisdipraseodynumoxid, Gadoliniumoxid und
Neodymoxid untersucht. Die Messwerte befinden sich in den Tabellen \ref{tab:dysprosium},
\ref{tab:seodynumoxid}, \ref{tab:gadolinium} und \ref{tab:neodym}. Dabei bezeichnen
$U_{\symup{o}}$ und $R_{\symup{o}}$ die Spannung bzw. den Widerstand ohne eingeführte
Probe und $U_{\symup{m}}$ und $R_{\symup{m}}$ die Spannung bzw. den Widerstand mit
eingeführter Probe. Da die Dichte von Trisdipraseodynumoxid nicht bestimmt werden
kann, können zu dieser Probe keine Rechnungen durchgeführt werden, da alle später
zu benutzenden Formeln abhängig von der Dichte des Materials sind.


\begin{table}[htp]
	\begin{center}
    \caption{Messwerte zu Dysprosiumoxid ($\ce{Dy2O3}$).}
    \label{tab:dysprosium}
		\begin{tabular}{ccccc}
		\toprule
			{$U_o/$mV} & {$U_m/$mV} & {$R_o/\symup{\Omega}$} & {$R_m/\symup{\Omega}$} & {$\Delta R/\symup{\Omega}$}\\
			\midrule
			16,00 & 34,00 & 3,55 & 1,64 & 1,91\\
			16,00 & 33,00 & 3,54 & 1,70 & 1,84\\
			16,00 & 33,50 & 3,62 & 1,76 & 1,86\\
		\bottomrule
		\end{tabular}
	\end{center}
\end{table}


\begin{table}[htp]
	\begin{center}
    \caption{Messwerte zu Trisdipraseodynumoxid ($\ce{C6O12Pr2}$).}
    \label{tab:seodynumoxid}
		\begin{tabular}{ccccc}
		\toprule
			{$U_o/$mV} & {$U_m/$mV} & {$R_o/\symup{\Omega}$} & {$R_m/\symup{\Omega}$} & {$\Delta R/\symup{\Omega}$}\\
			\midrule
			14.50 & 16.00 & 3.20 & 3.00 & 0.20\\
			16.00 & 16.00 & 3.14 & 3.07 & 0.06\\
			15.50 & 16.00 & 3.17 & 2.80 & 0.36\\
		\bottomrule
		\end{tabular}
	\end{center}
\end{table}


\begin{table}[htp]
	\begin{center}
    \caption{Messwerte zu Gadoliniumoxid ($\ce{Gd2O3}$).}
    \label{tab:gadolinium}
		\begin{tabular}{ccccc}
		\toprule
			{$U_o/$mV} & {$U_m/$mV} & {$R_o/\symup{\Omega}$} & {$R_m/\symup{\Omega}$} & {$\Delta R/\symup{\Omega}$}\\
			\midrule
			16,00 & 21,50 & 1,47 & 0,69 & 0,78\\
			17,50 & 21,50 & 1,47 & 0,73 & 0,74\\
			17,00 & 20,50 & 1,49 & 0,73 & 0,75\\
		\bottomrule
		\end{tabular}
	\end{center}
\end{table}



\begin{table}[htp]
	\begin{center}
    \caption{Messwerte zu Neodymoxid ($\ce{Nd2O3}$).}
    \label{tab:neodym}
		\begin{tabular}{ccccc}
		\toprule
			{$U_o/$mV} & {$U_m/$mV} & {$R_o/\symup{\Omega}$} & {$R_m/\symup{\Omega}$} & {$\Delta R/\symup{\Omega}$}\\
			\midrule
			16,50 & 18,00 & 1,60 & 1,49 & 0,11\\
			18,25 & 18,25 & 1,61 & 1,43 & 0,19\\
			17,75 & 18,00 & 1,61 & 1,48 & 0,14\\
		\bottomrule
		\end{tabular}
	\end{center}
\end{table}


Zur Bestimmung der Suszeptibilitäten der verschiedenen Proben muss zunächst
der reale Querschnitt $Q_{\symup{r}}$ der Proben bestimmt werden. Dies ist notwendig,
da die Proben aus staubförmigem Material bestehen und die in Kapitel \ref{sec:Theorie}
hergeleiteten Formeln nur für Dichten der Einkristalle der jeweiligen Porben
gültig sind. Der reale Querschnitt folgt dem Zusammenhang
\begin{align}
  Q_{\symup{r}}=\frac{m}{L\rho}
\end{align}
Dabei bezeichnet $m$ die Masse, $L$ die Länge und $\rho$ die Dichte der Probe. Die
Länge aller Proben beträgt $L=13,5\,$cm. Die verschiedenen Querschnitte und die zugrundeliegenden
Daten befinden sich in Tabelle \ref{tab:querschnitt}. Die Dichten werden dabei der
Versuchsanleitung \cite{Versuchsanleitung} entnommen.

\begin{table}[htp]
	\begin{center}
    \caption{Reale Querschnitte der Proben und zugrundeliegende Daten.}
    \label{tab:querschnitt}
		\begin{tabular}{cccc}
		\toprule
			&{$m/$g} & {$rho/\frac{\symup{g}}{\symup{cm}^3}$} & {$Q/\symup{mm}^2$}\\
			\midrule
			\ce{Dy2O3} & 15.10 & 7.80 & 14.34\\
			\ce{Gd2O3} & 14.08 & 7.40 & 14.09\\
			\ce{Nd2O3} & 9.00 & 7.24 & 9.21\\
		\bottomrule
		\end{tabular}
	\end{center}
\end{table}

Nun können die Suszeptibilitäten $\chi_{\symup{U}}$ und $\chi_{\symup{R}}$ aus
den Gleichungen \eqref{eqn:spannung} und \eqref{eqn:widerstand} berechnet werden.
Dabei wird für $U_{\symup{Br}}$ der gemäß den Gleichungen \eqref{eqn:mean} und \eqref{eqn:std} gemittelte
Wert von $U_{\symup{m}}$ und für $\Delta R$ der gleichermaßen gemittelte Wert
von $\Delta R$ eingesetzt. Die Speisespannung beträgt $U_{\symup{Sp}}=1 \,$V,
der Querschnitt der Spule beträgt $F=\SI{86.6e-6}{\meter\squared}$ und der Widerstand
ist $R_3=\SI{998}{\ohm}$. Damit ergeben
sich die in Tabelle \ref{tab:chiexp} dargestellten Werte für die Suszeptibilitäten.

\begin{table}[htp]
	\begin{center}
    \caption{Experimentell bestimmte Suzeptibilitäten $\chi_{\symup{U}}$ und $\chi_{\symup{U}}$.}
    \label{tab:chiexp}
		\begin{tabular}{ccc}
		\toprule
			& $\chi_{\symup{U}}$ & $\chi_{\symup{R}}$\\
			\midrule
			\ce{Dy2O3} &    0,00809\pm0,00010   &  0,0226\pm0,0004  \\
      \ce{Gd2O3}  &    0,00520\pm0,00012   &   0,00932\pm0,00021 \\
      \ce{Nd2O3}  &    0,00680\pm0,00004  &    0,0027\pm0,0006 \\
		\bottomrule
		\end{tabular}
	\end{center}
\end{table}

Diese sollen nun mit den Theoriewerten verglichen werden. Die Theoriewerte werden
mit Gleichung \eqref{eqn:theoriewert} berechnet. Die Quantenzahlen und der
Landé-Faktor für die Materialien, aus denen die Prben bestehen, wurden bereits
in Kapitel \ref{subsec:berechnungquantenzahlenundlande} berechnet und sind in
Tabelle \ref{tab:lsjg} dargestellt. Für das $N$ aus Gleichung \eqref{eqn:theoriewert}
gilt
\begin{equation}
  N=\frac{2 \rho N_{\symup{A}}}{M}
\end{equation}
Dabei ist $\rho$ die Dichte des Materials, $N_{\symup{A}}=\SI{6.022e23}{\per\mol}$
die Avogadrokonstante und $M$ die Molare Masse des Materials. Die berechneten
Theoriewerte und die zugrundeliegenden Daten sind in Tabelle \ref{tab:chitheo}
dargestellt. Die Molaren Massen werden dabei \cite{molaremasse} entnommen.

\begin{table}[htp]
	\begin{center}
    \caption{Theoriewerte $\chi_{\symup{theo}}$ für die Suszeptibilitäten.}
    \label{tab:chitheo}
		\begin{tabular}{ccc}
		\toprule
			&  $M/\frac{\symup{g}}{\symup{mol}}$ & $\chi_{\symup{theo}}$\\
			\midrule
			\ce{Dy2O3} &   372,998  &   0,02540 \\
      \ce{Gd2O3}  &  362,498   &   0,01378 \\
      \ce{Nd2O3}  &  336,482  &    0,00302\\
		\bottomrule
		\end{tabular}
	\end{center}
\end{table}

Die Abweichungen der experimentell bestimmten Werte $\chi_{\symup{U}}$ und
$\chi_{\symup{R}}$ vom Theoriewert $\chi_{\symup{theo}}$ sind in Tabelle
\ref{tab:abweichung} dargestellt.

\begin{table}[htp]
	\begin{center}
    \caption{Abweichung der Werte $\chi_{\symup{U}}$ und $\chi_{\symup{R}}$  von
    $\chi_{\symup{U}}$.}
    \label{tab:abweichung}
		\begin{tabular}{ccc}
		\toprule
			&  Abweichung $\chi_{\symup{U}}$  & Abweichung $\chi_{\symup{R}}$\\
			\midrule
			\ce{Dy2O3} &     &    \\
      \ce{C6O12Pr2} &  &    \\
      \ce{Gd2O3}  &    &    \\
      \ce{Nd2O3}  &    &    \\
		\bottomrule
		\end{tabular}
	\end{center}
\end{table}
