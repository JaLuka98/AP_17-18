\section{Diskussion}
\label{sec:Diskussion}
Die Messergebnisse sind insgesamt als konsistent zu bewerten.
Die Werte der statischen Methode können zwar nicht mit Literaturwerten verglichen werden,
da sie selbst mithilfe von Literaturwerten berechnet wurden, jedoch weisen sie
qualitativ die zu erwartenden Eigenschaften auf. Die Graph der Temperatur steigt für
den Aluminiumstab, welcher die größte Wärmeleitfähigkeit besitzt, am schnellsten und
für den Edelstahlstab, welcher die geringste Wärmeleitfähigkeit besitzt, am langsamsten.
Zudem sind die Werte für den Wärmestrom im Messingstab höher als im Edelstahlstab,
was ebenfalls durch die Literaturwerte der Wärmeleitfähigkeit zu erwarten ist.
Auffällig ist lediglich, dass der Graph für die Temperaturdifferenz am Messingstab
zunächst ein Maximum annimmt und daraufhin wieder abfällt. Ein möglicher Grund hierfür
ist, dass die Stromstärke am Power Supply während der Messung leicht sank, sodass
das Peltier-Element nicht gleichmäßig heizen konnte.


Die experimentell ermittelten Werte für die Wärmeleitfähigkeit durch die Auswertung
der Messung mit der dynamischen Methode sind:
\begin{align}
  \kappa_{\symup{Messing}} &= \SI{80(6)}{\watt\per\meter\per\kelvin}\,,\nonumber \\
  \kappa_{\symup{Aluminium}} &= \SI{193(29)}{\watt\per\meter\per\kelvin}\,,\nonumber \\
  \kappa_{\symup{Edelstahl}} &= \SI{12.3(08)}{\watt\per\meter\per\kelvin}\,.\nonumber
\end{align}
Auffällig ist hier lediglich der sehr große Fehler bei dem Wert für Aluminium. Der Grund dafür
ist wahrscheinlich ungenaues Ablesen von dem Graphen.
Die Abweichungen zu den Literaturwerten liegen insgesamt im Rahmen der Messungenauigkeit.
Diese ist bei der verwendeten Methode der Auswertung verhältnismäßig groß, da bereits
durch sehr geringe Ungenauigkeiten beim Ablesen vom Graphen große Fehler bei den
Messwerten und damit auch bei den Ergebnissen entstehen können.
