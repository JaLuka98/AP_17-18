\section{Diskussion}
\label{sec:Diskussion}
Der Lock-In-Verstärker ist laut Theorie in der Lage, schwache und eventuell
verrauschte Signale zu verstärken und durch Filterung deutlich aus dem Rauschen
sichtbar zu machen. Diese Eigenschaft wird bei diesem Versuch eindeutig bestätigt.
Die Signale haben mit $\SI{10}{\milli\volt}$ eine sehr geringe Amplitude im Vergleich
zur Referenzspannung mit $\SI{3,32}{\volt}$. Das Signal wird sogar auf Spannungen im Bereich
von $\SI{5}{\volt}$ verstärkt. Die Werte der Ausgangsspannung sind im Wesentlichen
unabhängig davon, ob das Messsignal verrauscht ist oder nicht. Daraus lässt sich
schließen, dass der Lock-In-Verstärker hier ein Rauschen sehr gut wegfiltern kann.

Die Abhängigkeit der Ausgangsspannung von dem Kosinus der Phasenverschiebung zwischen
Mess- und Referenzsignal wird in den Darstellungen der Messwerte und der Graphen
der Ausgleichsfunktionen gut deutlich. Die Parameter der Ausgleichsrechnung sind
nahezu deckungsgleich. Die nichtverschwindenden Phasenparameter $\delta_i$ sind
dadurch zu erklären, dass der Gang der Signale durch elektronische Bauteile diese
auch geringfügig verändert, schließlich wurde die Phasenverschiebung nur am
Anfang des Verstärkers auf null geregelt. Darüberhinaus können auch statistische
Fehler zu diesen kleinen Unsicherheiten führen.
Dass die Unsicherheiten der $U_{0,i}$ größer als die nominellen Werte ist hier unproblematisch,
da der Theoriewert für diesen Parameter ohnehin null ist. 
