\section{Auswertung}
\label{sec:Auswertung}

Die Messreihe der gemessenen Ereignisse $N_\text{z}$ in der Zeit $t$ in Abhängigkeit der
Dicke $d$ der absorbierenden Zinkplattenschicht ist in \ref{tab:zink} zu sehen.
Den $N_\text{z}$ wurde direkt ihre Unsicherheit $\sigma N_\mathrm{z} = \sqrt{N_\mathrm{z}}$
zugeordnet, die sich aus der Eigenschaft ergibt, dass der Prozess des radioaktiven Zerfalls
einer Poisson-Verteilung folgt. Dabei wird der Fehler auf zwei signifikante Stellen
gerundet, falls die erste signifikante Stellung eine eins oder zwei ist, anderfalls auf eine.
Stets wird er aufgerundet. Die Messergebnisse werden entsprechend gemäß des kaufmännischen
Rundens gerundet. Die Rohdaten sind REFERENZ????????????????????????? zu entnehmen.
Außerdem ist bereits die Aktivität abzüglich der Nullaktivität eingetragen.
Letztere beträgt $\SI{1.028}{\per\second}$, da bei einer Nullmessung ohne Strahler
1028 Ereignisse in $\SI{1000}{\second}$ auftraten.

\begin{table}[htp]
        \begin{center}
          \caption{Messwerte zur Absorption von Gammastrahlung durch Zink.}
          \label{tab:zink}
                \begin{tabular}{S[table-format=2.0] S[table-format=3.0] S[table-format=4.0,table-figures-uncertainty=3] S[table-format=3.1, table-figures-uncertainty=2]}
                \toprule
                        {$d/$mm} & {$t/$s} & {$N_\mathrm{z} \pm \sigma N_\mathrm{z}$} & {$(A_\mathrm{z}) \pm \sigma N_\mathrm{z})\cdot $s}\\
                        \midrule
                         2 &  40 &  5050 \pm  80 & 125.2 \pm 1.8 \\
                         4 &  60 &  7230 \pm  90 & 119.4 \pm 1.4 \\
                         6 &  80 &  8750 \pm 100 & 108.3 \pm 1.2 \\
                         8 & 100 & 10480 \pm 110 & 104.0 \pm 1.0 \\
                        10 & 100 &  9320 \pm 100 &  92.0 \pm 1.0 \\
                        12 & 100 &  8470 \pm 100 &  83.7 \pm 0.9 \\
                        14 & 100 &  7610 \pm  90 &  75.1 \pm 0.9 \\
                        16 & 100 &  7090 \pm  90 &  69.8 \pm 0.9 \\
                        18 & 100 &  6520 \pm  90 &  64.1 \pm 0.8 \\
                        20 & 100 &  5790 \pm  80 &  56.9 \pm 0.7 \\
                \bottomrule
                \end{tabular}
        \end{center}
\end{table}
