\section{Auswertung}
\label{sec:Auswertung}

\subsection{Untersuchung der Absorption von Gammastrahlung}
\label{subsec:auswertunggamma}

Die Messreihe der gemessenen Ereignisse $N_\text{z}$ in der Zeit $t$ in Abhängigkeit der
Dicke $d$ der absorbierenden Zinkplattenschicht ist in Tabelle \ref{tab:zink} zu sehen.
Den $N_\text{z}$ wurde direkt ihre Unsicherheit $\sigma_{N_\mathrm{z}} = \sqrt{N_\mathrm{z}}$
zugeordnet, die sich aus der Eigenschaft ergibt, dass der Prozess des radioaktiven Zerfalls
einer Poisson-Verteilung folgt. Dabei wird der Fehler auf zwei signifikante Stellen
gerundet, falls die erste signifikante Stellung eine eins oder zwei ist, anderfalls auf eine.
Stets wird er aufgerundet. Die Messergebnisse werden entsprechend gemäß des kaufmännischen
Rundens gerundet. Die Rohdaten sind angehängt.
Außerdem ist bereits die Aktivität abzüglich der Nullaktivität eingetragen.
Letztere beträgt $\SI{1.028}{\per\second}$, da bei einer Nullmessung ohne Strahler
1028 Ereignisse in $\SI{1000}{\second}$ auftraten.

\begin{table}[htp]
        \begin{center}
          \caption{Messwerte zur Absorption von Gammastrahlung durch Zink.}
          \label{tab:zink}
                \begin{tabular}{S[table-format=2.0] S[table-format=3.0] S[table-format=4.0,table-figures-uncertainty=3] S[table-format=3.1, table-figures-uncertainty=2]}
                \toprule
                        {$d/$mm} & {$t/$s} & {$N_\mathrm{Zn} \pm \sigma_{N_\mathrm{Zn}}$} & {$(A_\mathrm{z}) \pm \sigma_{N_\mathrm{Zn}})\,\cdot\, $s}\\
                        \midrule
                         2 &  40 &  5050 \pm  80 & 125.2 \pm 1.8 \\
                         4 &  60 &  7230 \pm  90 & 119.4 \pm 1.4 \\
                         6 &  80 &  8750 \pm 100 & 108.3 \pm 1.2 \\
                         8 & 100 & 10480 \pm 110 & 104.0 \pm 1.0 \\
                        10 & 100 &  9320 \pm 100 &  92.0 \pm 1.0 \\
                        12 & 100 &  8470 \pm 100 &  83.7 \pm 0.9 \\
                        14 & 100 &  7610 \pm  90 &  75.1 \pm 0.9 \\
                        16 & 100 &  7090 \pm  90 &  69.8 \pm 0.9 \\
                        18 & 100 &  6520 \pm  90 &  64.1 \pm 0.8 \\
                        20 & 100 &  5790 \pm  80 &  56.9 \pm 0.7 \\
                \bottomrule
                \end{tabular}
        \end{center}
\end{table}

Da das Absorptionsgesetz nach Gleichung \eqref{eqn:absorptionsgesetz} einen exponentiellen Zusammenhang darstellt,
wird für die Ausgleichsfunktion der Ansatz
\begin{equation*}
  A_\mathrm{Zn} = A_{0,\,\text{Zn}} \cdot \exp(-\mu_{\text{exp},\,\text{Zn}}\cdot d)
\end{equation*}
gewählt, wobei die Parameter die Aktivität ohne Absorption $A_{0,\,\text{Zn}}$ und der experimentell bestimmte Wert für den
Absorptionskoeffizient $\mu_{\text{exp},\,\text{Zn}}$ sind.
Für die konkreten Messwerte ergeben sich die Parameter zu
\begin{align*}
  A_{0,\,\text{Zn}} &= \SI{140(21)}{\per\second}\,,\\
  \mu_{\text{exp},\,\text{Zn}} &=  \SI{43.4(15)}{\per\meter}\,.
\end{align*}
\\
Die Messwerte und der Graph der Ausgleichsfunktion sind in Abbildung \ref{fig:zink}
dargestellt. Dabei ist die $y$-Achse logarithmiert.

\begin{figure}
  \centering
  \includegraphics{build/zink.pdf}
  \caption{Auftragung der gemessenen Aktivität in Abhängigkeit der Dicke des absorbierenden Zinks und Graph der Ausgleichsfunktion.}
  \label{fig:zink}
\end{figure}

Die Werte für die Messung für Eisen als Absorbermaterial sind in Tabelle \ref{tab:eisen}
zu sehen.

\begin{table}[htp]
        \begin{center}
          \caption{Messwerte zur Absorption von Gammastrahlung durch Eisen.}
          \label{tab:eisen}
                \begin{tabular}{cccc}
                \toprule
                        {$d/$mm} & {$t/$s} & {$N_\mathrm{Fe} \pm \sigma_{N_\mathrm{Fe}}$} & {$(A_\mathrm{Fe} \pm \sigma_{N_\mathrm{Fe}})\,\cdot\, $s}\\
                        \midrule
                         5 &  50 & 6040 \pm 80 & 119.7 \pm 1.6 \\
                        10 &  70 & 6540 \pm 90 &  92.4 \pm 1.2 \\
                        15 &  90 & 6395 \pm 80 &  70.0 \pm 0.9 \\
                        25 & 100 & 5786 \pm 80 &  56.8 \pm 0.8 \\
                        30 & 110 & 4971 \pm 80 &  44.2 \pm 0.7 \\
                        35 & 110 & 3785 \pm 70 &  33.4 \pm 0.6 \\
                        40 & 110 & 3099 \pm 60 &  27.1 \pm 0.5 \\
                        45 & 110 & 2568 \pm 60 &  22.3 \pm 0.5 \\
                        50 & 110 & 1874 \pm 50 &  16.0 \pm 0.4 \\
                        55 & 110 & 1354 \pm 40 &  11.3 \pm 0.4 \\
                        60 & 110 & 1004 \pm 40 &   8.1 \pm 0.3 \\
                        65 & 130 &  945 \pm 40 &   6.2 \pm 0.3 \\
                \bottomrule
                \end{tabular}
        \end{center}
\end{table}

Für Eisen wird für die Ausgleichsfunktion
\begin{equation*}
  A_\mathrm{Fe} = A_{0,\,\text{Fe}} \cdot \exp(-\mu_{\text{exp},\,\text{Fe}} \cdot d)
\end{equation*}
angesetzt, wobei die Parameter die entsprechende Bedeutung haben.
Für sie folgt nach Ausgleichsrechnung
\begin{align*}
  A_{0,\,\text{Fe}} &= \SI{145(5)}{\per\second}\,,\\
  \mu_{\text{exp},\,\text{Fe}} &=  \SI{42.5(18)}{\per\meter}\,.
\end{align*}
\\
Die Messwerte und der Graph der Ausgleichsfunktion sind in Abbildung \ref{fig:eisen}
zu sehen. Dabei wird die $y$-Achse erneut logarithmiert.

\begin{figure}
  \centering
  \includegraphics{build/eisen.pdf}
  \caption{Auftragung der gemessenen Aktivität in Abhängigkeit der Dicke des absorbierenden Eisens und Graph der Ausgleichsfunktion.}
  \label{fig:eisen}
\end{figure}

Es ist möglich, die experimentell bestimmten Werte für die Absorptionskoeffizienten mit
Theoriewerten zu vergleichen. Wird angenommen, dass die Absorption ausschließlich durch
Comptonstreuung geschieht, so lässt sich der Absorptionskoeffizient für diese Annahme
mit den Gleichungen \eqref{eqn:sigma_compton} und \eqref{eqn:mucompton} berechnen.
Für den verwendeten Strahler gilt $\epsilon = 1{,}295$. Außerdem wird für das molare Volumen
von Zink $\SI{9.157e-6}{\cubic\meter\per\mole}$ und für Eisen $\SI{7.0923e-6}{\cubic\meter\per\mole}$
verwendet. Diese Werte sind \cite{molarvolume} zu entnehmen. Es ergeben sich die folgenden Theoriewerte:
\begin{align*}
  \mu_{\text{com},\,\text{Zn}} &=  \SI{50.544}{\per\meter}\,,\\
  \mu_{\text{com},\,\text{Fe}} &=  \SI{56.557}{\per\meter}\,.
\end{align*}
Die relativen Fehler ergeben sich dann zu
\begin{align*}
  \frac{\mu_{\text{exp},\,\text{Zn}} - \mu_{\text{com},\,\text{Zn}}}{\mu_{\text{com},\,\text{Zn}}} &= -14{,}13 \% \,,\\
  \frac{\mu_{\text{exp},\,\text{Fe}} - \mu_{\text{com},\,\text{Fe}}}{\mu_{\text{com},\,\text{Fe}}} &= -24{,}85 \% \,.
\end{align*}

\subsection{Untersuchung der Absorption von Betastrahlung}
\label{subsec:auswertungbeta}

Die Messreihe der Ereignisse in Abhängigkeit der Dicke der absorbierenden Aluminiumplattenschicht
ist in Tabelle \ref{tab:aluminium} eingetragen. Auch sind die Unsicherheiten der Dicke, der Ereignisse und
die Aktivität mit Unsicherheit zu sehen. In diesem Versuchsteil wird ein Nulleffekt nicht berücksichtigt.

\begin{table}[htp]
        \begin{center}
          \caption{Messwerte zur Absorption von Betastrahlung durch Aluminium.}
          \label{tab:aluminium}
                \begin{tabular}{S[table-format=3.0,table-figures-uncertainty=2] S[table-format=3.0] S[table-format=4.0,table-figures-uncertainty=3] S[table-format=3.2, table-figures-uncertainty=3]}
                \toprule
                        {$(d \pm \sigma_\mathrm{d})/$mm} & {$t/$s} & {$N_\beta \pm \sigma_{N_\beta}$} & {$(A_\beta \pm \sigma_{N_\beta})\,\cdot\, $s}\\
                        \midrule
                        100         &  60 & 2430 \pm 50 & 40.6   \pm 0.9\\
                        125         &  60 &  576 \pm 24 &  9.60  \pm 0.40\\
                        \,\,\,\,\,\,\text{153 \pm \,0,5} &  60 &  630 \pm 26 & 10.50  \pm 0.50\\
                        160 \pm 1   &  60 &  362 \pm 20 &  6.03  \pm 0.40\\
                        200 \pm 1   & 200 &  416 \pm 21 &  2.08  \pm 0.11\\
                        253 \pm 1   & 484 &  417 \pm 21 &  0.86  \pm 0.05\\
                        302 \pm 1   & 620 &  408 \pm 21 &  0.66  \pm 0.04\\
                        338 \pm 5   & 650 &  460 \pm 22 &  0.71  \pm 0.04\\
                        400 \pm 1   & 660 &  403 \pm 21 &  0.61  \pm 0.04\\
                        444 \pm 2   & 600 &  420 \pm 21 &  0.70  \pm 0.04\\
                \bottomrule
                \end{tabular}
        \end{center}
\end{table}

Im Folgenden wird die Aktivität $A_\beta$ in Abhängigkeit der Massenbelegung $R = \rho d$
untersucht, wobei $\rho = \SI{2700}{\kilo\gram\per\meter\cubed}$ die Dichte von Aluminium ist.
Analog dem theoretisch zu erwartenden Verlauf der Absorptionskurve, zu sehen in Abbildung \ref{fig:absorptionskurve},
werden zwei getrennte lineare Ausgleichsrechnungen durchgeführt. Es ist darauf zu achten, die Aktivitäten
einheitenlos machen, was durch Multiplikation mit einer Sekunde geschieht, und zu logarithmieren. Die Fehler werden entsprechend behandelt. Der vierte und der fünfte Messwert
werden für die Ausgleichsrechnungen nicht berücksichtigt, da sich diese näherungsweise im gekrümmten Bereich
der Kurve befinden. Auf das Ausnehmen von Punkten und den Folgen daraus wird genauer in der Diskussion eingegangen.
Der Ansatz für die beiden Ausgleichsfunktionen ist
\begin{equation*}
  A_i(R) = a_i R + b_i
\end{equation*}
mit $i=1{,}2$, wobei $i=1$ die erste Gerade mit negativer Steigung bezeichnet. Der Untergrund
wird durch die Gerade mit $i=2$ beschrieben.
Die Ausgleichsrechnung ergibt konkret die Parameter
\begin{align*}
  a_1 &= \SI{-9(6)}{\meter\squared\per\kilo\gram}\,,\\
  b_1 &= \SI{5.9(22)}{}\,,\\
  a_2 &= \SI{0.0(28)}{\meter\squared\per\kilo\gram}\,,\\
  b_2 &= \SI{-0.40(28)}\,.
\end{align*}
Die Auftragung der Messwerte und die Graphen der Ausgleichsfunktionen sind in Abbildung \ref{fig:beta} zu sehen.
\begin{figure}
  \centering
  \includegraphics{build/beta.pdf}
  \caption{Auftragung der logarithmierten einheitenlosen Aktivität in Abhängigkeit der Massenbelegung des absorbierenden Aluminiums und Graph der Ausgleichsfunktionen.}
  \label{fig:beta}
\end{figure}
Der Schnittpunkt beider Ausgleichsfunktionen wird $R_\text{max}$ genannt und ergibt sich zu
\begin{equation*}
  R_\text{max} = \frac{b_2-b_1}{a_2-a_1} = \SI{0.68(25)}{\kilo\gram\per\meter\squared}\,.
\end{equation*}
Der Fehler dieser Größe wird durch Anwendung von Gleichung \eqref{eqn:gaussfehler}
zu
\begin{equation*}
  \sigma_{R_\text{max}} =
  \sqrt{\frac{\sigma_{a_{1}}^{2} \left(- b_{1} + b_{2}\right)^{2}}{\left(a_{1} - a_{2}\right)^{4}}
  + \frac{\sigma_{a_{2}}^{2} \left(- b_{1} + b_{2}\right)^{2}}{\left(a_{1} - a_{2}\right)^{4}}
  + \frac{\sigma_{b_{1}}^{2}}{\left(a_{1} - a_{2}\right)^{2}} + \frac{\sigma_{b_{2}}^{2}}{\left(a_{1} - a_{2}\right)^{2}}}
\end{equation*}
berechnet.

Mit Gleichung \eqref{eqn:emax} folgt für die maximale Energie des Strahlers
\begin{equation*}
  E_\text{max} = \SI{0.27(6)}{\mega\electronvolt}\,.
\end{equation*}
Erneut wird der Fehler mithilfe von Gleichung \eqref{eqn:gaussfehler} berechnet und beträgt
\begin{equation*}
  \sigma_{E_\text{max}} = 1.92 \sqrt{\frac{\sigma_{R}^{2} \left(R + 0.11\right)^{2}}{R^{2} + 0.22 R}} \,.
\end{equation*}
Der Theoriewert beträgt $\SI{0.294}{\mega\electronvolt}$ \cite{energy}, sodass sich der relative
Fehler des experimentell bestimmten Wertes zu $-8{,}16\%$ ergibt.
