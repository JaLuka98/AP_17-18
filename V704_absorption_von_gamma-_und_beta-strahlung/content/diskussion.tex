\section{Diskussion}
\label{sec:Diskussion}
Die Messung der Absorptionskoeffizienten über die Absorption von Gammastrahlung führt
zu signifikanten Abweichungen zu den theoretisch berechneten Comptonabsorptionskoeffizienten.
Dies kann dadurch erklärt werden, dass die Gammastrahlung auch auf andere Arten mit dem Material
wechselwirkt, zum Beispiel durch den Photoeffekt. Paarerzeugung ist ausgeschlossen,
da die Gammastrahlung nicht über ausreichend Energie verfügt. Außerdem können systematische
Fehler im Versuchsaufbau wie zum Beispiel bei der Justierung der Platten zu den Abweichungen führen.
Zuletzt können auch statistische Fehler zu diesen beitragen, da eine perfekt genaue Einstellung
der Plattendichte nicht gewährleistet werden kann es sich beim radioaktiven Zerfall um einen
Zufallsprozess handelt, der in endlicher Zeit keine perfekt genaue Messungen ermöglicht.
Es ist jedoch zu bemerken, dass die Abweichung bei Zink deutlich geringer als bei Eisen ist, sodass
davon ausgegangen werden kann, dass der Comptoneffekt bei der Absorption von Gammastrahlung
durch einen Zinkabsorber eine große Rolle spielt, bei Eisen scheint diese geringer zu sein.
Insgesamt ist allerdings zu erkennen, dass die aufgenommenen Messreihen in halblogarithmischen Diagrammen
in guter Näherung auf einer Geraden liegen, sodass das Messziel qualitativ erreicht wurde.\\
Die Untersuchung der Betastrahlung ist kritisch zu bewerten. Dies liegt daran, dass die Messwerte im gewissen Maße
von der theoretisch vorhergesagten Absorptionskurve abweichen. Sie fallen, wenn sie logarithmiert werden,
erst linear ab und erreichen nach einer Krümmung einen konstanten Wert, der die Untergrundstrahlung darstellt, und
folgen somit grob der erwarteten Verteilung. Die Domination des Absorptionsspektrums durch Bremsstrahlung
für große Energien konnte folglich qualitativ nachgewiesen werden. \\
Es soll kurz die Wahl der Punkte für die
Ausgleichsrechnungen erläutert werden. Es erscheint nicht sinnvoll, nur zwei Punkte für eine lineare Ausgleichsrechnung
zu wählen, da die Parameter nicht fehlerbehaftet sind und somit sehr sensitiv gegenüber systematischen und
statistischen Fehlern in den beiden Messungen sind. Deswegen ist es zu empfehlen, die erste Ausgleichsgerade
durch mindestens drei Punkte hindurchzulegen. Dabei wurden nur die ersten drei gewählt, da ansatzweise
zu erkennen ist, dass die nächsten Punkte im Bereich der Krümmung liegen, sodass die restlichen für
die Anpassungsfunktion des Untergrunds genommen werden. Es wird explizit angemerkt, dass insbesondere
der zweite und der dritte Messwert unerwartete Effekte zeigen; so ist es nicht zu erwarten, dass
die Absorption sich bei Erhöhung der Dicke wieder erhöht. Dieser Effekt ist wahrscheinlich groben
Messfehlern zuzuordnen, da die Messzeit und die Ereignisse beide groß genug waren, um die Auswirkungen
statistischer Fehler gering zu halten und systematische Fehler aufgrund fehlender Tendenz der Messwerte
recht sicher auszuschließen sind. Diese Ungenauigkeiten haben zur Folge, dass die herausgenommenen Messwerte
nicht rechts, sondern links des Graphen der ersten Ausgleichsfunktion ist, was aufgrund der Krümmung
der Theoriekurve nicht zu erwarten ist. Insgesamt ergeben diese Inkonsistenzen durchweg hohe
Fehler der Parameter. Trotz dieser nicht zu vernachlässigenden Probleme mit dem Datensatz
ergibt sich eine relative Abweichung vom Theoriewert, die durchaus innerhalb der Messungenauigkeit liegen kann. Allerdings
liegt der Theoriewert nicht innerhalb einer Standardabweichung des experimentell bestimmten
Wertes, sodass die Messung als nicht erfolgreich betrachtet werden muss.
