\section{Theorie}
\label{sec:Theorie}
In diesem Versuch werden grundsätzlich nur Festkörper untersucht. Diese bestehen
aus Atomen oder Molekülen, die über Ruhelagen auf ihren Gitterplätzen verfügen und
rückwirkenden Kräften ausgesetzt sind, falls sie aus der Gleichgewichtslage ausgelenkt
werden. Deswegen können im Modell als $n$ gekoppelte harmonische Oszillatoren
mit Schwingungsmöglichkeiten in allen drei Raumdimensionen beschrieben werden.
\subsection{Beschreibung in der klassischen Physik}
In der klassischen Physik, also ohne Betrachtung relativistischer und quantenmechanischer
Effekte, ist die Hamiltonfunktion identisch mit der Energie eines harmonischen
Oszillators und beträgt
\begin{equation}
  H = \frac{\symbf{p}^2}{2 m} + \frac{1}{2} m \omega^2 \symbf{r}^2\,.
  \label{eqn:hamilton}
\end{equation}
Dabei ist $\symbf{p}$ der klassische Impuls, $m$ bezeichnet die Masse des modellierten
Teilchens, $\omega$ die Kreisfrequenz der harmonischen Schwingung und $\symbf{r}$
den Ortsvektor im dreidimensionalen Raum. Der erste Term ist die klassische kinetische
Energie und der zweite Term ist die potenzielle Energie. Die Energie des gesamten
Systems lässt sich durch Summation über alle $n$ Gitterplätze erhalten.\newline
Zur weiteren Berechnung wird der Gleichverteilungssatz, auch Äquipartitionstheorem, zur Hilfe genommen:
Ist das betrachtete System mit der Hamiltonfunktion $H$ im thermischen Gleichgewicht, so gilt
\begin{equation}
  \Bigg \langle x_i \frac{\partial H}{\partial x_j} \Bigg \rangle = k_\text{B} T \delta_{ij}\,.
\end{equation}
Dabei bezeichnet $x_i$ entweder einen Ort oder einen Impuls, $k_\text{B}$ ist die Boltzmann-Konstante,
$T$ ist die Temperatur des Systems und $\delta_ij$ ist das Kronecker-Delta.
