\section{Theorie}
\label{sec:Theorie}
\subsection{Der Begriff der Wärmekapazität}
Wird die Temperatur eines Körpers um $\Delta T$ erhöht, ohne dass Arbeit geleistet
wird, so wird ihm die Wärmemenge $\Delta Q$ hinzugefügt. Dies kann durch die Gleichung
\begin{equation}
  \Delta Q = m c \Delta T
  \label{eqn:specificheat}
\end{equation}
ausgedrückt werden. Dabei ist $m$ die Masse des Körpers. Die Proportionalitätskonstante
$c$ wird dabei als spezifische Wärmekapazität bezeichnet und wird dem Material,
aus dem der Körper besteht, zugeordnet. \newline
Es ist auch möglich, die molare Wärmekapazität zu definieren. Dann lautet die
Beziehung zwischen Wärmemenge und Temperaturänderung
\begin{equation}
  \Delta Q = n C_\text{m} \Delta T
  \label{eqn:molarheat}
\end{equation}
mit der Stoffmenge $n$ und der molaren Wärmekapazität $C_\text{m}$. \newline
Es ist auch wichtig, unter welchen Bedingungen die Wärme dem Körper zugeführt wird.
Bei der spezifischen Wärmekapazität lässt sich zwischen Prozessen unterscheiden,
die bei konstantem Volumen oder konstantem Druck ablaufen. Bei ersterem gilt
dann für die spezifische Wärmekapazität bei konstantem Volumen
\begin{equation}
  C_V = \left(\frac{\symup{d}Q}{\symup{d}T}\right)_V = \left(\frac{\symup{d}U}{\symup{d}T}\right)_V\,.
  \label{eqn:cv}
\end{equation}
In diesem Versuch werden grundsätzlich nur Festkörper untersucht. Diese bestehen
aus Atomen oder Molekülen, die über Ruhelagen auf ihren Gitterplätzen verfügen und
rückwirkenden Kräften ausgesetzt sind, falls sie aus der Gleichgewichtslage ausgelenkt
werden. Deswegen können sie im Modell als $N$ gekoppelte harmonische Oszillatoren
mit Schwingungsmöglichkeiten in allen drei Raumdimensionen beschrieben werden.
\subsection{Beschreibung eines Festkörpers in der klassischen Physik}
In der klassischen Physik, also ohne Betrachtung relativistischer und quantenmechanischer
Effekte, ist die Hamiltonfunktion mit der Energie eines harmonischen
Oszillators identisch. Die Gesamtenergie kann für das vorliegende Modell ausgedrückt werden als
\begin{equation}
  H = \sum_{i=1}^N (\frac{\symbf{p}_i^2}{2 m} + \frac{1}{2} m \omega^2 \symbf{r}_i^2)\,.
  \label{eqn:hamilton}
\end{equation}
Dabei ist $\symbf{p}_i$ der klassische Impuls des $i$-ten Teilchens,
$m$ bezeichnet seine Masse, $\omega$ die Kreisfrequenz der harmonischen Schwingung und $\symbf{r}_i$
den Ortsvektor des $i$-ten Teilchens im dreidimensionalen Raum. Der erste Term ist die klassische kinetische
Energie und der zweite Term ist die potenzielle Energie. Die Energie des gesamten
Systems lässt sich durch Summation über alle $n$ Gitterplätze erhalten.\newline
Zur weiteren Berechnung wird der Gleichverteilungssatz, auch
Äquipartitionstheorem, zur Hilfe genommen.
Dieses besagt, dass der Mittelwert jedes in Ort oder Impuls quadratischen Terms
in der Hamiltonfunktion $k_\text{B} T / 2$ beträgt.
Dabei ist $k_\text{B}$ die Boltzmann-Konstante und $T$ die Temperatur des Systems.
Das bedeutet insbesondere, dass die mittlere Energie gleich über alle
Freiheitsgrade verteilt ist, sollten alle Terme der Hamiltonfunktion quadratisch
in Ort und Impuls eingehen. Einzelheiten zu diesem Satz und sein Beweis können
in \cite{Gleichverteilungssatz} nachgelesen werden.\newline
Identifiziert man nun die Energie auch mit der inneren Energie (vergleiche dazu \cite{failure}), folgt aus dem Gleichverteilungssatz
und Gleichung \eqref{eqn:hamilton} direkt für die innere Energie des Systems
\begin{equation}
  U = 3 N k_\text{B} T\,.
  \label{eqn:usolid}
\end{equation}
Nach Gleichung \eqref{eqn:cv} folgt sofort für die spezifische Wärmekapazität
bei konstantem Volumen
\begin{equation}
  C_V = 3 N k_\text{B}\,.
  \label{eqn:cvsolid}
\end{equation}
Es gilt nach \cite{failure} auch $R = N_\text{A} k_\text{B}$ mit der universellen Gaskonstante $R$ und der
Avogadro-Konstante $N_\text{A}$, sodass für die molare Wärmekapazität
\begin{equation}
  C_\text{m} = 3 R
  \label{eqn:dulongpetit}
\end{equation}
folgt. Identifiziert man das modellierte System mit dem Festkörper, so lässt sich
das Dulong-Petitsche Gesetz formulieren: \newline
Die Molare Wärmekapazität eines Festkörpers beträgt $3 R$. Insbesondere ist dieser
Wert unabhängig von der Temperatur und allgemein konstant.
\subsection{Quantenmechanische Betrachtung eines Festkörpers}
