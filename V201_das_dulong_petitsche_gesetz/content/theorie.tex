\section{Theorie}
\label{sec:Theorie}
\subsection{Der Begriff der Wärmekapazität}
Wird die Temperatur eines Körpers um $\Delta T$ erhöht, ohne dass Arbeit geleistet
wird, so wird ihm die Wärmemenge $\Delta Q$ hinzugefügt. Dies kann durch die Gleichung
\begin{equation}
  \Delta Q = m c \Delta T
  \label{eqn:specificheat}
\end{equation}
ausgedrückt werden. Dabei ist $m$ die Masse des Körpers. Die Proportionalitätskonstante
$c$ wird dabei als spezifische Wärmekapazität bezeichnet und wird dem Material,
aus dem der Körper besteht, zugeordnet. \newline
Es ist auch möglich, die molare Wärmekapazität zu definieren. Dann lautet die
Beziehung zwischen Wärmemenge und Temperaturänderung
\begin{equation}
  \Delta Q = n C_\text{m} \Delta T
  \label{eqn:molarheat}
\end{equation}
mit der Stoffmenge $n$ und der molaren Wärmekapazität $C_\text{m}$. \newline
Es ist auch wichtig, unter welchen Bedingungen die Wärme dem Körper zugeführt wird.
Bei der spezifischen Wärmekapazität lässt sich zwischen Prozessen unterscheiden,
die bei konstantem Volumen oder konstantem Druck ablaufen. Bei ersterem gilt
dann für die spezifische Wärmekapazität bei konstantem Volumen
\begin{equation}
  C_V = \left(\frac{\symup{d}Q}{\symup{d}T}\right)_V = \left(\frac{\symup{d}U}{\symup{d}T}\right)_V\,.
  \label{eqn:cv}
\end{equation}
In diesem Versuch werden grundsätzlich nur Festkörper untersucht. Diese bestehen
aus Atomen oder Molekülen, die über Ruhelagen auf ihren Gitterplätzen verfügen und
rückwirkenden Kräften ausgesetzt sind, falls sie aus der Gleichgewichtslage ausgelenkt
werden. Deswegen können sie im Modell als $N$ gekoppelte harmonische Oszillatoren
mit Schwingungsmöglichkeiten in allen drei Raumdimensionen beschrieben werden.

Wird bei der Messung mit konstantem Druck, statt mit konstantem Volumen gearbeitet,
so muss der Zusammenhang
\begin{align}
  C_{\symup{V}}&= C_{\symup{P}} -9\alpha^2 K V_0 T \\
  &=c_{\symup{k}}M-9\alpha^2 K\frac{M}{\rho}T_{\symup{m}}
  \label{eqn:druckvolumen}
\end{align}
berücksichtigt werden. Dabei ist $M$ die molare Masse, $\alpha$ ein linearer Ausdehnungskoeffizient,
$K$ das Kompressionsmodul, $\rho$ die Dichte der Probe und $T_{\symup{m}}$ die Mischtemperatur.
$C_{\symup{P}}$ bezeichnet die bei konstantem Druck ermittelte spezifische Wärme
des Probekörpers und $V_0$ das Molvolumen.

\subsection{Beschreibung eines Festkörpers in der klassischen Physik}
In der klassischen Physik, also ohne Betrachtung relativistischer und quantenmechanischer
Effekte, ist die Hamiltonfunktion mit der Energie eines harmonischen
Oszillators identisch. Die Gesamtenergie kann für das vorliegende Modell ausgedrückt werden als
\begin{equation}
  H = \sum_{i=1}^N \left(\frac{\symbf{p}_i^2}{2 m} + \frac{1}{2} m \omega^2 \symbf{r}_i^2\right)\,.
  \label{eqn:hamilton}
\end{equation}
Dabei ist $\symbf{p}_i$ der klassische Impuls des $i$-ten Teilchens,
$m$ bezeichnet seine Masse, $\omega$ die Kreisfrequenz der harmonischen Schwingung und $\symbf{r}_i$
den Ortsvektor des $i$-ten Teilchens im dreidimensionalen Raum. Der erste Term ist die klassische kinetische
Energie und der zweite Term ist die potenzielle Energie. Die Energie des gesamten
Systems lässt sich durch Summation über alle $n$ Gitterplätze erhalten.\newline
Zur weiteren Berechnung wird der Gleichverteilungssatz, auch
Äquipartitionstheorem, zur Hilfe genommen.
Dieses besagt, dass der Mittelwert jedes in Ort oder Impuls quadratischen Terms
in der Hamiltonfunktion $k_\text{B} T / 2$ beträgt.
Dabei ist $k_\text{B}$ die Boltzmann-Konstante und $T$ die Temperatur des Systems.
Das bedeutet insbesondere, dass die mittlere Energie gleich über alle
Freiheitsgrade verteilt ist, sollten alle Terme der Hamiltonfunktion quadratisch
in Ort und Impuls eingehen. Einzelheiten zu diesem Satz und sein Beweis können
in \cite{Gleichverteilungssatz} nachgelesen werden.\newline
Identifiziert man nun die Energie auch mit der inneren Energie (vergleiche dazu \cite{failure}), folgt aus dem Gleichverteilungssatz
und Gleichung \eqref{eqn:hamilton} direkt für die innere Energie des Systems
\begin{equation}
  U = 3 N k_\text{B} T\,.
  \label{eqn:usolid}
\end{equation}
Nach Gleichung \eqref{eqn:cv} folgt sofort für die spezifische Wärmekapazität
bei konstantem Volumen
\begin{equation}
  C_V = 3 N k_\text{B}\,.
  \label{eqn:cvsolid}
\end{equation}
Es gilt nach \cite{failure} auch $R = N_\text{A} k_\text{B}$ mit der universellen Gaskonstante $R$ und der
Avogadro-Konstante $N_\text{A}$, sodass für die molare Wärmekapazität
\begin{equation}
  C_\text{m} = 3 R
  \label{eqn:dulongpetit}
\end{equation}
folgt. Identifiziert man das modellierte System mit dem Festkörper, so lässt sich
das Dulong-Petitsche Gesetz formulieren: \newline
Die Molare Wärmekapazität eines Festkörpers beträgt $3 R$. Insbesondere ist dieser
Wert unabhängig von der Temperatur und allgemein konstant.
\subsection{Quantenmechanische Betrachtung eines Festkörpers}
In der Quantenmechanik ist die Energie des harmonischen Oszillators quantisiert.
Schwingt er mit der Kreisfrequenz $\omega$, so kann sich seine Gesamtenergie
nur in Paketen der Größe
\begin{equation}
  \Delta E = n \hbar \omega
  \label{eqn:equantum}
\end{equation}
ändern, wobei $n$ eine natürliche Zahl und $\hbar = h / 2 \pi$ das reduzierte Plancksche
Wirkungsquantum mit dem Planckschen Wirkungsquantum $h$ ist. Daraus folgt, dass
die zeitlich gemittelte Energie der harmonischen Oszillatoren im Festkörper
anders als in der klassischen Theorie von der Temperatur $T$ abhängt. Wird angenommen,
dass die Verteilung der Energien der Teilchen im Festkörper Boltzmann-verteilt ist,
ergibt sich für das zeitliche Mittel der inneren Energie der Oszillatoren
\begin{equation}
  \langle u_\text{qu} \rangle = \frac{\hbar \omega}{\exp(\hbar \omega / K_\text{B} T) - 1}\,.
  \label{eqn:uqu}
\end{equation}
Die mittlere Energie eines Festkörpers pro Mol ergibt sich dann zu
\begin{equation}
  \langle u_\text{qu} \rangle = \frac{3 N_\text{A} \hbar \omega}{\exp(\hbar \omega / K_\text{B} T) - 1}\,
  \label{eqn:uqu}
\end{equation}
Die mittlere Energie ist somit eine Funktion der Temperatur. Im Grenzwert sehr hoher
Temperaturen ergibt sich jedoch durch Taylorreihenentwicklung der obigen Gleichung
die Gleichheit mit dem klassischen Ergebnis $3 R T$ (siehe \eqref{eqn:cvsolid}). \newline
Abschließend lässt sich also formulieren, dass das Dulong-Petitsche Gesetz als Grenzfall
des quantenmechanischen Modells die molare Wärmekapazität von Festkörpern für hohe
Temperaturen zuverlässig vorhersagt. Dies ist gleichbedeutend damit, dass die quantenmechanischen
Oszillatoren genügend Energie besitzen, um viele ihrer Zustände zu durchlaufen.
Für tiefe Temperaturen versagt es jedoch. Die molare Wärmekapazität geht dann sogar gegen
null.

\subsection{Beschreibung der für die Auswertung notwendigen Zusammenhänge}

Zur Auswertung des Versuchs werden einige Formeln benötigt. Diese sollen im Folgenden kurz
erklärt werden. Das Messbehältnis ist dabei ein Kaloriemeter.

Da die von dem erhitzen Wasser abgegebene Wärmemenge unter der Annahme, dass keine
Arbeit verrichtet wird und keine Energie das System verlässt, genau so groß sein
muss, wie die von dem kalten Wasser und den Wänden des Kaloriemeters aufgenommene
Wärmemenge, lässt sich schreiben:
\begin{equation}
  (c_{\symup{w}}m_{\symup{x}}+c_{\symup{g}}m_{\symup{g}})(T_{\symup{m}}-T_{\symup{x}})
  =c_{\symup{w}}m_{\symup{y}}(T_{\symup{y}}-T_{\symup{m}}) \,.
\end{equation}
Dabei ist $c_{\symup{w}}$ die spezifische Wärmekapazität des Wassers,
$m_{\symup{x}}$ die Masse des Wassers im Kaloriemeter, $T_{\symup{x}}$ die
Temperatur des Wassers im Kaloriemeter, $m_{\symup{y}}$ die Masse des
im Becherglas zu erhitzenden Wassers, $T_{\symup{y}}$ die zugehörige Temperatur
und $T_{\symup{m}}$ die Mischtemperatur im Kaloriemeter nach dem Mischen. $c_{\symup{g}}m_{\symup{g}}$
ist die Wärmekapazität des Kaloriemeters und $c_{\symup{w}}$ ist die spezifische Wärmekapazität
von Wasser.
Wird diese Formel nach der Wärmekapazität des Kaloriemeters umgestellt, so ergibt sich:
\begin{equation}
  c_{\symup{g}}m_{\symup{g}}=\frac{c_{\symup{w}}m_{\symup{y}}(T_{\symup{y}}-T_{\symup{m}})
  -c_{\symup{w}}m_{\symup{x}}(T_{\symup{m}}-T_{\symup{x}})}{(T_{\symup{m}}-T_{\symup{x}})} \,.
  \label{eqn:kaloriemeter}
\end{equation}

Auch für die spezifischen Wärmekapazitäten der Probekörper lässt sich durch das Gleichsetzen der abgegebenen und der aufgenommenen
Wärmemengen eine Beziehung zur Berechnung der spezifischen Wärmekapazitäten bestimmen:
\begin{equation}
  c_{\symup{k}}=\frac{(c_{\symup{w}}m_{\symup{w}}+c_{\symup{g}}m_{\symup{g}})
  (T_{\symup{m}}-T_{\symup{w}})}{m_{\symup{k}}(T_{\symup{k}}-T_{\symup{m}})}\,.
  \label{eqn:proben}
\end{equation}
Dabei bezeichnet $c_{\symup{k}}$ die spezifische Wärmekapazität des Probekörpers,
$c_{\symup{w}}$ die spezifische Wärmekapazität von Wasser, $m_{\symup{w}}$ die
Masse des sich im Kaloriemeter befindenden Wassers, $T_{\symup{w}}$ dessen Temperatur
vor dem Wärmeaustausch und $T_{\symup{m}}$ dessen Temperatur nach dem Wärmeaustausch.
$c_{\symup{g}}m_{\symup{g}}$ bezeichnet erneut die Wärmekapazität des Kaloriemeters
und $m_{\symup{k}}$ ist die Masse des Probekörpers.
