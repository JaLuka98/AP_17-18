\section{Auswertung}
\label{sec:Auswertung}

\subsection{Bestimmung der Wärmekapazität des Kaloriemeters}
\label{sec:Auswertung_Kaloriemeter}

Da die von dem erhitzen Wasser abgegebene Wärmemenge unter der Annahme, dass keine
Arbeit verrichtet wird und keine Energie das System verlässt, genau so groß sein
muss, wie die von dem kalten Wasser und den Wänden des Kaloriemeters aufgenommene
Wärmemenge, lässt sich schreiben:
\begin{equation}
  (c_{\symup{w}}m_{\symup{x}}+c_{\symup{g}}m_{\symup{g}})(T_{\symup{m}}-T_{\symup{x}})
  =c_{\symup{w}}m_{\symup{y}}(T_{\symup{y}}-T_{\symup{m}}) \,.
\end{equation}
Dabei ist $c_{\symup{w}}$ die spezifische Wärmekapazität des Wassers,
$m_{\symup{x}}$ die Masse des Wassers im Kaloriemeter, $T_{\symup{x}}$ die
Temperatur des Wassers im Kaloriemeter, $m_{\symup{y}}$ die Masse des
im Becherglas zu erhitzenden Wassers, $T_{\symup{y}}$ die zugehörige Temperatur
und $T_{\symup{m}}$ die Mischtemperatur im Kaloriemeter nach dem Mischen. $c_{\symup{g}}m_{\symup{g}}$
ist die Wärmekapazität des Kaloriemeters und $c_{\symup{w}}$ ist die spezifische Wärmekapazität
von Wasser. Diese beträgt $\SI{4.18}{\joule\per\gram\per\kelvin}$ \cite{Versuchsanleitung}.
Wird diese Formel nach der Wärmekapazität des Kaloriemeters umgestellt, so ergibt sich:
\begin{equation}
  c_{\symup{g}}m_{\symup{g}}=\frac{c_{\symup{w}}m_{\symup{y}}(T_{\symup{y}}-T_{\symup{m}})
  -c_{\symup{w}}m_{\symup{x}}(T_{\symup{m}}-T_{\symup{x}})}{(T_{\symup{m}}-T_{\symup{x}})} \,.
  \label{eqn:kaloriemeter}
\end{equation}
Werden nun die gemessenen (und ggf. in passende Einheiten umgerechneten) Werte
\begin{align*}
  m_{\symup{x}} = \SI{340.77}{\gram}   \,,\\
  T_{\symup{x}} = \SI{294.75}{\kelvin} \,,\\
  m_{\symup{y}} = \SI{260.24}{\gram}   \,,\\
  T_{\symup{y}} = \SI{354.15}{\kelvin} \,,\\
  T_{\symup{m}} = \SI{319.05}{\kelvin} \,.\\
\end{align*}
in Gleichung \eqref{eqn:kaloriemeter} eingesetzt, so ergibt sich für die Kapazität
desselben zu
\begin{equation}
  c_{\symup{g}}m_{\symup{g}} = \SI{146.85}{\joule\per\kelvin} \,.\nonumber
\end{equation}


\subsection{Bestimmung der spezifischen Wärmekapazitäten verschiedener Stoffe}
\label{sec:Auswertung_stoffe}

Auch hier lässt sich durch das Gleichsetzen der abgegebenen und der aufgenommenen
Wärmemengen eine Beziehung zur Berechnung der spezifischen Wärmekapazitäten bestimmen:
\begin{equation}
  c_{\symup{k}}=\frac{(c_{\symup{w}}m_{\symup{w}}+c_{\symup{g}}m_{\symup{g}})
  (T_{\symup{m}}-T_{\symup{w}})}{m_{\symup{k}}(T_{\symup{k}}-T_{\symup{m}})}\,.
\end{equation}
Dabei bezeichnet $c_{\symup{k}}$ die spezifische Wärmekapazität des Probekörpers,
$c_{\symup{w}}$ die spezifische Wärmekapazität von Wasser, $m_{\symup{w}}$ die
Masse des sich im Kaloriemeter befindenden Wassers, $T_{\symup{w}}$ dessen Temperatur
vor dem Wärmeaustausch und $T_{\symup{m}}$ dessen Temperatur nach dem Wärmeaustausch.
$c_{\symup{g}}m_{\symup{g}}$ bezeichnet erneut die Wärmekapazität des Kaloriemeters
und $m_{\symup{k}}$ ist die Masse des Probekörpers.

Die Messwerte für Blei, sowie die zugehörigen berechneten spezifischen Wärmekapazitäten
befinden sich in Tabelle \ref{tab:blei}. Dabei werden die Temperaturen in Kelvin umgerechnet.
Die Masse der Bleiprobe beträgt $m_{\symup{Blei}}=\SI{530.75}{\gram}$. Die verwendete
Wassermenge beträgt $m_{\symup{w,Blei}}=\SI{593.40}{\gram}$.

\begin{table}
  \centering
  \caption{Messwerte für die Bleiprobe, sowie daraus berechnete Werte für die spezifische
  Wärmekapazität von Blei.}
  \label{tab:blei}
  \begin{tabular}{c c c c c c}
    \toprule
    & $T_{\symup{w}}/$K & $T_{\symup{k}}/$K & $T_{\symup{m}}/$K & $c_{\symup{Blei}}/\frac{J}{g K}$ \\
    \midrule
    Messung 1 & 295,65 & 363,15 & 297,35 & 0,128 \\
    Messung 2 & 297,35 & 372,25 & 299,05 & 0,115 \\
    Messung 3 & 299,15 & 372,05 & 300,85 & 0,118 \\
    \bottomrule
  \end{tabular}
\end{table}

Die spezifische Wärmekapazität wird nun mithilfe der Beziehung
\begin{equation}
  \overline{c} = \sum\limits_{i = 1}^N c_i
  \label{eqn:mean}
\end{equation}
gemittelt. Die empirische Standardabweichung ergibt sich durch den Zusammenhang
\begin{equation}
  \sigma_c = \sqrt{\frac{1}{N-1}
    \sum\limits_{i = 1}^N
    (c_i-\overline{c})^2} \,.
    \label{eqn:std}
\end{equation}

Damit ergibt sich die spezifische Wärmekapazität von Blei zu
\begin{equation}
  c_{\symup{Blei}} = \SI{0.120(0005)}{\joule\per\gram\per\kelvin} \,.\nonumber
\end{equation}
Der Literaturwert hierzu beträgt $c_{\symup{Blei,theo}}=\SI{0.130}{\joule\per\gram\per\kelvin}$ \cite{werte}.
Die relative Abweichung beträgt also $-7,69\%$.

Für die Aluminium- und Kupferprobe wird analog verfahren. Die Masse der Aluminiumprobe
beträgt $m_{\symup{Alu}}=\SI{156.23}{\gram}$ und die verwendete Wassermenge
beträgt $m_{\symup{w,Alu}}=\SI{600.77}{\gram}$. Die Werte für Aluminium sind in
Tabelle \ref{tab:alu} zu finden.

\begin{table}
  \centering
  \caption{Messwerte für die Aluminiumprobe, sowie daraus berechnete Werte für die spezifische
  Wärmekapazität von Aluminium.}
  \label{tab:alu}
  \begin{tabular}{c c c c c c}
    \toprule
    & $T_{\symup{w}}/$K & $T_{\symup{k}}/$K & $T_{\symup{m}}/$K & $c_{\symup{Alu}}/\frac{J}{g K}$ \\
    \midrule
    Messung 1 & 295,15 & 366,75 & 296,65 & 0,364 \\
    Messung 2 & 298,15 & 366,95 & 300,85 & 0,695 \\
    Messung 3 & 300,95 & 368,15 & 304,25 & 0,879 \\
    \bottomrule
  \end{tabular}
\end{table}

Der Mittelwert ergibt sich hier ebenfalls nach den Gleichungen \eqref{eqn:mean} und
\eqref{eqn:std} zu
\begin{equation}
  c_{\symup{Alu}} = \SI{0.65(021)}{\joule\per\gram\per\kelvin} \,.\nonumber
\end{equation}
Der Literaturwert liegt hier bei $c_{\symup{Alu,theo}}=\SI{0.92}{\joule\per\gram\per\kelvin}$ \cite{werte},
sodass sich die relative Abweichung des errechneten Wertes zum Literaturwert zu
$-29,35\%$ ergibt.


Die Kupferprobe besitzt eine Masse von $m_{\symup{Kupfer}}=\SI{237.61}{\gram}$.
Die hierbei verwendete Wassermenge beträgt $m_{\symup{w,Kupfer}}=\SI{619.57}{\gram}$.

\begin{table}
  \centering
  \caption{Messwerte für die Kupferprobe, sowie daraus berechnete Werte für die spezifische
  Wärmekapazität von Kupfer.}
  \label{tab:kupfer}
  \begin{tabular}{c c c c c c}
    \toprule
    & $T_{\symup{w}}/$K & $T_{\symup{k}}/$K & $T_{\symup{m}}/$K & $c_{\symup{Kupfer}}/\frac{J}{g K}$ \\
    \midrule
    Messung 1 & 295,05 & 368,05 & 296,85 & 0,291 \\
    Messung 2 & 297,15 & 369,95 & 298,75 & 0,259 \\
    Messung 3 & 298,75 & 372,25 & 300,55 & 0,289 \\
    \bottomrule
  \end{tabular}
\end{table}

Hier ist der Mittelwert nach den Gleichungen \eqref{eqn:mean} und
\eqref{eqn:std}
\begin{equation}
  c_{\symup{Kupfer}} = \SI{0.280(0015)}{\joule\per\gram\per\kelvin} \,.\nonumber
\end{equation}
Der Literaturwert für die spezifische Wärmekapazität von Kupfer beträgt
$c_{\symup{Alu,theo}}=\SI{0.39}{\joule\per\gram\per\kelvin}$ \cite{werte}.
Demnach ist die relative Abweichung $-28,21\%$.


\subsection{Bestimmung der Molwärmen der Stoffe}
\label{sec:molwaerme}

Die Molwärme der Stoffe lässt sich nach Gleichung \eqref{!!!} über den Zusammenhang
\begin{equation}
  C_{\symup{v}}=c_{\symup{k}}M-9\alpha^2 K\frac{M}{\rho}T_{\symup{m}}
\end{equation}
berechnen. Die Werte für den linearen Ausdehnungskoeffizienten $\alpha$,
das Kompressionsmodul $K$ bzw. die Kompressibilität $\kappa=\frac{1}{K}$, die molare Masse $M$ und die Dichte $\rho$
für die jeweiligen Materialien können der Versuchsanleitung \cite{Versuchsanleitung}
entnommen werden.

Damit ergeben sich für die Molwärmen die in Tabelle \ref{tab:molwaerme} dargestellten Werte.
Die Mittelwerte berechnen sich dabei mit den Gleichungen \eqref{eqn:mean} und
\eqref{eqn:std}.

\begin{table}
  \centering
  \caption{Berechnete Molwärmen der Stoffe.}
  \label{tab:molwaerme}
  \begin{tabular}{c c c c c c}
    \toprule
    & $C_{\symup{v,Blei}}/\frac{J}{mol\,K}$ & $C_{\symup{v,Alu}}/\frac{J}{mol\,K}$&
    $C_{\symup{v,Kupfer}}/\frac{J}{mol\,K}$ \\
    \midrule
    Messung 1   & 26,50 & 9,83  & 18,49 \\
    Messung 2   & 23,82 & 18,76 & 16,43 \\
    Messung 3   & 24,49 & 23,72 & 18,36 \\
    Mittelwert  & 24,9\pm1,1 & 17\pm6 & 17,8\pm0,9  \\
    \bottomrule
  \end{tabular}
\end{table}

 Die relativen Abweichungen bezogen auf den zu erwartenden Wert $3R\approx\SI{24,9434}{\joule\per\mol\per\kelvin}$
 betragen dann $-0,17\%$ für Blei, $-31,85\%$ für Aluminium und $-28,64\%$ für
 Kupfer.
