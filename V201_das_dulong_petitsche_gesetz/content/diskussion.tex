\section{Diskussion}
\label{sec:Diskussion}

Es lässt sich feststellen, dass die Abweichungen der Ergebnisse zu den Literaturwerten
im Rahmen der Messgungenauigkeit liegen. Diese ist bei diesem Versuch verhältnismäßig groß,
da die Zeitpunkte zu denen der Temperaturausgleich vollständig stattgefunden hat, nicht
eindeutig bestimmbar ist. Somit sind auch die Werte für die Mischtemperaturen $T_{\symup{m}}$
ungenau. Besonders bei den Messwerten für Aluminium zeigt sich, dass die erste Messung
sehr stark von den beiden darauf folgenden Messungen abweicht.
Ein weiterer Faktor, der die Mischtemperaturen $T_{\symup{m}}$ ungenau macht, ist die
nicht exakt bestimmbare Temperatur des Probekörpers. Lediglich die Wassertemperatur des
ihn umgebenden Wassers ist bestimmbar, jedoch kann die Temperatur des Probekörpers auch
leicht abweichen, wenn er beispielsweise nicht lange genug im Wasser erhitzt wird,
oder wenn das heiße Wasser ihn nicht vollständig umschließt.

Außerdem lässt sich feststellen, dass die Temperatur $T_{\symup{w}}$ des Wassers auch
nach Entnahme des Probekörpers noch weiter steigt, sodass die Temperatur
$T_{\symup{w}}$ des Wassers zu Beginn der nächsten Messreihe teilweise höher ist
als die Mischtemperatur $T_{\symup{m}}$. Ein möglicher Grund hierfür ist ein ungenaues
Messen der Mischtemperatur, sodass dieser Messwert zu früh aufgenommen wurde, also wenn
noch nicht der gesamte Wärmeaustausch erfolgt ist. Ein weiterer möglicher Grund hierfür
ist die nur sehr dürftig realisierte Durchmischung des Wassers. Es ist möglich, dass
sich die Wärme zum Messzeitpunkt noch nicht gleichmäßig im gesamten Medium ausgebreitet hat.
Ein weiterer möglicher Grund ist die Wärmekapazität des Deckels, die in der Rechnung
gar nicht berücksichtigt wird. Wird jedoch der Deckel auf das Kalorimeter gelegt, so findet
auch hier ein Wärmeaustausch statt, der das Wasser nachträglich weiter erwärmen könnte.

Die Messergebnisse für die Molwärme zeigen, dass die Werte für diese sich dem durch
das klassische Modell vorausgesagten Wert $3R$ annähern, jedoch kann wegen der vergleichsweise
großen Abweichungen durch Messunsicherheiten keine Aussage darüber getroffen werden, ob
eine Quantenmechanische Betrachtung hier zu genaueren Ergebnissen führen würde.
